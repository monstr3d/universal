\documentclass[10]{article}
%\documentclass[11pt]{book}
\usepackage{hyperref}
\usepackage{amsfonts,amssymb,amsmath,amsthm,cite}
%\usepackage{graphicx}
\usepackage[toc,page]{appendix}
\usepackage{nicefrac}
%% \usepackage[francais]{babel}
\usepackage[applemac]{inputenc}
\usepackage{amssymb, euscript}
\usepackage[matrix,arrow,curve]{xy}
\usepackage{graphicx}
\usepackage{tabularx}
\usepackage{float}
\usepackage{tikz}
\usepackage{slashed}
\usepackage{mathrsfs}
\usepackage{multirow}
\usepackage{rotating}

%\usepackage{mathtools}

\usetikzlibrary{matrix}
\usetikzlibrary{cd}

\usepackage{siunitx}

\usepackage{lmodern}
\usepackage[T1]{fontenc}
\usepackage[babel=true]{microtype}


\usepackage{amsfonts,cite}
\usepackage{graphicx}

%% \usepackage[francais]{babel}
\usepackage[applemac]{inputenc}


\usepackage[sc]{mathpazo}
\usepackage{environ}

\linespread{1.05}         % Palatino needs more leading (space between lines)


%\usepackage[usenames]{color}



\DeclareFontFamily{T1}{pzc}{}
\DeclareFontShape{T1}{pzc}{m}{it}{1.8 <-> pzcmi8t}{}
\DeclareMathAlphabet{\mathpzc}{T1}{pzc}{m}{it}
\DeclareMathOperator{\cosupp}{\mathfrak{cosupp}}            %% 
% the command for it is \mathpzc

\textwidth=140mm


% % % % % % % % % % % % % % % % % % % %
\theoremstyle{plain}
\newtheorem{prop}{Proposition}[section]
\newtheorem{prdf}[prop]{Proposition and Definition}
\newtheorem{lem}[prop]{Lemma}%[section]
\newtheorem{cor}[prop]{Corollary}%[section]
\newtheorem{thm}[prop]{Theorem}%[section]
\newtheorem{theorem}[prop]{Theorem}
\newtheorem{lemma}[prop]{Lemma}
\newtheorem{proposition}[prop]{Proposition}
\newtheorem{corollary}[prop]{Corollary}
\newtheorem{statement}[prop]{Statement}

\theoremstyle{definition}
\newtheorem{defn}[prop]{Definition}%[section]
\newtheorem{cordefn}[prop]{Corollary and Definition}%[section]
\newtheorem{empt}[prop]{}%[section]
\newtheorem{exm}[prop]{Example}%[section]
\newtheorem{rem}[prop]{Remark}%[section]
\newtheorem{prob}[prop]{Problem}
\newtheorem{conj}{Conjecture}       %% Hypothesis 1
\newtheorem{cond}{Condition}        %% Condition 1
%\newtheorem{axiom}[thm]{Axiom}           %% Axiom 1 modified
\newtheorem{fact}[prop]{Fact}
\newtheorem{ques}{Question}         %% Question 1
\newtheorem{answ}{Answer}           %% Answer 1
\newtheorem{notn}{Notation}        %% Notations are not numbered

\theoremstyle{definition}
\newtheorem{notation}[prop]{Notation}
\newtheorem{definition}[prop]{Definition}
\newtheorem{example}[prop]{Example}
\newtheorem{exercise}[prop]{Exercise}
\newtheorem{conclusion}[prop]{Conclusion}
\newtheorem{conjecture}[prop]{Conjecture}
\newtheorem{criterion}[prop]{Criterion}
\newtheorem{summary}[prop]{Summary}
\newtheorem{axiom}[prop]{Axiom}
\newtheorem{problem}[prop]{Problem}
%\theoremstyle{remark}
\newtheorem{remark}[prop]{Remark}

\numberwithin{equation}{section}
\newtheorem*{claim}{Claim}
\DeclareMathOperator{\Dom}{Dom}              %% domain of an operator
\newcommand{\Dslash}{{D\mkern-11.5mu/\,}}    %% Dirac operator
\newcommand{\trG}{{\rm tr}_G\;}
\newcommand{\trF}{{\rm tr}_{\cal F}\;}
\newcommand{\TR}{{\rm TR}\;}
\newcommand{\res}{{\rm res}\;}


\newcommand\ci{${\mathcal C}^{\infty}$}
\newcommand\CI{{\mathcal C}^{\infty}}
\newcommand\CIc{{\mathcal C}^{\infty}_{\text{c}}}

%\newcommand\myeq{\stackrel{\mathclap{\normalfont\mbox{def}}}{=}}
\newcommand{\rar}[1]{\stackrel{#1}{\longrightarrow}}
\newcommand{\lar}[1]{\stackrel{#1}{\longleftarrow}}

\newcommand{\nor}[1]{\left\Vert #1\right\Vert}    %\nor{x}=||x||
\newcommand{\norm}[1]{\left\| #1\right\|}    %\nor{x}=||x||
\newcommand{\vertiii}[1]{{\left\vert\kern-0.25ex\left\vert\kern-0.25ex\left\vert #1
		\right\vert\kern-0.25ex\right\vert\kern-0.25ex\right\vert}}
\newcommand{\Ga}{\Gamma}  
\newcommand{\coker}{\mathrm{coker}}                   %% short for  \Gamma
\newcommand{\Coo}{C^\infty}                  %% smooth functions
\newcommand{\dom}{\mathrm{dom}}                  %% smooth functions
\newcommand{\Cont}{C} 
\newcommand{\cl}{\overline} 
\newcommand{\Contc}{C_c} 
\newcommand{\Contb}{C_b} 
\newcommand{\Repi}{\mathrm{Rep}_{int}} 
\newcommand{\Rep}{\mathrm{Rep}} 
\newcommand{\hor}[0]{\mathrm{hor}}
\newcommand{\comp}{\operatorname{comp}}
\newcommand{\adb}{\operatorname{adb}}
\newcommand{\dist}{\operatorname{dist}}


% % % % % % % % % % % % % % % % % % % %


\usepackage[sc]{mathpazo}
\linespread{1.05}         % Palatino needs more leading (space between lines)

\newbox\ncintdbox \newbox\ncinttbox %% noncommutative integral symbols
\setbox0=\hbox{$-$} \setbox2=\hbox{$\displaystyle\int$}
\setbox\ncintdbox=\hbox{\rlap{\hbox
		to \wd2{\hskip-.125em \box2\relax\hfil}}\box0\kern.1em}
\setbox0=\hbox{$\vcenter{\hrule width 4pt}$}
\setbox2=\hbox{$\textstyle\int$} \setbox\ncinttbox=\hbox{\rlap{\hbox
		to \wd2{\hskip-.175em \box2\relax\hfil}}\box0\kern.1em}

\newcommand{\ncint}{\mathop{\mathchoice{\copy\ncintdbox}%
		{\copy\ncinttbox}{\copy\ncinttbox}%
		{\copy\ncinttbox}}\nolimits}  %% NC integral

%%% Repeated relations:
\newcommand{\xyx}{\times\cdots\times}      %% repeated product
\newcommand{\opyop}{\oplus\cdots\oplus}    %% repeated direct sum
\newcommand{\oxyox}{\otimes\cdots\otimes}  %% repeated tensor product
\newcommand{\wyw}{\wedge\cdots\wedge}      %% repeated exterior product
\newcommand{\subysub}{\subset\hdots\subset}      %% repeated subset
\newcommand{\supysup}{\supset\hdots\supset}      %% repeated supset
\newcommand\CC{\mathbb C}
\newcommand\NN{\mathbb N}
\newcommand\RR{\mathbb R}
\newcommand\ZZ{\mathbb Z}
\newcommand{\LGR}{\matcal L}
\newcommand{\rep}{\mathfrak{rep}}
\newcommand{\lift}{\mathfrak{lift}}
\newcommand{\desc}{\mathfrak{desc}}
\newcommand{\Cstar}{C^*}
\newcommand{\Cst}{C^*}
\newcommand{\Star}{*}
\newcommand{\PS}[1]{\Psi^{#1}(\GR;E)}

%%% Roman letters:
\newcommand{\id}{\mathrm{id}}                %% identity map
\newcommand{\Id}{\mathrm{Id}}                %% identity map
\newcommand{\pt}{\mathrm{pt ???}}                %% a point
\newcommand{\const}{\mathrm{const}}          %% a constant
\newcommand{\codim}{\mathrm{codim}}          %% codimension
\newcommand{\cyc}{\mathrm{cyclic}}  %% cyclic sum
\renewcommand{\d}{\mathrm{d}}       %% commutative differential
\newcommand{\dR}{\mathrm{dR}}       %% de~Rham cohomology
\newcommand{\proj}{\mathrm{proj}}                %% a projection
\newcommand*{\braket}[2]{\langle#1 {,~} #2\rangle}% right inner products
\newcommand*{\lbraket}[2]{\langle\!\langle#1{\mid}#2\rangle\!\rangle}% left inner products

\newcommand*{\Mult}{\mathcal M}% multiplier algebra
\newcommand{\Lt}{\mathcal{L}}                 %%\newcommand{\unitsv}[1]{#1^{(0)}}

\newcommand{\A}{\mathcal{A}}                 %%\newcommand{\unitsv}[1]{#1^{(0)}}
\newcommand{\units}{G^{(0)}}
\newcommand{\haars}{\{\lambda^{u}\}_{u\in\units}}
\newcommand{\shaars}{\{\lambda_{u}\}_{u\in\units}}
\newcommand{\haarsv}[2]{\{\lambda^{#2}_{#1}\}_{#2\in\unitsv{#1}}}
\newcommand{\haarv}[2]{\lambda^{#2}_{#1}}

\renewcommand{\a}{\alpha}                    %% short for  \alphapha
\DeclareMathOperator{\ad}{ad}                %% infml adjoint repn
\newcommand{\as}{\quad\mbox{as}\enspace}     %% `as' with spacing
\newcommand{\Aun}{\widetilde{\mathcal{A}}}   %% unital algebra
\newcommand{\B}{\mathcal{B}}                 %% space of distributions
\newcommand{\E}{\mathcal{E}}                 %% space of distributions
\renewcommand{\b}{\beta}                     %% short for \beta
\newcommand{\braCket}[3]{\langle#1\mathbin|#2\mathbin|#3\rangle}
%\newcommand{\braket}[2]{\langle#1\mathbin|#2\rangle} %% <w|z>
\newcommand{\C}{\mathbb{C}}                  %% complex numbers
\newcommand{\cc}{\mathbf{c}}                 %% Hochschild cycle
\DeclareMathOperator{\Cl}{C\ell}             %% Clifford algebra
\newcommand{\F}{\mathcal{F}}                 %% space of test functions
\newcommand{\G}{\mathcal{G}}                 %% 
\newcommand{\GR}{\mathcal{G}}                 %% 
\newcommand{\D}{\mathcal{D}}                 %% Moyal L^2-filtration
\renewcommand{\H}{\mathcal{H}}               %% Hilbert space
\newcommand{\half}{\tfrac{1}{2}}             %% small fraction  1/2
\newcommand{\hh}{\mathcal{H}}                %% Hilbert space
\newcommand{\hookto}{\hookrightarrow}        %% abbreviation
\newcommand{\Ht}{{\widetilde{\mathcal{H}}}}  %% Hilbert space of forms
\newcommand{\I}{\mathcal{I}}                 %% tracelike functions
\DeclareMathOperator{\Junk}{Junk}            %% the junk DGA ideal
\newcommand{\K}{\mathcal{K}}                 %% compact operators
\newcommand{\ket}[1]{|#1\rangle}             %% ket vector
\newcommand{\ketbra}[2]{|#1\rangle\langle#2|} %% rank one operator
\renewcommand{\L}{\mathcal{L}}               %% operator algebra
\newcommand{\La}{\Lambda}                    %% short for \Lambda
\newcommand{\la}{\lambda}                    %% short for \lambda
\newcommand{\lf}{L_f^\theta}                 %% left  mult operator
\newcommand{\M}{\mathcal{M}}                 %% Moyal multplr algebra
\newcommand{\Lb}{\mathcal{L}}                 %% Moyal multplr algebra
\newcommand{\mm}{\mathcal{M}^\theta}
%\newcommand{{{\star_{\theta}}}{{\mathchoice{\mathbin{\;|\;ar_{_\theta}}}
			%            {\mathbin{\;|\;ar_{_\theta}}}           %% Moyal
			%            {{\;|\;ar_\theta}}{{\;|\;ar_\theta}}}}    %% product
	\newcommand{\N}{\mathbb{N}}                  %%  integers
	\newcommand{\nb}{\nabla}                     %% gradient
	\newcommand{\Oh}{\mathcal{O}}                %% comm multiplier alg
	\newcommand{\om}{\omega}                     %% short for \omega
	\newcommand{\opp}{{\mathrm{op}}}             %% opposite algebra
	\newcommand{\ox}{\otimes}                    %% tensor product
	\newcommand{\eps}{\varepsilon}                    %% tensor product
	\newcommand{\otimesyox}{\otimes\cdots\otimes}    %% repeated tensor product
	\newcommand{\pa}{\partial}                   %% short for \partial
	\newcommand{\pd}[2]{\frac{\partial#1}{\partial#2}}%% partial derivative
	\newcommand{\piso}[1]{\lfloor#1\rfloor}      %% integer part
	\newcommand{\PsiDO}{\Psi~\mathrm{DO}}         %% pseudodiffl operators
	\newcommand{\Q}{\mathbb{Q}}                  %% rational numbers
	\newcommand{\R}{\mathbb{R}}                  %% real numbers
	\newcommand{\rdl}{R_\Dslash(\lambda)}        %% resolvent
	\newcommand{\roundbraket}[2]{(#1\mathbin|#2)} %% (w|z)
	\newcommand{\row}[3]{{#1}_{#2},\dots,{#1}_{#3}} %% list: a_1,...,a_n
	\newcommand{\sepword}[1]{\quad\mbox{#1}\quad} %% well-spaced words
	\newcommand{\set}[1]{\{\,#1\,\}}             %% set notation
	\newcommand{\Sf}{\mathbb{S}}                 %% sphere
	\newcommand{\uhor}[1]{\Omega^1_{hor}#1}
	\newcommand{\sco}[1]{{\sp{(#1)}}}
	\newcommand{\sw}[1]{{\sb{(#1)}}}
	\DeclareMathOperator{\spec}{sp}              %% spectrum
	\renewcommand{\SS}{\mathcal{S}}              %% Schwartz space
	\newcommand{\sss}{\mathcal{S}}               %% Schwartz space
	\DeclareMathOperator{\supp}{\mathfrak{supp}}            %% support
	\newcommand{\T}{\mathbb{T}}                  %% circle as a group
	\renewcommand{\th}{\theta}                   %% short for \theta
	\newcommand{\thalf}{\tfrac{1}{2}}            %% small* fraction 1/2
	\newcommand{\tihalf}{\tfrac{i}{2}}           %% small* fraction i/2
	\newcommand{\tpi}{{\tilde\pi}}               %% extended representation
	\DeclareMathOperator{\Tr}{Tr}                %% trace of operator
	\DeclareMathOperator{\tr}{tr}                %% trace of matrix
	\newcommand{\del}{\partial}                  %% short for  \partial
	\DeclareMathOperator{\tsum}{{\textstyle\sum}} %% small sum in display
	\newcommand{\V}{\mathcal{V}}                 %% test function space
	\newcommand{\vac}{\ket{0}}                   %% vacuum ket vector
	\newcommand{\vf}{\varphi}                    %% scalar field
	\newcommand{\w}{\wedge}                      %% exterior product
	\DeclareMathOperator{\wres}{wres}            %% density of Wresidue
	\newcommand{\x}{\times}                      %% cross
	\newcommand{\Z}{\mathbb{Z}}                  %% integers
	\newcommand{\7}{\dagger}                     %% short for + symbol
	\newcommand{\8}{\bullet}                     %% anonymous degree
	\renewcommand{\.}{\cdot}                     %% anonymous variable
	\renewcommand{\:}{\colon}                    %% colon in  f: A -> B
	
	%\newcommand{\sA}{\mathscr{A}}       %%
	\newcommand{\sA}{\mathcal{A}} 
	\newcommand{\sB}{\mathcal{B}}       %%
	\newcommand{\sC}{\mathcal{C}}       %%
	\newcommand{\sD}{\mathcal{D}}       %%
	\newcommand{\sE}{\mathcal{E}}       %%
	\newcommand{\sF}{\mathcal{F}}       %%
	\newcommand{\sG}{\mathcal{G}}       %%
	\newcommand{\sH}{\mathcal{H}}       %%
	\newcommand{\sI}{\mathcal{I}}       %%
	\newcommand{\sJ}{\mathcal{J}}       %%
	\newcommand{\sK}{\mathcal{K}}       %%
	\newcommand{\sL}{\mathcal{L}}       %%
	\newcommand{\sM}{\mathcal{M}}       %%
	\newcommand{\sN}{\mathcal{N}}       %%
	\newcommand{\sO}{\mathcal{O}}       %%
	\newcommand{\sP}{\mathcal{P}}       %%
	\newcommand{\sQ}{\mathcal{Q}}       %%
	\newcommand{\sR}{\mathcal{R}}       %%
	\newcommand{\sS}{\mathcal{S}}       %%
	\newcommand{\sT}{\mathcal{T}}       %%
	\newcommand{\sU}{\mathcal{U}}       %%
	\newcommand{\sV}{\mathcal{V}}       %%
	\newcommand{\sX}{\mathcal{X}}       %%
	\newcommand{\sY}{\mathcal{Y}}       %%
	\newcommand{\sZ}{\mathcal{Z}}       %%
	
	\newcommand{\Om}{\Omega}       %%
	
	
	\DeclareMathOperator{\ptr}{ptr}     %% Poisson trace
	\DeclareMathOperator{\Trw}{Tr_\omega} %% Dixmier trace
	\DeclareMathOperator{\vol}{Vol}     %% total volume
	\DeclareMathOperator{\Vol}{Vol}     %% total volume
	\DeclareMathOperator{\Area}{Area}   %% area of a surface
	\DeclareMathOperator{\Wres}{Wres}   %% (Wodzicki) residue
	
	\newcommand{\dd}[1]{\frac{\partial}{\partial#1}}   %% partial derivation
	\newcommand{\ddt}[1]{\frac{d}{d#1}}                %% derivative
	\newcommand{\inv}[1]{\frac{1}{#1}}                 %% inverse
	\newcommand{\sfrac}[2]{{\scriptstyle\frac{#1}{#2}}} %% tiny fraction
	
	\newcommand\VD{{\mathcal D}}
	\newcommand{\bA}{\mathbb{A}}       %%
	\newcommand{\bB}{\mathbb{B}}       %%
	\newcommand{\bC}{\mathbb{C}}       %%
	\newcommand{\bCP}{\mathbb{C}P}     %%
	\newcommand{\bD}{\mathbb{D}}       %%
	\newcommand{\bE}{\mathbb{E}}       %%
	\newcommand{\bF}{\mathbb{F}}       %%
	\newcommand{\bG}{\mathbb{G}}       %%
	\newcommand{\bH}{\mathbb{H}}       %%
	\newcommand{\bHP}{\mathbb{H}P}     %%
	\newcommand{\bI}{\mathbb{I}}       %%
	\newcommand{\bJ}{\mathbb{J}}       %%
	\newcommand{\bK}{\mathbb{K}}       %%
	\newcommand{\bL}{\mathbb{L}}       %%
	\newcommand{\bM}{\mathbb{M}}       %%
	\newcommand{\bN}{\mathbb{N}}       %%
	\newcommand{\bO}{\mathbb{O}}       %%
	\newcommand{\bOP}{\mathbb{O}P}     %%
	\newcommand{\bP}{\mathbb{P}}       %%
	\newcommand{\bQ}{\mathbb{Q}}       %%
	\newcommand{\bR}{\mathbb{R}}       %%
	\newcommand{\bRP}{\mathbb{R}P}     %%
	\newcommand{\bS}{\mathbb{S}}       %%
	\newcommand{\bT}{\mathbb{T}}       %%
	\newcommand{\bU}{\mathbb{U}}       %%
	\newcommand{\bV}{\mathbb{V}}       %%
	\newcommand{\bX}{\mathbb{X}}       %%
	\newcommand{\bY}{\mathbb{Y}}       %%
	\newcommand{\bZ}{\mathbb{Z}}       %%
	
	\newcommand{\bydef}{\stackrel{\mathrm{def}}{=}}          %% 
	\newcommand{\defeq}{\stackrel{\mathrm{def}}{=}}   
	
	
	
	\newcommand{\al}{\alpha}          %% short for  \alpha
	\newcommand{\bt}{\beta}           %% short for  \beta
	\newcommand{\Dl}{\Delta}          %% short for  \Delta
	\newcommand{\dl}{\delta}          %% short for  \delta
	\newcommand{\ga}{\gamma}          %% short for  \gamma
	\newcommand{\ka}{\kappa}          %% short for  \kappa
	\newcommand{\sg}{\sigma}          %% short for  \sigma
	\newcommand{\Sg}{\Sigma}          %% short for  \Sigma
	\newcommand{\Th}{\Theta}          %% short for  \Theta
	\renewcommand{\th}{\theta}        %% short for  \theta
	\newcommand{\vth}{\vartheta}      %% short for  \vartheta
	\newcommand{\ze}{\zeta}           %% short for  \zeta
	
	\DeclareMathOperator{\ord}{ord}     %% order of a PsiDO
	\DeclareMathOperator{\rank}{rank}   %% rank of a vector bundle
	\DeclareMathOperator{\sign}{sign}   %%
	\DeclareMathOperator{\sgn}{sgn}   %%
	\DeclareMathOperator{\chr}{char}   %%
	\DeclareMathOperator{\ev}{ev}       %% evaluation
	
	\newcommand{\Op}{\mathbf{Op}}
	\newcommand{\As}{\mathbf{As}}
	\newcommand{\Com}{\mathbf{Com}}
	\newcommand{\LLie}{\mathbf{Lie}}
	\newcommand{\Leib}{\mathbf{Leib}}
	\newcommand{\Zinb}{\mathbf{Zinb}}
	\newcommand{\Poiss}{\mathbf{Poiss}}
	
	\newcommand{\gX}{\mathfrak{X}}      %% vector fields
	\newcommand{\sol}{\mathfrak{so}}    %% special orthogonal Lie algebra
	\newcommand{\gm}{\mathfrak{m}}      %% maximal ideal
	
	
	\DeclareMathOperator{\Res}{Res}
	\DeclareMathOperator{\NCRes}{NCRes}
	\DeclareMathOperator{\Ind}{Ind}
	%% co/homology theories
	\DeclareMathOperator{\rH}{H}        %% any co/homology
	\DeclareMathOperator{\rC}{C}        %%  any co/chains
	\DeclareMathOperator{\rZ}{Z}        %% cycles
	\DeclareMathOperator{\rB}{B}        %% boundaries
	\DeclareMathOperator{\rF}{F}        %% filtration
	\DeclareMathOperator{\Gr}{gr}        %% associated graded object
	\DeclareMathOperator{\rHc}{H_{\mathrm{c}}}   %% co/homology with compact support
	\DeclareMathOperator{\drH}{H_{\mathrm{dR}}}  %% de Rham co/homology
	\DeclareMathOperator{\cechH}{\check{H}}    %% Cech co/homology
	\DeclareMathOperator{\rK}{K}        %% K-groups
	\DeclareMathOperator{\rKO}{KO}        %% real K-groups
	\DeclareMathOperator{\rKU}{KU}        %% unitary K-groups
	\DeclareMathOperator{\rKSp}{KSp}        %% symplectic K-groups
	\DeclareMathOperator{\rR}{R}        %% representation ring
	\DeclareMathOperator{\rI}{I}        %% augmentation ideal
	\DeclareMathOperator{\HH}{HH}       %% Hochschild co/homology
	\DeclareMathOperator{\HC}{HC}       %% cyclic co/homology
	\DeclareMathOperator{\HP}{HP}       %% periodic cyclic co/homology
	\DeclareMathOperator{\HN}{HN}       %% negative cyclic co/homology
	\DeclareMathOperator{\HL}{HL}       %% Leibniz co/homology
	\DeclareMathOperator{\KK}{KK}       %% KK-theory
	\DeclareMathOperator{\KKK}{\mathbf{KK}}       %% KK-theory as a category
	\DeclareMathOperator{\Ell}{Ell}       %% Abstract elliptic operators
	\DeclareMathOperator{\cd}{cd}       %% cohomological dimension
	\DeclareMathOperator{\spn}{span}       %% span
	\DeclareMathOperator{\linspan}{span} %% linear span (can't use \span)
	\newcommand{\blank}{-}   
	
	
	
	\newcommand{\twobytwo}[4]{\begin{pmatrix} #1 & #2 \\ #3 & #4 \end{pmatrix}}
	\newcommand{\CGq}[6]{C_q\!\begin{pmatrix}#1&#2&#3\\#4&#5&#6\end{pmatrix}}
	%% q-Clebsch--Gordan coefficients
	\newcommand{\cz}{{\bullet}}         %% anonymous degree
	\newcommand{\nic}{{\vphantom{\dagger}}} %% invisible dagger
	\newcommand{\ep}{{\dagger}}         %% abbreviation for + symbol
	\newcommand{\downto}{\downarrow}    %% right hand limit
	\newcommand{\isom}{\cong}          %% isomorphism
	\newcommand{\lt}{\triangleright}    %% a left  action
	\newcommand{\otto}{\leftrightarrow} %% bijection
	\newcommand{\rt}{\triangleleft}     %% a right action
	\newcommand{\semi}{\rtimes}         %% crossed product
	\newcommand{\tensor}{\otimes}       %% tensor product
	\newcommand{\cotensor}{\square}       %% cotensor product
	\newcommand{\trans}{\pitchfork}     %% transverse
	\newcommand{\ul}{\underline}        %% for sheaves
	\newcommand{\upto}{\uparrow}        %% left  hand limit
	\renewcommand{\:}{\colon}           %% colon in  f: A -> B
	\newcommand{\blt}{\ast}
	\newcommand{\Co}{C_{\bullet}}
	\newcommand{\cCo}{C^{\bullet}}
	\newcommand{\nbs}{\nabla^S}         %% spin connection
	\newcommand{\up}{{\mathord{\uparrow}}} %% `up' spinors
	\newcommand{\dn}{{\mathord{\downarrow}}} %% `down' spinors
	\newcommand{\updn}{{\mathord{\updownarrow}}} %% up or down
	
	%%% Bilinear enclosures:
	
	\newcommand{\bbraket}[2]{\langle\!\langle#1\stroke#2\rangle\!\rangle}
	%% <<w|z>>
	\newcommand{\bracket}[2]{\langle#1,\, #2\rangle} %% <w,z>
	\newcommand{\scalar}[2]{\langle#1,\,#2\rangle} %% <w,z>
	\newcommand{\poiss}[2]{\{#1,\,#2\}} %% {w,z}
	\newcommand{\dst}[2]{\langle#1,#2\rangle} %% distributions <u,\phi>
	\newcommand{\pairing}[2]{(#1\stroke #2)} %% right-linear pairing
	\def\<#1|#2>{\langle#1\stroke#2\rangle} %% \braket (Dirac notation)
	\def\?#1|#2?{\{#1\stroke#2\}}        %% left-linear pairing
	
	%%% Accent-like macros:
	
	\renewcommand{\Bar}[1]{\overline{#1}} %% closure operator
	\renewcommand{\Hat}[1]{\widehat{#1}}  %% short for \widehat
	\renewcommand{\Tilde}[1]{\widetilde{#1}} %% short for \widetilde
	
	
	\DeclareMathOperator{\bCl}{\bC l}   %% complex Clifford algebra
	
	%%% Small fractions in displays:
	
	\newcommand{\ihalf}{\tfrac{i}{2}}   %% small fraction  i/2
	\newcommand{\quarter}{\tfrac{1}{4}} %% small fraction  1/4
	\newcommand{\shalf}{{\scriptstyle\frac{1}{2}}}  %% tiny fraction  1/2
	\newcommand{\third}{\tfrac{1}{3}}   %% small fraction  1/3
	\newcommand{\ssesq}{{\scriptstyle\frac{3}{2}}} %% tiny fraction  3/2
	\newcommand{\sesq}{{\mathchoice{\tsesq}{\tsesq}{\ssesq}{\ssesq}}} %% 3/2
	\newcommand{\tsesq}{\tfrac{3}{2}}   %% small fraction  3/2
	
	
	%\newcommand\eqdef{\over set{\mathclap{\normalfont\mbox{def}}}{=}}
	\newcommand\eqdef{\over set{\mathrm{def}}{=}}
	
	
	%+++++++++++++++++++++++++++++++++++
	
	\newcommand{\word}[1]{\quad\text{#1}\enspace} %% well-spaced words
	\newcommand{\words}[1]{\quad\text{#1}\quad} %% better-spaced words
	\newcommand{\su}[1]{{\sp{[#1]}}}
	
	\def\<#1,#2>{\langle#1,#2\rangle}            %% bilinear pairing
	\def\ee_#1{e_{{\scriptscriptstyle#1}}}       %% basis projector
	\def\wick:#1:{\mathopen:#1\mathclose:}       %% Wick-ordered operator
	
	\newcommand{\opname}[1]{\mathop{\mathrm{#1}}\nolimits}
	
	\newcommand{\hideqed}{\renewcommand{\qed}{}} %% to suppress `\qed'
	
	
	%%%%%%%%%%%%%%%%%%%%%%%%%%%%%
	%% 2. Some internal machinery
	%%%%%%%%%%%%%%%%%%%%%%%%%%%%%
	
	\newbox\ncintdbox \newbox\ncinttbox %% noncommutative integral symbols
	\setbox0=\hbox{$-$}
	\setbox2=\hbox{$\displaystyle\int$}
	\setbox\ncintdbox=\hbox{\rlap{\hbox
			to \wd2{\box2\relax\hfil}}\box0\kern.1em}
	\setbox0=\hbox{$\vcenter{\hrule width 4pt}$}
	\setbox2=\hbox{$\textstyle\int$}
	\setbox\ncinttbox=\hbox{\rlap{\hbox
			to \wd2{\hskip-.05em\box2\relax\hfil}}\box0\kern.1em}
	
	\newcommand{\disp}{\displaystyle} %% short for  \displaystyle
	
	%\newcommand{\hideqed}{\renewcommand{\qed}{}} %% no `\qed' at end-proof
	
	\newcommand{\stroke}{\mathbin|}   %% (for `\bbraket' and such)
	\newcommand{\tribar}{|\mkern-2mu|\mkern-2mu|} %% norm bars: |||
	
	%%% Enclose one argument with delimiters:
	
	\newcommand{\bra}[1]{\langle{#1}\rvert} %% bra vector <w|
	\newcommand{\kett}[1]{\lvert#1\rangle\!\rangle} %% ket 2-vector |y>>
	\newcommand{\snorm}[1]{\mathopen{\tribar}{#1}%
		\mathclose{\tribar}}                 %% norm |||x|||
	
	
	\newcommand{\End}{\mathrm{End}}       %%
	\newcommand{\Endo}{\mathrm{End}}       %%
	\newcommand{\Ext}{\mathrm{Ext}}       %%
	\newcommand{\Hom}{\mathrm{Hom}}       %%
	\newcommand{\Mrt}{\mathrm{Mrt}}       %%
	\newcommand{\grad}{\mathrm{grad}}       %%
	\newcommand{\Spin}{\mathrm{Spin}}       %%
	\newcommand{\Ad}{\mathrm{Ad}}       %%
	\newcommand{\Pic}{\mathrm{Pic}}       %%
	\newcommand{\Aut}{\mathrm{Aut}}       %%
	\newcommand{\Inn}{\mathrm{Inn}}       %%
	\newcommand{\Out}{\mathrm{Out}}       %%
	\newcommand{\Homeo}{\mathrm{Homeo}}       %%
	\newcommand{\Diff}{\mathrm{Diff}}       %%
	\newcommand{\im}{\mathrm{im}}       %%
	
	
	\newcommand{\SO}{\mathrm{SO}}       %%
	\newcommand{\SU}{SU}       %%
	\newcommand{\gso}{\mathfrak{so}}    %% special orthogonal Lie algebra
	\newcommand{\gero}{\mathfrak{o}}    %% orthogonal Lie algebra
	\newcommand{\gspin}{\mathfrak{spin}} %% spin Lie algebra
	\newcommand{\gu}{\mathfrak{u}}      %% unitary Lie algebra
	\newcommand{\gsu}{\mathfrak{su}}    %% special unitary Lie algebra
	\newcommand{\gsl}{\mathfrak{sl}}    %% special linear Lie algebra
	\newcommand{\gsp}{\mathfrak{sp}}    %% symplectic linear Lie algebra
	
	%\newcommand{\bes}{\begin{equation}\begin{split}}
			%\newcommand{\ees}{\end{split}\end{equation}}
	%\NewEnviron{split.enviro}{%
		%	\begin{equation}\begin{split}
				%	\BODY
				%	\end{split}\end{equation}
		%$}
	\newenvironment{splitequation}{\begin{equation}\begin{split}}{\end{split}\end{equation}}
	
	%Begin equation split: Begin equation split = bes
	\newcommand{\bs}{\begin{split}}
		\newcommand{\es}{\end{split}}
	\newcommand{\be}{\begin{equation}}
		\renewcommand{\ee}{\end{equation}}
	\newcommand{\bea}{\begin{eqnarray}}
		\newcommand{\eea}{\end{eqnarray}}
	\newcommand{\bean}{\begin{eqnarray*}}
		\newcommand{\eean}{\end{eqnarray*}}
	\newcommand{\brray}{\begin{array}}
		\newcommand{\erray}{\end{array}}
	\newenvironment{equations}
	{\begin{equation}
			\begin{split}}
			{\end{split}
	\end{equation}}
	\newcommand{\Hsquare}{%
		\text{\fboxsep=-.2pt\fbox{\rule{0pt}{1ex}\rule{1ex}{0pt}}}%
	}
	\usetikzlibrary{calc,trees,positioning,arrows,chains,shapes.geometric,%
		decorations.pathreplacing,decorations.pathmorphing,shapes,%
		matrix,shapes.symbols}
	
	\usetikzlibrary{trees,positioning,shapes,shadows,arrows}
	
	
	\tikzset{
		basic/.style  = {draw, text width=2cm, drop shadow, font=\sffamily,     rectangle},
		root/.style   = {basic, rounded corners=2pt, thin, align=center,
			fill=green!30},
		level 2/.style = {basic, rounded corners=6pt, thin,align=center,     fill=green!60,
			text width=8em},
		level 3/.style = {basic, thin, align=left, fill=pink!60, text width=6.5em}
	}
	
	
	\title{Algebraic topology of $C^*$-algebras}
	
	\author
	{\textbf{Petr R. Ivankov*}\\
		e-mail: * monster.ivankov@gmail.com \\
	}
	
	
	\begin{document}
\maketitle  %\setlength{\parindent}{0pt}
\pagestyle{plain}

\begin{abstract}
	\noindent
\paragraph{}	Any $C^*$-algebra can be regarded as a generalization of locally compact Hausdorff topological space. Here we consider a generalization of fundamental group and (co)homology theory. In result one has invariants  of $C^*$-algebras such that: 
	\begin{itemize}
		\item for any commutative $C^*$-algebra $C_0\left(\sX \right)$ the invariants  coincide with the  $\sX$ ones,
		\item the theory is not trivial even for algebras having bad spectrum, e.g. containing two open sets only.
	\end{itemize}
	
\end{abstract}
\section{General Theory}
\subsection{The Gelfand space of $C^*$-algebra}

\begin{definition}\label{gelfand_space_defn}
	If $A$ is $C^*$-algebra then  the   \textit{Gelfand space} of $A$ is a set $\mathfrak{Gelfand}\left(A \right)$ of maximal left ideals of $A$ having the minimal topology  such that for any closed left ideal $I\subset A$ the set
	\be\label{gelfand_space_egn}
	\left\{\left.x \in \mathfrak{Gelfand}\left(A \right)\right| I \subsetneqq x\right\}	
	\ee
	is open.
\end{definition}



	\begin{lemma}\label{hered_ideal_lem}\cite{murphy}
	Let $A$ be a $C^*$-algebra.
	\begin{enumerate}
		\item[(i)] If $L$ is a closed left ideal in $A$ then $L\cap L^*$ is a hereditary $C^*$-subalgebra of $A$. The map $L \mapsto L\cap L^*$ is the bijection from the set of closed left deals of $A$ onto the the set of hereditary $C^*$-subalgebras of $A$.
		\item[(ii)] If $L_1, L_2$ are closed left ideals, then $L_1 \subseteq L_2$ is and only if $L_1\cap L_1^* \subset L_2\cap L_2^*$.
		\item[(iii)] If $B$ is a hereditary $C^*$-subalgebra of $A$, then the set 
		$$
		L\left(B \right) = \left\{\left.a \in A~\right| a^*a \in B \right\}
		$$
		is the unique closed left ideal of $A$ corresponding to $B$.
	\end{enumerate}
\end{lemma}
\begin{remark}
	Using the Lemma \ref{hered_ideal_lem} one can replace the closed left ideals in the Definition \ref{gelfand_space_defn} with right ones or hereditary $C^*$-subalgebras.
\end{remark}
\begin{thm}\label{gelfand-naimark_thm}\cite{arveson:c_alg_invt} (Commutative Gelfand-Na\u{\i}mark theorem). 
	Let $A$ be a commutative $C^*$-algebra and let $\mathcal{X}$ be the spectrum of A. There is the natural $*$-isomorphism $\gamma:A \xrightarrow{\cong} C_0(\mathcal{X})$.
\end{thm}
The following Lemma is a generalization of the Theorem \ref{gelfand-naimark_thm}.
\begin{lemma} (Generalized commutative Gelfand theorem).
	If $\sX$ is a  Hausdorff space then  is a natural  homeomorphism $\mathfrak{Gelfand}\left(C\left( \sX\right)  \right)\cong \sX$.
\end{lemma}
\begin{proof}
The set of left ideals of $C_0(\mathcal{X})$ naturally coincides with the set of two-sided ones.
\end{proof}
\subsection{Morphisms}

\begin{definition}\label{lrc_defn}\cite{matro:hcm}
	If $A$ is a $C^*$-algebra then a linear map $\la: A\to A$ is said to be a \textit{left centralizer} if
	\be
	\la\left(ab\right)= 	\la\left(a\right) b \quad \forall a, b \in A.
	\ee
	Similarly one defines a \textit{right centralizer}. Denote the spaces of left and right centralizers by $\mathbf{LC}(A)$ and  $\mathbf{RC}(A)$.
\end{definition}



\begin{lemma}\label{lolale_lem}
	Any injective homomorphism 	
	\bean
	\varphi_R: A \hookto \mathbf{RC}\left(\widetilde A\right)	\eean
	of $\C$-algebras	naturally yields a continuous mapping 
	\be\label{gelfand_map_eqn}
	\mathfrak{Gelfand}\left(\widetilde A \right)\xrightarrow{\mathfrak{Gelfand}\left(\varphi_R \right)}\mathfrak{Gelfand}\left(A \right)
	\ee
\end{lemma}
\begin{proof}
	For any $\widetilde x \in  \mathfrak{Gelfand}\left(\widetilde A \right)$ we define the ideal $x \subsetneqq A$ as the closure of the union 
	$$
	\bigcup \left\{I' \subsetneqq  A\left|  \quad \widetilde A\varphi_R\left(I' \right)  \subset \widetilde x \right.\right\}
	$$
	If $x$ is not a maximal nontrivial ideal  and $\widetilde x$ is the closure of $\widetilde A \varphi \left( I\right) $ then from $\widetilde x$ also is not maximal ideal. So the mapping $\widetilde x \mapsto x$ yields the natural map \eqref{gelfand_map_eqn}. From the equation 
	\eqref{gelfand_space_egn} it turns out that the mapping $\mathfrak{Gelfand}\left(\varphi_R \right)$ is continuous.
\end{proof}
\begin{remark}
	Similarly to the Lemma \ref{lolale_lem} a homomorphism $\varphi_L: A \hookto \mathbf{LC}\left(\widetilde A\right)$ yields a continuous mapping
	\bean
	\mathfrak{Gelfand}\left(\widetilde A \right)\xrightarrow{\mathfrak{Gelfand}\left(\varphi_L \right)}\mathfrak{Gelfand}\left(A \right)
	\eean
	
\end{remark}


\subsection{Coverings and fundamental group}
\subsubsection{Topologies of *-automorphisms groups}

If $A$ is a $C^*$-algebra then the group  $\mathrm{Aut}(A)$ of $*$-automorphisms carries several topologies making it into a topological group \cite{thomsem:ho_type_uhf}. The most important is {\it the topology of pointwise norm-convergence} based on the open sets
\begin{equation*}
	\left\{\left.\alpha \in \mathrm{Aut}(A) \ \right| \ \|\alpha(a)-a\| < 1 \right\}, \quad a \in A.
\end{equation*}
The other topology is the {\it uniform norm-topology} based on the open sets
\begin{equation}\label{aut_norm_eqn}
	\left\{\alpha \in \mathrm{Aut}(A) \ \left| \ \sup_{a \neq 0}\ \|a\|^{-1} \|\alpha(a)-a\| < \varepsilon \right. \right\}, \quad \varepsilon > 0
\end{equation}
which corresponds to following "norm"
\begin{equation}\label{uniform_norm_topology_formula_eqn}
	\|\alpha\|_{\text{Aut}} = \sup_{a \neq 0}\ \|a\|^{-1} \|\alpha(a)-a\| = \sup_{\|a\| =1}\  \|\alpha(a)-a\|.
\end{equation}
Above formula does not really means a norm because $\mathrm{Aut}\left(A\right)$ is not a vector space. Both of them can be used for our purposes.
\subsubsection{Basic definitions}
\begin{definition}\label{connected_c_a_defn}
	We say that a $C^*$-algebra $A$ is \textit{connected} if it cannot be represented as a direct sum  $A \cong A' \oplus A''$ of nontrivial $C^*$-algebras $A'$ and $A''$.
	
	% (the Gelfand spectrum of the center of $M\left( A\right) $ is connected). Let $A \subset B$ be a connected subalgebra. We say that $A$ is a \textit{connected component} of $B$ if  $1_{M\left( A\right) }$ lies in the center of $1_{M\left( B\right) }$.
\end{definition}

\begin{definition}\label{fin_quasi_defn}
	Let both  $A$ and  $\widetilde{A}$ be  connected $C^*$-algebras (cf. Definition \ref{connected_c_a_defn}), and let $\lift: A \hookto \widetilde{A}$ be an injective $*$-homomorphism of % connected	  
	$C^*$-algebras. Let $G$ be a finite  group of $*$-automorphisms of $\widetilde{A}$ such that 	$\lift\left(A\right) = \widetilde{A}^G\stackrel{\text{def}}{=}\left\{
	\left.a\in \widetilde{A}~\right|~ a = g a;~ \forall g \in G\right\}$.	We say that the triple $\left(A, \widetilde{A}, G \right)$ and/or the quadruple $\left(A, \widetilde{A}, G, \lift \right)$ and/or $*$-homomorphism $\lift: A \hookto \widetilde{A}$   is a \textit{noncommutative  quasi-covering}. We write
	\be\label{fin_cov_gr_eqn}
	G\left(\left.\widetilde{A}~\right| {A} \right) \stackrel{\text{def}}{=}  	G.
	\ee
\end{definition}

\begin{definition}\label{fin_pre_defn}
	Let   $A$ be an unital connected $C^*$-algebra  and let  $\widetilde{A}$ be  connected $C^*$-algebra (cf. Definition \ref{connected_c_a_defn}), and let $\lift: A \hookto M\left( \widetilde{A}\right) $ be an injective  $*$-homomorphism of % connected
	$C^*$-algebras such that following conditions hold:
	\begin{enumerate}
		\item[(a)] if $\Aut\left(\widetilde{A} \right)$ is a group of $*$-automorphisms of $\widetilde{A}$ then the group  
		\be\nonumber
		G \bydef \left\{ \left.g \in \Aut\left(\widetilde{A} \right)~\right|\forall a \in \lift \left( A\right) \quad ga = a\right\}
		\ee
		is discrete
		\item[(b)] 	$A = M\left( \widetilde{A}\right) ^G\stackrel{\text{def}}{=}\left\{\left.a\in \widetilde{A}~~\right|\forall g \in G\quad  a = g a\right\}$.
	\end{enumerate}
	We say that the triple $\left(A, \widetilde{A}, G \right)$ and/or the quadruple $\left(A, \widetilde{A}, G, \lift \right)$ and/or $*$-homomorphism $\lift: A \hookto \widetilde{A}$   is a \textit{noncommutative  pre-covering}. We write $G\left(\left.\widetilde A~\right|A \right)\bydef G$.
\end{definition}
	\begin{defn}\label{strict_topology_defn}\cite{pedersen:ca_aut}
	Let $A$ be a $C^*$-algebra.  The {\it strict topology} on the multiplier algebra $M(A)$ is the topology generated by seminorms 
	\be\label{strict_topology_norm_eqn}
	\vertiii{x}_a\bydef \|ax\| + \|xa\|,\quad a\in A.
	\ee
	If $\La$ is a directed set and $\left\{a_\la\in M\left( A\right) \right\}_{\la\in \La}$ is a net the we denote by $\bt\text{-}\lim_{\la\in\La }a_\la$ the limit of $\left\{a_\la \right\}$ with respect to the strict topology.
	If $x \in M(A)$  and a sequence of partial sums $\sum_{i=1}^{n}a_i$ ($n = 1,2, ...$), ($a_i \in A$) tends to $x$ in the strict topology then we shall write
	\begin{equation}\label{strict_topology_eqn}
		x = \bt\text{-}\sum_{i=1}^{\infty}a_i.
	\end{equation}
\end{defn}

\begin{definition}\label{evenly_defn}
	
	Let $\left(A, \widetilde{A}, G, \lift \right)$ be a noncommutative  pre-covering (cf. Definition \ref{fin_pre_defn})
	A connected hereditary $C^*$-subalgebra $B \subsetneqq A$ 
	is $\left(A, \widetilde{A}, G, \lift \right)$- \textit{evenly covered by} $\left(A, \widetilde{A}, G, \lift \right)$ if there is a hereditary $C^*$-subalgebra $\widetilde B \subset \widetilde A$ with a $*$-isomorphism $\lift^{\widetilde B}: B \cong \widetilde B$ such that
	\be\label{evenly_eqn}
	\forall b \in B \quad \lift\left( b\right) = \bt \text{-}\sum_{g\in G} g \lift^{\widetilde B}\left( b\right) 
	\ee
	where $\bt \text{-}\sum$ means the convergence with respect to the strict topology of $M\left( \widetilde A\right)$ (cf. the Definition \ref{strict_topology_defn} and the equation \eqref{strict_topology_eqn}) 	
\end{definition}

\begin{definition}\label{cov_unital_defn}
	
	A  noncommutative  pre-covering $\left(A, \widetilde{A}, G, \lift \right)$ (cf. Definition \ref{fin_pre_defn}) with unital $A$ is a {\it unital noncommutative covering} if for any $x \in \mathfrak{Gelfand}\left(A \right)$ there is a hereditary connected $C^*$-subalgebra of $B$ {evenly covered by} $\left(A, \widetilde{A}, G, \lift \right)$ with $B \in x$
\end{definition}

\begin{definition}\label{cov_defn}
	Let $\left(A, \widetilde{A}, G, \lift \right)$ be a quadruple such that both $A$ and $\widetilde{A}$ are $C^*$-algebras $G \subset \Aut\left( \widetilde{A}\right)$ is a discrete subgroup and $\lift: A \hookto M\left( \widetilde{A}\right)$ is an injective $*$-homomorphism. If there is an unital noncommutative covering $\left(B, \widetilde{B}, G, \widetilde\lift \right)$  with inclusions $A \subset B$ and $\widetilde A \subset \widetilde B$ such that:
	\begin{enumerate}
		\item [(a)] both $A$ and $\widetilde A$ are essential ideals of $B$ and $\widetilde B$,
		\item[(b)] $\lift \bydef \left.\widetilde\lift\right|_A$,
		\item[(c)] the action $G \times \widetilde B \to \widetilde B$ naturally comes from the $G \times \widetilde A \to \widetilde A$
	\end{enumerate}	
\end{definition}


	


\begin{definition}\label{fund_den}
	If $A$ is a $C^*$ -algebra then a noncommutative covering $\left(A, \widetilde{A}, G, \lift \right)$ is \textit{universal}, if for any noncommutative covering $\left(A, \widetilde{A}', G', \lift' \right)$ there is natural noncommutative covering $\left(\widetilde{A}', \widetilde{A}, G'', \lift'' \right)$. If the universal covering exists then $G$ is said to be the \textit{fundamental group} of $A$.
\end{definition}
	

\section{Applications}
\subsection{Fundamental group of commutative $C^*$-algebras}
\paragraph{} If $\sX$ is a connected, locally compact, Hausdorff space then there is a homeomorphism $\mathfrak{Gelfand}\left(C_0\left(\sX \right)  \right)\cong \sX$. If  $\left(C_0\left(\sX \right), \widetilde{A}, G, \lift \right)$ is a noncommutative covering then the Lemma \ref{lolale_lem} yield a continuous map
$$
\mathfrak{Gelfand}\left(\lift \right): \mathfrak{Gelfand}\left(\widetilde A \right)\to \sX.
$$
	
\begin{proposition}\label{hered_spectrum_prop}\cite{pedersen:ca_aut}
	If $B$ is a hereditary $C^*$-subalgebra of $A$ then the map $t \mapsto t\cap B$ is a homeomorphism between $\check A\setminus \mathrm{hull}\left( B\right)$ and $\check B$, where 
	$$
	\mathrm{hull}\left( B\right) = \left\{\left. x \in \hat A~\right|~ \rep_x\left(B \right)= \{0\} \right\} .
	$$ 
	Moreover  we have a commutative diagram:
	\\
	\begin{tikzcd}
		\hat A \setminus \mathrm{hull}\left(B \right)\arrow[d]\arrow[r, "\approx"] & \hat B\arrow[d]\\
		\check A \setminus \mathrm{hull}\left(B \right)\arrow[r, "\approx"] & \check B
	\end{tikzcd}
	\\ 	
\end{proposition}
	If $B\subset A$  is a hereditary $C^*$-subalgebra evenly covered by $\left(C_0\left(\sX \right), \widetilde{A}, G, \lift \right)$ (cf. Definition \ref{evenly_defn}) then there is a connected open subset $\sU \subset \sX$ with $B \cong \sU$. From the Definition \ref{evenly_defn} it follows that $\mathfrak{Gelfand}\left(\lift \right)^{-1}\left(\sU\right)$ is the disjoint union of homeomorphic to $\sU$ connected open subsets of  $\mathfrak{Gelfand}\left(\widetilde A \right)$, i.e. the map $\mathfrak{Gelfand}\left(\lift \right)$ is a covering. If $\widetilde A$ is not commutative then there is $\widetilde x \in \mathfrak{Gelfand}\left(\lift \right)$ with
	$$
\dim \widetilde A / \widetilde x > 1.	
	$$. From the condition (b) of the Definition \ref{fin_pre_defn} it follows that 
	$$
\dim C_0\left( \sX\right) / \mathfrak{Gelfand}\left(\lift \right)\left( \widetilde x\right) > 1.	
	$$
	It is impossible so $\widetilde A\cong C_0\left(\mathfrak{Gelfand}\left(\widetilde A\right) \right)$ is a commutative $C^*$-algebra. Thus there is an 1-1 correspondence between topological and noncommutative coverings of $C_0\left(\sX \right)$. From the Definition \ref{fund_den} that the fundamental group of $C_0\left(\sX \right)$ if it exists is isomorphic to $\pi_1\left( \sX\right)$. 
\subsection{Hausdorff blowing-up}
\begin{definition}\label{blowing_up_defn}
	For any $C^*$-algebra $A$ 
	an inclusion $C_0\left( \sX\right) \hookto M\left(A \right)$ into multiplier $C^*$-algebra  such that
	$$
	C_0\left( \sX\right)AC_0\left( \sX\right)
	$$
	is dense in $A$ is \textit{Hausdorff blowing-up}.
\end{definition}
The "blowing-up" word is inspired by following reasons.
\begin{itemize}
	\item Sometimes there is  the natural partially defined  surjective  map from  Hausdorff blowing-up to the spectrum.
	\item  In the algebraic geometry   "blowing-up" means  excluding of singular points  (cf. \cite{hartshorne:ag}).
\end{itemize}

From the Lemma \ref{lolale_lem} there is a continuous map $	\mathfrak{Gelfand}\left( A \right)\to \sX$. This map is used for calculating of related to  
$	\mathfrak{Gelfand}\left( A \right)$ invariants.
\subsection{Cohomology of continuous trace $C^*$-algebra}


\subsubsection{Gelfand space}
\begin{lemma}\label{hausdorff_spectrum_lem}\cite{rae:ctr_morita}
	Suppose $A$ is a $C^*$-algebra with Hausdorff spectrum $\mathcal{X}$.
	\begin{itemize}
		\item [(a)] If $a, b \in A$ and $\mathfrak{rep}_x\left(a \right)=  \mathfrak{rep}_x\left(b \right)$ for every $x \in  \mathcal{X}$, then $a = b$.
		\item[(b)] For each $a \in A$ the function $x \mapsto \left\|\mathfrak{rep}_x\left(a \right) \right\|$ is continuous on  $\mathcal{X}$, vanishes at infinity and has sup-norm equal to $\left\| a\right\|$. 
	\end{itemize}
\end{lemma}
	\begin{theorem}\label{irred_thm}\cite{pedersen:ca_aut}
	Let $\pi: A \to B\left(\H \right)$ be a nonzero representation of $C^*$-algebra $A$. The following conditions are equivalent:
	\begin{enumerate}
		\item [(i)] there are no non-trivial $A$-subspaces for $\pi$,
		\item[(ii)] the commutant of $\pi\left(A \right)$ is the scalar multipliers of 1,
		\item[(iii)] $\pi\left(A \right)$ is strongly dense in   $B\left(\H \right)$,
		\item[(iv)] for any two vectors $\xi, \eta \in \H$ with $\xi \neq 0$ there is $a \in A$ such that $\pi\left(a \right)\xi = \eta$,
		\item[(v)] each nonzero vector in $\H$ is cyclic for  $\pi\left(A \right)$,
		\item[(vi)]  $A \to B\left(\H \right)$ is spatially equivalent to a cyclic representation associated with a pure state of $A$.
	\end{enumerate} 
\end{theorem}
\begin{definition}\label{abelian_element_defn}\cite{pedersen:ca_aut}
	A positive element in $C^*$ - algebra $A$ is {\it Abelian} if subalgebra $xAx \subset A$ is commutative.
\end{definition}
\begin{definition}\label{type_I_defn}\cite{pedersen:ca_aut}
	We say that a $C^*$-algebra $A$ is \textit{of type} $I$ if each non-zero quotient of $A$ contains a non-zero
	Abelian element. If $A$ is even generated (as $C^*$-algebra) by its Abelian elements we say
	that it is \textit{of type} $I_0$.
\end{definition}
\begin{definition}\label{continuous_trace_c_alt_defn}\cite{rae:ctr_morita}
	%Definition 5.13. 
	A \textit{continuous-trace} $C^*$-\textit{algebra} is a $C^*$-algebra $A$ with Hausdorff
	spectrum $\sX$ such that, for each $x_0\in\sX$ there are a neighbourhood $\sU$ of $x_0$ and $a\in A$ such that $\rep_{ x}\left( a\right) $ is a rank-one projection for all $x \in \sU$.
\end{definition}
If $A$ is a continuous trace $C^*$-algebra  and $I\subsetneqq A$ is a closed  left ideal then from the Lemma \ref{hausdorff_spectrum_lem} it follows that  there are $\eps > 0$,  $~a \in A$ and an irreducible representation $\rho: A \to B\left( \H\right)$ with
\be\label{abelian_ineq_eqn}
\forall a' \in I \quad \left\|\rho\left(a - a' \right)  \right\| > \eps.
\ee
On the other hand $A$ is a $C^*$-algebra of type $I_0$. From this fact one can deduce that there is a satisfying to \eqref{abelian_ineq_eqn}  Abelian element. There is $\xi \in \H$ such that $\rho\left(a \right)= \xi \left\rangle \right\langle \xi$. From (v) of the Theorem \ref{irred_thm} it follows that $\H = \rho\left(A \right) \xi$. If $\xi \in \rho\left(I\right) \xi$ then $a \in I$. It is impossible so $\C\xi \cap \rho\left(I\right) \xi = \{0\}$. There are $\xi^\parallel \in  \rho\left(I\right) \xi$ such that if $\xi^\perp \bydef \xi - \xi^\parallel$ then $\xi^\perp\perp  \rho\left(I\right) \xi$. So for any closed left ideal there is one irreducible representation with $\rho\left( I\right)\xi \neq \H$. The spectrum $\sX$ of $A$ is  Hausdorff. If there are $x', x'' \in \sX$ with $x' \neq x''$ and $ \rho_{x'}\left(I\right)\xi' \neq \H'$, $\rho_{x''}\left(I\right)\xi'' \neq \H''$ then there is $f \in C_0\left( \sX\right)$ such that $f\left(x' \right)  = 0$ and  $f\left(x'' \right)  = 0$. If $I'\bydef I' + Af$ then $I'$ is a closed left ideal such that $I \subsetneqq I' \subsetneqq A$. So if $I$ is a maximal left ideal then there is a single point with $\rho\left(I\right) \xi\neq \H$. If codimension of $\rho\left(I\right) \xi\neq \H$ exceds 1 then there if an Abelian element $a'' \in A$ with
$$
\rho\left(I \right) \xi \subsetneqq \rho\left(I \right) \xi + \rho\left(a'' \right) \xi \subsetneqq \H.
$$
\begin{lemma}\label{ctr_gelfand_lem}
	If $A$ is a continuous trace $C^*$-algebra then any element of  $\mathfrak{Gelfand}\left(A \right) $ is given by a pair $\left(x, \xi \right)$ where
	\begin{enumerate}
		\item[(i)] $x$ is a point of the spectrum $\sX$ of $A$ which corresponds to the irreducible representation $\rep_x: A \to B\left( \H_x\right) $,
		\item[(ii)] $\xi \in \C P\left( \H_x\right)$ where $ \C P\left( \H_x\right)$ is a complex projective space.
	\end{enumerate} 
	
\end{lemma}
\begin{proof}
	There is the natural one-to-one correspondence between codimension one planes of any Hilbert space $\H$ and points of   $ \C P\left( \H\right)$.
\end{proof}
Suppose that $A$ is given by a locally trivial  fibre bundle $\F$ with fibre $\K = \K\left(\H \right)$. There is a subbundle $\E \subset \F$ such that fibers of $\E$ are  rank-one positive operators. It givers a bundle $\C P\left(\E \right)$ with fibre $\C P\left( \H\right)$. From the Lemma \ref{ctr_gelfand_lem}  it turns out that there is the natural set theoretic bijective map $\mathfrak{Gelfand}\left(A \right)\cong  \C P\left(\E \right)$.
If $\C P\left(\E \right)_{\mathfrak{Gelfand}}$ is $\C P\left(\E \right)$ supplied with Gelfand topology then there is a bijective continuous map
\be\label{pe_c_eqn}
\phi_{\E} : \C P\left(\E \right)_{\mathfrak{Classic}}\to \C P\left(\E \right)_{\mathfrak{Gelfand}}
\ee
where $\C P\left(\E \right)_{\mathfrak{Classic}}$ is supplied with the classic topology.
In particular if $\sX = \{x\}$ then $A = \K\left(\H \right)$ and $\mathfrak{Gelfand}\left(\K\left(\H \right) \right)\cong  \C P\left(\H \right)$. The subbase of $\mathfrak{Gelfand}\left(\K\left(\H \right)\right) $ contains all sets $\C P\left(\H \right) \setminus L$ where $L$ is a linear subspace of  $\C P\left(\H \right)$. Similarly to \eqref{pe_c_eqn} there is a continuous map 
$$
\phi_{\H} : \C P\left(\H \right)_{\mathfrak{Classic}}\to \C P\left(\H \right)_{\mathfrak{Gelfand}}
$$
Moreover if $\dim \H = n$ then there is a composition
$$
\C P^n_{\mathfrak{Classic}}\to \C P^n_{\mathfrak{Zariski}}\to \C P^n_{\mathfrak{Gelfand}}
$$
where the subscript $_{\mathfrak{Zariski}}$ means the Zariski topology.
If  $\dim \H = \infty$ and $\left\{L_0, L_1, ...\right\}$ is a set of mutually orthogonal codimension one spaces then
$$
\C P\left( \H\right) = \bigcup_{j=0}^\infty \left( \C P\left(\H \right) \setminus L_j\right) 
$$
Similarly 
$$
\C P^n = \bigcup_{j=0}^{n} \left( \C P^n \setminus L_j\right). 
$$
where $\left\{L_0,  ..., L_n\right\}$ is a set of mutually orthogonal codimension one spaces.
\begin{definition}\label{nerve_defn}\cite{spanier:at}
	Given a topological space $\sX$ and a collection $\mathscr W = \left\{\mathcal W\right\}$ of subsets of $\sX$, the \textit{nerve} of $\mathscr W$ denoted by $K\left( \mathscr W\right)$, is  the simplicial complex whose simplexes are finite nonempty subsets of  $\mathscr W$ with nonempty intersections. Thus the vertices of $K\left( \mathscr W\right)$ are nonempty elements of $\mathscr W$.
\end{definition}
\begin{theorem}\label{top_nerve_thm}\cite{bredon:sheaf}
	%PAGE 193 !!! DOWN !!!
	%4.13. Theorem. 
	Let $\mathscr A$ be a sheaf of Abelian groups  on $\sX$ and let $\mathscr U = \left\{\sU_\a\right\}_{\a \in I}$ an open covering  of $\sX$ having the property that $H^p\left( \sU_{\sigma};\mathscr A \right)= 0$  for $p > 0$ and
	all $\sigma \in K\left(\mathscr U \right)$ in the nerve of covering (cf. Definition \ref{nerve_defn}). Then there is a canonical isomorphism
	\be
	H^*\left(\sX, \mathscr A \right) \cong \check H^*\left( \mathscr U; \mathscr A\right). 
	\ee
\end{theorem}
If $F$ is an Abelian group and $\mathscr F$ is a corresponding constant sheaf on $\C P\left(\H \right)_{\mathfrak{Classic}}$  and $\mathscr W= \left\{ \C P\left(\H \right) \setminus L_j\right\}_{j = 0, 1, ... }$ or  $\mathscr W= \left\{ \C P^n\setminus L_j\right\}_{j = 0,..., n }$ then 
$H^p\left( \sU_{\sigma};\mathscr A \right)= 0$  for $p > 0$ and
all $\sigma \in K\left(\mathscr W\right)$ where n $K\left(\mathscr W\right)$ is the nerve of $\mathscr W$ (cf. Definition \ref{nerve_defn}). From the Theorem \ref{top_nerve_thm} it turns out that
$$
H^*\left(\C P^n_{\mathfrak{Classic}}, \mathscr F_{\mathfrak{Classic}} \right) \cong \check H^*\left( \mathscr W; \mathscr F_{\mathfrak{Classic}}\right)
$$
Taking into account that $\C P^n \setminus L_j$ is open in $\C P\left(\H \right)_{\mathfrak{Gelfand}}$ for any $j$ one has
$$
H^*\left(\C P\left(\H \right) _{\mathfrak{Classic}}, \mathscr F_{\mathfrak{Classic}} \right) \cong H^*\left(\C P\left(\H \right) _{\mathfrak{Gelfand}}, \mathscr F_{\mathfrak{Gelfand}}\right)
$$
The spectrum of $\K\left(\H \right)$ has the single point but cohomology of $\mathfrak{Gelfand}\left(\K\left(\H \right)  \right)$ are not trivial. 
The topology of $\C P\left(\E \right)_{\mathfrak{Gelfand}}$ is the hybrid of Hausdorff and non Hausdorff topology. The following Lemma is a consequence of the Theorem \ref{top_nerve_thm}.
\begin{lemma}
	Let $F$ is an Abelian group.
	Let $\pi :  \C P\left(\E \right)_{\mathfrak{Classic}}\to \sX$ be the natural surjective mapping 
	If $\sX = \bigcup_{\mathcal W \in \mathscr W}  \mathcal W$ where: 
	\begin{itemize}
		\item for any $\mathcal W \in \mathscr W$ one has $\pi^{-1}\left(\mathcal W \right) \cong \mathcal W\times \C P\left( \sH\right)_{\mathfrak{Classic}}$,
		\item $H^*\left( \sU,\mathscr F_{\mathfrak{Classic}} \right)= {0}$  for all $\sU \in K\left(\mathscr W \right)$ 
	\end{itemize}	
	then there is the natural isomorphism
	$$
	H^*\left(\mathfrak{Gelfand}\left(A \right),\mathscr F_{\mathfrak{Gelfand}}  \right) \cong H^*\left(\C P\left(\E \right) _{\mathfrak{Classic}}, \mathscr F_{\mathfrak{Classic}} \right) 
	$$ 
	where both $\mathscr F_{\mathfrak{Classic}}$ and $\mathscr F_{\mathfrak{Gelfang}}$ are corresponding to $F$ locally constant sheaves.
\end{lemma}






\section*{Acknowledgment}


\paragraph*{}
 Author would like to acknowledge members of the Moscow State University Seminars
"Noncommutative geometry and topology", "Algebras in analysis" leaded by professors A. S. Mishchenko and  A. Ya. Helemskii for a discussion
of this work. 


\begin{thebibliography}{10}
		\bibitem{arveson:c_alg_invt} W. Arveson. {\it An Invitation to $C^*$-Algebras}, Springer-Verlag. ISBN 0-387-90176-0, 1981.
	
		\bibitem{bredon:sheaf} Bredon, Glen E. (1997), \textit{Sheaf theory}. Graduate Texts in Mathematics, 170 (2nd ed.), Berlin, New York: Springer-Verlag.  ISBN 978-0-387-94905-5, MR 1481706 (oriented towards conventional topological applications), 1997.
	
	\bibitem{hartshorne:ag} Robin Hartshorne. {\it Algebraic Geometry.} Graduate Texts in Mathematics, Volume 52, 1977.
	
		
			\bibitem{matro:hcm} Manuilov V.M., Troitsky E.V. \textit{Hilbert $C^*$-modules}. % Publication Year: 2005. ISBN-10: 0-8218-3810-5 ISBN-13: 978-0-8218-3810-5 
	Translations of Mathematical Monographs, vol. 226, 2005.
	
	\bibitem{murphy}G.J. Murphy. {\it $C^*$-Algebras and Operator Theory.} Academic Press 1990.	
		\bibitem{pedersen:ca_aut}Gert K Pedersen. {\it $C^*$-algebras and their automorphism groups}. London ; New York : Academic Press, 1979.
	
	
	\bibitem{rae:ctr_morita} Iain Raeburn, Dana P. Williams. \textit{Morita Equivalence and Continuous-trace $C^*$-algebras}. American Mathematical Soc., 1998.
	
		\bibitem{spanier:at}
E.H. Spanier. {\it Algebraic Topology.} McGraw-Hill. New York. 1966.

		\bibitem{thomsem:ho_type_uhf} Klaus Thomsen. {\it The homotopy type of the group of automorphisms of a $UHF$-algebra}. Journal of Functional Analysis. Volume 72, Issue 1, May 1987.

\end{thebibliography}



	\end{document}
