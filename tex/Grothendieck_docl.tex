\documentclass{beamer}
\usepackage{amsmath,amssymb,amsthm,slashed, euscript}

\usepackage{tikz}

\textwidth=110mm


\title{Grothendieck topology of  $C^*$-algebras}
\institute
{
Algebras in analysis
}

\author{Petr R. Ivankov  }



\theoremstyle{plain}
\newtheorem{defn}{Definition}
\newtheorem{rem}{Remark}
\newtheorem{exm}{Example}
\newtheorem*{claim}{Claim}
\newtheorem{prop}{Proposition}
\newtheorem{empt}[prop]{}%[section]
\newtheorem{lem}{Lemma}%[section]
\newtheorem{thm}{Theorem}%[section]

\newcommand{\bean}{\begin{eqnarray*}}
\newcommand{\eean}{\end{eqnarray*}}


\newcommand{\A}{\mathcal{A}}
\newcommand{\be}{\begin{equation}}
\newcommand{\ee}{\end{equation}}
\newcommand{\Ga}{\Gamma}
\newcommand{\eps}{\varepsilon}                    %% tensor product
\newcommand{\B}{\mathcal{B}}
\newcommand{\Cc}{\mathcal{C}}
\newcommand{\C}{\mathbb{C}}
\newcommand{\D}{\mathcal{D}}
\newcommand{\G}{\mathcal{G}}
\newcommand{\Hc}{\mathcal{H}}
\newcommand{\Lc}{\mathcal{L}}
\newcommand{\Pc}{\mathcal{P}}
\newcommand{\Sc}{\mathcal{S}}
\newcommand{\U}{\mathcal{U}}
\newcommand{\rar}{\rightarrow}
\newcommand{\Ef}{\mathbb{E}}
\newcommand{\desc}{\mathfrak{desc}}


%Uppercase Gothic characters
\newcommand{\gtA}{\mathfrak{A}}
\newcommand{\gtB}{\mathfrak{B}}
\newcommand{\gtM}{\mathfrak{M}}
\newcommand{\gtN}{\mathfrak{N}}
\newcommand{\gtP}{\mathfrak{P}}
\newcommand{\gtS}{\mathfrak{S}}

%Lowercase Gothic characters
\newcommand{\gtf}{\mathfrak{f}}
\newcommand{\gtg}{\mathfrak{g}}

%Bold Characters
\newcommand{\Cb}{\mathbb{C}}
\newcommand{\Nb}{\mathbb{N}}
\newcommand{\Rb}{\mathbb{R}}
\newcommand{\Zb}{\mathbb{Z}}

%Uppercase Greek characters
\newcommand{\Gm}{\Gamma}
\newcommand{\Te}{\Theta}
\newcommand{\Om}{\Omega}
\newcommand{\s}{ }

%Lowercase Greek characters
\newcommand{\al}{\alpha}
\newcommand{\gm}{\gamma}
\newcommand{\dl}{\delta}
\newcommand{\sg}{\sigma}
\newcommand{\ph}{\varphi}
\newcommand{\te}{\theta}
\newcommand{\ze}{\zeta}
\newcommand{\lift}{\mathfrak{lift}}

\newcommand{\Id}{\mathrm{Id}}
\newcommand{\Aut}{\mathrm{Aut}}
\newcommand{\Coo}{{\mathrm{C}}^\infty}
\newcommand{\alg}{\mathrm{alg}}
\newcommand{\diag}{\mathrm{diag}}
\newcommand{\spinc}{\textbf{$spin^c$}}
\newcommand{\Hom}{\mathrm{Hom}}
\newcommand{\supp}{\mathrm{supp}}
\newcommand{\Ccl}{\mathbf{C}l}
\newcommand{\xto}{\xrightarrow}

\newcommand{\lto}{\longrightarrow}
\newcommand{\ox}{\otimes}
\newcommand{\nb}{\nabla}
\newcommand{\sS}{\mathcal{S}}
\newcommand{\Dn}{D\!\!\!\!/}
%\newcommand{\ij}{{i,j}}
\newcommand{\aC}{\ensuremath{\underline{\Cb}} }
\newcommand{\scp}[2]{\left\langle{#1},{#2}\right\rangle}
\newcommand{\op}[1]{J{#1}J^\dag}
\newcommand{\sA}{\mathcal{A}} 
\newcommand{\sB}{\mathcal{B}}       %%
\newcommand{\sC}{\mathcal{C}}       %%
\newcommand{\sD}{\mathcal{D}}       %%
\newcommand{\sE}{\mathcal{E}}       %%
\newcommand{\sF}{\mathcal{F}}       %%
\newcommand{\sG}{\mathcal{G}}       %%
\newcommand{\sH}{\mathcal{H}}       %%
\newcommand{\sI}{\mathcal{I}}       %%
\newcommand{\sJ}{\mathcal{J}}       %%
\newcommand{\sK}{\mathcal{K}}       %%
\newcommand{\sL}{\mathcal{L}}       %%
\newcommand{\sM}{\mathcal{M}}       %%
\newcommand{\sN}{\mathcal{N}}       %%
\newcommand{\sO}{\mathcal{O}}       %%
\newcommand{\sP}{\mathcal{P}}       %%
\newcommand{\sQ}{\mathcal{Q}}       %%
\newcommand{\sR}{\mathcal{R}}       %%
\newcommand{\sT}{\mathcal{T}}       %%
\newcommand{\sU}{\mathcal{U}}       %%
\newcommand{\sV}{\mathcal{V}}       %%
\newcommand{\sX}{\mathcal{X}}       %%
\newcommand{\sY}{\mathcal{Y}}       %%
\newcommand{\sZ}{\mathcal{Z}}       %%
\newcommand{\N}{\mathbb{N}}                  %% 

\renewcommand{\a}{\alpha}     
\newcommand{\la}{\lambda}     
\newcommand{\La}{\Lambda}
\newcommand{\bt}{\beta}           %% short for  \beta
 
    
\newcommand{\bydef}{\stackrel{\mathrm{def}}{=}}  
\newcommand{\hookto}{\hookrightarrow}        %% abbreviation
  
\begin{document}
%\titlepage

\begin{frame}
  \titlepage
\end{frame}
\begin{frame}
	\begin{defn}
		Let  $C$  be a category with pullbacks. Suppose that  for each object $U$ of $C$ there are  distinguished sets of families of morphisms $\left\{U_\iota \to U\right\}_{\iota\in I}$, called the \textit{coverings} of $U$, 
		satisfying the following axioms: 
		\begin{enumerate}
			\item[(a)] for any covering $\left\{U_\iota \to U\right\}_{\iota\in I}$ and any morphism $U \to V$ in $ C$, the fibre products 
			$U_\iota\times_U V$ exist, and $\left\{U_\iota\times_U V \to V\right\}_{\iota\in I}$ is a covering of $V$; 
			\item[(b)] if $\left\{U_\iota \to U \right\}_{\iota\in I}$ is a covering of $U$, and if for each $\iota \in I$, $\left\{V_{\iota j} \to U_\iota  \right\}_{j \in I_\iota}$	is a 
			covering of $U_\iota$, then the family $\left\{V_{\iota j }\to U\right\}_{\iota j}$ is a covering of $U$; 
			\item[(c)] for any $U$ in $C$, the family $\left\{U\xrightarrow{\Id}U \right\}$ consisting of a single map is a covering of $U$. 
		\end{enumerate}
		The system of coverings is then called a \alert{Grothendieck pretopology}.
		\end{defn}
\end{frame}
\begin{frame}
		\begin{example}\label{top_gro_exm}
		If $\sX$ is a topological space then one has a category of open subsets and their inclusions. If both $\sU_1, \sU_2 \subset \sX$ are open subsets then one can define a fibre product  by the following way
		$$
		\sU_1\times_\sX \sU_2 \bydef \sU_1 \cap \sU_2.	
		$$
		If $\sU \subset \sX$ is an open subset then we assume that a family  $\left\{\sU_\iota \subset \sU\right\}_{\iota\in I}$, is a covering of $\sU$ if and only if  $\sU = \cup_{\iota\in I}\sU_\iota$. This system of coverings is a specialization of Grothendieck topology. 
	\end{example}
\end{frame}
\begin{frame}
	\begin{definition}\label{etale_presheaf_defn}
	A \alert{presheaf of sets} on a category $ C$
	is a contravariant functor $ F$ from $ C$ to the category of sets. Thus,  $ F$
	to each object $U$ in $ C$ 
	attaches a set $ F\left(U \right)$ , and to each morphism $\varphi: U \to V$
	in $ C$, a map $ F\left(\varphi\right): F\left(V \right)\to  F\left(U \right)$. Note that the notion of a presheaf on $ C$ 
	does not depend on the 
	coverings. We sometimes denote $ F\left(\varphi\right)$ by $a \mapsto a|_U$.
	
\end{definition}
Similarly, a presheaf of (Abelian) groups or rings for the Grothendieck pretopology is a contravariant functor from
$ C$ to the category of (Abelian) groups or rings.
\end{frame}
\begin{frame}
\begin{definition}\label{etale_sheaf_defn}
	A \alert{sheaf} for the Grothendieck pretopology
	is a presheaf $ F$ 
	that satisfies the sheaf condition, that is a sequence 
	\be\label{etale_sheaf_eqn}
	 F \left(U\right) \to \prod_{\iota \in I}  F \left(U_\iota \right)\rightrightarrows  \prod_{\iota, j \in I\times I}  F \left(U_\iota \times_U U_j\right)
	\ee
	is exact for every covering $\left\{U_\iota \to U\right\}$. Thus $ F$ 
	is a sheaf if the map 
	$$
	f \mapsto \left\{f|_{U_\iota }\right\}: F \left(U\right) \to \prod_{\iota \in I}  F \left(U_\iota \right)
	$$
	identifies $ F\left( U\right)$ with the subset of the product consisting of families $\left\{f|_{U_\iota }\right\}$ such that 
	$$
	f_\iota|_{U_\iota \times_U U_j}= f_j|_{U_\iota \times_U U_j}
	$$
	for all $\iota, j \in I$. 
	When $\mathbf T$ 
	is the site arising from a topological space then these definitions coincide with the 
	topological ones. 
	
\end{definition}
\end{frame}

\begin{frame}
	
	\begin{definition}
	A category $C$  together with 
the Grothendieck  (pre)topology is called a \alert{site} $\mathbf T\bydef \left(  C, J\right)$. Then $\mathrm{Cat}\left( \mathbf T\right)\bydef  C$ denotes the underlying category. 
	\end{definition}
	
	
	\begin{definition}
	If $\mathbf T$ is a site then  a category  $\mathbf{Sh}\left( \mathbf T \right)$ of sheaves of sets is a \alert{Grothendieck topos}.
\end{definition}
	\begin{definition}\label{forget_sheaf_defn}
	There are categories $\mathbf{PreSh}\left(\mathbf T \right)$, $ \mathbf{Sh}\left(\mathbf T\right)$ of presheaves an sheaves. Moreover there is a \alert{forgetful functor} $\mathfrak{Forget} : \mathbf{Sh}\left(\mathbf T \right)\to \mathbf{PreSh}\left(\mathbf T \right)$.
\end{definition}


\end{frame}
\begin{frame}
	\begin{lem}\label{associated_sheaf_stmnt}
	There is an adjoint $\mathfrak{Ass} : \mathbf{PreSh}\left(\mathbf T  \right)\to \mathbf{Sh}\left(\mathbf T \right)$ to the  forgetful functor $\mathfrak{Forget} : \mathbf{Sh}\left(\mathbf T  \right)\to \mathbf{PreSh}\left(\mathbf T \right)$.
\end{lem}	
	\begin{defn}\label{associated_sheaf_defn}
	The given by the above lemma  \ref{associated_sheaf_stmnt} functor $\mathfrak{Ass} : \mathbf{PreSh}\left(\mathbf T  \right)\to \mathbf{Sh}\left(\mathbf T \right)$  is said to be an  \alert{associated sheaf functor}.
\end{defn}
\begin{lem}\label{site_enough_stmt}
	%8.13$
	If $\mathbf T$ is a site then a category $\mathbf {Ab}\left(\mathbf T \right)$ of sheaves of Abelian groups has enough injectives.
\end{lem}
\begin{definition}\label{grothenidieck_topos_defn}
	If $\mathbf T$ is a site then  a category  $\mathbf{Sh}\left( \mathbf T \right)$ of sheaves of sets is a \alert{Grothendieck topos}.
\end{definition}

\end{frame}
\begin{frame}
	\begin{definition}\label{frame_defn}
	A \alert{frame} is partially  ordered set with all joins and all finite meets which satisfies the infinite distributive law:
	$$
	x \wedge \left( \bigvee_\iota y_\iota \right) = \bigvee_\iota\left(x\wedge   y_\iota \right) 
	$$
	A \alert{frame homomorphism} 
	is a function which preserves finite meets and arbitrary joins. Frames and frame homomorphisms form a category $\mathbf{Frm}$.
\end{definition}
\begin{definition}\label{locale_defn}
	The category $\alert{ Locale}$ of \textit{locales} is the opposite of the category of frames
	$$
	\mathbf{ Locale} \bydef \mathbf{Frm}^\text{op}.
	$$
	If $X$ is  a locale  denote by $\mathcal O\left( X\right)$ the corresponding  frame.
\end{definition}
\end{frame}
\begin{frame}
The definition  of sheaves on a topological space depends
only on the lattice of open sets of that space, and so extends at once to
define sheaves on a locale $X$. Thus an "open" object $U$ is underlying category is said to be
covered by a family $\left\{U_\iota \right\}$ of "opens" of $X$ with each $U_\iota \le U$  if and only if $U= \bigvee_\iota V_\iota$ thus
\be\label{localic_site}
\left\{U_\iota\subset U \right\} \text{ covers } U \text{ if and only if } U= \bigvee_\iota V_\iota
\ee
The equation \eqref{localic_site} yields a Grothendieck pretopology, so one obtains a Grothendieck topology. In result we obtain a site and a Grothendieck topos.
\begin{definition}\label{localic_topos_defn}
	A Grothendieck topos obtained from a locale is said to be \alert{localic}.
\end{definition}
\end{frame}
\begin{frame}
		\begin{definition}\label{hered_defn}\
		A cone $M$ in the positive part of $C^*$-algebra $A$ is said to be \textit{hereditary} if $0 \le x \le y$, $y \in M$ implies $x \in M$ for each $x \in A$. A $C^*$-subalgebra $B$ of $A$ is \alert{hereditary} if $B_+$ is hereditary in $A_+$.
	\end{definition}
The set   $\mathbf{Hered}/A$ of hereditary subalgebra is  partially ordered.
We define a meet and arbitrary joins on $\mathbf{Hered}/A$ by the following way
\bean
\forall B', B'' \quad B' \wedge B'' \bydef B'\cap B'',~\\\label{join_eqn}
\forall  \left\{B_\iota\right\}_{\iota\in I} \subset  \mathbf{Hered}/A \\ \bigvee_{\iota \in I} B_\iota  \bydef \text{a generated by } \bigcup_{\iota \in I} B_\iota \text{ hereditary } C^*\text{-subalgebra of } A.~ 
\eean
\end{frame}
\begin{frame}
	\begin{lemma}\label{hered_ideal_lem}
	Let $A$ be a $C^*$-algebra.
	\begin{enumerate}
		\item[(i)] If $L$ is a closed left ideal in $A$ then $L\cap L^*$ is a hereditary $C^*$-subalgebra of $A$. The map $L \mapsto L\cap L^*$ is the bijection from the set of closed left ideals of $A$ onto the the set of hereditary $C^*$-subalgebras of $A$.
		\item[(ii)] If $L_1, L_2$ are closed left ideals, then $L_1 \subseteq L_2$ is and only if $L_1\cap L_1^* \subset L_2\cap L_2^*$.
		\item[(iii)] If $B$ is a hereditary $C^*$-subalgebra of $A$, then the set 
		\be\label{left_ideal_eqn}
		L\left(B \right) = \left\{\left.a \in A~\right| a^*a \in B\right\}
		\ee
		is the unique closed left ideal of $A$ corresponding to $B$.
	\end{enumerate}
\end{lemma}

\end{frame}
\begin{frame}
		\begin{theorem}\label{left_ideal_thm}
		%3.1.2. Theorem.
		If $L$ is a closed left ideal in a $C^*$-algebra $A$, then there
		is an increasing net $\left\{u_\la\right\}_{\la\in\La}$ of positive elements in the closed unit ball of
		$L$ such that $a = \lim_{\la\in \La}au_\la $ for all $a\in L$.
	\end{theorem}
\begin{lemma}\alert{Ivankov}.
If $A$ is a $C^*$-algebra then a partially ordered set  $\mathbf{Hered}/A$ of hereditary subalgebras of $A$ is a locale.
\end{lemma}
\begin{proof}
	One needs check a specialization of the equation	$$
	x \wedge \left( \bigvee_\iota y_\iota \right) = \bigvee_\iota\left(x\wedge   y_\iota \right) 
	$$
or equivalently if   $\left\{B_\iota\right\}_{\iota\in I} \subset  \mathbf{Hered}/A$ and $C \in \mathbf{Hered}/A$ then
$
B \cap C = B^\cap 
$ 
where  $B$ is a generated by $\bigcup_{\iota \in I} B_\iota$ \text{ hereditary } $C^*${-subalgebra of } $A$ and $B^\cap$ is a generated by $\bigcup_{\iota \in I}\left(  B_\iota\cap C\right) $ \text{ hereditary } $C^*${-subalgebra of } $A$.	
\end{proof}	
\end{frame}
\begin{frame}
 If $L\left( B\right)$, $L\left( C\right)$ and $L\left( B_\iota \right)$ are given by the equation \eqref{left_ideal_eqn} closed left ideals then $L\left(B\cap C \right) = L(B)\cap L(C)$, $~L\left(B_\iota \cap C \right)= L\left(B_\iota \right)\cap L\left( C \right)$. Moreover $L\left(B\right)$ is the $C^*$-norm closure of an algebraic sum $\sum_{\iota\in I} L\left( B_\iota\right)$. For any $\iota \in I$ one has $L\left(B_\iota\right)\cap L\left(C\right)\subset L\left(B\right)\cap L\left(C\right)$ it follows that $\sum_{\iota\in I} \left( L\left(B_\iota\right)\cap L\left(C\right)\right) \subset L\left(B\right)\cap L\left(C\right)$. Since the left ideal  $L\left(B\right)\cap L\left(C\right)$ is $C^*$-norm closed the $C^*$-norm closure of an algebraic sum $\sum_{\iota\in I} \left( L\left(B_\iota\right)\cap L\left(C\right)\right)$ is a subset of  the intersection $L\left(B\right)\cap L\left(C\right)$ i.e. a generated by a union $\cup_\iota \left( B_\iota\cup C\right)$ hereditary subalgebra is a subalgebra of the intersection $B\cap C$.
 If $\eps > 0$ and $ b\in L\left(C\right)\cap L\left(B\right)$ then from the from the above theorem it follows that there is a positive $u \in L\left(C\right)$ such that 
 \bean
 \left\|u \right\| \le 1,\\
 \left\|b - b u\right\|< \frac{\eps}{2}.
 \eean 
\end{frame}
\begin{frame}

 Since the set $\sum_{\iota\in I}  L\left(B_\iota\right)$ is dense in $L\left(B\right)$ there is a sum $\sum_{k = 1}^n b_{k}$ such that
 \bean
 \forall k\in \{1,..., n\}\quad \exists \iota_k\in I \quad b_k \in L\left( B_{\iota_k}\right)  ,\\
 \left\|b - \sum_{k = 1}^n b_k \right\|< \frac{\eps}{2}.
 \eean
 Applying the triangle identity one has
 $$
 \left\|b -  \left( \sum_{k = 1}^n b_k\right)  u\right\|	\le 	\left\|\left( b -   \sum_{k = 1}^n b_k\right)  u\right\|+ \left\|b -   b  u\right\|< \left\| b -   \sum_{k = 1}^n b_k\right\|\left\|  u\right\|+ \frac{\eps}{2}< \eps.
 $$
 From the following circumstances:
 \begin{itemize}
 	\item  the number $\eps$ is arbitrary small,
 	\item $\left( \sum_{k = 1}^n b_k\right)  u\in \sum_{\iota\in I} \left( L\left(B_\iota\right)\cap L\left(C\right)\right)$
 \end{itemize}
 we conclude that $b$ lies in the $C^*$-norm closure of $\sum_{\iota\in I} \left( L\left(B_\iota\right)\cap L\left(C\right)\right)$. So $L\left(B\right)\cap L\left(C\right)$ equals to the $C^*$-norm closure of $\sum_{\iota\in I} \left( L\left(B_\iota\right)\cap L\left(C\right)\right)$ and $B\cap C$ is a generated by the union $\cup_{\iota\in I} \left(B_\iota\cap C \right)$ hereditary $C^*$-algebra. Thus 
 a family $\left\{B_\iota\cap C \subset B\cap C\right\}_{\iota\in I}$ is a {covering} of $B\cap C$ (cf. Definition \ref{hered_cov_defn}). $L\left( B\right)$, $L\left( C\right)$ and $L\left( B_\iota \right)$ are given by the equation \eqref{left_ideal_eqn} closed left ideals then $L\left(B\cap C \right) = L(B)\cap L(C)$, $~L\left(B_\iota \cap C \right)= L\left(B_\iota \right)\cap L\left( C \right)$ (cf. Lemma \ref{hered_ideal_lem}). Moreover $L\left(B\right)$ is the $C^*$-norm closure of an algebraic sum $\sum_{\iota\in I} L\left( B_\iota\right)$. For any $\iota \in I$ one has $L\left(B_\iota\right)\cap L\left(C\right)\subset L\left(B\right)\cap L\left(C\right)$ it follows that $\sum_{\iota\in I} \left( L\left(B_\iota\right)\cap L\left(C\right)\right) \subset L\left(B\right)\cap L\left(C\right)$. Since the left ideal  $L\left(B\right)\cap L\left(C\right)$ is $C^*$-norm closed the $C^*$-norm closure of an algebraic sum $\sum_{\iota\in I} \left( L\left(B_\iota\right)\cap L\left(C\right)\right)$ is a subset of  the intersection $L\left(B\right)\cap L\left(C\right)$ i.e. a generated by a union $\cup_\iota \left( B_\iota\cup C\right)$ hereditary subalgebra is a subalgebra of the intersection $B\cap C$. If $\eps > 0$ and $ b\in L\left(C\right)\cap L\left(B\right)$ then from the from the Theorem \ref{left_ideal_thm} it follows that there is a positive $u \in L\left(C\right)$ such that 
 \bean
 \left\|u \right\| \le 1,\\
 \left\|b - b u\right\|< \frac{\eps}{2}.
 \eean 
\end{frame}
\begin{frame}
	
	Since the set $\sum_{\iota\in I}  L\left(B_\iota\right)$ is dense in $L\left(B\right)$ there is a sum $\sum_{k = 1}^n b_{k}$ such that
	\bean
	\forall k\in \{1,..., n\}\quad \exists \iota_k\in I \quad b_k \in L\left( B_{\iota_k}\right)  ,\\
	\left\|b - \sum_{k = 1}^n b_k \right\|< \frac{\eps}{2}.
	\eean
	Applying the triangle identity one has
	$$
	\left\|b -  \left( \sum_{k = 1}^n b_k\right)  u\right\|	\le 	\left\|\left( b -   \sum_{k = 1}^n b_k\right)  u\right\|+ \left\|b -   b  u\right\|< \left\| b -   \sum_{k = 1}^n b_k\right\|\left\|  u\right\|+ \frac{\eps}{2}< \eps.
	$$
	From the following circumstances:
	\begin{itemize}
		\item  the number $\eps$ is arbitrary small,
		\item $\left( \sum_{k = 1}^n b_k\right)  u\in \sum_{\iota\in I} \left( L\left(B_\iota\right)\cap L\left(C\right)\right)$
	\end{itemize}
	we conclude that $b$ lies in the $C^*$-norm closure of $\sum_{\iota\in I} \left( L\left(B_\iota\right)\cap L\left(C\right)\right)$. So $L\left(B\right)\cap L\left(C\right)$ equals to the $C^*$-norm closure of $\sum_{\iota\in I} \left( L\left(B_\iota\right)\cap L\left(C\right)\right)$.
\end{frame}
\begin{frame}
\begin{definition}\alert{Ivankov}.
	The explained above locale  $\mathfrak{Locale}\left(A \right)$ is said to be a \alert{Gelfand} $A$-\alert{locale}. Grothendieck topology $J_A$  is said to be a  \alert{Gelfand} $A$-\alert{topology}. A corresponding  site  $\mathbf{T}_A\bydef \left(\mathbf{Hered}/A, J_A\right)$ is said to be \alert{Gelfand}  $A$ -\alert{site} A Grothendieck topos $\mathfrak{Topos}(A)\bydef \mathbf{Sh}\left(\mathbf{T}_A \right)$ is said to be the \alert{Gelfand} $A$-\alert{topos}.
\end{definition}
\end{frame}

\begin{frame}
	\begin{definition}\label{lrc_defn}
	If $A$ is a $C^*$-algebra then a linear map $\la: A\to A$ is said to be a \alert{left centralizer} if
	\be
	\la\left(ab\right)= 	\la\left(a\right) b \quad \forall a, b \in A.
	\ee
	Similarly one defines a \alert{right} centralizer. Denote the spaces of left and right centralizers by $\mathbf{LC}(A)$ and  $\mathbf{RC}(A)$.
\end{definition}
\begin{lem}
	If $\rho\in  \mathbf{RC}(A)$ then $\rho^*\in  \mathbf{LC}(A)$ where $\rho^*\left(a \right)\bydef \left(\rho\left( a^* \right) \right)^*$. 
\end{lem}


\begin{lem}\label{lrc_lem}
	Each left centralizer, and each right centralizer is bounded.
\end{lem}

\end{frame}
\begin{frame}
\begin{definition}\label{hereditary_full_defn}
\alert{Ivankov}.	If both be $A$ and $\widetilde{A}$ be $C^*$-algebras then  a bounded ring homomorphism of $\C$-algebras
	\bean
	\varphi_R: A \hookto \mathbf{RC}\left(\widetilde A\right)
	\eean
	(where $\mathbf{RC}\left(\widetilde A\right)$ is a space of right centralizers cf. Definition \ref{lrc_defn}) is said to be a \alert{hereditary full} if  a left ideal $\widetilde{A} \varphi\left(A \right)$  is dense in $\widetilde{A}$.
	
	Similarly $\varphi_L: A \hookto \mathbf{LC}\left(\widetilde A\right)$ if is said to be a \alert{hereditary full} if  a right ideal $ \varphi\left(A \right)\widetilde{A}$  is dense in $\widetilde{A}$.
	
\end{definition}
\end{frame}
\begin{frame}
\begin{lemma}\label{locale_lem} 
\alert{Ivankov}.	Any hereditary full homomorphism $\varphi_R: A \hookto \mathbf{RC}\left(\widetilde A\right)$ (or $\varphi_L: A \hookto \mathbf{LC}\left(\widetilde A\right)$)  yields a morphism of locales,
	$$
f: \mathfrak{Locale}\left( \widetilde A\right)\xrightarrow{ }\mathfrak{Locale}\left( A\right).
	$$
so there is  a natural localic geometric  morphism of toposes 
	$$\mathfrak{Topos}\left( \widetilde A\right)\xrightarrow{}\mathfrak{Topos}\left( A\right).$$ 
\end{lemma}
\end{frame}


\begin{frame}
\begin{proof}
	Consider frames $\mathfrak{Frame}\left(  A\right)$ and   $\mathfrak{Frame}\left( \widetilde A\right)$. One can define a morphism of frames
	\be
	\begin{split}
		f^{-1}:\mathfrak{Frame}\left(  A\right)\to \mathfrak{Frame}\left( \widetilde A\right);\\
		B \mapsto \quad \text{a generated by } \varphi_R\left(B \right)^*\widetilde{A} \varphi_R\left(B \right) \text{ hereditary subalgebra of } \widetilde A,\\
		\text{or}\\ B \mapsto \quad \text{a generated by } \varphi_L\left(B \right)\widetilde{A} \varphi_L\left(B \right)^* \text{ hereditary subalgebra of } \widetilde A.
	\end{split}
	\ee
This morphism  of frames naturally yields  a morphism of locales
	$$
	f: \mathfrak{Locale}\left( \widetilde A\right)\to\mathfrak{Locale}\left( A\right).
	$$
	
\end{proof}
\end{frame}


\begin{frame}
	%TAKESAKI
\begin{lemma}
If $f$ is a bounded continuous function defined on an open dense 
subset $G$ of a stonean space $\Om$, then $f$ can be extended to a continuous function 
defined on the whole space $\Om$. Therefore, the Stone-\v{C}ech compactification 
of any open dense subset of Q is the whole space $\Om$ itself. 
\end{lemma}


\end{frame}
\end{document}























