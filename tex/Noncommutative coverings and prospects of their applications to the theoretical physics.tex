\documentclass{beamer}
\usepackage{amsmath,amssymb,amsthm,slashed, euscript}

\textwidth=110mm
\mode<presentation>
{
%	\usetheme{Berlin}
	% or ...
	
	\setbeamercovered{transparent}
	% or whatever (possibly just delete it)
}


\textwidth=110mm


\title{Noncommutative coverings and prospects of their applications to the theoretical physics}
\author{Petr R. Ivankov \inst{1} }
\institute
{
	\inst{1}
	CSTS "Dinamika" , Moscow, Russia\\  \ \\
Geometry, Groups, Operator Algebras, and Integrability 2022\\
(June 27–July 2, 2022, Lomonosov Moscow State University, Moscow) \\ \ \\
	( arXiv:1904.13130
	)
}
%Max-Planck-Institut f\ddot ur Mathematik, Bonn}
\date{}

\theoremstyle{plain}
\newtheorem{defn}{Definition}
\newtheorem{rem}{Remark}
\newtheorem{exm}{Example}
\newtheorem*{claim}{Claim}
\newtheorem{prop}{Proposition}
\newtheorem{empt}[prop]{}%[section]
\newtheorem{lem}{Lemma}%[section]
\newtheorem{thm}{Theorem}%[section]



\newcommand{\A}{\mathcal{A}}
\newcommand{\be}{\begin{equation}}
\newcommand{\ee}{\end{equation}}
\newcommand{\Ga}{\Gamma}
\newcommand{\B}{\mathcal{B}}
\newcommand{\Cc}{\mathcal{C}}
\newcommand{\C}{\mathbb{C}}
\newcommand{\D}{\mathcal{D}}
\newcommand{\G}{\mathcal{G}}
\newcommand{\Hc}{\mathcal{H}}
\newcommand{\Lc}{\mathcal{L}}
\newcommand{\Pc}{\mathcal{P}}
\newcommand{\Sc}{\mathcal{S}}
\newcommand{\U}{\mathcal{U}}
\newcommand{\rar}{\rightarrow}
\newcommand{\Ef}{\mathbb{E}}


%Uppercase Gothic characters
\newcommand{\gtA}{\mathfrak{A}}
\newcommand{\gtB}{\mathfrak{B}}
\newcommand{\gtM}{\mathfrak{M}}
\newcommand{\gtN}{\mathfrak{N}}
\newcommand{\gtP}{\mathfrak{P}}
\newcommand{\gtS}{\mathfrak{S}}

%Lowercase Gothic characters
\newcommand{\gtf}{\mathfrak{f}}
\newcommand{\gtg}{\mathfrak{g}}

%Bold Characters
\newcommand{\Cb}{\mathbb{C}}
\newcommand{\Nb}{\mathbb{N}}
\newcommand{\Rb}{\mathbb{R}}
\newcommand{\Zb}{\mathbb{Z}}

%Uppercase Greek characters
\newcommand{\Gm}{\Gamma}
\newcommand{\Te}{\Theta}
\newcommand{\Om}{\Omega}
\newcommand{\s}{ }

%Lowercase Greek characters
\newcommand{\al}{\alpha}
\newcommand{\gm}{\gamma}
\newcommand{\dl}{\delta}
\newcommand{\sg}{\sigma}
\newcommand{\ph}{\varphi}
\newcommand{\te}{\theta}
\newcommand{\ze}{\zeta}

\newcommand{\Id}{\mathrm{Id}}
\newcommand{\Aut}{\mathrm{Aut}}
\newcommand{\Coo}{{\mathrm{C}}^\infty}
\newcommand{\alg}{\mathrm{alg}}
\newcommand{\diag}{\mathrm{diag}}
\newcommand{\spinc}{\textbf{$spin^c$}}
\newcommand{\Hom}{\mathrm{Hom}}
\newcommand{\supp}{\mathrm{supp}}
\newcommand{\Ccl}{\mathbf{C}l}
\newcommand{\xto}{\xrightarrow}

\newcommand{\lto}{\longrightarrow}
\newcommand{\ox}{\otimes}
\newcommand{\nb}{\nabla}
\newcommand{\sS}{\mathcal{S}}
\newcommand{\Dn}{D\!\!\!\!/}
%\newcommand{\ij}{{i,j}}
\newcommand{\aC}{\ensuremath{\underline{\Cb}} }
\newcommand{\scp}[2]{\left\langle{#1},{#2}\right\rangle}
\newcommand{\op}[1]{J{#1}J^\dag}
\newcommand{\sA}{\mathcal{A}} 
\newcommand{\sB}{\mathcal{B}}       %%
\newcommand{\sC}{\mathcal{C}}       %%
\newcommand{\sD}{\mathcal{D}}       %%
\newcommand{\sE}{\mathcal{E}}       %%
\newcommand{\sF}{\mathcal{F}}       %%
\newcommand{\sG}{\mathcal{G}}       %%
\newcommand{\sH}{\mathcal{H}}       %%
\newcommand{\sI}{\mathcal{I}}       %%
\newcommand{\sJ}{\mathcal{J}}       %%
\newcommand{\sK}{\mathcal{K}}       %%
\newcommand{\sL}{\mathcal{L}}       %%
\newcommand{\sM}{\mathcal{M}}       %%
\newcommand{\sN}{\mathcal{N}}       %%
\newcommand{\sO}{\mathcal{O}}       %%
\newcommand{\sP}{\mathcal{P}}       %%
\newcommand{\sQ}{\mathcal{Q}}       %%
\newcommand{\sR}{\mathcal{R}}       %%
\newcommand{\sT}{\mathcal{T}}       %%
\newcommand{\sU}{\mathcal{U}}       %%
\newcommand{\sV}{\mathcal{V}}       %%
\newcommand{\sX}{\mathcal{X}}       %%
\newcommand{\sY}{\mathcal{Y}}       %%
\newcommand{\sZ}{\mathcal{Z}}       %%
\newcommand{\N}{\mathbb{N}}                  %% 

\renewcommand{\a}{\alpha}     
\newcommand{\la}{\lambda}     
\newcommand{\La}{\Lambda}
\newcommand{\bt}{\beta}           %% short for  \beta
 
    
\newcommand{\bydef}{\stackrel{\mathrm{def}}{=}}  
\newcommand{\hookto}{\hookrightarrow}        %% abbreviation
  
\begin{document}
%\titlepage
\begin{frame}
  \titlepage
\end{frame}
\begin{frame}
There are theories of coverings of $C^*$-algebras which can be included into a following list:
\begin{itemize}
	\item Coverings of commutative $C^*$-algebras.  \alert{C. Canlubo, A. Pavlov, E. Troitsky}
	\item Coverings of $C^*$-algebras of foliations  \alert{Moto O'uchi,   Xiaolu Wang}.
	\item  Coverings of noncommutative tori \alert{C. Canlubo, K. Schwieger, S. Wagner}.
	\item {The double covering of the quantum group $SO_q(3)$} \alert{Dijkhuizen Mathijs S., P. Podle\'{s}}. 
\end{itemize}
This work is devoted to a single general theory which includes all theories of this list, i.e. we develop a  system of axioms which can be applied for every element of the list.
\end{frame}
	\section{Noncommutative finite-fold coverings}

\begin{frame}
	   \begin{definition}\label{pre_defn} \alert{P. Ivankov}.
		Let $\pi: A \hookto \widetilde{A}$ be an injective *-homomorphism of connected  $C^*$-algebras such that following conditions hold:
		\begin{enumerate}
			\item[(a)] If $\Aut\left(\widetilde{A} \right)$ is a group of *-automorphisms of $\widetilde{A}$ then the group  
			$
			G \bydef \left\{ \left.g \in \Aut\left(\widetilde{A} \right)~\right|~ ga = a;~~\forall a \in A\right\}
			$
			is finite.
			\item[(b)] 	\be\label{cond_b_eqn}
			A \cong \widetilde{A}^G\stackrel{\text{def}}{=}\left\{\left.a\in \widetilde{A}~~\right|~ a = g a;~ \forall g \in G\right\}.\ee
		\end{enumerate}
		We say that the quadruple $\left(A, \widetilde{A}, G, \pi \right)$ and/or *-homomorphism $\pi: A \to \widetilde{A}$   is a \textit{noncommutative finite-fold  pre-covering}. 
	\end{definition}
\begin{lem}  \alert{P. Ivankov}.
If $\left(A, \widetilde{A}, G, \pi \right)$ is noncommutative finite-fold  pre-covering such that
\begin{itemize}
	\item $A$ is commutative,
	\item $\widetilde{A}$ is a finitely generated $A$-module.
\end{itemize}
Then $\widetilde{A}$ is a commutative $C^*$-algebra.
\end{lem}
\end{frame}

\begin{frame}
\begin{example}
Let $\widetilde{\sX} \to \sX$ be a covering of locally compact Hausdorff spaces.
 $A \bydef C_0\left(\mathcal{X}\right)$ and $\widetilde{A}\bydef C_0\left(\mathcal{X}\right)$ then then $\widetilde{A}$ is a finitely generated $A$-module and the group
			\be\nonumber
G \bydef \left\{ \left.g \in \Aut\left(\widetilde{A} \right)~\right|~ ga = a;~~\forall a \in A\right\}
\ee
is finite.
\end{example}\begin{example}
If $\mathcal{X}$ is a locally compact Hausdorff space, $A = C_0\left(\mathcal{X}\right)$ and $\widetilde{A}= A\otimes \mathbb{M}_n\left(\mathbb{C} \right)$  then $\widetilde{A}$ is a finitely generated $A$-module and however the group
\be\nonumber
G \bydef \left\{ \left.g \in \Aut\left(\widetilde{A} \right)~\right|~ ga = a;~~\forall a \in A\right\}
\ee

is not finite.
\end{example}
\end{frame}
\begin{frame}

	\begin{theorem}\alert{A. Pavlov, E Troitsky}
		Suppose $\mathcal X$ and $\mathcal Y$ are compact Hausdorff connected spaces and $p :\mathcal  Y \to \mathcal X$
is a continuous surjection. If $C(\mathcal Y )$ is a projective finitely generated Hilbert module over
$C(\mathcal X)$ with respect to the action
\begin{equation*}
(f\xi)(y) = f(y)\xi(p(y)), ~ f \in  C(\mathcal Y ), ~ \xi \in  C(\mathcal X),
\end{equation*}
then $p$ is a finite-fold  covering.
	\end{theorem}


\begin{corollary}\alert{P. Ivankov}
	If $\left(A, \widetilde{A}, G, \pi \right)$ is noncommutative finite-fold  pre-covering such that
	\begin{itemize}
		\item $A$ is commutative,
		\item Both $A$ and $\widetilde{A}$ are unital.
		\item $\pi$ is unital 
	\item $\widetilde{A}$	is a finitely generated projecive $A$-module.
	\end{itemize}
	then $\pi$ corresponds to a finite-fold covering $ \widetilde{\mathcal  X}\to \mathcal  X$.
\end{corollary}
\end{frame}
\begin{frame}
\begin{definition}
	\alert{P. Ivankov}
	  	Let $\left(A, \widetilde{A}, G, \pi \right)$ be a  noncommutative finite-fold  pre-covering. Suppose both $A$ and  $\widetilde{A}$ are unital. We say that $\left(A, \widetilde{A}, G, \pi \right)$ is an \textit{unital noncommutative finite-fold  covering} if \\ $\pi$ is unital and $\widetilde{A}$ is a finitely generated projective  $A$-module.
\end{definition}
\begin{corollary}\alert{P. Ivankov}
	Any unital noncommutative finite-fold  covering $\left(A, \widetilde{A}, G, \pi \right)$ of a commutative $C^*$-algebra $A$ corresponds to a finite-fold covering $ \widetilde{\mathcal  X}\to \mathcal  X$.
\end{corollary}
\end{frame}

\begin{frame}

\begin{definition}
	A covering $ p:\widetilde{\mathcal  X}\to \mathcal  X$ is said to be \textit{transitive} if for any $x \in \mathcal  X$ the group
	$$
	G\left(\left.\widetilde{\mathcal  X} \right|\mathcal  X\right)= \left\{ \left.g \in \text{Homeo}\left(\widetilde{\mathcal X} \right)~\right|~ p(g\widetilde{x}) = p(\widetilde{x});~~\forall \widetilde{x} \in \widetilde{\mathcal  X}\right\} 
	$$
	transitively acts on $p^{-1}\left(x\right)$.
\end{definition}
\begin{fact}
	If the space $ \mathcal  X$ is locally path connected then any transitive covering  of $ \mathcal  X$  is regular and vice versa.
\end{fact}
\begin{fact}
	A covering $ p:\widetilde{\mathcal  X}\to \mathcal  X$ is {transitive} if and only if there is the natural homeomoprphism
	$$
\mathcal  X \cong \widetilde{\mathcal  X}/		G\left(\left.\widetilde{\mathcal  X} \right|\mathcal  X\right).
	$$
	
\end{fact}
\end{frame}
\begin{frame}
\begin{lemma}
	\alert{P. Ivankov}
	If $\mathcal  X$ is a connected, compact, Hausdorff space then there is a 1-1 correspondence between finite-fold transitive coverings of $\mathcal  X$ and unital noncommutative finite-fold  coverings of $C\left(\mathcal  X\right)$.
\end{lemma}

\begin{proof}
We already know that any  unital noncommutative finite-fold  covering corresponds to the finite-fold covering of $\widetilde{\mathcal  X}\to \mathcal  X$. One should prove that the covering is transitive. From the condition \eqref{cond_b_eqn} it follows that
$$
C\left(\mathcal  X\right)\cong C\left(\widetilde{\mathcal  X}\right)^{G\left(\left.\widetilde{\mathcal  X} \right|\mathcal  X\right)}.
$$
The above equation is equivalent to
$$
{\mathcal  X}\cong\widetilde{\mathcal  X}/ G\left(\left.\widetilde{\mathcal  X} \right|\mathcal  X\right).
$$
so the covering is transitive. The converse statement can be proved similarly.
\end{proof}
\end{frame}
\begin{frame}


\begin{definition}\label{top_cov_comp_defn}
	\alert{P. Ivankov}. 	A   covering $p: \widetilde{   \mathcal X } \to \mathcal X$ is said to be a \textit{ covering with compactification} if there are compactifications ${   \mathcal X } \hookto {   \mathcal Y }$ and $\widetilde{   \mathcal X } \hookto \widetilde{   \mathcal Y }$ such that:
	\begin{itemize}
		\item There is a finite-fold  covering $\widetilde{p}:\widetilde{   \mathcal Y }\to {   \mathcal Y }$,
		\item The covering $p$ is the restriction of $\widetilde{p}$, i.e. $p = \widetilde{p}|_{\widetilde{   \mathcal X }}$.
	\end{itemize}
\end{definition}
\begin{definition}\label{fin_comp_defn}\alert{P. Ivankov}
	Let $\left(A, \widetilde{A}, G, \pi \right)$ be a noncommutative finite-fold  pre-covering such  that following conditions hold:
	\begin{enumerate}
		\item[(a)] 
		There are unitizations $A \hookto B$, $\widetilde{A} \hookto \widetilde{B}$.
		%	\item[(b)]$A = B\bigcap \widetilde{A}$,
		\item[(b)] There is an %(strong) 
		unital  noncommutative finite-fold covering	$\left(B ,\widetilde{B}, G, \widetilde{\pi} \right)$ such that $\pi = \widetilde{\pi}|_A$ (or, equivalently $A \cong \widetilde{A}\cap B$) and the action $G \times\widetilde{A} \to \widetilde{A}$ is induced by $G \times\widetilde{B} \to \widetilde{B}$.
	\end{enumerate}
	We say that the  quadruple $\left(A, \widetilde{A}, G, \pi \right)$ is a
	\textit{noncommutative finite-fold covering with unitization}. 
\end{definition}

\end{frame}
\begin{frame}
\begin{example}
	
	Let $\mathcal X  = \C \setminus \{0\}$ be a complex plane with punctured 0, which is parametrized by the complex variable $z$. 
	If $\widetilde{   \mathcal X } \cong \mathcal X$ then for any natural $n>1$ there is a finite-fold covering 
	\begin{equation*}
	p: \widetilde{   \mathcal X } \to \mathcal X,
	\qquad	z \mapsto z^n.
	\end{equation*}
	which is not a covering with compactification. Really let  $\mathcal X \to\mathcal Y$ be any compactification. If both $\left\{z'_\al \in \mathcal X\right\}$, 	$\left\{z''_\al \in \mathcal X\right\}$ are  nets such that $\lim_{\al}\left|z'_\al\right|=\lim_\al\left|z''_\al\right| = 0$ then form $\lim_{\al}\left|z'_\al-z''_\al\right|= 0$ it turns out that
	$
	x_0 = \lim_{\al} z'_\al = \lim_{\al} z''_\al \in \mathcal Y
	$. 	It turns out $\left|\widetilde{p}^{-1}\left(x_0 \right) \right|=1$. However $\widetilde{p}$ is an $n$-fold covering and if $n >1$ then  $\left|\widetilde{p}^{-1}\left(x_0 \right) \right|=n>1$.
\end{example}
\end{frame}
\begin{frame}
\begin{lemma}
	\alert{P. Ivankov}. If $\mathcal X$ is a locally compact, connected, Hausdorff space then there is a one-to-one correspondence between transitive finite-fold coverings with compactifications and of $\mathcal X$ and noncommutative finite-fold coverings with unitization of $C_0\left(\mathcal X\right)$.
\end{lemma}
\begin{definition}
	\alert{P. Ivankov}.	A   noncommutative finite-fold  pre-covering $\left(A, \widetilde{A}, G, \pi \right)$ is said to be  a \textit{noncommutative finite-fold covering} if there is an increasing net $\left\{u_\lambda\right\}_{\lambda\in\Lambda}\subset M\left( A\right)_+ $  of positive elements such that
	\begin{enumerate}
		\item[(a)] There is the limit 
		$
		\beta\text{-}\lim_{\lambda \in \Lambda} u_\lambda = 1_{M\left(A \right) }
		$
		in the strict topology of $M\left(A \right)$,
		\item[(b)]  If for all   $\lambda\in\Lambda$ both $A_\lambda$ and  $\widetilde A_\lambda$ are $C^*$-norm completions  of $u_\lambda A u_\lambda$ and  $u_\lambda\widetilde{A}u_\lambda$ respectively then for every $\lambda\in\Lambda$ a quadruple
		$
		\left(A_\lambda, \widetilde{A}_\lambda, G, \left.\pi\right|_{A_\lambda} :A_\lambda\to \widetilde{A}_\lambda\right)	
		$
		is a noncommutative finite-fold covering with unitization. The action 	$G \times \widetilde{A}_\lambda\to \widetilde{A}_\lambda$, is the restriction on $\widetilde{A}_\lambda\subset \widetilde{   A}$ of the action $G\times  \widetilde{A}\to \widetilde{A}$.
	\end{enumerate}
	
\end{definition}
\end{frame}
\begin{frame}
\begin{lem}
	\alert{P. Ivankov}. 	If $\mathcal X$ is a connected, locally connected, locally compact, Hausdorff space  then any finite-fold transitive covering of $\sX$ corresponds to a noncommutative finite-fold  covering of $C_0\left(\sX \right)$. 
\end{lem}
\end{frame}


\section{Infinite noncommutative coverings}
\begin{frame}
	\begin{rem}
		There is a complicated definition of infinite noncommutative coverings  complete which  takes about 40 pages. One can find it here https://arxiv.org/abs/1904.13130. 
	\end{rem}
	\begin{definition}
		A group $G$ is said to be \alert{residually finite} if for each  element  $g\in G$ there is a finite group $K$ and a homomorphism $\varphi: G\to K$ such that $\varphi\left( g\right) \neq 1$.
	\end{definition}
\begin{theorem}\alert{P. Ivankov}
If $\sX$ is a connected, locally connected, locally compact, Hausdorff space then there is a natural 1-1 correspondence between infinite noncommutative coverings of $C_0\left(\sX \right)$ and transitive coverings $\widetilde \sX \to \sX$ such that the group  $G\left(\left.\widetilde \sX \right| \sX\right)$ of covering transformations  is residually finite.
\end{theorem}
\end{frame}


\begin{frame}
	There is a very complicated construction of infinite noncommutative coverings.
	\end{frame}
\begin{frame}
	\begin{center}
		\Huge
		Expect further results.
		\newline
		I appreciate if you join this promising research.
		\newline
		\alert{Thank You!}
	\end{center}
\end{frame}




















\end{document}























