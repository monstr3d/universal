\documentclass[10]{article}
%\documentclass[11pt]{book}
\usepackage{hyperref}
\usepackage{amsfonts,amssymb,amsmath,amsthm,cite}
\usepackage{graphicx}
\usepackage[toc,page]{appendix}
\usepackage{nicefrac}
%% \usepackage[francais]{babel}
\usepackage[applemac]{inputenc}
\usepackage{amssymb, euscript}
\usepackage[matrix,arrow,curve]{xy}
\usepackage{graphicx}
\usepackage{tabularx}
\usepackage{float}
\usepackage{tikz}
\usepackage{slashed}
\usepackage{mathrsfs}
\usepackage{multirow}

%\usepackage{mathtools}

\usetikzlibrary{matrix}

\usepackage[T1]{fontenc}
\usepackage{amsfonts,cite}
\usepackage{graphicx}

%% \usepackage[francais]{babel}
\usepackage[applemac]{inputenc}


\usepackage[sc]{mathpazo}
\usepackage{environ}

\linespread{1.05}         % Palatino needs more leading (space between lines)


%\usepackage[usenames]{color}



\DeclareFontFamily{T1}{pzc}{}
\DeclareFontShape{T1}{pzc}{m}{it}{1.8 <-> pzcmi8t}{}
\DeclareMathAlphabet{\mathpzc}{T1}{pzc}{m}{it}
% the command for it is \mathpzc

\textwidth=140mm


% % % % % % % % % % % % % % % % % % % %
\theoremstyle{plain}
\newtheorem{prop}{Proposition}[section]
\newtheorem{prdf}[prop]{Proposition and Definition}
\newtheorem{lem}[prop]{Lemma}%[section]
\newtheorem{cor}[prop]{Corollary}%[section]
\newtheorem{thm}[prop]{Theorem}%[section]
\newtheorem{theorem}[prop]{Theorem}
\newtheorem{lemma}[prop]{Lemma}
\newtheorem{proposition}[prop]{Proposition}
\newtheorem{corollary}[prop]{Corollary}
\newtheorem{statement}[prop]{Statement}

\theoremstyle{definition}
\newtheorem{defn}[prop]{Definition}%[section]
\newtheorem{cordefn}[prop]{Corollary and Definition}%[section]
\newtheorem{empt}[prop]{}%[section]
\newtheorem{exm}[prop]{Example}%[section]
\newtheorem{rem}[prop]{Remark}%[section]
\newtheorem{prob}[prop]{Problem}
\newtheorem{conj}{Conjecture}       %% Hypothesis 1
\newtheorem{cond}{Condition}        %% Condition 1
%\newtheorem{axiom}[thm]{Axiom}           %% Axiom 1 modified
\newtheorem{fact}[prop]{Fact}
\newtheorem{ques}{Question}         %% Question 1
\newtheorem{answ}{Answer}           %% Answer 1
\newtheorem{notn}{Notation}        %% Notations are not numbered

\theoremstyle{definition}
\newtheorem{notation}[prop]{Notation}
\newtheorem{definition}[prop]{Definition}
\newtheorem{example}[prop]{Example}
\newtheorem{exercise}[prop]{Exercise}
\newtheorem{conclusion}[prop]{Conclusion}
\newtheorem{conjecture}[prop]{Conjecture}
\newtheorem{criterion}[prop]{Criterion}
\newtheorem{summary}[prop]{Summary}
\newtheorem{axiom}[prop]{Axiom}
\newtheorem{problem}[prop]{Problem}
%\theoremstyle{remark}
\newtheorem{remark}[prop]{Remark}

\numberwithin{equation}{section}
\newtheorem*{claim}{Claim}
\DeclareMathOperator{\Dom}{Dom}              %% domain of an operator
\newcommand{\Dslash}{{D\mkern-11.5mu/\,}}    %% Dirac operator


%\newcommand\myeq{\stackrel{\mathclap{\normalfont\mbox{def}}}{=}}
\newcommand{\nor}[1]{\left\Vert #1\right\Vert}    %\nor{x}=||x||
\newcommand{\vertiii}[1]{{\left\vert\kern-0.25ex\left\vert\kern-0.25ex\left\vert #1
		\right\vert\kern-0.25ex\right\vert\kern-0.25ex\right\vert}}
\newcommand{\Ga}{\Gamma}  
\newcommand{\coker}{\mathrm{coker}}                   %% short for  \Gamma
\newcommand{\Coo}{C^\infty}                  %% smooth functions
% % % % % % % % % % % % % % % % % % % %


\usepackage[sc]{mathpazo}
\linespread{1.05}         % Palatino needs more leading (space between lines)

\newbox\ncintdbox \newbox\ncinttbox %% noncommutative integral symbols
\setbox0=\hbox{$-$} \setbox2=\hbox{$\displaystyle\int$}
\setbox\ncintdbox=\hbox{\rlap{\hbox
		to \wd2{\hskip-.125em \box2\relax\hfil}}\box0\kern.1em}
\setbox0=\hbox{$\vcenter{\hrule width 4pt}$}
\setbox2=\hbox{$\textstyle\int$} \setbox\ncinttbox=\hbox{\rlap{\hbox
		to \wd2{\hskip-.175em \box2\relax\hfil}}\box0\kern.1em}

\newcommand{\ncint}{\mathop{\mathchoice{\copy\ncintdbox}%
		{\copy\ncinttbox}{\copy\ncinttbox}%
		{\copy\ncinttbox}}\nolimits}  %% NC integral

%%% Repeated relations:
\newcommand{\xyx}{\times\cdots\times}      %% repeated product
\newcommand{\opyop}{\oplus\cdots\oplus}    %% repeated direct sum
\newcommand{\oxyox}{\otimes\cdots\otimes}  %% repeated tensor product
\newcommand{\wyw}{\wedge\cdots\wedge}      %% repeated exterior product
\newcommand{\subysub}{\subset\hdots\subset}      %% repeated subset
\newcommand{\supysup}{\supset\hdots\supset}      %% repeated supset
\newcommand{\rep}{\mathfrak{rep}}
\newcommand{\lift}{\mathfrak{lift}}
\newcommand{\desc}{\mathfrak{desc}}
%%% Roman letters:
\newcommand{\id}{\mathrm{id}}                %% identity map
\newcommand{\Id}{\mathrm{Id}}                %% identity map
\newcommand{\pt}{\mathrm{pt}}                %% a point
\newcommand{\const}{\mathrm{const}}          %% a constant
\newcommand{\codim}{\mathrm{codim}}          %% codimension
\newcommand{\cyc}{\mathrm{cyclic}}  %% cyclic sum
\renewcommand{\d}{\mathrm{d}}       %% commutative differential
\newcommand{\dR}{\mathrm{dR}}       %% de~Rham cohomology
\newcommand{\proj}{\mathrm{proj}}                %% a projection



\newcommand*{\Mult}{\mathcal M}% multiplier algebra

\newcommand{\A}{\mathcal{A}}                 %%\newcommand{\unitsv}[1]{#1^{(0)}}
\newcommand{\units}{G^{(0)}}
\newcommand{\haars}{\{\lambda^{u}\}_{u\in\units}}
\newcommand{\shaars}{\{\lambda_{u}\}_{u\in\units}}
\newcommand{\haarsv}[2]{\{\lambda^{#2}_{#1}\}_{#2\in\unitsv{#1}}}
\newcommand{\haarv}[2]{\lambda^{#2}_{#1}}

\renewcommand{\a}{\alpha}                    %% short for  \alphapha
\DeclareMathOperator{\ad}{ad}                %% infml adjoint repn
\newcommand{\as}{\quad\mbox{as}\enspace}     %% `as' with spacing
\newcommand{\Aun}{\widetilde{\mathcal{A}}}   %% unital algebra
\newcommand{\B}{\mathcal{B}}                 %% space of distributions
\newcommand{\E}{\mathcal{E}}                 %% space of distributions
\renewcommand{\b}{\beta}                     %% short for \beta
\newcommand{\braCket}[3]{\langle#1\mathbin|#2\mathbin|#3\rangle}
\newcommand{\braket}[2]{\langle#1\mathbin|#2\rangle} %% <w|z>
\newcommand{\C}{\mathbb{C}}                  %% complex numbers
\newcommand{\CC}{\mathcal{C}}                %% space of distributions
\newcommand{\cc}{\mathbf{c}}                 %% Hochschild cycle
\DeclareMathOperator{\Cl}{C\ell}             %% Clifford algebra
\newcommand{\F}{\mathcal{F}}                 %% space of test functions
\newcommand{\G}{\mathcal{G}}                 %% 
\newcommand{\D}{\mathcal{D}}                 %% Moyal L^2-filtration
\renewcommand{\H}{\mathcal{H}}               %% Hilbert space
\newcommand{\half}{\tfrac{1}{2}}             %% small fraction  1/2
\newcommand{\hh}{\mathcal{H}}                %% Hilbert space
\newcommand{\hookto}{\hookrightarrow}        %% abbreviation
\newcommand{\Ht}{{\widetilde{\mathcal{H}}}}  %% Hilbert space of forms
\newcommand{\I}{\mathcal{I}}                 %% tracelike functions
\DeclareMathOperator{\Junk}{Junk}            %% the junk DGA ideal
\newcommand{\K}{\mathcal{K}}                 %% compact operators
\newcommand{\ket}[1]{|#1\rangle}             %% ket vector
\newcommand{\ketbra}[2]{|#1\rangle\langle#2|} %% rank one operator
\renewcommand{\L}{\mathcal{L}}               %% operator algebra
\newcommand{\La}{\Lambda}                    %% short for \Lambda
\newcommand{\la}{\lambda}                    %% short for \lambda
\newcommand{\lf}{L_f^\theta}                 %% left mult operator
\newcommand{\M}{\mathcal{M}}                 %% Moyal multplr algebra
\newcommand{\mm}{\mathcal{M}^\theta}
%\newcommand{{{\star_{\theta}}}{{\mathchoice{\mathbin{\;|\;ar_{_\theta}}}
			%            {\mathbin{\;|\;ar_{_\theta}}}           %% Moyal
			%            {{\;|\;ar_\theta}}{{\;|\;ar_\theta}}}}    %% product
	\newcommand{\N}{\mathbb{N}}                  %% nonnegative integers
	\newcommand{\NN}{\mathcal{N}}                %% a Moyal algebra
	\newcommand{\nb}{\nabla}                     %% gradient
	\newcommand{\Oh}{\mathcal{O}}                %% comm multiplier alg
	\newcommand{\om}{\omega}                     %% short for \omega
	\newcommand{\opp}{{\mathrm{op}}}             %% opposite algebra
	\newcommand{\ox}{\otimes}                    %% tensor product
	\newcommand{\eps}{\varepsilon}                    %% tensor product
	\newcommand{\otimesyox}{\otimes\cdots\otimes}    %% repeated tensor product
	\newcommand{\pa}{\partial}                   %% short for \partial
	\newcommand{\pd}[2]{\frac{\partial#1}{\partial#2}}%% partial derivative
	\newcommand{\piso}[1]{\lfloor#1\rfloor}      %% integer part
	\newcommand{\PsiDO}{\Psi~\mathrm{DO}}         %% pseudodiffl operators
	\newcommand{\Q}{\mathbb{Q}}                  %% rational numbers
	\newcommand{\R}{\mathbb{R}}                  %% real numbers
	\newcommand{\rdl}{R_\Dslash(\lambda)}        %% resolvent
	\newcommand{\roundbraket}[2]{(#1\mathbin|#2)} %% (w|z)
	\newcommand{\row}[3]{{#1}_{#2},\dots,{#1}_{#3}} %% list: a_1,...,a_n
	\newcommand{\sepword}[1]{\quad\mbox{#1}\quad} %% well-spaced words
	\newcommand{\set}[1]{\{\,#1\,\}}             %% set notation
	\newcommand{\Sf}{\mathbb{S}}                 %% sphere
	\newcommand{\uhor}[1]{\Omega^1_{hor}#1}
	\newcommand{\sco}[1]{{\sp{(#1)}}}
	\newcommand{\sw}[1]{{\sb{(#1)}}}
	\DeclareMathOperator{\spec}{sp}              %% spectrum
	\renewcommand{\SS}{\mathcal{S}}              %% Schwartz space
	\newcommand{\sss}{\mathcal{S}}               %% Schwartz space
	\DeclareMathOperator{\supp}{\mathfrak{supp}}            %% support
	\newcommand{\T}{\mathbb{T}}                  %% circle as a group
	\renewcommand{\th}{\theta}                   %% short for \theta
	\newcommand{\thalf}{\tfrac{1}{2}}            %% small* fraction 1/2
	\newcommand{\tihalf}{\tfrac{i}{2}}           %% small* fraction i/2
	\newcommand{\tpi}{{\tilde\pi}}               %% extended representation
	\DeclareMathOperator{\Tr}{Tr}                %% trace of operator
	\DeclareMathOperator{\tr}{tr}                %% trace of matrix
	\newcommand{\del}{\partial}                  %% short for  \partial
	\DeclareMathOperator{\tsum}{{\textstyle\sum}} %% small sum in display
	\newcommand{\V}{\mathcal{V}}                 %% test function space
	\newcommand{\vac}{\ket{0}}                   %% vacuum ket vector
	\newcommand{\vf}{\varphi}                    %% scalar field
	\newcommand{\w}{\wedge}                      %% exterior product
	\DeclareMathOperator{\wres}{wres}            %% density of Wresidue
	\newcommand{\x}{\times}                      %% cross
	\newcommand{\Z}{\mathbb{Z}}                  %% integers
	\newcommand{\7}{\dagger}                     %% short for + symbol
	\newcommand{\8}{\bullet}                     %% anonymous degree
	\renewcommand{\.}{\cdot}                     %% anonymous variable
	\renewcommand{\:}{\colon}                    %% colon in  f: A -> B
	
	%\newcommand{\sA}{\mathscr{A}}       %%
	\newcommand{\sA}{\mathcal{A}} 
	\newcommand{\sB}{\mathcal{B}}       %%
	\newcommand{\sC}{\mathcal{C}}       %%
	\newcommand{\sD}{\mathcal{D}}       %%
	\newcommand{\sE}{\mathcal{E}}       %%
	\newcommand{\sF}{\mathcal{F}}       %%
	\newcommand{\sG}{\mathcal{G}}       %%
	\newcommand{\sH}{\mathcal{H}}       %%
	\newcommand{\sI}{\mathcal{I}}       %%
	\newcommand{\sJ}{\mathcal{J}}       %%
	\newcommand{\sK}{\mathcal{K}}       %%
	\newcommand{\sL}{\mathcal{L}}       %%
	\newcommand{\sM}{\mathcal{M}}       %%
	\newcommand{\sN}{\mathcal{N}}       %%
	\newcommand{\sO}{\mathcal{O}}       %%
	\newcommand{\sP}{\mathcal{P}}       %%
	\newcommand{\sQ}{\mathcal{Q}}       %%
	\newcommand{\sR}{\mathcal{R}}       %%
	\newcommand{\sS}{\mathcal{S}}       %%
	\newcommand{\sT}{\mathcal{T}}       %%
	\newcommand{\sU}{\mathcal{U}}       %%
	\newcommand{\sV}{\mathcal{V}}       %%
	\newcommand{\sX}{\mathcal{X}}       %%
	\newcommand{\sY}{\mathcal{Y}}       %%
	\newcommand{\sZ}{\mathcal{Z}}       %%
	
	\newcommand{\Om}{\Omega}       %%
	
	
	\DeclareMathOperator{\ptr}{ptr}     %% Poisson trace
	\DeclareMathOperator{\Trw}{Tr_\omega} %% Dixmier trace
	\DeclareMathOperator{\vol}{Vol}     %% total volume
	\DeclareMathOperator{\Vol}{Vol}     %% total volume
	\DeclareMathOperator{\Area}{Area}   %% area of a surface
	\DeclareMathOperator{\Wres}{Wres}   %% (Wodzicki) residue
	
	\newcommand{\dd}[1]{\frac{\partial}{\partial#1}}   %% partial derivation
	\newcommand{\ddt}[1]{\frac{d}{d#1}}                %% derivative
	\newcommand{\inv}[1]{\frac{1}{#1}}                 %% inverse
	\newcommand{\sfrac}[2]{{\scriptstyle\frac{#1}{#2}}} %% tiny fraction
	
	\newcommand{\bA}{\mathbb{A}}       %%
	\newcommand{\bB}{\mathbb{B}}       %%
	\newcommand{\bC}{\mathbb{C}}       %%
	\newcommand{\bCP}{\mathbb{C}P}     %%
	\newcommand{\bD}{\mathbb{D}}       %%
	\newcommand{\bE}{\mathbb{E}}       %%
	\newcommand{\bF}{\mathbb{F}}       %%
	\newcommand{\bG}{\mathbb{G}}       %%
	\newcommand{\bH}{\mathbb{H}}       %%
	\newcommand{\bHP}{\mathbb{H}P}     %%
	\newcommand{\bI}{\mathbb{I}}       %%
	\newcommand{\bJ}{\mathbb{J}}       %%
	\newcommand{\bK}{\mathbb{K}}       %%
	\newcommand{\bL}{\mathbb{L}}       %%
	\newcommand{\bM}{\mathbb{M}}       %%
	\newcommand{\bN}{\mathbb{N}}       %%
	\newcommand{\bO}{\mathbb{O}}       %%
	\newcommand{\bOP}{\mathbb{O}P}     %%
	\newcommand{\bP}{\mathbb{P}}       %%
	\newcommand{\bQ}{\mathbb{Q}}       %%
	\newcommand{\bR}{\mathbb{R}}       %%
	\newcommand{\bRP}{\mathbb{R}P}     %%
	\newcommand{\bS}{\mathbb{S}}       %%
	\newcommand{\bT}{\mathbb{T}}       %%
	\newcommand{\bU}{\mathbb{U}}       %%
	\newcommand{\bV}{\mathbb{V}}       %%
	\newcommand{\bX}{\mathbb{X}}       %%
	\newcommand{\bY}{\mathbb{Y}}       %%
	\newcommand{\bZ}{\mathbb{Z}}       %%
	
	\newcommand{\bydef}{\stackrel{\mathrm{def}}{=}}          %% 
	
	
	\newcommand{\al}{\alpha}          %% short for  \alpha
	\newcommand{\bt}{\beta}           %% short for  \beta
	\newcommand{\Dl}{\Delta}          %% short for  \Delta
	\newcommand{\dl}{\delta}          %% short for  \delta
	\newcommand{\ga}{\gamma}          %% short for  \gamma
	\newcommand{\ka}{\kappa}          %% short for  \kappa
	\newcommand{\sg}{\sigma}          %% short for  \sigma
	\newcommand{\Sg}{\Sigma}          %% short for  \Sigma
	\newcommand{\Th}{\Theta}          %% short for  \Theta
	\renewcommand{\th}{\theta}        %% short for  \theta
	\newcommand{\vth}{\vartheta}      %% short for  \vartheta
	\newcommand{\ze}{\zeta}           %% short for  \zeta
	
	\DeclareMathOperator{\ord}{ord}     %% order of a PsiDO
	\DeclareMathOperator{\rank}{rank}   %% rank of a vector bundle
	\DeclareMathOperator{\sign}{sign}   %%
	\DeclareMathOperator{\sgn}{sgn}   %%
	\DeclareMathOperator{\chr}{char}   %%
	\DeclareMathOperator{\ev}{ev}       %% evaluation
	
	
	\newcommand{\Op}{\mathbf{Op}}
	\newcommand{\As}{\mathbf{As}}
	\newcommand{\Com}{\mathbf{Com}}
	\newcommand{\LLie}{\mathbf{Lie}}
	\newcommand{\Leib}{\mathbf{Leib}}
	\newcommand{\Zinb}{\mathbf{Zinb}}
	\newcommand{\Poiss}{\mathbf{Poiss}}
	
	\newcommand{\gX}{\mathfrak{X}}      %% vector fields
	\newcommand{\sol}{\mathfrak{so}}    %% special orthogonal Lie algebra
	\newcommand{\gm}{\mathfrak{m}}      %% maximal ideal
	
	
	\DeclareMathOperator{\Res}{Res}
	\DeclareMathOperator{\NCRes}{NCRes}
	\DeclareMathOperator{\Ind}{Ind}
	%% co/homology theories
	\DeclareMathOperator{\rH}{H}        %% any co/homology
	\DeclareMathOperator{\rC}{C}        %%  any co/chains
	\DeclareMathOperator{\rZ}{Z}        %% cycles
	\DeclareMathOperator{\rB}{B}        %% boundaries
	\DeclareMathOperator{\rF}{F}        %% filtration
	\DeclareMathOperator{\Gr}{gr}        %% associated graded object
	\DeclareMathOperator{\rHc}{H_{\mathrm{c}}}   %% co/homology with compact support
	\DeclareMathOperator{\drH}{H_{\mathrm{dR}}}  %% de Rham co/homology
	\DeclareMathOperator{\cechH}{\check{H}}    %% Cech co/homology
	\DeclareMathOperator{\rK}{K}        %% K-groups
	\DeclareMathOperator{\rKO}{KO}        %% real K-groups
	\DeclareMathOperator{\rKU}{KU}        %% unitary K-groups
	\DeclareMathOperator{\rKSp}{KSp}        %% symplectic K-groups
	\DeclareMathOperator{\rR}{R}        %% representation ring
	\DeclareMathOperator{\rI}{I}        %% augmentation ideal
	\DeclareMathOperator{\HH}{HH}       %% Hochschild co/homology
	\DeclareMathOperator{\HC}{HC}       %% cyclic co/homology
	\DeclareMathOperator{\HP}{HP}       %% periodic cyclic co/homology
	\DeclareMathOperator{\HN}{HN}       %% negative cyclic co/homology
	\DeclareMathOperator{\HL}{HL}       %% Leibniz co/homology
	\DeclareMathOperator{\KK}{KK}       %% KK-theory
	\DeclareMathOperator{\KKK}{\mathbf{KK}}       %% KK-theory as a category
	\DeclareMathOperator{\Ell}{Ell}       %% Abstract elliptic operators
	\DeclareMathOperator{\cd}{cd}       %% cohomological dimension
	\DeclareMathOperator{\spn}{span}       %% span
	\DeclareMathOperator{\linspan}{span} %% linear span (can't use \span)
	\newcommand{\blank}{-}   
	
	
	
	\newcommand{\twobytwo}[4]{\begin{pmatrix} #1 & #2 \\ #3 & #4 \end{pmatrix}}
	\newcommand{\CGq}[6]{C_q\!\begin{pmatrix}#1&#2&#3\\#4&#5&#6\end{pmatrix}}
	%% q-Clebsch--Gordan coefficients
	\newcommand{\cz}{{\bullet}}         %% anonymous degree
	\newcommand{\nic}{{\vphantom{\dagger}}} %% invisible dagger
	\newcommand{\ep}{{\dagger}}         %% abbreviation for + symbol
	\newcommand{\downto}{\downarrow}    %% right hand limit
	\newcommand{\isom}{\cong}          %% isomorphism
	\newcommand{\lt}{\triangleright}    %% a left action
	\newcommand{\otto}{\leftrightarrow} %% bijection
	\newcommand{\rt}{\triangleleft}     %% a right action
	\newcommand{\semi}{\rtimes}         %% crossed product
	\newcommand{\tensor}{\otimes}       %% tensor product
	\newcommand{\cotensor}{\square}       %% cotensor product
	\newcommand{\trans}{\pitchfork}     %% transverse
	\newcommand{\ul}{\underline}        %% for sheaves
	\newcommand{\upto}{\uparrow}        %% left hand limit
	\renewcommand{\:}{\colon}           %% colon in  f: A -> B
	\newcommand{\blt}{\ast}
	\newcommand{\Co}{C_{\bullet}}
	\newcommand{\cCo}{C^{\bullet}}
	\newcommand{\nbs}{\nabla^S}         %% spin connection
	\newcommand{\up}{{\mathord{\uparrow}}} %% `up' spinors
	\newcommand{\dn}{{\mathord{\downarrow}}} %% `down' spinors
	\newcommand{\updn}{{\mathord{\updownarrow}}} %% up or down
	
	%%% Bilinear enclosures:
	
	\newcommand{\bbraket}[2]{\langle\!\langle#1\stroke#2\rangle\!\rangle}
	%% <<w|z>>
	\newcommand{\bracket}[2]{\langle#1,\, #2\rangle} %% <w,z>
	\newcommand{\scalar}[2]{\langle#1,\,#2\rangle} %% <w,z>
	\newcommand{\poiss}[2]{\{#1,\,#2\}} %% {w,z}
	\newcommand{\dst}[2]{\langle#1,#2\rangle} %% distributions <u,\phi>
	\newcommand{\pairing}[2]{(#1\stroke #2)} %% right-linear pairing
	\def\<#1|#2>{\langle#1\stroke#2\rangle} %% \braket (Dirac notation)
	\def\?#1|#2?{\{#1\stroke#2\}}        %% left-linear pairing
	
	%%% Accent-like macros:
	
	\renewcommand{\Bar}[1]{\overline{#1}} %% closure operator
	\renewcommand{\Hat}[1]{\widehat{#1}}  %% short for \widehat
	\renewcommand{\Tilde}[1]{\widetilde{#1}} %% short for \widetilde
	
	
	\DeclareMathOperator{\bCl}{\bC l}   %% complex Clifford algebra
	
	%%% Small fractions in displays:
	
	\newcommand{\ihalf}{\tfrac{i}{2}}   %% small fraction  i/2
	\newcommand{\quarter}{\tfrac{1}{4}} %% small fraction  1/4
	\newcommand{\shalf}{{\scriptstyle\frac{1}{2}}}  %% tiny fraction  1/2
	\newcommand{\third}{\tfrac{1}{3}}   %% small fraction  1/3
	\newcommand{\ssesq}{{\scriptstyle\frac{3}{2}}} %% tiny fraction  3/2
	\newcommand{\sesq}{{\mathchoice{\tsesq}{\tsesq}{\ssesq}{\ssesq}}} %% 3/2
	\newcommand{\tsesq}{\tfrac{3}{2}}   %% small fraction  3/2
	
	
	%\newcommand\eqdef{\overset{\mathclap{\normalfont\mbox{def}}}{=}}
	\newcommand\eqdef{\overset{\mathrm{def}}{=}}
	
	
	%+++++++++++++++++++++++++++++++++++
	
	\newcommand{\word}[1]{\quad\text{#1}\enspace} %% well-spaced words
	\newcommand{\words}[1]{\quad\text{#1}\quad} %% better-spaced words
	\newcommand{\su}[1]{{\sp{[#1]}}}
	
	\def\<#1,#2>{\langle#1,#2\rangle}            %% bilinear pairing
	\def\ee_#1{e_{{\scriptscriptstyle#1}}}       %% basis projector
	\def\wick:#1:{\mathopen:#1\mathclose:}       %% Wick-ordered operator
	
	\newcommand{\opname}[1]{\mathop{\mathrm{#1}}\nolimits}
	
	\newcommand{\hideqed}{\renewcommand{\qed}{}} %% to suppress `\qed'
	
	
	%%%%%%%%%%%%%%%%%%%%%%%%%%%%%
	%% 2. Some internal machinery
	%%%%%%%%%%%%%%%%%%%%%%%%%%%%%
	
	\newbox\ncintdbox \newbox\ncinttbox %% noncommutative integral symbols
	\setbox0=\hbox{$-$}
	\setbox2=\hbox{$\displaystyle\int$}
	\setbox\ncintdbox=\hbox{\rlap{\hbox
			to \wd2{\box2\relax\hfil}}\box0\kern.1em}
	\setbox0=\hbox{$\vcenter{\hrule width 4pt}$}
	\setbox2=\hbox{$\textstyle\int$}
	\setbox\ncinttbox=\hbox{\rlap{\hbox
			to \wd2{\hskip-.05em\box2\relax\hfil}}\box0\kern.1em}
	
	\newcommand{\disp}{\displaystyle} %% short for  \displaystyle
	
	%\newcommand{\hideqed}{\renewcommand{\qed}{}} %% no `\qed' at end-proof
	
	\newcommand{\stroke}{\mathbin|}   %% (for `\bbraket' and such)
	\newcommand{\tribar}{|\mkern-2mu|\mkern-2mu|} %% norm bars: |||
	
	%%% Enclose one argument with delimiters:
	
	\newcommand{\bra}[1]{\langle{#1}\rvert} %% bra vector <w|
	\newcommand{\kett}[1]{\lvert#1\rangle\!\rangle} %% ket 2-vector |y>>
	\newcommand{\snorm}[1]{\mathopen{\tribar}{#1}%
		\mathclose{\tribar}}                 %% norm |||x|||
	
	
	\newcommand{\End}{\mathrm{End}}       %%
	\newcommand{\Ext}{\mathrm{Ext}}       %%
	\newcommand{\Hom}{\mathrm{Hom}}       %%
	\newcommand{\Mrt}{\mathrm{Mrt}}       %%
	\newcommand{\grad}{\mathrm{grad}}       %%
	\newcommand{\Spin}{\mathrm{Spin}}       %%
	\newcommand{\Ad}{\mathrm{Ad}}       %%
	\newcommand{\Pic}{\mathrm{Pic}}       %%
	\newcommand{\Aut}{\mathrm{Aut}}       %%
	\newcommand{\Inn}{\mathrm{Inn}}       %%
	\newcommand{\Out}{\mathrm{Out}}       %%
	\newcommand{\Homeo}{\mathrm{Homeo}}       %%
	\newcommand{\Diff}{\mathrm{Diff}}       %%
	\newcommand{\im}{\mathrm{im}}       %%
	
	
	\newcommand{\SO}{\mathrm{SO}}       %%
	\newcommand{\SU}{SU}       %%
	\newcommand{\gso}{\mathfrak{so}}    %% special orthogonal Lie algebra
	\newcommand{\gero}{\mathfrak{o}}    %% orthogonal Lie algebra
	\newcommand{\gspin}{\mathfrak{spin}} %% spin Lie algebra
	\newcommand{\gu}{\mathfrak{u}}      %% unitary Lie algebra
	\newcommand{\gsu}{\mathfrak{su}}    %% special unitary Lie algebra
	\newcommand{\gsl}{\mathfrak{sl}}    %% special linear Lie algebra
	\newcommand{\gsp}{\mathfrak{sp}}    %% symplectic linear Lie algebra
	
	%\newcommand{\bes}{\begin{equation}\begin{split}}
			%\newcommand{\ees}{\end{split}\end{equation}}
	%\NewEnviron{split.enviro}{%
		%	\begin{equation}\begin{split}
				%	\BODY
				%	\end{split}\end{equation}
		%$}
	\newenvironment{splitequation}{\begin{equation}\begin{split}}{\end{split}\end{equation}}
	
	%Begin equation split: Begin equation split = bes
	\newcommand{\bs}{\begin{split}}
		\newcommand{\es}{\end{split}}
	\newcommand{\be}{\begin{equation}}
		\renewcommand{\ee}{\end{equation}}
	\newcommand{\bea}{\begin{eqnarray}}
		\newcommand{\eea}{\end{eqnarray}}
	\newcommand{\bean}{\begin{eqnarray*}}
		\newcommand{\eean}{\end{eqnarray*}}
	\newcommand{\brray}{\begin{array}}
		\newcommand{\erray}{\end{array}}
	\newenvironment{equations}
	{\begin{equation}
			\begin{split}}
			{\end{split}
	\end{equation}}
	\newcommand{\Hsquare}{%
		\text{\fboxsep=-.2pt\fbox{\rule{0pt}{1ex}\rule{1ex}{0pt}}}%
	}
	
	\title{Weak fundamental group}
	
	\author
	{\textbf{Petr R. Ivankov*}\\
		e-mail: * monster.ivankov@gmail.com \\
	}
	
	\begin{document}
		
		\maketitle  %\setlength{\parindent}{0pt}
		\pagestyle{plain}
		
		
		%\vspace{1 in}
		
		
		%\noindent
		
		\begin{abstract}
 The classical notion.
		\end{abstract}
		
		
		%\end{abstract}
		
		\section{General theory}
		\paragraph{}
		The notion of the fundamental group  is relevant to path connected spaces (cf. Definition \ref{top_path_connected_defn}). Here we consider a generalized notion which can be applied for more general connected spaces.
		\begin{definition}\label{top_weakly_simply_connected_defn} 
			A connected space $\sX$ is said to be \textit{weakly simply connected} if for any covering $p: \widetilde \sX\to \sX$ the space $\sX$ is evenly covered by $p$ (cf. Definition \ref{top_covering_defn})
		\end{definition}
		\begin{remark}
			Any connected, path connected, simply connected  space is weakly simply connected.
		\end{remark}
		\begin{remark}
			The Definition \ref{simply_connected_defn} can be regarded as a generalization of \ref{top_weakly_simply_connected_defn} one (cf. Theorem \ref{top_fin_thm} below).
		\end{remark}
		
		\begin{defn}\label{top_weakly_semi1_defn}\cite{spanier:at}
			An open connected subset $\sU\subset \sX$ of a topological space is said to be \textit{semilocally proper} it is evenly covered by any covering $\widetilde{\sX}\to\sX$ (cf. Definition \ref{top_covering_defn}).
			A space $\sX$ is said to be \textit{weakly semilocally 1-connected} if for  every point $x$ and there is a semilocally proper neighborhood.
		\end{defn} 
		\begin{remark}
			Any connected, path connected, locally path connected, semilocally 1-connected (cf. Definition \ref{top_semi1_defn}),  space is weakly semilocally 1-connected.
		\end{remark}
		\begin{definition}\label{top_weak_path_defn}
			Let $\sX$ be a topological space.	A \textit{weak path on} $\sX$ is a finite sequence $\left(\sU_1, ..., \sU_n\right)$ of open  connected subsets of $\sX$ such that $\sU_j \cap \sU_{j+1} \neq \emptyset$ for any $j = 1, ..., n - 1$. A weak path $\left(\sU_1, ..., \sU_n\right)$ is said to be \textit{weakly semilocally 1-connected} if $\sU_j$ is semilocally proper for all $j = 1,..., n$.
		\end{definition}
		\begin{empt}\label{top_path_inv_empt}
			There is an involution * on the set of paths such that
			\be\label{top_path_inv_eqn}
			\left(\sU_{1},...,\sU_{n}\right)^*\bydef \left(\sU_{n},...,\sU_{1}\right).
			\ee
		\end{empt}
		\begin{definition}\label{top_path_comp_defn}
			If both $\mathfrak p'=\left(\sU'_{1},...,\sU'_{k}\right)$ and $\mathfrak p''=\left(\sU''_{1},...,\sU_{l}''\right)$ are paths such that $\sU'_{k} \cap \sU''_{1}\neq\emptyset$ then we say that a pair $\left( \mathfrak p', \mathfrak p''\right)$ is \textit{composable}, and the path $\mathfrak p=\left(\sU'_{1},...,\sU'_{k},\sU''_{_1},...,\sU_{l}'' \right)$  is said to be the \textit{composition} of $\mathfrak p'$ and $\mathfrak p''$. We write
			\be\label{top_path_comp_eqn}
			\mathfrak p'\circ \mathfrak p''\bydef\mathfrak p.
			\ee
		\end{definition}
		\begin{remark}\label{top_path_ass_rem}
			The given by the Definition \ref{top_path_comp_defn} composition is associative, i.e.  if both pairs $\left( \mathfrak p', \mathfrak p''\right)$ and $\left( \mathfrak p'', \mathfrak p'''\right)$ are composable then one has
			\be\label{top_path_ass_eqn}
			\left( \mathfrak p'\circ \mathfrak p''\right)  \circ \mathfrak p''' = \mathfrak p'\circ \left( \mathfrak p'' \circ \mathfrak p'''\right) .
			\ee 
		\end{remark}
		
		%\begin{rem}
		%	There is groupoid of paths with composition and inversion given by the equations \eqref{top_path_comp_eqn} and \eqref{top_path_inv_eqn} respectively.
		%\end{rem}
		If $p: \widetilde{\sX} \to \sX$ is a covering and $\widetilde{\mathfrak p}\bydef \left(\widetilde\sU_1, ..., \widetilde\sU_n\right)$ is weak path on $\widetilde\sX$ then $\mathfrak p\bydef \left(p\left( \widetilde\sU_1\right) , ..., p\left( \widetilde\sU_n\right) \right)$ is a weak path on $\sX$.
		\begin{defn}\label{top_path_desc_defn}
			In the above situation we say that $\mathfrak p$ is a $p$-\textit{descent} of $\widetilde{\mathfrak p}$. We write
			\be\label{top_path_desc_eqn}
			\mathfrak p\bydef \desc_p\left( \widetilde{\mathfrak p}\right). 
			\ee
		\end{defn}
		\begin{definition}\label{top_gen_path_defn}
			Let $\sX$ be a connected, locally connected, locally compact, Hausdorff space.
			Let $\mathfrak{A}\bydef\left\{\sU_\a\right\}_{\a \in \mathscr A}$ be a family of connected open subsets of $\sX$, such that $\sX = \cup~ \sU_\a$ and  a closure of $\sU_\a$ is compact for all $\a\in \mathscr A$. A weak path $\left(\sU_{\a_1},...,\sU_{\a_n}\right)$ (cf. Definition \ref{top_weak_path_defn}) on $\sX$  is said to be an  $\mathfrak{A}$-\textit{path} if  $\sU_j \in \mathfrak{A}$ for all $j = 1,...,n-1$.
		\end{definition}
		
		
		
		\begin{lemma}\label{top_gen_path_lem}
			Consider the situation of the Definition \ref{top_gen_path_defn}. For any $\sU_{\a'}, \sU_{\a''} \in \mathfrak{A}$ there is a $\left\{\sU_\a\right\}$-{path} $\left(\sU_{\a_1},...,\sU_{\a_n}\right)$ such that $\sU_{\a_1}=\sU_{\a'}$ and $\sU_{\a_n}=\sU_{\a''}$ .
		\end{lemma}
		
		\begin{proof}
			Denote by
			\bean
			\mathfrak{A}_{\a'}\bydef \left\{\sU_{\a''}\in \mathfrak A\left|\exists \text{ path }\left(\sU_{\a_1},...,\sU_{\a_n}\right)\quad\a_1 = \a',\quad \a_n = \a'' \right.\right\},\\
			\mathscr A'\bydef \left\{\left.\a \in \mathscr A\right| \sU_{\a} \in\mathfrak{A}_{\a'} \right\}.
			\eean
			If $\mathscr A'=\mathscr A$ then this lemma is proven. If not then  and 
			$$
			\left( \bigcup_{\bt \in\mathscr A'}\sU_{\bt}\right) \bigcap \left( \bigcup_{\a \in\mathscr A\setminus\mathscr A'}\sU_{\a}\right)\neq \emptyset,
			$$
			then there is  $\bt \in\mathscr A'$ and $\a \in\mathscr A\setminus\mathscr A'$ such that $\sU_{\bt}\cap \sU_{\a}\neq \emptyset$. According to the definition of $\mathfrak{A}_{\a'}$ there is a path $\left(\sU_{\a_1},...,\sU_{\a_n}\right)$ such that $\a_1 = \a',\quad \a_n = \bt$. So one has a path $\left(\sU_{\a_1},...,\sU_{\a_n}, \sU_{\a} \right)$, it follows that $\a \in \mathscr A'$ and one has a contradiction. It turns out that 
			\bean
			\left( \bigcup_{\bt \in\mathscr A'}\sU_{\bt}\right) \bigcap \left( \bigcup_{\a \in\mathscr A\setminus\mathscr A'}\sU_{\a}\right)= \emptyset.
			\eean
			Both unions $\bigcup_{\bt \in\mathscr A'}\sU_{\bt}$ and  $\bigcup_{\a \in\mathscr A\setminus\mathscr A'}\sU_{\a}$ of open sets are open, so one has a disjoint union
			$$
			\sX = \left( \bigcup_{\bt \in\mathscr A'}\sU_{\bt}\right) \bigsqcup \left( \bigcup_{\a \in\mathscr A\setminus\mathscr A'}\sU_{\a}\right),
			$$
			i.e.  $\sX$ is not connected. It contradicts with our assumption.	From this contradiction it turns out that $\mathscr A' = \mathscr A$.  
		\end{proof}
		
		\begin{corollary}\label{top_connected_union_cor}
			If $\sX$ is a connected, locally connected, locally compact, Hausdorff space  then there is a  directed set $\La$ and  family $\left\{\sU_\la\right\}_{\la\in\La}$ of connected open subsets of $\sX$ such that
			\begin{itemize}
				\item 
				$$
				\sX = \bigcup_{\la\in \La}\sU_\la,
				$$
				\item a closure of $\sU_\la$ is compact for each $\la\in \La$,
				\item
				\be\label{top_lmit_eqn}
				\forall \mu, \nu \in \La  \quad \sU_\mu \cap \sU_\nu \neq \emptyset,
				\ee
				\item
				\be\label{top_lmord_eqn}
				\forall \mu, \nu \in \La \quad \mu \le \nu\quad\Leftrightarrow \quad \sU_\mu \subset \sU_\nu.
				\ee
			\end{itemize}
		\end{corollary}
		\begin{proof}
			Consider a situation of the Lemma \ref{top_gen_path_lem}. Select $\a_0 \in \mathscr A$. For any $\a\in \mathscr A$ we select a path $\left\{\sU_\a\right\}$-{path} $\left(\sU_{\a_1},...,\sU_{\a_n}\right)$ such that $\sU_{\a_1}=\sU_{\a_0}$ and $\sU_{\a_n}=\sU_{\a}$ (cf Lemma \ref{top_gen_path_lem}). For all $\a \in \mathscr A$ a closure of a finite union $\sV_{\a}\bydef \bigcup \sU_{\a_j}$ is compact because a closure of $\sU_{\a_j}$ is compact for every $j = 1,..., n$. From $\sU_{\a_j}\cap \sU_{\a_{j+1}}\neq \emptyset$ for all $j=1,...,n-1$ it turns out that $\sV_{\a}$ is connected, and taking into account $\bigcup_{\a\in \mathscr A}\sU_\a = \sX$ one has 
			\bean
			\forall \a\in \mathscr A \quad  \sU_\a\subset \sV_\a\quad \Rightarrow\quad \bigcup_{\a\in \mathscr A}\sU_\a\subset \bigcup_{\a\in \mathscr A}\sV_\a \quad \Rightarrow\quad\bigcup_{\a\in \mathscr A}\sV_\a= \sX.
			\eean 
			If $\left\{\sU_\la\right\}_{\la \in \La}$ is a family of all finite unions  $$\sU_\la\bydef \bigcup_{j=1}^n \sV_{\a_j}$$ then from $\sU_{\a_0} \subset \sV_{\a}$ it follows that $\sU_\la$ is connected for all $\la \in \La$ and the condition \ref{top_lmit_eqn} holds.
			The condition \eqref{top_lmord_eqn} follows from $\sU_\a \subset \sU_\la$ for all $\la\in \La$.
			A union of two finite unions is also finite union so one has
			$$
			\forall \mu, \nu \in \La\quad \exists \la \in \La\quad \sU_\mu \cup \sU_\nu = \sU_\la.
			$$
			So if we consider the given by \eqref{top_lmord_eqn} pre-order then $\La$ is a directed set.	
		\end{proof}
		
		\begin{defn}\label{top_closed_path_defn}
			If $\left(\sX, x_0\right)$ is a pointed space then a path $\left(\sU_1, ..., \sU_n\right)$ on $\sX$ is said to be $x_0$-\textit{path} if $x_0\in\sU_1$.  The  $x_0$-{path} is said to be \textit{closed} if $x_0\in\sU_n$
		\end{defn}
		\begin{rem}
			Any pair of closed path is composable.
		\end{rem}
		\begin{exercise}\label{top_path_lift_exer}
			Let $p: \left(\widetilde\sX, \widetilde x_0\right)\to\left(\sX, x_0\right)$ be a pointed covering, and let $\mathfrak{A}\bydef\left\{\sU_\a\right\}_{\a \in \mathscr A}$ 
			be a family of connected all open subsets of $\sX$ evenly covered by $p$ (cf. Definition \ref{top_covering_defn}). Let  ${\mathfrak p}\bydef \left(\sU_1, ..., \sU_n\right)$ be a $\mathfrak{A}$-{path}, such that $x_0\in\sU_1$. 
			Prove that there is an unique  $\widetilde{\mathfrak p}\bydef \left(\widetilde \sU_1, ...,\widetilde \sU_n\right)$ such that $\widetilde x_0 \in \widetilde \sU_1$ and ${\mathfrak p}$ is the $p$-descent of $\widetilde{\mathfrak p}$.
		\end{exercise}
		
		\begin{definition}\label{top_path_lift_defn}
			The given by the Exercise  \ref{top_path_lift_exer} path $\widetilde{\mathfrak p}$ is said to be the $p$-\textit{lift} of $\mathfrak p$. We write
			\be\label{top_path_lift_eqn}
			\lift_p~\mathfrak p\bydef\widetilde{\mathfrak p}.
			\ee 
		\end{definition}
		\begin{definition}\label{top_contactible_path_defn}
			Let 
			$\left(\sX, x_0\right)$ be  a pointed space such that  $\sX$ is a connected, locally connected and weakly semilocally 1-connected space.  A closed  semilocally 1-connected path $x_0$-path $\left(\sU_1, ..., \sU_n\right)$ (cf. Definition \ref{top_weak_path_defn} and \ref{top_closed_path_defn}) is said to be \textit{contractible} if for any pointed covering  $p: \left(\widetilde\sX, \widetilde x_0\right)\to\left(\sX, x_0\right)$ a $p$-{lift} of $\mathfrak p$ is closed (cf. Definitions \ref{top_closed_path_defn} and \ref{top_path_lift_defn}).
		\end{definition}
		
		\begin{exercise}\label{top_path_inv_exer} 
			Prove that 	if $\mathfrak p$ is a closed semilocally 1-connected $x_0$-path then both compositions $\mathfrak p^* \circ \mathfrak p$ and $\mathfrak p \circ \mathfrak p^*$ are contractible.
		\end{exercise}
		\begin{definition}
			We say that both closed, weakly semilocally 1-connected  $x_0$-paths $\mathfrak p_1$ and $\mathfrak p_2$ are \textit{homotopically equivalent} if a composition $\mathfrak p_1 \circ \mathfrak p^*_2$ is contractible.
		\end{definition}
		
		
		From the Remark \ref{top_path_ass_rem} and the Exercise \ref{top_path_inv_exer} it follows that the set of classes of closed weakly semilocally 1-connected $x_0$-paths modulo homotopical equivalence  is a group with  a composition  and an inversion given by $\circ$ and $*$ operations respectively.
		\begin{definition}\label{top_weak_fundamental_group_defn}
			If
			$\left(\sX, x_0\right)$ is  a pointed space, such that $\sX$ is a connected, locally connected and weakly semilocally 1-connected space
			then the described above group is said to be a \textit{weak fundamental group} of $\sX$. We denote it by $\pi_1^{\mathrm{w}}\left(\sX, x_0 \right) $.
		\end{definition}
		
		\begin{exercise}\label{top_fundamental_group_mor_exer}
			Let  $p: \left( \widetilde{\sX}, \widetilde{x}_0\right) \to \left(\sX, x_0\right)$ be a pointed covering. Prove that the $p$-descent (cf. Definition  \ref{top_path_desc_defn}) yields an injective homomorphism $\pi_1^{\mathrm{w}}\left(p \right): \pi_1^{\mathrm{w}}\left(\widetilde\sX, \widetilde x_0 \right)\hookto \pi_1^{\mathrm{w}}\left(\sX, x_0 \right)$. Prove a generalization of the Theorem \ref{top_fundamental_group_mor_thm}.
		\end{exercise}
		\begin{exercise}
			Let $\sX$ be a connected, locally path-connected  and  semilocally 1-connected (cf. Definition \ref{top_semi1_defn}) Hausdorff space. Prove that there is a natural isomorphism
			\be
			\pi_1^{\mathrm{w}}\left(\sX, x_0 \right)\cong \pi_1\left(\sX, x_0 \right)
			\ee
			between the weak fundamental group, and the explained in the Remark \ref{top_homotopy_group_rem} one.
		\end{exercise}
		Let $\left(\sX, x_0\right)$ be  a pointed space, such that $\sX$ is a connected, locally connected and weakly semilocally 1-connected space. 
		For any $x \in \sX$ denote by $\mathfrak{Paths} \left( x_0, x\right)$ a set of all semilocally 1-connected $x_0$-paths such that $$
		\left( \sU_1, ..., \sU_n\right)\in \mathfrak{Paths} \left( x_0, x\right) \quad \Leftrightarrow \quad x_0 \in \sU_1 \quad x \in \sU_n.
		$$
		There is an equivalence relation $\sim_{\left( x_0, x\right)}$ on $\mathfrak{Paths} \left( x_0, x\right)$ such that
		$$
		\mathfrak p_1, \mathfrak p_2 \in  \mathfrak{Paths} \left( x_0, x\right)~~ \mathfrak p_1\sim_{\left( x_0, x\right)} \mathfrak p_2 ~~\Leftrightarrow ~~ \mathfrak p^*_1\circ  \mathfrak p_2 \text{ is contractible (cf. Definition \ref{top_contactible_path_defn})}.
		$$
		A class of equivalence of $\mathfrak p \in \mathfrak{Paths} \left( x_0, x\right)$ will be denoted by $$\left[\mathfrak p\right]_{\left( x_0, x\right)}\in \mathfrak{Paths} \left( x_0, x\right)/\sim_{\left( x_0, x\right)}.$$
		Let $\widetilde \sX$ be a set of pairs $\left(x, \left[\mathfrak p\right]_{\left( x_0, x\right)}\right)$ where $$\left[\mathfrak p\right]_{\left( x_0, x\right)}\in  \mathfrak{Paths} \left( x_0, x\right)/\sim_{\left( x_0, x\right)}.$$ There is a natural map 
		\be
		\begin{split}
			\widetilde p: \widetilde \sX\to \sX,\\
			\left(x, \left[\mathfrak p\right]_{\left( x_0, x\right)}\right)\mapsto x.
		\end{split}
		\ee
		For any $\mathfrak p = \left( \sU_1, ..., \sU_n\right)\in \mathfrak{Paths} \left( x_0, x\right)$ we define a set 
		\be\label{top_u_path_eqn}
		\widetilde \sU_{\mathfrak p} \bydef \left\{\left.  \left( x',  \left[\mathfrak p\right]_{\left( x_0, x'\right)}\right)\in \widetilde{\sX} \right|x' \in \sU_n\right\}.
		\ee
		\begin{lemma}\label{top_uni_top_lem}
			If $\sX$ is a connected,  weakly semilocally 1-connected (cf. Definition \ref{top_weakly_semi1_defn}) space then a family of sets given by the equation \eqref{top_u_path_eqn} a {basis} for a topology on $\widetilde\sX$. 
		\end{lemma}
		\begin{proof}
			One needs check conditions (a) and (b) of the Definition \ref{top_base_defn}. The condition (a) is evident, let us check the condition (b). Let both ${\mathfrak p'}\bydef \left( \sU'_1, ..., \sU'_{n'}\right)$ and ${\mathfrak p''}\bydef \left( \sU''_1, ..., \sU''_{n''}\right)$ be paths  such that there is $\widetilde x \in \widetilde \sU_{\mathfrak p'}\cap \widetilde \sU_{\mathfrak p''}$. It follows   that $\widetilde p\left(\widetilde x \right) \in \sU'_{n'}\cap \sU''_{n''}$. From 
			\eqref{top_u_path_eqn} it turns out that
			\be\label{top_up_path_eqn}
			\left[\mathfrak p'\right]_{\left( x_0, \widetilde p\left(\widetilde x \right)\right)}= \left[\mathfrak p''\right]_{\left( x_0, \widetilde p\left(\widetilde x \right)\right)}
			\ee
			From the Definition \ref{top_weakly_semi1_defn} it turns out that there is a weakly 1-connected neighborhood $\sU$ of $p\left(\widetilde x \right)$ which is a subset of $\sU'_{n'}\cap \sU''_{n''}$. If ${\mathfrak p}\bydef \left( \sU'_1, ..., \sU'_{n'}, \sU\right)$ then from \eqref{top_up_path_eqn} one has
			\be\label{top_upp_path_eqn}
			\left[\mathfrak p\right]_{\left( x_0, \widetilde p\left(\widetilde x \right)\right)}=\left[\mathfrak p'\right]_{\left( x_0, \widetilde p\left(\widetilde x \right)\right)}= \left[\mathfrak p''\right]_{\left( x_0, \widetilde p\left(\widetilde x \right)\right)}.
			\ee
			From \eqref{top_upp_path_eqn} it follows that $\widetilde x \in \widetilde \sU_{\mathfrak p}$ and  $\widetilde\sU_{\mathfrak p}\subset \widetilde\sU_{\mathfrak p'}\cap \widetilde\sU_{\mathfrak p''}$.
		\end{proof}
		\begin{lemma}
			If we consider the given by the Lemma \ref{top_uni_top_lem} topology on $\widetilde \sX$ then the map $p: \widetilde \sX \to \sX$ is a covering.
		\end{lemma}
		\begin{proof}
			For any  $x \in \sX$  there is an open weakly 1-connected neighborhood $\sU$.  If $\widetilde x \in p^{-1}\left(x\right)$ then $\widetilde{x}= \left( x,  \left[\mathfrak p\right]_{\left( x_0, x\right)}\right)$. If $\mathfrak p= \left( \sU_1, ..., \sU_{n}\right)$ and $\mathfrak p'= \left( \sU_1, ..., \sU_{n}, \sU\right)$ and $\widetilde \sU_{\mathfrak p'}$ is given by  \eqref{top_u_path_eqn} then $p\left( \widetilde \sU_{\mathfrak p'}\right)= \sU$ and $\widetilde \sU_{\mathfrak p'}$ is mapped homeomorphically onto $\sU$. So for all $\widetilde x \in p^{-1}\left(x\right)$ one can find an open neighborhood $\widetilde \sU_{\widetilde x}$ which is is mapped homeomorphically onto $\sU$, i.e. $\sU$ is evenly covered by $p$.
		\end{proof}
		Let $p': \widetilde \sX'\to \sX$ is a covering then we define a map $\widetilde p': \widetilde \sX \to \widetilde \sX'$ by a following way. For any $\widetilde{x}= \left( x,  \left[\mathfrak p\right]_{\left( x_0, x\right)}\right)$ we define a unique $p'$-lift  $\left( \widetilde\sU'_1, ..., \widetilde\sU'_{n}\right)$ of $\mathfrak p = \left( \sU_1, ..., \sU_{n}\right)$. There is a unique $\widetilde x' \in \widetilde \sX'$ such that $\left\{\widetilde x'\right\}= p^{-1}\left( x\right) \cap \widetilde\sU'_{n}$. So one can define a map 
		\be\label{top_from_uni_defn}
		\begin{split}
			\widetilde p': \widetilde \sX \to \widetilde \sX',\\
			\widetilde x \mapsto  \widetilde x'.
		\end{split}
		\ee
		\begin{exercise}\label{top_uni_cov_exer}
			Prove that the given by \eqref{top_from_uni_defn} map is a covering.
		\end{exercise}
		From the Exercise \ref{top_uni_cov_exer} it follows that $\widetilde \sX$ is a universal covering space of $\sX$ (cf. Definition \ref{top_universal_covering_defn}).
		So one has the following Theorem.
		\begin{theorem}\label{top_simply_con_cov_thm}If $\sX$ is a connected, locally connected, weakly semilocally 1-connected space then there is an universal covering space for $\sX$.
		\end{theorem}
		
		\begin{remark}
			The Theorem \ref{top_simply_con_cov_thm} is a generalization of the Lemma \ref{top_simply_con_cov_lem}.
		\end{remark}
		\begin{exercise}\label{top_weak_covering_iso_exer}
			Let $\left(\sX, x_0\right)$ be  a pointed space, such that $\sX$ is a connected, locally connected and weakly semilocally 1-connected space, and let $p: \widetilde\sX\to\sX$ is an universal covering. Prove that there is a following natural isomorphism
			$$
			\pi^{\text{w}}_1\left(\sX, x_0\right)\cong G\left(\left. \widetilde \sX\right|\sX \right)  
			$$
			where $\pi^{\text{w}}_1\left(\sX, x_0\right)$ is a weak fundamental group (cf. Definition \ref{top_weak_fundamental_group_defn}) and $G\left(\left. \widetilde \sX\right|\sX \right)$ is a group of covering transformations (cf. Definition \ref{top_covering_transformations_group_defn}). 
			
		\end{exercise}
		\section{Example}
		\begin{lemma}\label{top_lex_square_lem}
			The ordered square $I^2_o$ (cf. \ref{top_lex_square_empt}) is weakly simply connected (cf. Definition \ref{top_weakly_simply_connected_defn}).
		\end{lemma}
		
		\begin{proof}
			It is known that the space $I^2_o$ is locally connected. 
			Let $p: \widetilde \sX \to I^2_o$ be a covering. 
			For any $x \in I^2_o$ we select an open connected neighborhood $\sU_x$ evenly covered by $p$ (cf. Definition \ref{top_covering_defn}). 
			From the Lemma \ref{top_gen_path_lem} it follows that there is a $\left\{\sU_x\right\}_{x \in \sX}$-path $\left( \sU_1, ..., \sU_{n}\right)$ 
			(cf. Definition \ref{top_gen_path_defn})  from $\left(0,0\right)$ to $\left(1,1\right)$. 
			If there is $x \in I^2_o\setminus\cup_{j = 1}^n\sU_j$ then both sets $I^2_o\setminus \{x\}$ and $\cup_{j = 1}^n\sU_j$ are not connected (cf. \cite{counter_topology}). 
			However the set $\cup_{j = 1}^n\sU_j$ is connected, so one has $I^2_o=\cup_{j = 1}^n\sU_j$. Let $\widetilde x_0$ be such that $p\left(\widetilde x_0 \right) = \left(0,0\right)$. Let $\widetilde \sU_1$ be a connected component of $p^{-1}\left( \sU_1\right)$ such that $\widetilde x_0\in \widetilde \sU_1$. There is a homeomorphism $\phi_1 : \sU_1\xrightarrow{\cong}  \widetilde \sU_1$. If $x_1 \in \sU_1\cap \sU_2$ the there is a unique $\widetilde x_1 \in \widetilde \sU_1$ such that $p\left(\widetilde x_1 \right)= x_1$.  Let $\widetilde \sU_2$ be a connected component of $p^{-1}\left( \sU_2\right)$ such that $\widetilde x_1\in \widetilde \sU_2$. There is a homeomorphism $\sU_2\xrightarrow{\cong}  \widetilde \sU_2$ and $\widetilde \sU_1\cap \widetilde \sU_2\neq \emptyset$. Having homeomorphisms $\sU_1\xrightarrow{\cong}  \widetilde \sU_1$ and  $\sU_2\xrightarrow{\cong}  \widetilde \sU_2$ one has a homeomorphism $\phi_2:\sU_1\cup \sU_2 \xrightarrow{\cong}  \widetilde \sU_1\cup \widetilde \sU_2$. Continuing we obtain a continuous map 
			$$
			\phi^0_n: \bigcup^n_{j = 1} \sU_j = I^2_o\to \widetilde \sX
			$$
			such that $p \circ \phi^0_n = \Id_{I^2_o}$.
			If $\widetilde \sX_0 \bydef \phi^0_n\left(  I^2_o\right)$ then a restriction $p|_{\widetilde \sX_0} : \widetilde \sX_0 \to I^2_o$ is a homeomorphism. If $\widetilde x_1$ is such that $\widetilde x_1\neq \widetilde x_0$ and  $p\left(\widetilde x_1 \right)= (0,0)$ then one can find a subspace  $\widetilde \sX_1 \subset \widetilde \sX$ such that $\widetilde x_1\in \widetilde \sX_1$ and a restriction $p|_{\widetilde \sX_1} : \widetilde \sX_1 \to I^2_o$ is homeomorphism. For any $x \in I^2_o$ the number of points of $p^{-1}\left(x \right)\cap \left(\widetilde \sX_0\cup \widetilde \sX_1 \right)$ equals to one or two. The set $\left\{ x \in I^2_o~\left| ~ \left|p^{-1}\left(x \right)\cap \left(\widetilde \sX_0\cup \widetilde \sX_1 \right) \right|=2 \right.\right\}$ is open, so the the set $\left\{ x \in I^2_o~\left| ~ \left|p^{-1}\left(x \right)\cap \left(\widetilde \sX_0\cup \widetilde \sX_1 \right) \right|=1 \right.\right\}$ is closed. If 
			$$
			x' \bydef \inf \left\{ x \in I^2_o~\left| ~ \left|p^{-1}\left(x \right)\cap \left(\widetilde \sX_0\cup \widetilde \sX_1 \right) \right|=1 \right.\right\}
			$$ 
			then $\left|p^{-1}\left(x' \right)\cap \left(\widetilde \sX_0\cup \widetilde \sX_1 \right) \right|=1$ because the set\\ $\left\{ x \in I^2_o~\left| ~ \left|p^{-1}\left(x \right)\cap \left(\widetilde \sX_0\cup \widetilde \sX_1 \right) \right|=1 \right.\right\}$ is closed. Let $\widetilde x'\in \widetilde \sX_0\cup \widetilde \sX_1$ be the unique point  such that $p\left( \widetilde x' \right) = x'$, and let $\sU'$ be a open connected set  evenly covered by $p$, and let $\widetilde \sU'$ be a connected component of $p^{-1}\left(\sU'\right)$ such that $\widetilde x'\in \widetilde \sU'$. There is $x'' \in \sU'$ such that $x'' < x'$ and $$\left|p^{-1}\left(x'' \right)\cap \left(\widetilde \sX_0\cup \widetilde \sX_1 \right) \right|=2.$$ From the above equation it follows that $\sU'$ is not evenly covered by $p$. From this contradiction one has $\left|p^{-1}\left(x \right)\cap \left(\widetilde \sX_0\cup \widetilde \sX_1 \right) \right|=2$ for all $x\in I^2_0$. It turns out that  $\widetilde \sX_0\cap \widetilde \sX_1= \emptyset$.  If a union $\widetilde \sX_0\cup \widetilde \sX_1$ is not disjoint then $\widetilde \sX_0$ is open but not closed or $\widetilde \sX_1$ is open but not closed. If $\widetilde \sX_0$ is open but not closed and the  $\widetilde \sX_0\cup \widetilde \sX_1$ is not disjoint then there is a convergent  net $\left\{\widetilde x_\a\right\}\subset \widetilde \sX_0$ such that $\lim \widetilde x_\a\in \widetilde \sX_1$. However from $\widetilde x_\a = \phi^0_n \circ p\left(\widetilde x_\a \right)$ it turns out that $\lim \widetilde x_\a= \phi^0_n\left(  \lim p\left( \widetilde x_\a\right) \right) \in \widetilde \sX_0$. If follows that $\widetilde \sX_0\cup \widetilde \sX_1=\widetilde \sX_0\sqcup \widetilde \sX_1$. By the similar way one can proof that $p^{-1}\left( I^2_o\right)$ is a disjoint union of homeomorphic to $I^2_o$ sets, i.e. $I^2_o$ is evenly covered by $p$.
		\end{proof}
		
		\begin{exercise}\label{top_lex_square_exer}
			Let  $I^2_o$ be ordered square (cf. \ref{top_lex_square_empt}). Denote by $x_0 \bydef (0,0), x_1 \bydef (1,1)\in I^2_o$. Let $\sX \bydef I^2_o/\sim_\sX$ where $x_0 \sim_\sX x_1$.
			\begin{enumerate}
				\item Prove that $\sX$ is a connected, locally connected, compact, Hausdorff space, but $\sX$ is not path connected (cf. \cite{counter_topology}).
				\item Similarly to the proof of the Lemma \ref{top_lex_square_lem} prove that the space $\sX$ is weakly semilocally 1-connected (cf. Definition \ref{top_weakly_semi1_defn}).
			\end{enumerate}
		\end{exercise}
		\begin{exercise}\label{top_lex_square1_exer}
			Let us use the notation of the Exercise \ref{top_lex_square_exer}. Let $\sX_j$ be a homeomorphic to $I^2_o$ space for any $j\in \Z$. Let $x^j_0 , x^j_1 \in \sX_j$ be points which correspond to $x_0 \bydef (0,0), x_1 \bydef (1,1)\in I^2_o$. Let $\sim_{\widetilde \sX}$ be an equivalence relation on $\bigsqcup_{j\in \Z} \sX_j$ such that
			$$
			\forall j \in \Z \quad x^j_1 \sim_{\widetilde \sX}  x^{j+1}_0.
			$$
			Let $\widetilde{\sX}\bydef \bigsqcup_{j\in \Z} \sX_j/ \sim_{\widetilde \sX}$.
			\begin{enumerate}
				\item Prove that $\widetilde\sX$ is a connected, locally connected, locally compact, Hausdorff space, but $\widetilde\sX$ is not path connected (cf. \cite{counter_topology}).
				\item Similarly to the proof of the Lemma \ref{top_lex_square_lem} prove that the space $\widetilde\sX$ is weakly simply connected (cf. Definition \ref{top_weakly_simply_connected_defn}).
			\end{enumerate}
		\end{exercise}
		
		\begin{exercise}\label{top_lex_square2_exer}
			Let us use the notation of the Exercises \ref{top_lex_square_exer}, \ref{top_lex_square1_exer}.   for any $n \in \Z$ define a homeomorphism  $\varphi_n^\sqcup$ of $\sqcup\sX_j$ which homeomorphically  maps $\sX_j$ onto $\sX_{j + n}$. Prove following statements:
			\begin{enumerate}
				\item for any $n\in \Z$ the homeomorphism $\varphi_n^\sqcup$ induces a homeomorphism $\varphi_n$ of $\widetilde \sX$,
				\item a family of $\left\{\varphi_n\right\}$ induces an action $\Z\times \widetilde \sX \to \widetilde{\sX}$ such that $\Z$ is a properly discontinuous group of homeomorphisms (cf. Definition \ref{top_properly_disc_defn}),
				\item there is a natural isomorphism $\sX = \widetilde\sX /\Z$.
			\end{enumerate}
		\end{exercise}
		\begin{example}
			Let us use the notation of the Exercises \ref{top_lex_square_exer}, \ref{top_lex_square1_exer}. From the Theorem \ref{top_cov_fact_thm} it follows that there is a covering $p: \widetilde\sX\to \sX$. Since $\widetilde\sX$ is simply 1-connected the coveting is universal. If  $n \in \Z$ and $\widetilde x_n \bydef x^n_0 / \sim_{\widetilde \sX}\in \widetilde \sX$ then $p\left(\widetilde x_n \right) = x_0 \bydef p\left(\widetilde x_0 \right)$ for all $n \in \Z$. One can regard $p$ as a pointed covering $\left(\widetilde\sX, \widetilde x_0 \right)\to  \left(\sX, x_0 \right)$. If $\mathfrak p = \left( \sU_1, ..., \sU_k\right)$ is a semilocally 1-connected closed $x_0$-path on $\sX$ and  $\widetilde{\mathfrak p} = \left( \widetilde\sU_1, ..., \widetilde\sU_k\right)$ is its $p$-lift, then there is a unique $n \in \Z$ such that $\widetilde x_n \in \widetilde\sU_k$. One can proof that  the map
			$$
			\mathfrak	p\mapsto n
			$$
			yields a homomorphism $h: \pi^{\text{w}}_1\left(\sX \right)  \to \Z$. Since the covering $p$ is universal a path $\mathfrak p$ is contractible if and only  $\lift_p \mathfrak p$ is contractible, i.e. $h\left(e \right) = 0$ where $e$ is a unity of $\pi^{\text{w}}_1\left(\sX \right)$. So the map $h$ is injective. 	From the Lemma \ref{top_gen_path_lem} it follows that for all $k \in \Z$ there is a path
			$\widetilde {\mathfrak p} = \left(\widetilde \sU_1, ..., \widetilde\sU_k\right)$ such that
			\begin{itemize}
				\item $\widetilde x_0\in \widetilde \sU_1, \quad \widetilde x_n \subset \widetilde \sU_k$, 
				\item the set $\widetilde \sU_j$ is weakly 1-connected.
			\end{itemize}
			If $\mathfrak p = \desc_p \widetilde {\mathfrak p}$ is a $p$ descent  of $\widetilde {\mathfrak p}$ (cf. Definition \ref{top_path_desc_defn}) then  the path $\mathfrak p$ is closed and
			$h\left(\left[\mathfrak p\right]\right)= n$. So the homomorphism $h$ is surjective. In result one has an isomorphism $\pi^{\text{w}}_1\left(\sX \right)\cong\Z$. However the explained in the Remark \ref{top_homotopy_group_rem} fundamental group $\pi_1\left(\sX \right)$ is trivial since $\sX$ is not path connected.
		\end{example}
		
	\begin{appendices}
	\section{Topology}
	\begin{defn}\label{top_semi1_defn}\cite{spanier:at}
	A space $\sX$ is said to by \textit{semilocally} 1-\textit{connected} if every point $x_0$ has a neighborhood $N$ such that $\pi_1\left( N, x_0\right) \to \pi_1\left(\sX, x_0 \right)$ is trivial. 
\end{defn} 

	\begin{defn}\label{top_covering_transformations_group_defn}\cite{spanier:at}
		Let $p: \mathcal{\widetilde{X}} \to \mathcal{X}$ be a covering.  A self-equivalence is a homeomorphism $f:\mathcal{\widetilde{X}}\to\mathcal{\widetilde{X}}$ such that $p \circ f = p$. This group of such homeomorphisms is said to be the {\it group of covering transformations} of $p$ or the {\it covering group}. Denote by $G\left( \mathcal{\widetilde{X}}~|~\mathcal{X}\right)$ this group.
	\end{defn}
	
	\begin{definition}\label{top_properly_disc_defn}\cite{spanier:at}
		A group $G$ of homeomorphisms of a topological space $\sY$ is said to be  
		\textit{discontinuous} if the orbits of $G$ in $\sY$ are discrete subsets of $\sY$. $G$ is \textit{properly 
			discontinuous} if for $y \in \sY$ there is an open neighborhood $\sU$ of $y$ in $\sY$ such 
		that if $g, g' \in  G$ and $g\sU$ meets $g'\sU$, then $g = g'$. $G$ acts \textit{ without fixed points} 
		if the only element of $G$ having fixed points is the identity element. The 
		following are clear. 
		A properly discontinuous group of homeomorphisms is discontinuous 
		and acts without fixed points. 
	\end{definition}
	
	\begin{definition}\label{top_path_connected_defn}\cite{munkres:topology}
		%Definition. 
		Given points $x$ and $y$ of the space $\sX$, a \textit{path} in $\sX$ from $x$ to $y$ is a continuous map $f: \left[a, b\right]\to \sX$ of some closed interval in the real line into $\sX$, such
		that $f(a) = x$ and $f(b) = y$. A space $\sX$ is said to be \textit{path connected} if every pair of
		points of $\sX$ can be joined by a path in $\sX$.
	\end{definition}
	
		\begin{defn}\label{top_covering_defn}\cite{spanier:at}
		Let $\widetilde{\pi}: \widetilde{\mathcal{X}} \to \mathcal{X}$ be a continuous map. An open subset $\mathcal{U} \subset \mathcal{X}$ is said to be {\it evenly covered } by $\widetilde{\pi}$ if $\widetilde{\pi}^{-1}(\mathcal U)$ is the disjoint union of open subsets of $\widetilde{\mathcal{X}}$ each of which is mapped homeomorphically onto $\mathcal{U}$ by $\widetilde{\pi}$. A continuous map $\widetilde{\pi}: \widetilde{\mathcal{X}} \to \mathcal{X}$ is called a {\it covering projection} if each point $x \in \mathcal{X}$ has an open neighborhood evenly covered by $\widetilde{\pi}$. $\widetilde{\mathcal{X}}$ is called the {
			\it covering space} and $\mathcal{X}$ the {\it base space} of the covering.
	\end{defn}
	
\begin{theorem}\label{top_fundamental_group_mor_thm}\cite{spanier:at}
%2 theoeem 
Let $p: \widetilde{\sX}\to \sX$ a fibration with unique path lifting. Let $\sX$ be path connected and let $x_0\in \sX$. Then  there is  a monomorphism $\psi$ of $G\left( \left.\widetilde{\sX} \right|\sX\right) $ to 
the quotient group $N\left(\pi_1\left( p\right) \left( \pi_1\left(\widetilde\sX,\widetilde x_0 \right)\right)\right)   /\pi_1\left(\widetilde\sX,\widetilde x_0 \right)$ where $N\left(\pi_1\left( p\right) \left( \pi_1\left(\widetilde\sX,\widetilde x_0 \right)\right)\right)$ is a maximal among subgroups $G\subset \pi_1\left(\sX, x_0 \right)$ such that $\pi_1\left(\widetilde\sX,\widetilde x_0 \right)$ is a normal subgroup of $G$.	If $\sX$ is also locally path connected, 
$\psi$ is an isomorphism. 
\end{theorem}

	
	\end{appendices}	
		
		
		
		\begin{thebibliography}{10}
			
			
\bibitem{munkres:topology} James R. Munkres. {\it Topology.} Prentice Hall, Incorporated, 2000.

\bibitem{spanier:at}
E.H. Spanier. {\it Algebraic Topology.} McGraw-Hill. New York. 1966.

\bibitem{switzer:at} Switzer R M, {\it Algebraic Topology - Homotopy and Homology}, Springer. 2002.

\bibitem{counter_topology}	Lynn Arthur Steen,
J. Arthur Seebach. \textit{Counterexamples in topology}, Springer-Verlag,
1970.
			
			
			
		\end{thebibliography}
		
		
		
		
	\end{document}
	
	
