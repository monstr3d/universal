\documentclass{beamer}
\usepackage{amsmath,amssymb,amsthm,slashed, euscript}
\usepackage{graphicx}
\usepackage{mathrsfs}  

\textwidth=110mm


\title{Algebraic topology of $C^*$-algebras}
\institute
{
Noncommutative geometry and topology
}

\author{Petr R. Ivankov  }



\theoremstyle{plain}
\newtheorem{defn}{Definition}
\newtheorem{rem}{Remark}
\newtheorem{exm}{Example}
\newtheorem*{claim}{Claim}
\newtheorem{prop}{Proposition}
\newtheorem{empt}[prop]{}%[section]
\newtheorem{lem}{Lemma}%[section]
\newtheorem{thm}{Theorem}%[section]



\newcommand{\A}{\mathcal{A}}
\newcommand{\be}{\begin{equation}}
\newcommand{\ee}{\end{equation}}
	\newcommand{\bean}{\begin{eqnarray*}}
	\newcommand{\eean}{\end{eqnarray*}}
\newcommand{\Ga}{\Gamma}
\newcommand{\B}{\mathcal{B}}
\newcommand{\Cc}{\mathcal{C}}
\newcommand{\C}{\mathbb{C}}
\newcommand{\D}{\mathcal{D}}
\newcommand{\G}{\mathcal{G}}
\newcommand{\Hc}{\mathcal{H}}
\newcommand{\Lc}{\mathcal{L}}
\newcommand{\Pc}{\mathcal{P}}
\newcommand{\Sc}{\mathcal{S}}
\newcommand{\U}{\mathcal{U}}
\newcommand{\rar}{\rightarrow}
\newcommand{\Ef}{\mathbb{E}}
\newcommand{\desc}{\mathfrak{desc}}
\newcommand{\rep}{\mathfrak{rep}}


%Uppercase Gothic characters
\newcommand{\gtA}{\mathfrak{A}}
\newcommand{\gtB}{\mathfrak{B}}
\newcommand{\gtM}{\mathfrak{M}}
\newcommand{\gtN}{\mathfrak{N}}
\newcommand{\gtP}{\mathfrak{P}}
\newcommand{\gtS}{\mathfrak{S}}

%Lowercase Gothic characters
\newcommand{\gtf}{\mathfrak{f}}
\newcommand{\gtg}{\mathfrak{g}}

%Bold Characters
\newcommand{\Cb}{\mathbb{C}}
\newcommand{\Nb}{\mathbb{N}}
\newcommand{\Rb}{\mathbb{R}}
\newcommand{\Zb}{\mathbb{Z}}

%Uppercase Greek characters
\newcommand{\Gm}{\Gamma}
\newcommand{\Te}{\Theta}
\newcommand{\Om}{\Omega}
\newcommand{\s}{ }

%Lowercase Greek characters
\newcommand{\al}{\alpha}
\newcommand{\gm}{\gamma}
\newcommand{\dl}{\delta}
\newcommand{\sg}{\sigma}
\newcommand{\ph}{\varphi}
\newcommand{\te}{\theta}
\newcommand{\ze}{\zeta}
\newcommand{\lift}{\mathfrak{lift}}

\newcommand{\Id}{\mathrm{Id}}
\newcommand{\Aut}{\mathrm{Aut}}
\newcommand{\Coo}{{\mathrm{C}}^\infty}
\newcommand{\alg}{\mathrm{alg}}
\newcommand{\diag}{\mathrm{diag}}
\newcommand{\spinc}{\textbf{$spin^c$}}
\newcommand{\Hom}{\mathrm{Hom}}
\newcommand{\supp}{\mathrm{supp}}
\newcommand{\Ccl}{\mathbf{C}l}
\newcommand{\xto}{\xrightarrow}

\newcommand{\lto}{\longrightarrow}
\newcommand{\ox}{\otimes}
\newcommand{\nb}{\nabla}
\newcommand{\sS}{\mathcal{S}}
\newcommand{\Dn}{D\!\!\!\!/}
%\newcommand{\ij}{{i,j}}
\newcommand{\aC}{\ensuremath{\underline{\Cb}} }
\newcommand{\scp}[2]{\left\langle{#1},{#2}\right\rangle}
\newcommand{\op}[1]{J{#1}J^\dag}
\newcommand{\K}{\mathcal{K}} 
\newcommand{\F}{\mathcal{F}} 
\newcommand{\E}{\mathcal{E}} 
\newcommand{\sA}{\mathcal{A}} 
\newcommand{\sB}{\mathcal{B}}       %%
\newcommand{\sC}{\mathcal{C}}       %%
\newcommand{\sD}{\mathcal{D}}       %%
\newcommand{\sE}{\mathcal{E}}       %%
\newcommand{\sF}{\mathcal{F}}       %%
\newcommand{\sG}{\mathcal{G}}       %%
\newcommand{\sH}{\mathcal{H}}       %%
\newcommand{\sI}{\mathcal{I}}       %%
\newcommand{\sJ}{\mathcal{J}}       %%
\newcommand{\sK}{\mathcal{K}}       %%
\newcommand{\sL}{\mathcal{L}}       %%
\newcommand{\sM}{\mathcal{M}}       %%
\newcommand{\sN}{\mathcal{N}}       %%
\newcommand{\sO}{\mathcal{O}}       %%
\newcommand{\sP}{\mathcal{P}}       %%
\newcommand{\sQ}{\mathcal{Q}}       %%
\newcommand{\sR}{\mathcal{R}}       %%
\newcommand{\sT}{\mathcal{T}}       %%
\newcommand{\sU}{\mathcal{U}}       %%
\newcommand{\sV}{\mathcal{V}}       %%
\newcommand{\sX}{\mathcal{X}}       %%
\newcommand{\sY}{\mathcal{Y}}       %%
\newcommand{\sZ}{\mathcal{Z}}       %%
\newcommand{\N}{\mathbb{N}}                  %% 

\renewcommand{\a}{\alpha}     
\newcommand{\la}{\lambda}     
\newcommand{\La}{\Lambda}
\newcommand{\bt}{\beta}           %% short for  \beta
 
    
\newcommand{\bydef}{\stackrel{\mathrm{def}}{=}}  
\newcommand{\hookto}{\hookrightarrow}        %% abbreviation
  
\begin{document}

\begin{frame}
  \titlepage
\end{frame}

\begin{frame}
\begin{theorem}\alert{Pavlov, Troisky}
	Suppose $\mathcal X$ and $\mathcal Y$ are compact Hausdorff connected spaces and $p :\mathcal  Y \to \mathcal X$
	is a continuous surjection. If $C(\mathcal Y )$ is a projective finitely generated Hilbert module over
	$C(\mathcal X)$ with respect to the action
	\begin{equation*}
		(f\xi)(y) = f(y)\xi(p(y)),\quad  f \in  C(\mathcal Y ), \quad  \xi \in  C(\mathcal X),
	\end{equation*}
	then $p$ is a finite-fold  covering.
\end{theorem}
	There is a counterexample  to the Theorem. Alexandru Chirvasitu. {\it Non-commutative branched covers and bundle unitarizability}, arXiv:2409.03531v1, 2024.
	

\end{frame}
\begin{frame}
Development of the (co)homology theory of $C^*$-algebras such that: 
\begin{itemize}
	\item for any commutative $C^*$-algebra $C\left(\sX \right)$ it coincides with  (co)homology theory of $\sX$.
	\item the theory is not trivial even for algebras having bag spectrum, e.g. containing two open sets only.
\end{itemize}
\end{frame}
\section{Gelfand space}
\begin{frame}

\begin{definition}\label{gelfand_space_defn}\alert{Ivankov}.
	If $A$ is $C^*$-algebra then  the   \alert{Gelfand space} of $A$ is a set $\mathfrak{Gelfand}\left(A \right)$ of maximal left ideals of $A$ having the minimal topology  such that for any left ideal $I\subset A$ the set
	\be\label{gelfand_space_egn}
	\left\{\left.x \in \mathfrak{Gelfand}\left(A \right)\right| I \subsetneqq x\right\}	
	\ee
	is open.
\end{definition}


\begin{thm}\label{gelfand-naimark_thm} (Commutative Gelfand-Na\u{\i}mark theorem). 
	Let $A$ be a commutative $C^*$-algebra and let $\mathcal{X}$ be the spectrum of A. There is the natural $*$-isomorphism $\gamma:A \xrightarrow{\cong} C_0(\mathcal{X})$.
\end{thm}


\begin{lemma}\alert{Ivankov}. (Generalized commutative Gelfand theorem) 
	If $\sX$ is a compact Hausdorff space then  is a natural  homeomorphism $\mathfrak{Gelfand}\left(C\left( \sX\right)  \right)\cong \sX$.
\end{lemma}
\end{frame}
\begin{frame}

\section{Morphisms}
	\begin{definition}\label{lrc_defn}
	If $A$ is a $C^*$-algebra then a linear map $\la: A\to A$ is said to be a \alert{left centralizer} if
	\be
	\la\left(ab\right)= 	\la\left(a\right) b \quad \forall a, b \in A.
	\ee
	Similarly one defines a \alert{right} centralizer. Denote the spaces of left and right centralizers by $\mathbf{LC}(A)$ and  $\mathbf{RC}(A)$.
\end{definition}


\begin{lemma}\alert{Ivankov}
	Any injective homomorphism 	
	\bean
	\varphi_R: A \hookto \mathbf{RC}\left(\widetilde A\right)	\eean
	of $\C$-algebras	naturally yields a continuous mapping 
	\be\label{gelfand_map_eqn}
	\mathfrak{Gelfand}\left(\widetilde A \right)\xrightarrow{\mathfrak{Gelfand}\left(\varphi_R \right)}\mathfrak{Gelfand}\left(A \right)
	\ee
	\end{lemma}
\end{frame}

\section{Continuous trace $C^*$-algebras}
\begin{frame}
\begin{lemma}\label{ctr_gelfand_lem}\alert{Ivankov}
	If $A$ is a continuous trace $C^*$-algebra then any $x \in \mathfrak{Gelfand}\left(A \right) $ is given by a pair $\left(x, \xi \right)$ where
	\begin{enumerate}
		\item[(i)] $x$ is a point of the spectrum $\sX$ of $A$ which corresponds to the representation $\rep_x: A \to B\left( \sH_x\right) $,
		\item[(ii)] $\xi \in \C P\left( \sH_x\right)$ where $ \C P\left( \sH_x\right)$ is a complex projective space.
	\end{enumerate} 
	
\end{lemma}

\end{frame}
\begin{frame}
Suppose that $A$ is given by a locally trivial  fibre bundle $\F$ with fibre $\K = \K\left(\sH \right)$. There is a subbundle $\E \subset \F$ such that fibers of $\E$ are  rank-one positive operators. It givers a bundle $\C P\left(\E \right)$ with fibre $\C P\left( \sH\right)$. From the Lemma \ref{ctr_gelfand_lem}  it turns out that there is the natural set theoretic bijective map $\mathfrak{Gelfand}\left(A \right)\cong  \C P\left(\E \right)$.
If $\C P\left(\E \right)_{\mathfrak{Gelfand}}$ is $\C P\left(\E \right)$ supplied with Gelfand topology then there is a bijective continuous map
\be\label{pe_c_eqn}
\phi_{\E} : \C P\left(\E \right)_{\mathfrak{Classic}}\to \C P\left(\E \right)_{\mathfrak{Gelfand}}
\ee
where $\C P\left(\E \right)_{\mathfrak{Classic}}$ is supplied with the classic topology.
In particular if $\sX = \{x\}$ then $A = \K\left(\sH \right)$ and $\mathfrak{Gelfand}\left(\K\left(\sH \right) \right)\cong  \C P\left(\sH \right)$. The subbase of $\mathfrak{Gelfand}\left(\K\left(\sH \right)\right) $ contains all sets $\C P\left(\sH \right) \setminus L$ where $L$ is a linear subspace of  $\C P\left(\sH \right)$. Similarly to \eqref{pe_c_eqn} there is a continuous map 
$$
\phi_{\sH} : \C P\left(\sH \right)_{\mathfrak{Classic}}\to \C P\left(\sH \right)_{\mathfrak{Gelfand}}
$$
Moreover if $\dim \sH = n$ then there is a composition
$$
\C P^n_{\mathfrak{Classic}}\to \C P^n_{\mathfrak{Zariski}}\to \C P^n_{\mathfrak{Gelfand}}
$$
where the subscript $_{\mathfrak{Zariski}}$ means the Zariski topology.
\end{frame}
\begin{frame}
If  $\dim \sH = \infty$ and $\left\{L_0, L_1, ...\right\}$ is a set of mutually orthogonal codimension one spaces then
$$
\C P\left( \sH\right) = \bigcup_{j=0}^\infty \left( \C P\left(\sH \right) \setminus L_j\right) 
$$
Similarly 
$$
\C P^n = \bigcup_{j=0}^{n} \left( \C P^n \setminus L_j\right) 
$$
	\begin{definition}\label{nerve_defn}
	Given a set $X$ and a collection $\mathscr W = \left\{W\right\}$ of subsets of $X$, the \textit{nerve} of $\mathscr W$ denoted by $K\left( \mathscr W\right)$, is  the simplicial complex whose simplexes are finite nonempty subsets of  $\mathscr W$ with nonempty intersections. Thus the vertices of $K\left( \mathscr W\right)$ are nonempty elements of $\mathscr W$.
\end{definition}
\begin{theorem}\label{top_nerve_thm}
	%PAGE 193 !!! DOWN !!!
	%4.13. Theorem. 
	Let $\mathscr A$ be a sheaf of Abelian groups  on $\sX$ and let $\mathscr U = \left\{\sU_\a\right\}_{\a \in I}$ an open covering  of $\sX$ having the property that $H^p\left( \sU_{\sigma};\mathscr A \right)= 0$  for $p > 0$ and
	all $\sigma \in K\left(\mathscr U \right)$ in the nerve of covering. Then there is a canonical isomorphism
	$
	H^*\left(\sX, \mathscr A \right) \cong \check H^*\left( \mathscr U; \mathscr A\right). 
	$
\end{theorem}
\end{frame}
\begin{frame}
If $F$ is an Abelian group and $\mathscr F$ is a corresponding constant sheaf on $\C P\left(\sH \right)_{\mathfrak{Classic}}$  and $\mathscr W= \left\{ \C P\left(\sH \right) \setminus L_j\right\}_{j = 0, 1, ... }$ or  $\mathscr W= \left\{ \C P^n\setminus L_j\right\}_{j = 0,..., n }$ then 
$H^p\left( \sU_{\sigma};\mathscr A \right)= 0$  for $p > 0$ and
all $\sigma \in K\left(\mathscr W\right)$ where n $K\left(\mathscr W\right)$ is the nerve of $\mathscr W$. From the above Theorem it turns out that
$$
H^*\left(\C P^n_{\mathfrak{Classic}}, \mathscr A \right) \cong \check H^*\left( \mathscr W; \mathscr A\right)
$$
Taking into account that $\C P^n \setminus L_j$ is open in $\C P\left(\sH \right)_{\mathfrak{Gelfand}}$ for any $j$ one has
$$
H^*\left(\C P\left(\sH \right) _{\mathfrak{Classic}}, \mathscr F_{\mathfrak{Classic}} \right) \cong H^*\left(\C P\left(\sH \right) _{\mathfrak{Gelfand}}, \mathscr F_{\mathfrak{Gelfand}} \right)
$$
The spectrum of $\K\left(\sH \right)$ has the single point but cohomology of $\mathfrak{Gelfand}\left(\K\left(\sH \right)  \right)$ are not trivial. 
\end{frame}
\begin{frame}
The topology of $\C P\left(\E \right)_{\mathfrak{Gelfand}}$ is the hybrid of Hausdorff topology of $\sX$ and non Hausdorff topology of $\C P\left(\sH \right) _{\mathfrak{Gelfand}}$ 
\begin{lemma}\alert{Ivankov}
	Let $F$ is an Abelian group.
	Let $\pi :  \C P\left(\E \right)_{\mathfrak{Classic}}\to \sX$ be the natural surjective mapping 
	If $\sX = \bigcup_{\mathcal W \in \mathscr W}  \mathcal W$ where: 
	\begin{itemize}
		\item for any $\mathcal W \in \mathscr W$ one has $\pi^{-1}\left(\mathcal W \right) \cong \mathcal W\times \C P\left( \sH\right)_{\mathfrak{Classic}}$,
		\item $H^*\left( \sU,\mathscr F_{\mathfrak{Classic}} \right)= {0}$  for all $\sU \in K\left(\mathscr W \right)$ 
	\end{itemize}	
	then there is the natural isomorphism
	$$
	H^*\left(\mathfrak{Gelfand}\left(A \right),\mathscr F_{\mathfrak{Gelfand}}  \right) \cong H^*\left(\C P\left(\E \right) _{\mathfrak{Classic}}, \mathscr F_{\mathfrak{Classic}} \right) 
	$$ 
	where both $\mathscr F_{\mathfrak{Classic}}$ and $\mathscr F_{\mathfrak{Gelfang}}$ are corresponding to $F$ locally constant sheaves.
\end{lemma}

\end{frame}
\section{Coverings}
\begin{frame}
	\begin{definition}\label{top_covering_defn} 
	Let $\widetilde{\pi}: \widetilde{\mathcal{X}} \to \mathcal{X}$ be a continuous map. An open subset $\mathcal{U} \subset \mathcal{X}$ is said to be \alert{ evenly covered } by $\widetilde{\pi}$ if $\widetilde{\pi}^{-1}(\mathcal U)$ is the disjoint union of open subsets of $\widetilde{\mathcal{X}}$ each of which is mapped homeomorphically onto $\mathcal{U}$ by $\widetilde{\pi}$. A continuous map $\widetilde{\pi}: \widetilde{\mathcal{X}} \to \mathcal{X}$ is called a \alert{covering} if each point $x \in \mathcal{X}$ has an open neighbourhood evenly covered by $\widetilde{\pi}$. $\widetilde{\mathcal{X}}$ is called the {
		\it covering space} and $\mathcal{X}$ the {\it base space} of the covering.
\end{definition}
\end{frame}

\begin{frame}
	Let $A$ be a $C^*$-algebra. The group $\mathrm{Aut}\left(\widetilde{A}\right)$ of *-automorphisms carries (at least) two different topologies making it into a topological group  The most important is \alert{the topology of pointwise norm-convergence} based on the open sets
\begin{equation*}
	\left\{\left.\alpha \in \mathrm{Aut}(A) \ \right| \ \|\alpha(a)-a\| < 1 \right\}, \quad a \in A.
\end{equation*}
The other topology is the \alert{uniform norm-topology} based on the open sets
\begin{equation}\label{aut_norm_eqn}
	\left\{\alpha \in \mathrm{Aut}(A) \ \left| \ \sup_{a \neq 0}\ \|a\|^{-1} \|\alpha(a)-a\| < \varepsilon \right. \right\}, \quad \varepsilon > 0
\end{equation}
which corresponds to following "norm"
\begin{equation}\label{uniform_norm_topology_formula_eqn}
	\|\alpha\|_{\text{Aut}} = \sup_{a \neq 0}\ \|a\|^{-1} \|\alpha(a)-a\| = \sup_{\|a\| =1}\  \|\alpha(a)-a\|.
\end{equation}
Above formula does not really means a norm because $\mathrm{Aut}\left(A\right)$ is not a vector space. Henceforth the uniform norm-topology will be considered only.
\end{frame}
\begin{frame}

  \begin{definition}\label{connected_c_a_defn}
	We say that a $C^*$-algebra $A$ is \alert{connected} if it cannot be represented as a direct sum  $A \cong A' \oplus A''$ of nontrivial $C^*$-algebras $A'$ and $A''$.
	
	% (the Gelfand spectrum of the center of $M\left( A\right) $ is connected). Let $A \subset B$ be a connected subalgebra. We say that $A$ is a \textit{connected component} of $B$ if  $1_{M\left( A\right) }$ lies in the center of $1_{M\left( B\right) }$.
\end{definition}
\end{frame}
\begin{frame}

\begin{definition}\label{fin_pre_defn}\alert{Ivankov}
	Let   $A$ be an  connected $C^*$-algebra  and let  $\widetilde{A}$ be  connected $C^*$-algebra , and let $\lift: A \hookto M\left( \widetilde{A}\right) $ be an injective  $*$-homomorphism of % connected
	$C^*$-algebras such that following conditions hold:
	\begin{enumerate}
		\item[(a)] if $\Aut\left(\widetilde{A} \right)$ is a group of $*$-automorphisms of $\widetilde{A}$ then the group  
		\be\nonumber
		G \bydef \left\{ \left.g \in \Aut\left(\widetilde{A} \right)~\right|\forall a \in \lift \left( A\right) \quad ga = a\right\}
		\ee
		is discrete
		\item[(b)] 	$A = \widetilde{A}^G\stackrel{\text{def}}{=}\left\{\left.a\in \widetilde{A}~~\right|\forall g \in G\quad  a = g a\right\}$.
	\end{enumerate}
	We say that the triple $\left(A, \widetilde{A}, G \right)$ and/or the quadruple $\left(A, \widetilde{A}, G, \lift \right)$ and/or $*$-homomorphism $\lift: A \hookto \widetilde{A}$   is a \alert{noncommutative finite-fold  pre-covering}. We write $G\left(\left.\widetilde A~\right|A \right)\bydef G$.
\end{definition}
\end{frame}
\begin{frame}

\begin{definition}\label{evenly_defn}\alert{Ivankov}
	
	Let $\left(A, \widetilde{A}, G, \lift \right)$ be a noncommutative  pre-covering.	A connected hereditary $C^*$-subalgebra $B \subset A$ 
	is $\left(A, \widetilde{A}, G, \lift \right)$- \alert{evenly covered by} $\left(A, \widetilde{A}, G, \lift \right)$ if there is a hereditary $C^*$-subalgebra $\widetilde B \subset \widetilde A$ with a $*$-isomorphism $\lift^{\widetilde B}: B \cong \widetilde B$ such that
	\be\label{evenly_eqn}
	\forall b \in B \quad \lift\left( b\right) = \bt \text{-}\sum_{g\in G} g \lift^{\widetilde B}\left( b\right) 
	\ee
	where $\bt \text{-}\sum$ means the convergence with respect to the strict topology of $M\left( \widetilde A\right)$
\end{definition}
\end{frame}
\begin{frame}
\begin{definition}\label{cov_unital_defn}\alert{Ivankov}
	
	A  noncommutative  pre-covering $\left(A, \widetilde{A}, G, \lift \right)$  with unital $A$ is a \alert{unital noncommutative covering} if for any $x \in \mathfrak{Gelfand}\left(A \right)$ there is a hereditary connected $C^*$-subalgebra of $B$ {evenly covered by} $\left(A, \widetilde{A}, G, \lift \right)$ with $B \in x$
\end{definition}




\begin{definition}\label{cov__defn}\alert{Ivankov}
	
	A  noncommutative finite-fold pre-covering $\left(A, \widetilde{A}, G, \lift \right)$  \alert{noncommutative  covering} if there is 
	if there is an unital noncommutative covering $\left(B, \widetilde{B}, G, \widetilde\lift \right)$  with inclusions $A \subset B$ and $\widetilde A \subset \widetilde B$ such that:
	\begin{enumerate}
		\item [(a)] both $A$ and $B$ are essential ideals of $B$ and $\widetilde B$,
		\item[(b)] $\lift \bydef \left.\widetilde\lift\right|_A$,
		\item[(c)] the action $G \times \widetilde B \to \widetilde B$ naturally comes from the $G \times \widetilde A \to \widetilde A$
	\end{enumerate}	
	\end{definition}
\end{frame}
\begin{frame}
	\alert{Thank you}
\end{frame}
\end{document}























