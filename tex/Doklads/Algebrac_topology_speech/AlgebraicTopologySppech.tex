\documentclass{beamer}
\usepackage{amsmath,amssymb,amsthm,slashed, euscript}
\usepackage{graphicx}


\textwidth=110mm


\title{Algebraic topology of $C^*$-algebras}
\institute
{
Algebras in analysis
}

\author{Petr R. Ivankov  }



\theoremstyle{plain}
\newtheorem{defn}{Definition}
\newtheorem{rem}{Remark}
\newtheorem{exm}{Example}
\newtheorem*{claim}{Claim}
\newtheorem{prop}{Proposition}
\newtheorem{empt}[prop]{}%[section]
\newtheorem{lem}{Lemma}%[section]
\newtheorem{thm}{Theorem}%[section]



\newcommand{\A}{\mathcal{A}}
\newcommand{\be}{\begin{equation}}
\newcommand{\ee}{\end{equation}}
	\newcommand{\bean}{\begin{eqnarray*}}
	\newcommand{\eean}{\end{eqnarray*}}
\newcommand{\Ga}{\Gamma}
\newcommand{\B}{\mathcal{B}}
\newcommand{\Cc}{\mathcal{C}}
\newcommand{\C}{\mathbb{C}}
\newcommand{\D}{\mathcal{D}}
\newcommand{\G}{\mathcal{G}}
\newcommand{\Hc}{\mathcal{H}}
\newcommand{\Lc}{\mathcal{L}}
\newcommand{\Pc}{\mathcal{P}}
\newcommand{\Sc}{\mathcal{S}}
\newcommand{\U}{\mathcal{U}}
\newcommand{\rar}{\rightarrow}
\newcommand{\Ef}{\mathbb{E}}
\newcommand{\desc}{\mathfrak{desc}}


%Uppercase Gothic characters
\newcommand{\gtA}{\mathfrak{A}}
\newcommand{\gtB}{\mathfrak{B}}
\newcommand{\gtM}{\mathfrak{M}}
\newcommand{\gtN}{\mathfrak{N}}
\newcommand{\gtP}{\mathfrak{P}}
\newcommand{\gtS}{\mathfrak{S}}

%Lowercase Gothic characters
\newcommand{\gtf}{\mathfrak{f}}
\newcommand{\gtg}{\mathfrak{g}}

%Bold Characters
\newcommand{\Cb}{\mathbb{C}}
\newcommand{\Nb}{\mathbb{N}}
\newcommand{\Rb}{\mathbb{R}}
\newcommand{\Zb}{\mathbb{Z}}

%Uppercase Greek characters
\newcommand{\Gm}{\Gamma}
\newcommand{\Te}{\Theta}
\newcommand{\Om}{\Omega}
\newcommand{\s}{ }

%Lowercase Greek characters
\newcommand{\al}{\alpha}
\newcommand{\gm}{\gamma}
\newcommand{\dl}{\delta}
\newcommand{\sg}{\sigma}
\newcommand{\ph}{\varphi}
\newcommand{\te}{\theta}
\newcommand{\ze}{\zeta}
\newcommand{\lift}{\mathfrak{lift}}

\newcommand{\Id}{\mathrm{Id}}
\newcommand{\Aut}{\mathrm{Aut}}
\newcommand{\Coo}{{\mathrm{C}}^\infty}
\newcommand{\alg}{\mathrm{alg}}
\newcommand{\diag}{\mathrm{diag}}
\newcommand{\spinc}{\textbf{$spin^c$}}
\newcommand{\Hom}{\mathrm{Hom}}
\newcommand{\supp}{\mathrm{supp}}
\newcommand{\Ccl}{\mathbf{C}l}
\newcommand{\xto}{\xrightarrow}

\newcommand{\lto}{\longrightarrow}
\newcommand{\ox}{\otimes}
\newcommand{\nb}{\nabla}
\newcommand{\sS}{\mathcal{S}}
\newcommand{\Dn}{D\!\!\!\!/}
%\newcommand{\ij}{{i,j}}
\newcommand{\aC}{\ensuremath{\underline{\Cb}} }
\newcommand{\scp}[2]{\left\langle{#1},{#2}\right\rangle}
\newcommand{\op}[1]{J{#1}J^\dag}
\newcommand{\sA}{\mathcal{A}} 
\newcommand{\sB}{\mathcal{B}}       %%
\newcommand{\sC}{\mathcal{C}}       %%
\newcommand{\sD}{\mathcal{D}}       %%
\newcommand{\sE}{\mathcal{E}}       %%
\newcommand{\sF}{\mathcal{F}}       %%
\newcommand{\sG}{\mathcal{G}}       %%
\newcommand{\sH}{\mathcal{H}}       %%
\newcommand{\sI}{\mathcal{I}}       %%
\newcommand{\sJ}{\mathcal{J}}       %%
\newcommand{\sK}{\mathcal{K}}       %%
\newcommand{\sL}{\mathcal{L}}       %%
\newcommand{\sM}{\mathcal{M}}       %%
\newcommand{\sN}{\mathcal{N}}       %%
\newcommand{\sO}{\mathcal{O}}       %%
\newcommand{\sP}{\mathcal{P}}       %%
\newcommand{\sQ}{\mathcal{Q}}       %%
\newcommand{\sR}{\mathcal{R}}       %%
\newcommand{\sT}{\mathcal{T}}       %%
\newcommand{\sU}{\mathcal{U}}       %%
\newcommand{\sV}{\mathcal{V}}       %%
\newcommand{\sX}{\mathcal{X}}       %%
\newcommand{\sY}{\mathcal{Y}}       %%
\newcommand{\sZ}{\mathcal{Z}}       %%
\newcommand{\N}{\mathbb{N}}                  %% 

\renewcommand{\a}{\alpha}     
\newcommand{\la}{\lambda}     
\newcommand{\La}{\Lambda}
\newcommand{\bt}{\beta}           %% short for  \beta
 
    
\newcommand{\bydef}{\stackrel{\mathrm{def}}{=}}  
\newcommand{\hookto}{\hookrightarrow}        %% abbreviation
  
\begin{document}
%\titlepage
\begin{frame}
  \titlepage
\end{frame}
\begin{frame}
		\begin{definition}\label{lattice_defn}\cite{johnstone:stone_spaces} 
	
	A \alert{join-semilattice} is a poset which supports for any finite set both  least upper bounds. Similarly we define 	\alert{meet-semilattice},	 A \alert{lattice} is a poset which supports for any finite set both  least upper bounds and greatest lower  bounds. 
\end{definition}
\begin{definition}\label{ideal_defn}\cite{johnstone:stone_spaces} 
	A subset $I$ of a join-semilattice $A$ is said to be an {\it ideal} if
	\begin{enumerate}
		\item [(a)] $I$ is a sub-join-semilattice of $A$; i.e. $0\in A$, and $a, b \in I$ imply  	$a \vee b \in I$; and
		\item [(b)] $I$ is a lower set; i.e. $a \in I$ and $b \le a$ imply $b \in I$.  
	\end{enumerate}
\end{definition}
\begin{definition}\label{filter_defn}\cite{johnstone:stone_spaces} 
	A subset $\mathfrak F$ of a meet-semilattice $A$ is said to be an {\it filter} if
	\begin{enumerate}
		\item [(a)] $\mathfrak F$ is a sub-meet-semilattice of $A$; i.e. $1\in A$, and $a, b \in \mathfrak F$  imply  	$a \wedge b \in \mathfrak F$; and
		\item [(b)] $\mathfrak F$ is a lower set; i.e. $a \in \mathfrak F$ and $ a\le b$ imply $b \in \mathfrak F$.  
	\end{enumerate}
\end{definition}
\end{frame}
\begin{frame}
		\begin{definition}\label{ultra_filter_defn}\cite{johnstone:stone_spaces} 
	A maximal filter is an \alert{ultrafilter}.
\end{definition}

\begin{lemma}\label{top_ultra_thm}\cite{johnstone:stone_spaces}
	One has:
	\begin{enumerate}
		\item [(a)] 	A topological space $\sX$ is Hausdorff if and only if every ultrafilter on $\sX$ has at most one limit.
		\item[(b)]   	A topological space $\sX$ is compact if and only if every ultrafilter has at least one limit.
	\end{enumerate} 
	
\end{lemma}
\begin{empt}
	From the Zorn's lemma (cf. Theorem \ref{zorn_thm}) it follows that any filter is a subset of an ultrafilter.
	\end{empt}

\end{frame}
\begin{frame}
	
	If $A$ is $C^*$-algebra then from the Lemma \ref{hered_ideal_lem} it follows that the a meet-semilattices (cf. Definition \ref{lattice_defn}) or closed left, right ideals and hereditary $C^*$-subalgebras are isomorphic.
If  $\mathfrak{Gelfand}\left(A \right)$  is a set of ultrafilters of these the meet-semilattices then denote by
\be\label{gelfand_eqn}
\begin{split}
	\mathfrak{Gelfand}\left(A \right)_{I} \bydef \left\{\left. x \in \mathfrak{Gelfand}\left(A \right)\right| I \in x\right\},\\
	\mathfrak{Gelfand}\left(A \right)_{B} \bydef \left\{\left. x \in \mathfrak{Gelfand}\left(A \right)\right| B \in x\right\}
\end{split}
\ee
where $I$ is a one-sided ideal, $B$ is a hereditary $C^*$-algebra.	
\begin{lemma}
	There is the natural topology on  $\mathfrak{Gelfand}\left(A \right)$ generated by the sets \eqref{gelfand_eqn}
\end{lemma}
	\begin{definition}\label{gelfand_space_defn}\alert{Ivankov}
	Under the above hypothesis $\mathfrak{Gelfand}\left(A \right)$ is the \alert{Gelfand's space} of $A$.
\end{definition}

\end{frame}
\begin{frame}
\begin{theorem}\label{gelfand-naimark_thm}\cite{arveson:c_alg_invt} (Commutative Gelfand-Na\u{\i}mark theorem). 
	Let $A$ be a commutative $C^*$-algebra and let $\mathcal{X}$ be the spectrum of A. There is the natural $*$-isomorphism $\gamma:A \xrightarrow{\cong} C_0(\mathcal{X})$.
\end{theorem}


\begin{lemma} (Generalized commutative Gelfand theorem).
	If $\sX$ is a compact Hausdorff space then  is a natural  homeomorphism $\mathfrak{Gelfand}\left(C\left( \sX\right)  \right)\cong \sX$.
\end{lemma}
\end{frame}
\begin{frame}
\begin{definition}\label{hereditary_extension_defn}
	If both be $A$ and $\widetilde{A}$ be $C^*$-algebras then  a *-homomorphism
	\bean
	\varphi: A \hookto M\left(\widetilde A\right)
	\eean
	is  \alert{hereditary full} if a set $\varphi\left( A\right) \widetilde{A} \varphi\left(A \right)$ is dense in  $\widetilde A$. Equivalently a left ideal $\widetilde{A} \varphi\left(A \right)$  of $\widetilde{A}$ is dense in $\widetilde{A}$.
\end{definition}



\begin{lemma}\label{lolale_lem}
	Any hereditary full  *-homomorphism 	
	\bean
	\varphi: A \hookto M\left(\widetilde A\right)
	\eean
	naturally yields a continuous mapping 
	\bean
	\mathfrak{Gelfand}\left(\widetilde A \right)\xrightarrow{\mathfrak{Gelfand}\left(\varphi \right)}\mathfrak{Gelfand}\left(A \right)
	\eean
\end{lemma}

\end{frame}
\end{document}























