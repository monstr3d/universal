\documentclass{beamer}
\usepackage{amsmath,amssymb,amsthm,slashed, euscript}
\usepackage{mathrsfs} 

\usepackage{tikz}
\usepackage{tikz-cd}

\textwidth=110mm


\title{Algebraic topology of $C^*$-algebras}
\institute
{
Noncommutative geometry and topology
}

\author{Petr R. Ivankov  }



\theoremstyle{plain}
\newtheorem{defn}{Definition}
\newtheorem{rem}{Remark}
\newtheorem{exm}{Example}
\newtheorem*{claim}{Claim}
\newtheorem{prop}{Proposition}
\newtheorem{empt}[prop]{}%[section]
\newtheorem{lem}{Lemma}%[section]
\newtheorem{thm}{Theorem}%[section]



\newcommand{\A}{\mathcal{A}}
\newcommand{\be}{\begin{equation}}
\newcommand{\ee}{\end{equation}}
	\newcommand{\bean}{\begin{eqnarray*}}
	\newcommand{\eean}{\end{eqnarray*}}
\newcommand{\Ga}{\Gamma}
\newcommand{\B}{\mathcal{B}}
\newcommand{\Cc}{\mathcal{C}}
\newcommand{\C}{\mathbb{C}}
\newcommand{\R}{\mathbb{R}}
\newcommand{\D}{\mathcal{D}}
\newcommand{\G}{\mathcal{G}}
\newcommand{\Hc}{\mathcal{H}}
\newcommand{\Lc}{\mathcal{L}}
\newcommand{\Pc}{\mathcal{P}}
\newcommand{\Sc}{\mathcal{S}}
\newcommand{\U}{\mathcal{U}}
\newcommand{\rar}{\rightarrow}
\newcommand{\Ef}{\mathbb{E}}
\newcommand{\desc}{\mathfrak{desc}}
\newcommand{\rep}{\mathfrak{rep}}
\newcommand{\eps}{\varepsilon}


%Uppercase Gothic characters
\newcommand{\gtA}{\mathfrak{A}}
\newcommand{\gtB}{\mathfrak{B}}
\newcommand{\gtM}{\mathfrak{M}}
\newcommand{\gtN}{\mathfrak{N}}
\newcommand{\gtP}{\mathfrak{P}}
\newcommand{\gtS}{\mathfrak{S}}

%Lowercase Gothic characters
\newcommand{\gtf}{\mathfrak{f}}
\newcommand{\gtg}{\mathfrak{g}}

%Bold Characters
\newcommand{\Cb}{\mathbb{C}}
\newcommand{\Nb}{\mathbb{N}}
\newcommand{\Rb}{\mathbb{R}}
\newcommand{\Zb}{\mathbb{Z}}

%Uppercase Greek characters
\newcommand{\Gm}{\Gamma}
\newcommand{\Te}{\Theta}
\newcommand{\Om}{\Omega}
\newcommand{\s}{ }

%Lowercase Greek characters
\newcommand{\al}{\alpha}
\newcommand{\gm}{\gamma}
\newcommand{\dl}{\delta}
\newcommand{\sg}{\sigma}
\newcommand{\ph}{\varphi}
\newcommand{\te}{\theta}
\newcommand{\ze}{\zeta}
\newcommand{\lift}{\mathfrak{lift}}

\newcommand{\Id}{\mathrm{Id}}
\newcommand{\Aut}{\mathrm{Aut}}
\newcommand{\Coo}{{\mathrm{C}}^\infty}
\newcommand{\alg}{\mathrm{alg}}
\newcommand{\diag}{\mathrm{diag}}
\newcommand{\spinc}{\textbf{$spin^c$}}
\newcommand{\Hom}{\mathrm{Hom}}
\newcommand{\supp}{\mathrm{supp}}
\newcommand{\Ccl}{\mathbf{C}l}
\newcommand{\xto}{\xrightarrow}

\newcommand{\lto}{\longrightarrow}
\newcommand{\ox}{\otimes}
\newcommand{\nb}{\nabla}
\newcommand{\sS}{\mathcal{S}}
\newcommand{\Dn}{D\!\!\!\!/}
%\newcommand{\ij}{{i,j}}
\newcommand{\aC}{\ensuremath{\underline{\Cb}} }
\newcommand{\scp}[2]{\left\langle{#1},{#2}\right\rangle}
\newcommand{\op}[1]{J{#1}J^\dag}
\newcommand{\K}{\mathcal{K}} 
\newcommand{\F}{\mathcal{F}} 
\newcommand{\E}{\mathcal{E}} 
\newcommand{\sA}{\mathcal{A}} 
\newcommand{\sB}{\mathcal{B}}       %%
\newcommand{\sC}{\mathcal{C}}       %%
\newcommand{\sD}{\mathcal{D}}       %%
\newcommand{\sE}{\mathcal{E}}       %%
\newcommand{\sF}{\mathcal{F}}       %%
\newcommand{\sG}{\mathcal{G}}       %%
\newcommand{\sH}{\mathcal{H}}       %%
\newcommand{\sI}{\mathcal{I}}       %%
\newcommand{\sJ}{\mathcal{J}}       %%
\newcommand{\sK}{\mathcal{K}}       %%
\newcommand{\sL}{\mathcal{L}}       %%
\newcommand{\sM}{\mathcal{M}}       %%
\newcommand{\sN}{\mathcal{N}}       %%
\newcommand{\sO}{\mathcal{O}}       %%
\newcommand{\sP}{\mathcal{P}}       %%
\newcommand{\sQ}{\mathcal{Q}}       %%
\newcommand{\sR}{\mathcal{R}}       %%
\newcommand{\sT}{\mathcal{T}}       %%
\newcommand{\sU}{\mathcal{U}}       %%
\newcommand{\sV}{\mathcal{V}}       %%
\newcommand{\sX}{\mathcal{X}}       %%
\newcommand{\sY}{\mathcal{Y}}       %%
\newcommand{\sZ}{\mathcal{Z}}       %%
\newcommand{\N}{\mathbb{N}}                  %% 

\renewcommand{\a}{\alpha}     
\newcommand{\la}{\lambda}     
\newcommand{\La}{\Lambda}
\newcommand{\bt}{\beta}           %% short for  \beta
 
    
\newcommand{\bydef}{\stackrel{\mathrm{def}}{=}}  
\newcommand{\hookto}{\hookrightarrow}        %% abbreviation
  
\begin{document}

\begin{frame}
  \titlepage
\end{frame}

\begin{frame}
\begin{theorem}\textit{Pavlov, Troisky}
	Suppose $\mathcal X$ and $\mathcal Y$ are compact Hausdorff connected spaces and $p :\mathcal  Y \to \mathcal X$
	is a continuous surjection. If $C(\mathcal Y )$ is a projective finitely generated Hilbert module over
	$C(\mathcal X)$ with respect to the action
	\begin{equation*}
		(f\xi)(y) = f(y)\xi(p(y)),\quad  f \in  C(\mathcal Y ), \quad  \xi \in  C(\mathcal X),
	\end{equation*}
	then $p$ is a finite-fold  covering.
\end{theorem}
	There is a counterexample  to the Theorem. Alexandru Chirvasitu. \textit{ Non-commutative branched covers and bundle unitarizability}, arXiv:2409.03531v1, 2024.
	

\end{frame}
\begin{frame}
Development of the (co)homology theory of $C^*$-algebras such that: 
\begin{itemize}
	\item for any commutative $C^*$-algebra $C\left(\sX \right)$ it coincides with  (co)homology theory of $\sX$.
	\item the theory is not trivial even for algebras having one-point  spectrum.
\end{itemize}
\end{frame}
\section{Introduction}

\begin{frame}
	
	\begin{figure}
		\includegraphics[width=\linewidth]{PICTURE.png}
%		\caption{A boat.}
%		\label{fig:boat1}
	\end{figure}
Irrational rotation $C^*$-algebra $C^*_r\left( M, \mathcal F\right)$ and its one-sided ideal.
\end{frame}
\begin{frame}
		\begin{definition}\label{upper_defn}
	Let $A$ be a partially ordered set, $S$ a subset of $A$. We say an element $a \in A$
	is a \textit{ meet} (or \textit{ greatest lower bound}) for $S$, and write $a = \wedge S$, if 
	\begin{enumerate}
		\item [(a)] $a$ is an lower bound for $S$, i.e. $a \le s$ for all $s \in S$, and 
		\item [(b)] if $b$ satisfies $\forall \in S\quad  b \le s$, then $b \le a$. 
	\end{enumerate}
	The antisymmetry axiom  ensures that the join of $S$, if it exists, is 
	unique. If $S$ is a two-element set $\{s, t\}$, we write $s \wedge t$ for $\wedge \{s, t\}$ and if $S$	is the empty set $\emptyset$, we write $0$ for  $\wedge \emptyset$.
\end{definition}

\begin{definition}\label{lattice_defn}
	A \textit{meet-semilattice} is a partially ordered set which supports for any finite set the  greatest lower bound.
\end{definition}
\begin{definition}\label{filter_defn}
	A subset $\mathfrak F$ of a meet-semilattice $A$ is said to be an {\it filter} if
	\begin{enumerate}
		\item [(a)] $\mathfrak F$ is a sub-meet-semilattice of $A$; i.e. $1\in A$, and $a, b \in \mathfrak F$  imply  	$a \wedge b \in \mathfrak F$; and
		\item [(b)] $\mathfrak F$ is a lower set; i.e. $a \in \mathfrak F$ and $ a\le b$ imply $b \in \mathfrak F$.  
	\end{enumerate}
\end{definition}
\end{frame}
\section{Gelfand space}
\begin{frame}
		Any homomorphism $\phi : L' \to L''$ of {semi}-{lattices} yields a map
		of filers
		\be\label{filtermap_eqn}
		\left\{I_\la\right\}_{\la \in \La}	\mapsto \text{ minimal filter containing } \left\{\phi\left( I_\la\right) \right\}_{\la \in \La}
		\ee
\begin{definition}\label{ultra_filter_defn}
	A maximal filter is an \textit{ultrafilter}.
\end{definition}		
		\begin{example}\label{space_semi_exm}
			If $\sX$ is a topological space then the $\sX$-\textit{semi}-\textit{lattice} is  a  meet-semilattice $\mathfrak{Lattice}\left( \sX\right)$ such that elements of $\mathfrak{Lattice}\left( \sX\right)$ are open subsets of $\sX$ and one has
			\be\label{x_lat_eqn}
			\begin{split}
				\sU' \wedge \sU''\bydef \sU' \cap \sU'',\\
				\sU' \le \sU'' \quad \Leftrightarrow\quad \sU'' \subset \sU',\\
				0 \bydef \emptyset.
			\end{split}
			\ee
			A set of neighborhoods of a point of Hausdorff space is an ultrafilter.
		\end{example}
	\label{ultra_filter_rem}
			From the Zorn’s lemma it follows that any filter is a subset of an	ultrafilter.

		
	\end{frame}

	\begin{frame}
\begin{definition}\label{lattice_a_defn}
	If $A$ be a $C^*$-algebra then $A$-\textit{semi}-\textit{lattice} is a meet-semilattice of closed left ideals such that 
	\be\label{meet_eqn}
	\begin{split}
		L' \wedge L''\bydef L' \cap L'',\\
		L' \le L'' \quad \Leftrightarrow\quad L' \subset L'',\\
		0 \bydef \{0\}\subset A
	\end{split}
	\ee
	We denote this semilattice by $\mathfrak{Lattice}\left( A\right)$ and denote by $\mathfrak{Filters}\left( A\right)$  a set of filers.
\end{definition}
\begin{definition}\label{ultrafilters_space_defn}
	If $\mathfrak{Ultrafilters}\left(A \right)$ is a set of ultrafilters of the meet-semilattice $\mathfrak{Lattice}\left( A\right)$ then for any closed left ideal $L \subset A$ denote by
	$$
	\mathfrak{Ultrafilters}\left(A \right)_L	\bydef \left\{\left. x \in \mathfrak{Ultrafilters}\left(A \right) \right| \exists L' \in x\quad L'\cap L = \{0\} \right\}
	$$
	The \textit{space of} $A$-\textit{ultrafilters} is a set $\mathfrak{Ultrafilters}\left(A \right)$ with a  the smallest topology o such that all sets $\mathfrak{Ultrafilters}\left(A \right)_L$ are open.
	
\end{definition}
\end{frame}
\begin{frame}

\begin{rem}
If $A = C_0\left( \sX\right)$ then $\mathfrak{Lattice}\left( A\right)\cong \mathfrak{Lattice}\left(\sX\right)$, $\mathfrak{Filters}\left( A\right)\cong \mathfrak{Filters}\left(\sX\right)$ and $\mathfrak{Ultrafilters}\left( A\right)\cong \mathfrak{Ultrafilters}\left(\sX\right)$.
\end{rem}
\begin{thm}
	One has:
	\begin{enumerate}
		\item [(a)] 	A topological space $\sX$ is Hausdorff if and only if every ultrafilter on $\sX$ has at most one limit.
		\item[(b)]   	A topological space $\sX$ is compact if and only if every ultrafilter has at least one limit.
	\end{enumerate} 
\end{thm}	
\begin{rem}
	If $sX$ is locally compact then there are "infinite ultrafilters" which do not correspond to points of $\sX$.
\end{rem}
\end{frame}
\begin{frame}
Here we would like exclude "infinite ultrafilters".
	\begin{thm}\label{pedersen_ideal_thm}  
	% THEOREM 5.6.1
	For each $C^*$-algebra $A$ there is a dense hereditary ideal $K(A)$,
	which is minimal among dense ideals.
	\end{thm}
\begin{definition}\label{pedersen_ideal_defn}
	The ideal $K\left( A\right) $  is said to be the {\it Pedersen's ideal of $A$}. %Henceforth Pedersen's ideal shall be denoted by $K(A)$.
	\end{definition}
	One has $K\left( C_0\left( \sX\right) \right) = C_c\left(\sX \right)$ 
	\begin{definition}\label{gelfand_space_defn}({\it Ivankov})
		An ultrafilter  $x \in \mathfrak{Ultrafilters}\left(A \right)$ is a \textit{finite point} if there is a nontrivial element $a\in K\left( A\right)\setminus\{0\}$ such that 
		$$
		Aa \in x.
		$$
		The \textit{Gelfand space} $\mathfrak{Gelfand}\left(A \right)$ of $C^*$-algebra $A$ is a topological subspace  of the {space of} $A$-{ultrafilters} (cf. Definition \ref{ultrafilters_space_defn})
		$$
		\mathfrak{Gelfand}\left(A \right)\bydef \left\{\left. x \in \mathfrak{Ultrafilters}\left(A \right)\right| \exists L \in x \quad L \text{ is a finite point}\right\}.
		$$
	\end{definition}
	
\end{frame}
\begin{frame}
	\begin{thm}\label{gelfand-naimark_thm} (Commutative Gelfand-Na\u{\i}mark theorem). 
		Let $A$ be a commutative $C^*$-algebra and let $\mathcal{X}$ be the spectrum of A. There is the natural $*$-isomorphism $\gamma:A \xrightarrow{\cong} C_0(\mathcal{X})$.
	\end{thm}
\begin{lemma}\label{lem_g_lem} ({\it Ivankov}) (Generalized commutative Gelfand theorem).
	If $\sX$ is a locally compact, Hausdorff space then  is a natural  homeomorphism $ \mathfrak{Gelfand}_\sX: \mathfrak{Gelfand}\left(C_0\left( \sX\right)  \right)\cong \sX$.
	\end{lemma}
\end{frame}
\section{Morphisms}
\begin{frame}
	\begin{definition}\label{lrc_defn}
	If $A$ is a $C^*$-algebra then a linear map $\la: A\to A$ is said to be a \textit{left centralizer} if
	\be
	\la\left(ab\right)= 	\la\left(a\right) b \quad \forall a, b \in A.
	\ee
	Similarly one defines a \textit{right} centralizer. Denote the spaces of left and right centralizers by $\mathbf{LC}(A)$ and  $\mathbf{RC}(A)$.
\end{definition}
If both $A$ and $\widetilde{A}$ are $C^*$-algebra then any injective homomorphism 	
\bean
\varphi_R: A \hookto \mathbf{RC}\left(\widetilde A\right)	
\eean
of $\C$-algebras yields a homomorphism of lattices
\be\label{lat_mor_eqn}
\begin{split}
	\mathfrak{Lattice}\left(\widetilde  A\right) \xrightarrow{\mathfrak{Lattice}\left(\varphi_R \right)}\mathfrak{Lattice} \left( A\right),\\
	\widetilde{L} \mapsto \bigcap \left\{L \subset A \left| \widetilde{L}\subset \widetilde{A}\varphi_R\left( L\right) \right.\right\}.
\end{split}
\ee
\end{frame}
\begin{frame}
On the other hand the homomorphism $\mathfrak{Lattice}\left(\varphi_R \right)$ yields a mapping of filters
\be\label{fil_mor_eqn}
\begin{split}
	\mathfrak{Filters}\left(\widetilde  A\right) \xrightarrow{\mathfrak{Filters}\left(\varphi_R \right)}\mathfrak{Filters} \left( A\right),\\
	\widetilde{x} \mapsto \text{ the filter generated by }\left\{\mathfrak{Lattice}\left(\varphi_R \right)\left(\widetilde L \right)  \left| \widetilde{L}\in \widetilde{ x} \right.\right\}.
\end{split}
\ee
\begin{lemma}\label{good_lem}
	If the homomorphism $\varphi_R$ is injective then the map \eqref{fil_mor_eqn}  induces a continuous mapping
	\be\label{ultra_mor_eqn}
	\begin{split}
		\mathfrak{Ultrafilters}\left(\widetilde  A\right) \xrightarrow{\mathfrak{Ultrafilters}\left(\varphi_R \right)}\mathfrak{Ultrafilters} \left( A\right)
	\end{split}
	\ee
\end{lemma}
\begin{definition}\label{good_defn}
	The injective homomorphism $\varphi_R$ is \textit{good} if the given by \eqref{ultra_mor_eqn} mapping $\mathfrak{Ultrafilters}\left(\varphi_R \right)$ yields the natural continuous  map
	\bean
	\mathfrak{Gelfand}\left(\widetilde A \right)\xrightarrow{\mathfrak{Gelfand}\left(\varphi_R \right)}\mathfrak{Gelfand}\left(A \right).
	\eean
\end{definition}
\end{frame}
\begin{frame}

\begin{lemma}\textit{Ivankov}
	Any injective homomorphism 	
	\bean
	\varphi_R: A \hookto \mathbf{RC}\left(\widetilde A\right)	\eean
	of $\C$-algebras	naturally yields a continuous mapping 
	\be\label{gelfand_map_eqn}
	\mathfrak{Gelfand}\left(\widetilde A \right)\xrightarrow{\mathfrak{Gelfand}\left(\varphi_R \right)}\mathfrak{Gelfand}\left(A \right)
	\ee
	\end{lemma}
\end{frame}
\section{$C^*$-algebra of compact operators}
\begin{frame}
If $\sH$ is a Hilbert space and $A = \K\left(\sH \right)$ then for any  left ideal $L\subset $ there is a closed $\C$-linear subspace $V_L \subset \K\left(\sH \right)$ such that
\bean
L = \left\{\left.a \in \K\left(\sH \right)\right| a V_L = \{0\}\right\}.
\eean
If $L$ is maximal then $V_L$ is a one dimensional space. Any ultrafilter is principal, generated by maximal ideal. The set of one-dimensional subspaces of $\sH$ is a complex projective space  $\C P\left(\sH \right)$, i.e. there is the natural set theoretic bijective map  $\mathfrak{Gelfand}\left(\K\left(\sH \right) \right)\cong  \C P\left(\sH \right)$. The topology  of $\mathfrak{Gelfand}\left(\K\left(\sH \right)\right) $ contains all sets $\C P\left(\sH \right) \setminus V$ where $V$ is a linear projective  $\C$ subspace of  $\C P\left(\sH \right)$. There identity map yields the  continuous map 
$$
\phi_{\sH} : \C P\left(\sH \right)_{\mathfrak{Hausdorff}}\to \C P\left(\sH \right)_{\mathfrak{Gelfand}}
$$
where $\C P_{\mathfrak{Hausdorff}}\left(\sH \right) $ is the projective  space with   Hausdorff  topology and  $\C P\left(\sH \right)_{\mathfrak{Gelfand}}$ is the same set  with the topology of the Gelfand space.
\end{frame}
\begin{frame}
If $\dim \sH = n < \infty$  then $\C P\left(\sH \right)= \C P^n$ and there is the Zariski topology on $ \C P^n$ such that closed subsets are given by polynomial equations. The closed sets of the Gelfand space are given by linear equations, so the Zariski topology is finer then Gelfand one.
There is a composition of continuous maps
$$
\C P^n_{\mathfrak{Hausdorff}}\to \C P^n_{\mathfrak{Zariski}}\to \C P^n_{\mathfrak{Gelfand}}
$$
where the subscript $_{\mathfrak{Zariski}}$ means the Zariski topology.
If  $\dim \sH = \infty$ and $\left\{L_0, L_1, ...\right\}$ is a set of mutually orthogonal  codimension one projective subspaces of $\C P\left(\sH \right)$ then
$$
\C P\left( \sH\right) = \bigcup_{j=0}^\infty \left( \C P\left(\sH \right) \setminus L_j\right) 
$$
\end{frame}
\begin{frame}
Similarly 
$$
\C P^n = \bigcup_{j=0}^{n} \left( \C P^n \setminus L_j\right). 
$$
where $\left\{L_0,  ..., L_n\right\}$ is a set of mutually orthogonal  codimension one subspaces of $\C P^n$. where $\left\{L_0,  ..., L_n\right\}$ is a set of mutually orthogonal codimension one projective subspaces of $\C P^n$.
\begin{definition}\label{nerve_defn}
	Given a set $X$ and a collection $\mathscr W = \left\{W\right\}$ of subsets of $X$, the \textit{nerve} of $\mathscr W$ denoted by $K\left( \mathscr W\right)$, is  the simplicial complex whose simplexes are finite nonempty subsets of  $\mathscr W$ with nonempty intersections. Thus the vertices of $K\left( \mathscr W\right)$ are nonempty elements of $\mathscr W$.
\end{definition}
\end{frame}
\begin{frame}
\begin{theorem}\label{top_nerve_thm}
	%PAGE 193 !!! DOWN !!!
	%4.13. Theorem. 
	Let $\mathscr A$ be a sheaf of Abelian groups  on $\sX$ and let $\mathscr U = \left\{\sU_\a\right\}_{\a \in I}$ an open covering  of $\sX$ having the property that $H^p\left( \sU_{\sigma};\mathscr A \right)= 0$  for $p > 0$ and
	all $\sigma \in K\left(\mathscr U \right)$ in the nerve of coverin. Then there is a canonical isomorphism
	\be
	H^*\left(\sX, \mathscr A \right) \cong \check H^*\left( \mathscr U; \mathscr A\right). 
	\ee
	\end{theorem}
If $F$ is an Abelian group and $\mathscr F$ is a corresponding constant sheaf on $\C P\left(\sH \right)_{\mathfrak{Hausdorff}}$  and $\mathscr W= \left\{ \C P\left(\sH \right) \setminus L_j\right\}_{j = 0, 1, ... }$ or  $\mathscr W= \left\{ \C P^n\setminus L_j\right\}_{j = 0,..., n }$ then
\be\label{nerve_eqn}
\begin{split}
	\forall \sigma \in K\left(\mathscr W\right)\quad \sU_{\sigma} = \sH \setminus \bigcup_{j = 0}^{n_\sigma} L^\sigma_j,\\
	\forall j = 1,..., n \quad L^\sigma_j \text{ is a linear subspace of } \sH \quad \text{codim}_\R ~ L^\sigma_j  \ge 2,
\end{split}
\ee
\end{frame}
\begin{frame}
From the equation \eqref{nerve_eqn} it follows that 
\be\label{nerve_0_eqn}
\begin{split}
	\mathfrak{Lattice}\left(\sU_{\sigma} \right) \cong \mathfrak{Lattice}\left(\sH' \right),\\
	\forall p > 0 \quad \forall \sigma \in K\left(\mathscr W\right) \quad H^p\left( \sU_{\sigma};\mathscr A \right)= 0.
\end{split}
\ee
where $\sH'$ is a Hilbert space and the isomorphism of semi-lattices comes from the inclusion 
$\sU_{\sigma}\subset \sH'$. If $\mathscr F$ is a corresponding to $F$ constant sheaf on $\C P\left(\sH \right)_{\mathfrak{Hausdorff}}$  and $\mathscr W= \left\{ \C P\left(\sH \right) \setminus L_j\right\}_{j = 0, 1, ... }$ or  $\mathscr W= \left\{ \C P^n\setminus L_j\right\}_{j = 0,..., n }$ then 
$H^p\left( \sU_{\sigma};\mathscr A \right)= 0$  for $p > 0$ and
all $\sigma \in K\left(\mathscr W\right)$ where n $K\left(\mathscr W\right)$ is the nerve of $\mathscr W$. From the above theorem \it turns out that
$$
H^*\left(\C P^n_{\mathfrak{Hausdorff}}, \mathscr A \right) \cong \check H^*\left( \mathscr W; \mathscr A\right)
$$
Taking into account that $\C P^n \setminus L_j$ is open in $\C P\left(\sH \right)_{\mathfrak{Gelfand}}$ for any $j$ one has
$$
H^*\left(\C P\left(\sH \right) _{\mathfrak{Hausdorff}}, \mathscr F_{\mathfrak{Hausdorff}} \right) \cong H^*\left(\C P\left(\sH \right) _{\mathfrak{Gelfand}}, \mathscr F_{\mathfrak{Gelfand}}\right)
$$
The spectrum of $\K\left(\sH \right)$ has the single point but cohomology of $\mathfrak{Gelfand}\left(\K\left(\sH \right)  \right)$ are not trivial. 
\end{frame}


\section{Hausdorff blowing-up}
\begin{frame}

\begin{definition}\label{blowing_defn}
If $A$ is a $C^*$-algebra then an inclusion $C_0\left( \sY\right) \subset M\left(A \right)$ is \textit{Hausdorff blowing-up} of $A$ if  both sets
\be\label{blowing_eqn}
\begin{split}
	C_c\left( \sY\right)A \bydef \left\{fa| f \in C_c\left( \sY\right)\quad a \in A \right\},\\
	AC_c\left( \sY\right) \bydef \left\{af| f \in C_c\left( \sY\right)\quad a \in A \right\}
\end{split}
\ee
are dense in $A$.
\end{definition}

\begin{rem}\label{blowing_rem}
	$C_c\left( \sY\right)A$ is dense in $A$ if and only if $AC_c\left( \sY\right)$ is dense in $A$ (cf. equations \eqref{blowing_eqn}), i.e. both equations \eqref{blowing_eqn} are equivalent.
\end{rem}
The "blowing-up" word is inspired by following reasons.
\begin{itemize}
	\item Sometimes there is  the natural partially defined  surjective  map from  Hausdorff blowing-up to the spectrum.
	\item  In the algebraic geometry   "blowing-up" means  excluding of singular points.
\end{itemize}
\end{frame}
\begin{frame}
\begin{definition}\label{blowing_ideals_au_ua_defn}
	Let  $ C_0\left( \sY\right)\subset  M\left( A\right) $ be  Hausdorff blowing-up of $A$ (cf. Definition \ref{blowing_defn}), and let $\sU \subset \sY$ be an open subset. Both   left and right  closed ideals $A_\sU$  and $_\sU A$ of $A$ generated by sets 	$AC_0\left( \sU\right)$ and $C_0\left( \sU\right)A$ are the \textit{left} $\sU$-\textit{ideal} and the \textit{right} $\sU$-\textit{ideal} respectively. A hereditary $C^*$-subalgebra of $A$
	\be\label{blowing_hereditary_u_eqn} 
	\begin{split}
		_\sU A_\sU \bydef	~	_\sU A\cap  A_\sU = A^*	_\sU \cap  A_\sU\\ %(\text{cf. Definition \ref{hered_defn} and the Lemma \ref{hered_bab_lem}}).
	\end{split}
	\ee	
	is the $ \sU$-\textit{subalgebra}.
	
\end{definition}
	\begin{definition}\label{blowing_support_defn}
	If $C_0\left( \sY\right) \subset M\left(A \right)$ is    {Hausdorff blowing-up} of $A$,  $a \in A$ and
	$
	\sU_a \bydef\bigcap 
	\left\{\left.{\sU} \subset \sX\right| a\in~_\sU A_{\sU} \right\}
	$
	then the  closure $\sV_a$  of $\sU_a$ is said to be the \textit{support} of $a$. We write $\supp~ a \bydef \sV_a$.
\end{definition}
\end{frame}
\begin{frame}
	\begin{lemma}\label{blowing_pedersen_compact_lem}
	If  $C_0\left( \sY\right)\hookto M\left( A\right)$ is Hausdorff blowing-up and $a\in A$ belongs to the Pedersen's ideal $K\left(A \right)$  then the support of $a$  is compact.
\end{lemma}
	From the above Lemma it follows that the injective homomorphism $\varphi_\sX : C_0\left(\sX \right) \hookto M\left(A \right)$ is good  \ref{good_defn}). So there is the natural surjective continuous map
\be\label{phi_a_eqn}
\phi_A : \mathfrak{Gelfand}\left(A \right) \to \sX.
\ee
So one has
from the Remark \eqref{phi_a_eqn} it follows that
\begin{enumerate}		\item [(i)] 
	if $\mathscr A$ is sheaf of  Abelian group in $\sY$ then we have a homomorphism $H^q\left(\sY, \mathscr A \right) \xrightarrow{} H^q \left(\mathfrak{Gelfand}\left(A \right) , \phi_A^*\mathscr A\right)$ for each $q$ which is functorial in $f$ and natural in $\mathscr A$.
	\item [(ii)] If $\mathscr B$ is a sheaf of Abelian groups in $\sX$ then we have a spectral sequence (Leray spectral sequence) $H^p\left(\mathfrak{Gelfand}\left( A\right) , R^q\left(\phi_A \right) _*\left(\mathscr B \right) \right)\Rightarrow H^{p + q}\left(\sX, \mathscr B \right)$ which is natural in $ \mathscr B$.
\end{enumerate}
\end{frame}


\section{Continuous trace $C^*$-algebras}
\begin{frame}
\begin{definition}\label{abelian_element_defn}
	A positive element in $C^*$ - algebra $A$ is {\it Abelian} if subalgebra $xAx \subset A$ is commutative.
\end{definition}
\begin{definition}\label{continuous_trace_c_alt_defn}
	%Definition 5.13. 
	A \textit{continuous-trace} $C^*$-\textit{algebra} is a $C^*$-algebra $A$ with Hausdorff
	spectrum $\sX$ such that, for each $x_0\in\sX$ there are a neighbourhood $\sU$ of $x_0$ and $a\in A$ such that $\rep_{ x}\left( a\right) $ is a rank-one projection for all $x \in \sU$.
	\end{definition}
	\end{frame}
	\begin{frame}
If $A$ is a continuous trace $C^*$-algebra  and $L\subsetneqq A$ is a closed  left ideal then   there are $\eps > 0$,  $~a \in A\setminus L$ and an irreducible representation $\rho: A \to B\left( \sH\right)$ with
\be\label{abelian_ineq_eqn}
\forall a' \in L \quad \left\|\rho\left(a - a' \right)  \right\| > \eps.
\ee
On the other hand $A$ is a $C^*$-algebra of type $I_0$. From this fact one can deduce that $a$ is a satisfying to \eqref{abelian_ineq_eqn}  Abelian element. There is $\xi \in \sH$ such that $\rho\left(a \right)= \xi \left\rangle \right\langle \xi$. One has $\sH = \rho\left(A \right) \xi$ since $\rho$ is irreducible. If $\xi \in \rho\left(L\right) \xi$ then $a \in L$. It is impossible so $\C\xi \cap \rho\left(L\right) \xi = \{0\}$. There are $\xi^\parallel \in  \rho\left(L\right) \xi$ such that if $\xi^\perp \bydef \xi - \xi^\parallel$ then $\xi^\perp\perp  \rho\left(L\right) \xi$. So for any closed left ideal there is one irreducible representation with $\rho\left( L\right)\xi \neq \sH$. The spectrum $\sX$ of $A$ is  Hausdorff. If there are $x', x'' \in \sX$ with $x' \neq x''$ and $ \rho_{x'}\left(L\right)\xi' \neq \sH'$, $\rho_{x''}\left(L\right)\xi'' \neq \sH''$ then there is $f \in C_0\left( \sX\right)$ such that $f\left(x' \right)  = 0$ and  $f\left(x'' \right)  = 0$. If $L'\bydef L' + Af$ then $L'$ is a closed left ideal such that $L \subsetneqq L' \subsetneqq A$. So if $L$ is a maximal left ideal then there is a single point with $\rho\left(L\right) \xi\neq \sH$. If codimension of $\rho\left(L\right) \xi\neq \sH$ exceeds 1 then there if an Abelian element $a'' \in A$ with
$$
\rho\left(L \right) \xi \subsetneqq \rho\left(L \right) \xi + \rho\left(a'' \right) \xi \subsetneqq \sH.
$$
	\end{frame}
\begin{frame}

\begin{lemma}\label{ctr_gelfand_lem}
	If $A$ is a continuous trace $C^*$-algebra then any element of  $\mathfrak{Gelfand}\left(A \right) $ is given by a pair $\left(x, \xi \right)$ where
	\begin{enumerate}
		\item[(i)] $x$ is a point of the spectrum $\sX$ of $A$ which corresponds to the irreducible representation $\rep_x: A \to B\left( \sH_x\right) $,
		\item[(ii)] $\xi \in \C P\left( \sH_x\right)$ where $ \C P\left( \sH_x\right)$ is a complex projective space.
	\end{enumerate} 
	
\end{lemma}
\begin{lemma}\label{ctr_bundle_prop}
	%Proposition 5.59. 
	Let $\sH$ be a separable infinite-dimensional Hilbert space. If $A$
	is a stable separable continuous-trace $C^*$-algebra with spectrum $\sX$, there is a locally
	trivial bundle $\left( \F, \pi,\sX\right) $ with fibre $\K(\sH)$ and structure group $\Aut \left( \K(\sH)\right)$ such that $A$ is $C_0\left( \sX\right)$ -isomorphic to the space of sections $\Ga_0\left(\sX \right)$
	%  and ?(A) is the Dixmier-Douady class ?(X) ofthe bundle discussed on page 109. Indeed, the assignment A 7? X induces a bijec-tion between the C0(T)-isomorphism classes of such algebras and the isomorphismclasses of such bundles.
\end{lemma}
\end{frame}
\begin{frame}

Suppose that $A$ is given by a locally trivial  fibre bundle $\F$ with fibre $\K = \K\left(\sH \right)$. There is a subbundle $\E \subset \F$ such that fibers of $\E$ are  rank-one positive operators. It givers a bundle $\C P\left(\E \right)$ with fibre $\C P\left( \sH\right)$. From the Lemma \ref{ctr_gelfand_lem}  it turns out that there is the natural set theoretic bijective map $\mathfrak{Gelfand}\left(A \right)\cong  \C P\left(\E \right)$.
If $\C P\left(\E \right)_{\mathfrak{Gelfand}}$ is $\C P\left(\E \right)$ supplied with Gelfand topology then there is a bijective continuous map
\be\label{pe_c_eqn}
\phi_{\E} : \C P\left(\E \right)_{\mathfrak{Hausdorff}}\to \C P\left(\E \right)_{\mathfrak{Gelfand}}
\ee
\end{frame}
\begin{frame}

\begin{lemma}
	Let $F$ is an Abelian group.
	Let $\pi :  \C P\left(\E \right)_{\mathfrak{Hausdorff}}\to \sX$ be the natural surjective mapping 
	If $\sX = \bigcup_{\mathcal W \in \mathscr W}  \mathcal W$ where: 
	\begin{itemize}
		\item for any $\mathcal W \in \mathscr W$ one has $\pi^{-1}\left(\mathcal W \right) \cong \mathcal W\times \C P\left( \sH\right)_{\mathfrak{Hausdorff}}$,
		\item $H^p\left( \sU,\mathscr F_{\mathfrak{Hausdorff}} \right)= {0}$  for all $\sU \in K\left(\mathscr W \right)$ and $p> 0$.
	\end{itemize}	
	then there is the natural isomorphism
	$$
	H^*\left(\mathfrak{Gelfand}\left(A \right),\mathscr F_{\mathfrak{Gelfand}}  \right) \cong H^*\left(\C P\left(\E \right) _{\mathfrak{Hausdorff}}, \mathscr F_{\mathfrak{Hausdorff}} \right) 
	$$ 
	where both $\mathscr F_{\mathfrak{Hausdorff}}$ and $\mathscr F_{\mathfrak{Gelfang}}$ are corresponding to $F$ locally constant sheaves.
\end{lemma}


\end{frame}
\section{Fundamental group of commutative $C^*$-algebras}
\begin{frame}
	If $\sX$ is a connected, locally compact, Hausdorff space then there is a homeomorphism $\mathfrak{Gelfand}\left(C_0\left(\sX \right)  \right)\cong \sX$. If  $\left(C_0\left(\sX \right), \widetilde{A}, G, \lift \right)$ is a noncommutative covering then the Lemma \ref{lolale_lem} yield a continuous map
	$$
	\mathfrak{Gelfand}\left(\lift \right): \mathfrak{Gelfand}\left(\widetilde A \right)\to \sX.
	$$
	
	\begin{lemma}\label{hered_spectrum_prop}
		If $B$ is a hereditary $C^*$-subalgebra of $A$ then the map $t \mapsto t\cap B$ is a homeomorphism between $\check A\setminus \mathrm{hull}\left( B\right)$ and $\check B$, where 
		$$
		\mathrm{hull}\left( B\right) = \left\{\left. x \in \hat A~\right|~ \rep_x\left(B \right)= \{0\} \right\} .
		$$ 
		
	\end{lemma}
	If $B\subset A$  is a hereditary $C^*$-subalgebra evenly covered by $\left(C_0\left(\sX \right), \widetilde{A}, G, \lift \right)$ (cf. Definition \ref{evenly_defn}) then there is a connected open subset $\sU \subset \sX$ with $B \cong \sU$. From the Definition \ref{evenly_defn} it follows that $\mathfrak{Gelfand}\left(\lift \right)^{-1}\left(\sU\right)$ is the disjoint union of homeomorphic to $\sU$ connected open subsets of  $\mathfrak{Gelfand}\left(\widetilde A \right)$, i.e. the map $\mathfrak{Gelfand}\left(\lift \right)$ is a covering. 
\end{frame}
\begin{frame}
	If $\widetilde A$ is not commutative then there is $\widetilde x \in \mathfrak{Gelfand}\left(\lift \right)$ with
	$$
	\dim \widetilde A / \widetilde x > 1.	
	$$. From the condition (b) of the Definition \ref{fin_pre_defn} it follows that 
	$$
	\dim C_0\left( \sX\right) / \mathfrak{Gelfand}\left(\lift \right)\left( \widetilde x\right) > 1.	
	$$
	It is impossible so $\widetilde A\cong C_0\left(\mathfrak{Gelfand}\left(\widetilde A\right) \right)$ is a commutative $C^*$-algebra. Thus there is an 1-1 correspondence between topological and noncommutative coverings of $C_0\left(\sX \right)$. From the Definition \ref{fund_den} that the fundamental group of $C_0\left(\sX \right)$ if it exists is isomorphic to $\pi_1\left( \sX\right)$.
\end{frame}

\section{Coverings}
\begin{frame}
	\begin{definition}\label{top_covering_defn} 
	Let $\widetilde{\pi}: \widetilde{\mathcal{X}} \to \mathcal{X}$ be a continuous map. An open subset $\mathcal{U} \subset \mathcal{X}$ is said to be \textit{ evenly covered } by $\widetilde{\pi}$ if $\widetilde{\pi}^{-1}(\mathcal U)$ is the disjoint union of open subsets of $\widetilde{\mathcal{X}}$ each of which is mapped homeomorphically onto $\mathcal{U}$ by $\widetilde{\pi}$. A continuous map $\widetilde{\pi}: \widetilde{\mathcal{X}} \to \mathcal{X}$ is called a \textit{covering} if each point $x \in \mathcal{X}$ has an open neighbourhood evenly covered by $\widetilde{\pi}$. $\widetilde{\mathcal{X}}$ is called the {
		\it covering space} and $\mathcal{X}$ the \textit{ base space} of the covering.
\end{definition}
\end{frame}

\begin{frame}
	Let $A$ be a $C^*$-algebra. The group $\mathrm{Aut}\left(\widetilde{A}\right)$ of *-automorphisms carries (at least) two different topologies making it into a topological group  The most important is \textit{the topology of pointwise norm-convergence} based on the open sets
\begin{equation*}
	\left\{\left.\alpha \in \mathrm{Aut}(A) \ \right| \ \|\alpha(a)-a\| < 1 \right\}, \quad a \in A.
\end{equation*}
The other topology is the \textit{uniform norm-topology} based on the open sets
\begin{equation}\label{aut_norm_eqn}
	\left\{\alpha \in \mathrm{Aut}(A) \ \left| \ \sup_{a \neq 0}\ \|a\|^{-1} \|\alpha(a)-a\| < \varepsilon \right. \right\}, \quad \varepsilon > 0
\end{equation}
which corresponds to following "norm"
\begin{equation}\label{uniform_norm_topology_formula_eqn}
	\|\alpha\|_{\text{Aut}} = \sup_{a \neq 0}\ \|a\|^{-1} \|\alpha(a)-a\| = \sup_{\|a\| =1}\  \|\alpha(a)-a\|.
\end{equation}
Above formula does not really means a norm because $\mathrm{Aut}\left(A\right)$ is not a vector space. Henceforth the uniform norm-topology will be considered only.
\end{frame}
\begin{frame}

  \begin{definition}\label{connected_c_a_defn}
	We say that a $C^*$-algebra $A$ is \textit{connected} if it cannot be represented as a direct sum  $A \cong A' \oplus A''$ of nontrivial $C^*$-algebras $A'$ and $A''$.
	
	% (the Gelfand spectrum of the center of $M\left( A\right) $ is connected). Let $A \subset B$ be a connected subalgebra. We say that $A$ is a \textit{connected component} of $B$ if  $1_{M\left( A\right) }$ lies in the center of $1_{M\left( B\right) }$.
\end{definition}
\end{frame}
\begin{frame}

\begin{definition}\label{fin_pre_defn}\textit{Ivankov}
	Let   $A$ be an  connected $C^*$-algebra  and let  $\widetilde{A}$ be  connected $C^*$-algebra , and let $\lift: A \hookto M\left( \widetilde{A}\right) $ be an injective  $*$-homomorphism of % connected
	$C^*$-algebras such that following conditions hold:
	\begin{enumerate}
		\item[(a)] if $\Aut\left(\widetilde{A} \right)$ is a group of $*$-automorphisms of $\widetilde{A}$ then the group  
		\be\nonumber
		G \bydef \left\{ \left.g \in \Aut\left(\widetilde{A} \right)~\right|\forall a \in \lift \left( A\right) \quad ga = a\right\}
		\ee
		is discrete
		\item[(b)] 	$A = \widetilde{A}^G\stackrel{\text{def}}{=}\left\{\left.a\in \widetilde{A}~~\right|\forall g \in G\quad  a = g a\right\}$.
	\end{enumerate}
	We say that the triple $\left(A, \widetilde{A}, G \right)$ and/or the quadruple $\left(A, \widetilde{A}, G, \lift \right)$ and/or $*$-homomorphism $\lift: A \hookto \widetilde{A}$   is a \textit{noncommutative finite-fold  pre-covering}. We write $G\left(\left.\widetilde A~\right|A \right)\bydef G$.
\end{definition}
\end{frame}
\begin{frame}

\begin{definition}\label{evenly_defn}\textit{Ivankov}
	
	Let $\left(A, \widetilde{A}, G, \lift \right)$ be a noncommutative  pre-covering.	A connected hereditary $C^*$-subalgebra $B \subset A$ 
	is $\left(A, \widetilde{A}, G, \lift \right)$- \textit{evenly covered by} $\left(A, \widetilde{A}, G, \lift \right)$ if there is a hereditary $C^*$-subalgebra $\widetilde B \subset \widetilde A$ with a $*$-isomorphism $\lift^{\widetilde B}: B \cong \widetilde B$ such that
	\be\label{evenly_eqn}
	\forall b \in B \quad \lift\left( b\right) = \bt \text{-}\sum_{g\in G} g \lift^{\widetilde B}\left( b\right) 
	\ee
	where $\bt \text{-}\sum$ means the convergence with respect to the strict topology of $M\left( \widetilde A\right)$
\end{definition}
\end{frame}
\begin{frame}
\begin{definition}\label{cov_unital_defn}\textit{Ivankov}
	
	A  noncommutative  pre-covering $\left(A, \widetilde{A}, G, \lift \right)$  with unital $A$ is a \textit{unital noncommutative covering} if for any $x \in \mathfrak{Gelfand}\left(A \right)$ there is a hereditary connected $C^*$-subalgebra of $B$ {evenly covered by} $\left(A, \widetilde{A}, G, \lift \right)$ with $B \in x$
\end{definition}




\begin{definition}\label{cov__defn}\textit{Ivankov}
	
	A  noncommutative finite-fold pre-covering $\left(A, \widetilde{A}, G, \lift \right)$  \textit{noncommutative  covering} if there is 
	if there is an unital noncommutative covering $\left(B, \widetilde{B}, G, \widetilde\lift \right)$  with inclusions $A \subset B$ and $\widetilde A \subset \widetilde B$ such that:
	\begin{enumerate}
		\item [(a)] both $A$ and $B$ are essential ideals of $B$ and $\widetilde B$,
		\item[(b)] $\lift \bydef \left.\widetilde\lift\right|_A$,
		\item[(c)] the action $G \times \widetilde B \to \widetilde B$ naturally comes from the $G \times \widetilde A \to \widetilde A$
	\end{enumerate}	
	\end{definition}
\end{frame}
\begin{frame}
	\textit{Thank you}
\end{frame}
\end{document}



\begin{thebibliography}{10}
	\bibitem{johnstone:topos}P.T. Johnstone. \textit{Topos Theory}, L. M. S. Monographs no. 10, Academic Press, 1977.
	
	\bibitem{arveson:c_alg_invt} W. Arveson. \textit{ An Invitation to $C^*$-Algebras}, Springer-Verlag. ISBN 0-387-90176-0, 1981.
	
	
	\bibitem{johnstone:stone_spaces} Peter Johnstone. \textit{Stone Spaces} Cambridge Studies in Advanced Mathematics 3, Cambridge University Press (1982, 1986).
	
	\bibitem{counter_ex} 	Alexandru Chirvasitu. \textit{ Non-commutative branched covers and bundle unitarizability}, arXiv:2409.03531v1, 2024.
	
	%\bibitem{lie_groupoids_coov}  CORRECT TURKISH
	
	%I. I en, M. G rsoy, A.  zcan \textit{Coverings of Lie groupoids} 	Turk J Math, 35 (2011) , 207   218. 	c T  UB ITAK, doi:10.3906/mat-0902-2, 2011.
	\bibitem{alekseev_bytsko:wilson_nc_tori} Anton Alekseev, Andrei Bytsko, \textit{ 	Wilson lines on noncommutative tori}. arXiv:hep-th/0002101, 2000.
	
	\bibitem{apt_mult}	Charles A. Akemann, Gert K. Pedersen, Jun Tomiyama. \textit{Multipliers of $C^*$-algebras}. Journal of Functional Analysis Volume 13, Issue 3, July 1973, Pages 277-301, 1973.
	
	
	\bibitem{Ambjorn:2000cs}  J.~Ambjorn, Y.~M.~Makeenko, J.~Nishimura and R.~J.~Szabo,  \textit{ Lattice gauge fields and discrete noncommutative Yang-Mills theory}.  JHEP {\bf 0005}, 023 hep-th/0004147, 2000.
	
	\bibitem{af:rings_cat_mod} Anderson F.W., Fuller K.R. \textit{Rings and categories of modules}, Graduate Text in Mathematics, Springer Verlag, N.Y. 1974.
	%\bibitem{akhi:fa}	N. I. Akhiezer, I. M. Glazman. \textit{Theory of Linear Operators in Hilbert Space}. Dover  Publications, New York, 1961, 1963.
	
	\bibitem{antoine:part_o} Antoine, Jean-Pierre, Inoue, Atsushi, Trapani, C.  \textit{Partial $*$- Algebras and Their Operator Realizations}. SPRINGER-SCIENCE+BUSINESS MEDIA, B.V. 2002.
	
	
	\bibitem{antoine:part_s} J.-P. Antoine, F. Mathot, \textit{Partial $*$-algebras of closed operators and their commutants. I. General
		structure}. Ann. Inst. H. Poincar  46, 299 324 (1987), 1987.
	
	\bibitem{arveson:c_alg_invt} W. Arveson. \textit{ An Invitation to $C^*$-Algebras}, Springer-Verlag. ISBN 0-387-90176-0, 1981.
	
	
	\bibitem{atiyah_b_s}
	M. F. Atiyah, R. Bott and A. Shapiro, \textit{Clifford Modules}, Topology {\bf 3} (1964), 3--38.  1964.
	
	
	\bibitem{atiyah:kt}\textit{$K$-theory}
	By Michael Atiyah
	W. A. BENJAMIN, INC. New York, Amsterdam 1967
	Work for these note is was partially supported by NSF Grant GP-1217
	
	
	%\bibitem{ant_azz_scan:flat_k}Paolo Antonini, Sara Azzali, Georges Skandalis \textit{ Flat bundles, von Neumann algebras and $K$-theory with $\mathbb{M}hbb{R}/\mathbb{M}hbb{Z}$-coefficients}, arXiv:1308.0218, 2013.
	
	\bibitem{auslander:galois} M. Auslander; I. Reiten; S.O. Smal\o{}. \textit{Galois actions on rings and finite Galois coverings}. Mathematica Scandinavica (1989), Volume: 65, Issue: 1, page 5-32, ISSN: 0025-5521; 1903-1807/e , 1989.
	
	\bibitem{bfss} T. Banks, W. Fischler, S.H. Shenker and L. Susskind, Phys.
	Rev. {\bf
		D55} (1997) 5112 [{\tt hep-th/9610043}].
	
	\bibitem{bss} T. Banks, N. Seiberg and S.H. Shenker, Nucl. Phys. {\bf B490}
	(1997) 91
	[{\tt hep-th/9612157}].
	
	
	
	
	%\bibitem{bezandry_diagana:bound_unbound}Paul H. Bezandry, Toka Diagana \textit{ Bounded and Unbounded Linear Operators}, in \textit{ Almost Periodic Stochastic Processes}, Springer, 2011.
	
	%\bibitem{ballentine:qm} Leslie E Ballentine. \textit{ Quantum Mechanics: A Modern Development.} World Scientific Publishing Co. Pte. Ltd. 2000.
	\bibitem{bass} H. Bass. \textit{ Algebraic K-theory.} W.A. Benjamin, Inc. 1968. 
	
	
	\bibitem{becker:sting_m}Katrin Becker, Melanie Becker, John H. Schwarz. \textit{String Theory and M-Theory: A Modern Introduction}. Cambridge University Press. 2007.
	
	\bibitem{blackadar:ko} B. Blackadar. \textit{ K-theory for Operator Algebras}, Second edition. Cambridge University Press. 1998.
	
	\bibitem{blackadar:oa} B. Blackadar. \textit{Operator Algebras: Theory of C$*$-Algebras and von Neumann Algebras}, (Encyclopaedia of Mathematical Sciences), Springer,  2006.
	
	\bibitem{blackadar:shape_theory} B. Blackadar, \textit{ Shape theory for $C^*$-algebras}, Math. Scand. 56 , 249-275, 1985.
	
	\bibitem{blecher_merdy} David P. Blecher, Christian Le Merdy. \textit{Operator algebras and their modules - an operator space approach}, CLARENDON PRESS - OXFORD, 2004.
	
	%\bibitem{blecher:hilb_gen} D.P. Blecher. \textit{ A generalization of Hilbert modules}, J.Funct. An. 136, 365-421 1996.
	
	\bibitem{bogachev_measure_v1}V. I. Bogachev. \textit{ Measure Theory} (volume 1). Springer-Verlag, Berlin, 2007.
	
	\bibitem{bogachev_measure_v2}V. I. Bogachev. \textit{ Measure Theory}. (volume 2). Springer-Verlag, Berlin, 2007.
	
	\bibitem{bogopolsky:group_theory}Oleg Bogopolski. \textit{Introduction to Group Theory}. European Mathematical Society. 2008.
	
	
	\bibitem{bourbaki_sp:gt} N. Bourbaki, \textit{ Elements of Mathematics. General Topology}, Part 1. \newline HERMANN, \'{E}DITEURS DES SCIENCES ET DAS ARTS \newline 115 Boulevard Saint-Germain. Paris \newline ADDISON-WESLEY PUBLISHING COMPANY. \newline Reading, Massachusets - Palo Ito - London - Don Mills, Ontario \newline A translation of \newline \'{E}L\'{E}MENTS DE MATH\'{E}MATIQUE, TOPOLOGIE G\'{E}N\'{E}RALE, \newline originally published in French by Hermann, Paris. 1966.
	
	\bibitem{bredon:topology_geometry} Bredon, Glen E. \textit{Topology and geometry} (Graduate texts in mathematics) Corr. 2nd print Edition, 1993
	
	
	\bibitem{bredon:sheaf} Bredon, Glen E. (1997), \textit{Sheaf theory}. Graduate Texts in Mathematics, 170 (2nd ed.), Berlin, New York: Springer-Verlag.  ISBN 978-0-387-94905-5, MR 1481706 (oriented towards conventional topological applications), 1997.
	
	
	\bibitem{brickell_clark:diff_m} F. Brickell and R. S. Clark.
	\textit{ Differentiable manifolds; An introduction.} London; New York: V. N. Reinhold Co., 1970.
	
	
	
	
	
	\bibitem{brown:proper_groupoids}Jonathan Henry Brown. \textit{Proper actions of groupoids on $C^*$-algebras}. arXiv:0907.5570, 2009.
	
	\bibitem{brown:stable} Lawrence G. Brown, Philip Green, and Marc A. Rieffel. \textit{Stable isomorphism and strong Morita equivalence of $C^*$-algebras}. Pacific J. Math., Volume 71, Number 2 (1977), 349-363. 1977.
	
	
	
	\bibitem{brzezinsky:flat_co}Tomasz Brzezinski \textit{Flat connections and (co)modules}, arXiv:math/0608170, 2006.
	\bibitem{candel:foliI}Alberto Candel, Lawrence Conlon. \textit{Foliations I}. Graduate Studies in Mathematics, American Mathematical Society (1999), 1999.
	
	\bibitem{uni_groupoid_ca}Alcides Buss, Rohit Holkar, Ralf Meyer. \textit{A universal property for groupoid C$*$-algebras. I}. math.arXiv:1612.04963v2, 2018.
	
	
	
	
	\bibitem{candel:foliII}Alberto Candel, Lawrence Conlon. \textit{Foliations II}. American Mathematical Society; 1 edition (April 1 2003), 2003.
	
	\bibitem{chun-yen:separability} Chun-Yen Chou. \textit{ Notes on the Separability of $C^*$-Algebras.} TAIWANESE JOURNAL OF MATHEMATICS Vol. 16, No. 2, pp. 555-559, April 2012. This paper is available online at http://journal.taiwanmathsoc.org.tw , 2012.
	
	%\bibitem{mont:hopf-morita} M. Cohen, D. Fischman, S. Montgomery. \textit{Hopf Galois extensions, smash products, and Morita equivalence}. Journal of Algebra Volume 133, Issue 2, September 1990, Pages 351-372, 1990.
	
	\bibitem{clarisson:phd} Clarisson Rizzie Canlubo. \textit{Non-commutative Covering Spaces and Their Symmetries}. PhD thesis. University of Copenhagen. 2017.
	
	\bibitem{connes:foli_survey} A. Connes. \textit{A survey of foliations and operator algebras}. Operator algebras and applications, Part 1, pp. 521-628, Proc. Sympos. Pure Math., 38, Amer. Math. Soc, Providence, R.I., 1982; MR 84m:58140. 1982.
	
	%	\bibitem{cds} A. Connes, M.R. Douglas and A. Schwarz, J. High Energy Phys.{\bf 9802}(1998) 003 [{\tt hep-th/9711162}].
	\bibitem{matrix_tori} Alain Connes,   Michael R. Douglas and Albert Schwarz \textit{ Noncommutative Geometry and Matrix Theory: Compactification on Tori}, arXiv:hep-th/9711162, 1998.
	
	\bibitem{connes:ncg94} Alain Connes. \textit{ Noncommutative Geometry}, Academic Press, San Diego, CA,  661 p., ISBN 0-12-185860-X, 1994.
	
	\bibitem{connes:c_alg_dg} Alain Connes. \textit{ $C^*$-algebras and differential geometry}. arXiv:hep-th/0101093, 2001.
	
	
	
	\bibitem{connes_landi:isospectral} Alain Connes, Giovanni Landi. \textit{ Noncommutative Manifolds the Instanton Algebra and Isospectral Deformations}, arXiv:math/0011194, 2001.
	
	\bibitem{connes_lott:particle} Connes, Alain, Lott, John. \textit{Particle models and noncommutative geometry}, Nuclear Physics B - Proceedings Supplements (1991/01) 18(2): 29-47, 1991
	
	\bibitem{connes_rieffel:nc_ym}A. Connes, Marc A. Rieffel \textit{Yang-Mills for noncommutative two-tori}. 1987 30 pages Published in: Contemp. Math. 62 (1987) 237-266, In *Li, M. (ed.) et al.: Physics in non-commutative world, vol. 1* 46-62. 1987.
	
	%\bibitem{conway:fa}Conway, John B. (1990). \textit{A Course in Functional Analysis}. Graduate Texts in Mathematics. Vol. 96 (2nd ed.). New York: Springer-Verlag. ISBN 978-0-387-97245-9. OCLC 21195908. 1990.
	
	\bibitem{cra_moe:nhaus} M. Crainic and I. Moerdijk. \textit{A remark on sheaf  theory for non-Hausdorff manifolds}. Tech. Report 1119, Utrecht University, 1999.
	
	\bibitem{cuntz:k_c_a}  J. Cuntz. \textit{$K$-theory for certain $C^*$-algebras}, Ann. of Math. (2) 113:1, 1981.
	
	%\bibitem{vandaele:dqg} A. Van Daele. \textit{Discrete Quantum Groups.} Academic Press, San Diego, CA,  661 p., ISBN 0-12-185860-X, 1994. Journal of Algebra Volume 180, Issue 2, March 1996, Pages 431-444, 1996.
	
	
	\bibitem{dabrowski:product}
	Ludwik D\k{a}browski, Giacomo Dossena. \textit{Product of real spectral triples}. International Journal of Geometric Methods in Modern Physics, Volume 08, Issue 08, December 2011.
	
	\bibitem{dix:profinite} John D. Dixon, Edward W. Formanek, John C. Poland, Luis Ribes. \textit{Profinite completion and isomorphic finite quotients}. Journal of Pure and Applied Algebra 23 (1982) 227-23 1, 1982.
	
	\bibitem{dimca:sheaves} Alexandru Dimca. \textit{Sheaves in Topology} Springer Science \& Business Media, Mar 12, 2004. 
	
	\bibitem{Driver} B.~Driver \textit{ Classifications of bundle connection pairs by parallel translation and lassos}.
	J.\ Funct.\ Anal.\ {\bf 83}, no.\ 1, (1989).
	\bibitem{engelking:general_topology} Ryszard Engelking. \textit{General topology}, PWN, Warsaw. 1977.
	
	\bibitem{varilly_bondia:phobos}Jos\'e M.  Gracia-Bond\'{\i}a, Joseph C.  V\'arilly.  \textit{Algebras of Distributions suitable for phase-space quantum mechanics. I}. Escuela de Matem\'{a}tica, Universidad de Costa Rica, San Jos\'e, Costa Rica J. Math. Phys 29 (1988), 869-879, 1988.
	
	%\bibitem{varilly_bondia:deimos} J. C. V\'arilly and J. M. Gracia-Bond\'{\i}a, \textit{Algebras of distributions suitable for phase-space quantum mechanics II: Topologies on the Moyal algebra}, J. Math. Phys. {\bf 29} (1988), 880--887. 1988.
	
	%\bibitem{hajac:s_conn}	P. M. Hajac, \textit{Strong connections on quantum principal bundles},Commun. Math. Phys. {\bf 182} (1996), 579--617. 1996.	
	
	%\bibitem{bruckler:tensor} Franka Miriam Br\"uckler. \textit{ Tensor products of $C^*$-algebras, operator spaces and Hilbert $C^*$-modules}. Mathematical Communications 4(1999), 1999.
	
	\bibitem{do_carmo:rg} Manfredo P. do Carmo. \textit{ Riemannian Geometry.} Birkh\"auser, 1992.
	
	
	\bibitem{chakraborty_pal:quantum_su_2} Partha Sarathi Chakraborty,  Arupkumar Pal. \textit{Equivariant spectral triples on the quantum $SU(2)$ group}. arXiv:math/0201004v3, 2002.
	
	\bibitem{chakraborty_pal:inv_hom} Partha Sarathi Chakraborty,  Arupkumar Pal. \textit{An invariant for homogeneous spaces of compact quantum groups}. Advances in Mathematics. 301 (2016) 258  2016.
	
	%\bibitem{chang:fermionic} Ee Chang-Young, Hiroaki Nakajima, Hyeonjoon Shin. \textit{ Fermionic $T$-duality and Morita Equivalence}, arXiv:1101.0473, 2011.
	
	%\bibitem{morita_hopf_galois}S. Caenepeel, S. Crivei, A. Marcus, M. Takeuchi. \textit{ Morita equivalences induced by bimodules over   Hopf-Galois extensions.}arXiv:math/0608572, 2007.
	
	
	
	%\bibitem{cheng_li:gauge}Cheng, T.-P.; Li, L.-F. \textit{ Gauge Theory of Elementary Particle Physics}. Oxford University Press. ISBN 0-19-851961-3. 1983.
	
	%\bibitem{connes:gravity}A. Connes. \textit{ Gravity coupled with matter and foundation of noncommutative geometry \}, Commun. Math. Phys. 182 (1996), 155 176. 1996.
		
		%\bibitem{connes:c_alg_dg} Alain Connes. \textit{ $C^*$-algebras and differential geometry}. arXiv:hep-th/0101093, 2001.
		
		
		
		%\bibitem{connes_marcolli:motives}
		%Alain Connes, Matilde Marcolli. \textit{ Noncommutative Geometry, Quantum Fields and Motives},  American Mathematical Society, Colloquium Publications, 2008.
		
		% \bibitem{connes_moscovici:local_index} A. Connes and H. Moscovici, \textit{ The local index theorem in noncommutative geometry"}. Geom. and Funct. Anal., 1996.
		
		
		%\bibitem{cuntz_quillen:alg_ext} Joachim Cuntz, Daniel Quillen.  \textit{ Algebra extensions and nonsingularity}, J. Amer. Math. Soc. 8 251-289, 1995
		
		%\bibitem{davis_kirk_at}James F. Davis. Paul Kirk. \textit{ Lecture Notes in Algebraic Topology}. Department of Mathematics, Indiana University, Blooming- ton, IN 47405, 2001.
		
		\bibitem{dijkhuizen:so_doublecov} Dijkhuizen, Mathijs S. \textit{The double covering of the quantum group $SO_q(3)$}. Rend. Circ. Mat. Palermo (2) Suppl. (1994), 47-57. MR1344000, Zbl 0833.17019. 1994.
		
		\bibitem{dixmier_a_r} Jacques Dixmier \textit{ Les C$*$-alg\`{e}bres et leurs repr\'esentations} 2e \'ed. Gauthier-Villars in Paris 1969.
		
		
		
		\bibitem{dixmier_ca}Jacques Dixmier. \textit{$C^*$-Algebras}. University of Paris VI, North-Holland Publishing Company, 1977.
		
		\bibitem{dixmier_tr}J. Dixmier. \textit{ Traces sur les $C^*$-algebras}. Ann. Inst. Fourier, 13, 1(1963), 219-262, 1963.
		
		\bibitem{dosi:multi} Anar Dosi. \textit{Local operator spaces, unbounded operators and multinormed $C^*$-algebras}.      Journal of Functional Analysis 255(7):1724-1760. October 2008.
		
		
		\bibitem{effros:loc_conv} E.G. Effros, C. Webster \textit{ Operator analogues of locally convex spaces}, in: Operator Algebras and Applications,Samos 1996, in: NATO Adv. Sci. Inst. Ser. C Math. Phys. Sci., vol. 495, Kluwer, Amsterdam, 1997.
		
		\bibitem{elliot:an} Elliott H. Lieb, Michael Loss. \textit{Analysis}, American Mathematical Soc., 2001.
		
		
		\bibitem{fell:operator_fields}J. M. G. Fell. \textit{The structure of algebras of operator fields}. Acta Math. Volume 106, Number 3-4 (1961), 233-280. 1961.
		
		\bibitem{quasi_star_many} M. Fragoulopoulou, C. Trapani S. Triolo \textit{Locally convex quasi $*$-algebras with sufficiently many $*$-representations}.  J. Math. Anal. Appl. 388 (2012) 1180 1193, 2012.
		
		\bibitem{quasi_star} Maria Fragoulopoulo, Camillo Trapani \textit{Locally Convex	Quasi $*$-Algebras 	and their	Representations}. Springer Nature Switzerland AG, 2020.
		
		
		\bibitem{moyal_spectral} V. Gayral, J. M. Gracia-Bond\'{i}a, B. Iochum, T. Sch\"{u}cker, J. C. Varilly. \textit{ Moyal Planes are Spectral Triples}. arXiv:hep-th/0307241, 2003.
		
		\bibitem{godement:sheaf} Roger Godement, \textit{Topologie Alg brique et Th orie des Faisceaux}. Actualit s Sci. Ind. No. 1252. Publ. Math. Univ. Strasbourg. No. 13 Hermann, Paris. 1958.
		
		\bibitem{cinfty_manifolds} Juan A. Navarro Gonz lez, Juan B. Sancho de Salas. $\Coo$-\textit{Differentiable Spaces}. Springer.  2003.
		
		
		
		%\bibitem{gilkey:odd_space}P.B. Gilkey. \textit{ The eta invariant and the $K$-theory of odd dimensional spherical space forms}.Inventiones mathematicae, Springer-Verlag, 1984.
		
		
		%\bibitem{nicolas_ginoux:dirac_spectrum}Nicolas Ginoux. \textit{ The Dirac Spectrum.}Springer, Jun 11, 2009.
		
		\bibitem{goldblatt:topoi} Robert Goldblatt. \textit{Topoi: The Categorial Analysis of Logic}. Revised edition of XLVII 445. Studies in logic and the foundations of mathematics, vol. 98. North-Holland, Amsterdam, New York, and Oxford, 1984, xvi + 551 pp. 1984.
		
		\bibitem{varilly_bondia} Jos\'e M. Gracia-Bondia, Joseph C. Varilly, Hector Figueroa, \textit{ Elements of Noncommutative Geometry}, Springer, 2001.
		
		\bibitem{green_schwarz_witten:superstring} \textit{ Superstring Theory: Volume 2, Loop Amplitudes, Anomalies and Phenomenology}. (Cambridge Monographs on Mathematical Physics) by Michael B. Green, John H. Schwarz, Edward Witten. 1988.
		
		
		%\bibitem{gross_gauge}David J. Gross. \textit{ Gauge Theory-Past, Present, and Future} Joseph Henry Luborutoties, Ainceton University, Princeton, NJ 08544, USA. (Received November 3,1992).
		
		%\bibitem{ful:gr_repr} Fulton William, Harris Joe. \textit{ Representation theory. A first course} Graduate Texts in Mathematics, Readings in Mathematics 129, New York: Springer-Verlag. 1991.
		
		\bibitem{had:ntk} Tom Hadfield. \textit{K-homology of the rotation algebras} $A_\th$. arXiv:math/0112235, 2001.
		
		
		\bibitem{hartshorne:ag} Robin Hartshorne. \textit{ Algebraic Geometry.} Graduate Texts in Mathematics, Volume 52, 1977.
		
		
		
		%\bibitem{halmos:set} Paul R.  Halmos \textit{ Naive Set Theory.} D. Van Nostrand Company, Inc., Prineston, N.J., 1960.
		
		
		%\bibitem{helemsky:qfa} A. Ya. Helemsky. \textit{ Quantum Functional Analysis. Non-Coordinate Approach.} Providence, R.I. : American Mathematical Society, 2010.
		
		
		\bibitem{hajac:toknotes}
		\textit{ Lecture notes on noncommutative geometry and quantum groups}, Edited by Piotr M. Hajac.
		
		\bibitem{hamermesh:group} M. Hamermesh. \textit{Group Theory and its Applications to Physical Problems}, Addison-Wesley, 1962.
		
		
		\bibitem{harzheim:os} Egbert Harzheim. \textit{Ordered sets}.
		Springer Science+Business Media, Inc. 2005.
		
		
		
		
		%\bibitem{isaacs:auto} I.M. Isaacs. \textit{Automorphisms of matrix algebras over   commutative rings}. Linear Algebra and its Applications, Volume 31, June 1980, Pages 215-231, 1980.
		
		\bibitem{hilsum_scandalis:stab}Hilsum, Michel; Skandalis, Georges. \textit{Stabilit  des $C^*$-alg bres de feuilletages}. Annales de l'Institut Fourier, Volume 33 (1983) no. 3, p. 201-208, 1983. 
		
		\bibitem{hatcher:at}Allen Hatcher, \textit{Algebraic topology}. Cambridge University Presses, Cambridge, 2002. 
		
		\bibitem{hatcher:kt}\textit{Vector Bundles \& $K$-Theory}. Version 2.2, November 2017. Copyright c 2003.
		Paper or electronic copies for noncommercial use may be made freely without explicit permission from the author.
		All other rights reserved.
		
		
		\bibitem{horm:I}  Lars H\"ormander. \textit{The Analysis of Linear Partial Differential Operators I. Distribution Theory and Fourier Analysis}. Springer Verlag. 1990.
		
		%\bibitem{ivankov:ms}Petr Ivankov. \textit{Moduli space of noncommutative flat connections and finite-fold nomcommutative coverings}.{J. Phys.: Conf. Series}  \textbf{1194} 12051, 2018.
		
		
		\bibitem{ikkt} N. Ishibashi, H. Kawai, Y. Kitazawa and A. Tsuchiya, Nucl.
		Phys. {\bf
			B498} (1997) 467 [{\tt hep-th/9612115}] 1997.
		
		\bibitem{jensen_thomsen:kk}Jensen, K. K. and Thomsen, K. \textit{Elements of KK-theory.} (Mathematics: Theory and Applications,
		Birkhauser, Basel-Boston-Berlin 1991), viii + 202 pp. 3 7643 3496 7, sFr. 98. 1991.
		
		%\bibitem{johnstone:topos}P.T. Johnstone. Topos Theory, L. M. S. Monographs no. 10, Academic Press 1977.
		
		%\bibitem{kakariadis:corr}Evgenios T.A. Kakariadis, Elias G. Katsoulis, \textit{ Operator algebras and $C^*$-correspondences: A survey.} 	arXiv:1210.6067, 2012.
		
		\bibitem{landi:nm_nt} {G. Landi}, {F. Lizzi} and {R.J. Szabo}. \textit{From Large $N$ Matrices to the  Noncommutative Torus}.  DSM--QM462 DSF--40/99 NBI--HE--99--48 hep--th/9912130 December 1999.
		
		\bibitem{new_matrix} Giovanni Landi, Fedele Lizzi and Richard J. Szabo. \textit{ A New Matrix Model for Noncommutative Field Theory}.  arXiv:hep-th/0309031v1, 2003.
		
		\bibitem{kahn:glo_an} Donald W. Kahn. \textit{Introduction to Global Analysis}.  ACADEMIC PRESS
		A Subsidiary of Harcourt Brace Jovanovich, Publishers
		New York London Toronto Sydney San Francisco. 1980.  
		
		
		\bibitem{kasch:mr} \textit{Modules and Rings}. A translation of \textit{Moduln und Ringe}. German text by F. Kasch, Ludwig-Maximilian University, Munich, Germany, Translation and editing by D. A. R. WALLACE University of Stirling, Stirling, Scotland 1982 ACADEMIC PRESS. A Subsidiary of Harcourt Brace Jovanovich, Publishers LONDON, NEW YORK, PARIS, SAN DIEGO, SAN FRANCISCO, S\~{A}O PAULO, SYDNEY, TOKYO, TORONTO.  1982.
		
		\bibitem{kaplansky:certain} I. Kaplanky. \textit{The structure of certain operator algebras.} Trans. Amer. Math. Soc., 70 (1951), 219-255. 1951.
		
		
		
		%\bibitem{kaku:loc}Kaku, M. \textit{ Locality in the gauge-covariant field theory of strings}. Phys. Lett. 162B, 97. Kaku, M. 1986.
		
		%\bibitem{karaali:ha} Gizem Karaali \textit{ On Hopf Algebras and Their Generalizations}, arXiv:math/0703441, 2007.
		
		\bibitem{karoubi:k} M. Karoubi. \textit{ K-theory, An Introduction.} Springer-Verlag. 1978.
		
		\bibitem{kelley:gt} John L. Kelley. \textit{ General Topology
		}. Springer, 1975. 
		
		
		\bibitem{kl-sch} Klimyk, A. \& Schmuedgen, K. {\sl Quantum Groups	and their Representations}, Springer, New York, 1998.
		
		
		
		%\bibitem{kastler:connes_lott} Daniel Kastler, Thomas Schucker, \textit{ The Standard Model a la Connes-Lott}, arXiv:hep-th/9412185, 1994.
		
		
		
		\bibitem{kobayashi_nomizu:diff_geom} S. Kobayashi, K. Nomizu. \textit{ Foundations of Differential Geometry}. Volume 1. Interscience publishers a division of John Willey \& Sons, New York - London. 1963.
		
		
		
		\bibitem{krajewski:finite}
		Thomas Krajewski. \textit{Classification of finite spectral triples}. Journal of Geometry and Physics Volume 28, Issues 1  November 1998, Pages 1-30, 1998.
		
		\bibitem{kreyszig:fa} Erwin Kreyszig. \textit{Introductory Functional Analysis with Applications}.  New York, N.Y. : Wiley, 1978.
		\bibitem{Lance:Hilbert_modules}E. Christopher Lance, \textit{Hilbert $C^*$-modules}, London Mathematical Society Lecture Note Series, vol. 210, Cambridge University Press, Cambridge, 1995. DOI 10.1017/CBO9780511526206 MR, 1995.
		
		
		\bibitem{kurosh:lga} A. G. Kurosh. \textit{Lectures on General Algebra}. PERGAMON PRESS.
		OXFORD . LONDON   EDINBURGH   NEW YORK  
		PARIS   FRANKFURT. 1965.
		
		
		\bibitem{kurat:topI} Kazimierz Kuratowski. \textit{Topology}. Volume 1. Academic Press, 1966
		
		
		
		\bibitem{lance:so} E. Christopher Lance. \textit{The compact quantum group $SO(3)_q$}. Journal of Operator Theory, Vol. 40, No. 2 (Fall 1998), pp. 295-307, 1998.
		
		\bibitem{lang} S. Lang. Algebra. Addison-Wesley Publishing Company, Reading, Mass. 1965.
		
		%\bibitem{lazar_tailor:mo}A. J. Lazar and D. C. Taylor, \textit{Multipliers of Pedersen's ideal}, Mem. Amer. Math. Soc No. 169, 1976. (1976).
		
		
		
		\bibitem{lawson_m}
		H. B. Lawson, Jr. and M.-L. Michelsohn, \textit{ Spin Geometry}, Princeton
		Univ. Press, Princeton, NJ, 1989. 
		\bibitem{lee:smooth_manifolds} John M. Lee: \textit{Introduction to Smooth Manifolds} Published 2003, Springer: Graduate Texts in Mathematics ISBN 0-387-95495-3. 2003.
		
		\bibitem{matro:hcm} Manuilov V.M., Troitsky E.V. \textit{Hilbert $C^*$-modules}. % Publication Year: 2005. ISBN-10: 0-8218-3810-5 ISBN-13: 978-0-8218-3810-5 
		Translations of Mathematical Monographs, vol. 226, 2005.
		
		
		
		
		\bibitem{bram:atricle}Bram Mesland. \textit{ Unbounded bivariant $K$-theory and correspondences in noncommutative geometry}. arXiv:0904.4383, 2009.
		
		\bibitem{meyer:unb_repr} Ralf Meyer, \textit{Representations of $*$-algebras by unbounded operators: $C^*$-hulls, local-global principle, and induction} arXiv:1607.04472, 2017.
		
		\bibitem{milne:etale}J.S. Milne. \textit{ \'Etale cohomology.} Princeton Univ. Press.  1980.
		
		\bibitem{milne:lec} J.S. Milne \textit{Lectures on \'Etale Cohomology} Version 2.21 March 22, 2013.
		%\bibitem{miyashita_fin_outer_gal} Y\^oichi Miyashita, \textit{ Finite outer Galois theory of noncommutative rings}. Department of Mathematics, Hokkaido, University, 1966.
		
		%\bibitem{miyashita_infin_outer_gal} Y\^oichi Miyashita, \textit{ Locally finite outer Galois theory}. Department of Mathematics, Hokkaido, University, 1967.
		
		%\bibitem{muhly_williams:groupoid_ctr}  Paul S. Muhly and Dana P. Williams. \textit{ continuous trace  groupoid $C^*$-algebras.}, Math. Scand. 1990
		\bibitem{renault:gropoid_equiv}P. S. Muhly, Jean Renault, Dana P. Williams, \emph{Equivalence and isomorphism for groupoid $C^*$ - algebras} Journal of operator theory. January 1987
		
		
		
		\bibitem{MW08}
		Paul~S. Muhly and Dana~P. Williams, \emph{Renault's equivalence theorem for groupoid crossed products}, New York Journal of Mathematics \textbf{3} (2008), 1--87. 2008;
		
		\bibitem{munkres:topology} James R. Munkres. \textit{ Topology.} Prentice Hall, Incorporated, 2000.
		
		\bibitem{neshv:non_haudorff} Sergey Neshveyev and Gaute Schwartz. \textit{Non-Hausdorff \'etale groupoids and $C^*$-algebras of left  cancellative monoids}. M\"unster J. of Math. 16 (2023), 147 175, 2023.
		
		\bibitem{murphy}G.J. Murphy. \textit{ $C^*$-Algebras and Operator Theory.} Academic Press 1990.
		\bibitem{Paschke:73}
		William~L. Paschke. \emph{Inner product modules over {B}{$^\ast$}-algebras},
		Transactions of the American Mathematical Society \textbf{182} (1973),
		443--468. 1973.
		
		\bibitem{ouchi:cov_fol}Moto O'uchi \textit{Coverings of foliations and associated $C^*$-algebras}. Mathematica Scandinavica Vol. 58 (1986), pp. 69-76. 1986. 
		
		
		
		
		\bibitem{pavlov_troisky:cov} Alexander Pavlov, Evgenij Troitsky. \textit{ Quantization of branched coverings.}   Russ. J. Math. Phys. (2011) 18: 338. doi:10.1134/S1061920811030071, 2011.
		
		%\bibitem{pedersen:semi}Gert  Kj rg rd  Pedersen.	\textit{Applications of weak* semicontinuity in $C^*$-algebra theory}. Duke Math. J. Volume 39, Number 3 (1972), 431-450. 1972.
		\bibitem{parta:two_approaches_ym}
		Partha Sarathi Chakraborty, Satyajit Guin. \textit{ Equivalence of Two Approaches to Yang-Mills on Non-commutative Torus }.
		arXiv:1304.7616v1, 2013.
		
		
		
		\bibitem{pedersen:ca_aut}Gert Kj rg rd Pedersen. \textit{ $C^*$-algebras and their automorphism groups}. London ; New York : Academic Press, 1979.
		
		%\bibitem{pierce:ass} Richard S. Pierce. \textit{Associative algebras}. Springer-Verlag, 1982.
		
		\bibitem{pedersen:mea_c} Gert Kj rg rd Pedersen.  \textit{Measure Theory for $C^*$-Algebras}. Mathematica Scandinavica (1966) Volume: 19, page 131-145
		ISSN: 0025-5521; 1903-1807/e, 1966.
		
		\bibitem{phillips:inv_lim_app}
		N. Christopher Phillips \textit{ Inverse Limits of $C^*$ - algebras and Applications.} 
		University of California at Los Angeles, Los Angeles,  CA 90024
		Edited by David E. Evans, Masamichi Takesaki
		Publisher: Cambridge University Press
		DOI: https://doi.org/10.1017/CBO9780511662270.011
		pp 127-186, Print publication year: 1989.
		
		
		\bibitem{phillips:ped_id}N. C. Phillips. \textit{A new approach to the multipliers of Pedersen's ideal}. Proc. American Mathematical Society, Volume 104, Number 3, November 1988.
		
		\bibitem{phillips:inv_lim} N. Christopher Phillips. \textit{ Inverse Limits of $C^*$ - algebras.} Journal of Operator Theory Vol. 19, No. 1 (Winter 1988), pp. 159-195. 1988.
		
		
		\bibitem{phillips:nt_at}  \textit{Every simple higher dimensional noncommutative torus is an AT algebra}. arXiv:math/0609783, 2006.
		
		
		\bibitem{plymen:mor_s}
		R. J. Plymen, ``Strong Morita equivalence, spinors and symplectic
		spinors'', J. Oper. Theory {\bf 16} (1986), 305--324. 1986.
		
		\bibitem{podles:so_su} Piotr Podle\'{s}. \textit{Symmetries of quantum spaces. Subgroups and quotient spaces of quantum SU(2) and SO(3) groups.} Comm. Math. Phys. Volume 170, Number 1 (1995), 1-20. 1995.
		
		\bibitem{rae:ctr_morita} Iain Raeburn, Dana P. Williams. \textit{Morita Equivalence and Continuous-trace $C^*$-algebras}. American Mathematical Soc., 1998.
		
		\bibitem{reed_simon:mp_1}Michael Reed, Barry Simon. \textit{ Methods of modern mathematical physics 1: Functional Analysis}. Academic Press, 1972.
		
		\bibitem{Rieffel:74a}
		Marc~A. Rieffel, \emph{Induced representations of {C}{$^\ast$}-algebras},
		Advances in Mathematics \textbf{13} (1974), 176--257. 1974.
		
		
		\bibitem{renault:gropoid_ca} Jean Renault, \emph{A groupoid approach to {$C\sp{\ast} $}-algebras}, Lecture Notes in Mathematics, vol. 793, Springer, Berlin, 1980. 
		
		
		%\bibitem{Rieffel:74b}Marc~A. Rieffel. \emph{{M}orita equivalence for {C}{$^\ast$}-algebras and	{W}{$^\ast$}-algebras}, Journal of Pure and Applied Algebra \textbf{5}(1974), 51--96. 1974.
		
		%\bibitem{adams:infinite_loop_spaces} J. F. Adams. \textit{ Infinite loop spaces}. Ann. of Math. Studies no. 90, Princeton Univ. Press, Princeton, N. J., 1978
		
		%\bibitem{phillips:c_infty_loop} N. Christopher Philllips. \textit{ $C^{\infty}$ Loop Algebras and Noncommutative Bott Periodicity}. Transactions of the American Matematical Society, Volume 325, Number 2, June 1991
		
		%\bibitem{sitarz:equiv} Andrzej Sitarz \textit{ Equivariant spectral triples}, Noncommutative Geometry and Quantum Groups (Piotr M. Hajac and Wieslaw Pusz, eds.), Banach Center Publ., vol 61, Polish Acad. Sci., pp. 231-268,  Warsaw 2003
		
		%\bibitem{cuntz:o_n} J. Cuntz, \textit{ Simple $C^*$ - algebras generated by isometries}, Comm. Math. Phys. 57:2, 1977
		
		%\bibitem{cuntz:k_o_n} J. Cuntz, \textit{$K$ - theory of certain $C^*$ - algebras}, Ann. of Math. (2), 113:1 1981
		
		%\bibitem{Cohn:68} Paul~Moritz Cohn. \textit{ {M}orita equivalence and duality}, Queen Mary College   Mathematics Notes, Dillon's Q.M.C.\ Bookshop, London, 1968.
		
		%\bibitem{bourbaki_sp:gt} N. Bourbaki, \textit{ General Topology}. Chapters 1-4, Springer, Sep 18, 1998
		
		%\bibitem{williams_sp:morita_cont_trace_alg} Iain Raeburn, Dana P. Williams. \textit{ Morita Equivalence and Continuous-Trace $C^*$-Algebras}. American Mathematical Soc., 1998
		
		%\bibitem{dixmier_tr}J.Dixmier. \textit{ Traces sur les $C^*$-algebras}. Ann. Inst. Fourier, 13, 1(1963), 219-262, 1963
		
		\bibitem{baum_higson_schik:kh}Paul Baum, Nigel Higson, and Thomas Schick. \textit{ On the Equivalence of Geometric and Analytic $K$-Homology}. Pure and Applied Mathematics Quarterly Volume 3, Number 1 (Special Issue: In honor of Robert MacPherson, Part 3 of 3) 1-24, 2007.
		
		%\bibitem{meyer:morita} Ralf Meyer. \textit{ Morita Equivalence In Algebra And Geometry.} math.berkeley.edu/~alanw/277papers/meyer.tex, 1997
		
		%\bibitem{rumynin_hopf_galois_ci} Dmitriy Rumynin  \textit{ Hopf-Galois extensions with central invariants.}  arXiv:q-alg/9707021 1997
		
		%\bibitem{rieffel_finite_g} Marc A. Reiffel, \textit{ Actions of Finite Groups on $C^*$ - Algebras}. 	Department of Mathematics University of California Berkeley. Cal. 94720 U.S.A. 1980.
		
		
		\bibitem{rieffel_morita} Marc A. Reiffel, \textit{ Morita equivalence for $C^*$-algebras and $W^*$-algebras }, Journal of Pure and Applied Algebra 5 (1974), 51-96. 1974.
		
		\bibitem{Rieffel:76}
		Marc A. Reiffel, \emph{Strong {M}orita equivalence of certain transformation group
			{C}{$^\ast$}-algebras}, {M}athematische {A}nnalen \textbf{222} (1976), 7--22. 1976.
		
		\bibitem{Rieffel:irrat}
		Marc A. Reiffel, 
		\textit{$C^*$-algebras associated with irrational rotations}.
		Pacific J. Math. 93(2): 415-429 (1981). 1981.
		
		%\bibitem{dixmier_douady_d} Claude Schochet, \textit{ Dixmier-Douady for Dummies}. 	arXiv:0902.2025 2009.
		
		
		%\bibitem{rieffel:fin_act} Marc A. Rieffel. \textit{ Actions of finite groups on $C^*$-algebras.} Mathematica Scandinavica, Vol. 47, No. 1 (December 12, 1980), pp. 157-176, 1980.
		
		\bibitem{rtf:qlie} Reshetikhin, N. YU., Takhtadzhyan, L.A., Faddeev, L.D., \textit{Quantization of Lie groups and Lie algebras}. Leningrad Math. J., 1 (1) (1990), 193-225. 1990.
		
		
		\bibitem{rotman:ag} Rotman, Joseph , \textit{ An Introduction to Algebraic Topology}. Part of the Graduate Texts in Mathematics book series (GTM, volume 119), 1988.
		
		%\bibitem{rieffel_finite_g} Marc A. Reiffel. \textit{ Actions of Finite Groups on $C^*$ - Algebras}. 	Department of Mathematics University of California Berkeley. Cal. 94720 U.S.A. 1980.
		
		
		%\bibitem{Rieffel74} M.~A. Rieffel. \textit{ Morita equivalence for {$C\sp{\ast} $}-algebras and {$W\sp{\ast}	$}-algebras}. \newblock { J. Pure Appl. Algebra} {\bf 5} (1974), 51--96. 1974.
		
		%\bibitem{ros:ctr}Jonathan Rosenberg. \textit{Continuous-trace algebras from the bundle theoretic point of view}. Journal of the Australian Mathematical Society, Volume 47, Issue 3 December 1989 , pp. 368-381, 1989.
		
		
		
		\bibitem{ruan:real_comp} Ruan, Z.J. \textit{Complexifications of real operator spaces}. Illinois J. Math.
		Volume 47, Number 4 (2003), 1047-1062. 2003.
		
		\bibitem{ruan:real_os} Ruan, Z.J. \textit{On Real Operator Spaces}. Acta Math Sinica 19, 485 496 (2003). https://doi.org/10.1007. 2003.
		
		
		
		\bibitem{rudin:fa}Walter Rudin. \textit{Functional Analysis}, Second Edition, McGraw-Hill, Inc. New York St. Louis San Francisco Auckland Bogota Caracas Hamburg Lisbon London Madrid Mexico Milan Montreal New Delhi Paris San Juan Sao Paulo Singapore Sydney Tokyo Toronto, 1991.
		
		
		\bibitem{rudin:pa} Rudin, Walter. \textit{ Principles of mathematical analysis}. (3rd. ed.), McGraw-Hill, ISBN 978-0-07-054235-8. 1976.
		
		%\bibitem{ros_scho:kt_uct} Jonathan Rosenberg, Claude Schochet, \textit{ The K\"unneth theorem and the universal coefficient theorem for Kasparov's generalized K -functor}, Duke Math. J. Volume 55, Number 2 1987.
		
		\bibitem{schenkel:nc_parallel}
		Alexander Schenkel. \textit{ Module parallel transports in fuzzy gauge theory.},  arXiv:1201.4785, 2013.
		
		
		\bibitem{schwieger:nt_cov}
		Kay Schwieger, Stefan Wagner. \textit{Noncommutative Coverings of Quantum Tori}. 	arXiv:1710.09396 [math.OA], 2017.
		
		\bibitem{counter_topology}	Lynn Arthur Steen,
		J. Arthur Seebach. \textit{Counterexamples in topology}, Springer-Verlag,
		1970.
		
		\bibitem{sw1} N. Seiberg and E. Witten, J. High Energy Phys. {\bf 9909}
		(1999) 032
		[{\tt hep-th/9908142}].
		
		
		
		\bibitem{Sengupta} A.~Sengupta. \textit{ Gauge invariant functions of connections}.  Proc.\ Am.\ Math.\ Soc.\ {\bf 121}, 897-905 (1994).
		
		\bibitem{discrete_crossed}Adam Sierakowski. \textit{Discrete Crossed product $C^*$-algebras}. Ph.D. Thesis. University of Copenhagen   Department of Mathematical Sciences   2009.
		
		\bibitem{sol_tro:ca_op} Yu. P. Solovyov, E. V. Troitsky. \textit{$C^*$-Algebras and Elliptic Operators in Differential Topology}. (vol. 192 of Translations of Mathemetical Monographs) Amer. Math. Soc, Providence, RI, 2000 (Revised English translation). 2000.
		
		%\bibitem{drag:ineq}  Silvestru Sever Dragomir. \textit{Inequalities for Functions of Selfadjoint Operators on Hilbert Spaces}.  arXiv:1203.166, 2012,
		
		\bibitem{spanier:at}
		E.H. Spanier. \textit{ Algebraic Topology.} McGraw-Hill. New York. 1966.
		
		\bibitem{sudo:ntk} Takahiro Sudo. \textit{K-Homology of Continuous Fields of Noncommutative Tori}. Nihonkai Math. J. Vol.19(2008), 1-19, 2008.
		
		\bibitem{sudo:k_ctr} Takahiro Sudo. \textit{$K$-theory of $C^*$-algebras of locally trivial continuous fields}. Commun. Korean Math. Soc. 2005 Vol. 20, No. 1, 79-92 Printed March 1, 2005.
		
		
		\bibitem{switzer:at} Switzer R M, \textit{ Algebraic Topology - Homotopy and Homology}, Springer. 2002.
		
		
		\bibitem{takeda:inductive} Zir\^{o} Takeda. \textit{Inductive limit and infinite direct product of operator algebras.} Tohoku Math. J. (2) 	Volume 7, Number 1-2 (1955), 67-86. 1955.
		
		%\bibitem{takesaki:oa_ii} Takesaki, Masamichi. \textit{ Theory of Operator Algebras II}. Encyclopaedia of Mathematical Sciences, 2003.
		
		
		\bibitem{thomsem:ho_type_uhf} Klaus Thomsen. \textit{ The homotopy type of the group of automorphisms of a $UHF$-algebra}. Journal of Functional Analysis. Volume 72, Issue 1, May 1987.
		
		\bibitem{torsten:sheaves} Torsten Wedhorn. \textit{Manifolds, sheaves, and cohomology}. (Springer Studium Mathematik - Master) (Englisch) Taschenbuch . August 2016.
		
		
		
		%\bibitem{takeuchi:inf_out_cov}Takeuchi, Yasuji \textit{ Infinite outer Galois theory of non commutative rings} Osaka J. Math. Volume 3, Number 2, 1966.
		
		%\bibitem{inikolaev:c_bundles} Igor Nikolaev, \textit{ Topology of the $C^*$ algebra bundles}. Centre interuniversitaire de recherche en g\'eom\'etrie diff\'erentielle et topologie UQAM Montr\'eal H3C 3P8 Canada 1999.
		
		
		
		
		
		
		%\bibitem{thompsen:homtop}
		%Klaus Thompsen. \textit{ Homotopy classes of * - homomorphisms between stable $C^*$ - algebras and their multiplier algebras.} Duke Matematical Journal (C) August 1990.
		
		
		
		
		%\bibitem{blackadar:oa}
		%B. Blackadar. \textit{ Operator Algebras Theory of $C^*$ Algebras and von Neumann Algebras}. Springer-Verlag Berlin Heidelberg 2006
		
		
		
		
		%\bibitem{murre:fund}
		%J.P. Murre. \textit{ Lectures on An Introduction to Grothendieck's  Theory of the Fundamental Group.} Notes by S. Anantharaman, Tata Institute of Fundamental Research, Bombay, 1967.
		
		
		
		%\bibitem{connes_marcolli::motives} Alain Connes Matilde Marcolli. \textit{ Noncommutative Geometry, Quantum Fields and Motives.} Preliminatry version. www.alainconnes.org/docs/bookwebfinal.pdf
		
		%\bibitem{mesland::unbounded_biviariant} Bram Mesland. \textit{ Unbounded biviariant $K$-theory and correspondences in noncommutative geometry}. arXiv:0904.4383. 2009.
		
		
		
		%\bibitem{connes:ng} A. Connes. \textit{ Noncommutative Geometry.} Academic Press, London, 1994.
		
		%\bibitem{faith:I} C. Faith. Algebra: \textit{ Rings, Modules and Cathegories I}. Springer-Verlag 1973
		
		
		
		
		\bibitem{varilly:noncom} J.C. V\'arilly. \textit{ An Introduction to Noncommutative Geometry}. EMS. 2006.
		
		%\bibitem{voic:dual} D. V. Voiculescu. \textit{Dual algebraic structures on operator algebras related to free products}. J. Operator Theory 17 (1987) 85-98. | MR 873463 | Zbl 0656.46058, 1987.
		
		\bibitem{wagner:pb} S. Wagner. \textit{On noncommutative principal bundles with finite abelian structure group.} J. Noncommut. Geom., 8(4):987  2014.
		
		\bibitem{structure_of_standard} Wai Mee Ching. \textit{The structure of standard $C^*$-algebras and their representations}. Pacific J. Math. 67(1): 131-153 (1976).
		
		Profinite Groups
		\bibitem{wilson:profinite} John S. Wilson. \textit{Profinite Groups}. A Clarendon Press Publication, 1999.
		
		\bibitem{xiaolu:foli_cov} Xiaolu Wang. \textit{On the Relation Between $C^*$-Algebras of Foliations and Those of Their Coverings}. Proceedings of the American Mathematical Society Vol. 102, No. 2 (Feb., 1988), pp. 355-360, 1988.
		
		\bibitem{wittenD} E. Witten, Nucl. Phys. {\bf B460} (1996) 335 [{\tt
			hep-th/9510135}].
		
		
		\bibitem{wegge_olsen} N. E. Wegge-Olsen. \textit{$K$-Theory and $C^*$-Algebras: A Friendly Approach.} Oxford University Press,
		Oxford, England, 1993.
		
		
		\bibitem{weil:basic_number_theory}Andre Weil. \textit{ Basic Number Theory}. Springer 1995.
		
		%\bibitem{Partha_quantum_su} Partha Sarathi Chakraborty, Arupkumar Pal. \textit{ Equivariant spectral triples on the quantum $SU(2)$ group.} arXiv:math.KT/0201004, 2003.
		
		%\bibitem{geom_anal_k_homology} Paul Baum, Nigel Higson, and Thomas Schick \textit{ On the Equivalence of Geometric and Analytic K-Homology} arXiv:math/0701484, 2009.
		
		%\bibitem{blackadar:kocalg_neumann} B. Blackadar \textit{ Operator Algebras Theory of C* - Algebras and von Neumann Algebras}. Springer-Verlag Berlin Heidelberg 2006.
		
		
		
		
		
		
		
		%\bibitem{connesdebois:3dsphere}
		%A. Connes, M. Dubois-Violette. Moduli space and structure of
		%noncommutative 3-spheres.  LPT-ORSAY 03-34 ; IHES/M/03/56.
		%Lett.Math.Phys. 66 91-121. 2003.
		
		%\bibitem{conneslandi:isospectal}
		%A. Connes, G. Landi. Noncommutative Manifolds the Instanton Algebra
		%and Isospectral Deformations, math.QA/0011194, 2000.
		
		%\bibitem{suprsym:qt}
		%J. Fr\"ohlich, O. Grandjean, A. Recknagel. Supersymmetric Quantum
		%Theory and (Non-Commutative) Differential Geometry, ETH-TH/96-45
		%1996.
		
		
		
		
		%\bibitem{reconstr}
		%A. Rennie, J.C. V\'arilly. Reconstruction of Manifolds in
		%Noncommutative Geomery. \newline arXiv:math/0610418v3 [math.OA] 24
		%Mar 2007.
		
		
		%\bibitem{varilly:lecture}
		%J.C. V\'arilly. Dirac operators and Spectral Geometry. Lecture notes
		%by Pave{\l} Witkowsky from Warshaw Noncommutative Geometry, January
		%2006.%http://ncg.mimuw.edu.pl/index.phpoption=com_docman&task=doc_download&gid=10&Itemid=58
		
		\bibitem{TFB2}%[Wil07]
		Dana~P. Williams, \emph{Crossed products of {$C{\sp \ast}$}-algebras},
		Mathematical Surveys and Monographs, vol. 134, American Mathematical Society, Providence, RI, 2007. MR2288954 (2007m:46003), 2007.
		
		
		%\bibitem{wolf:const_curv} Wolf, J. \textit{ Spaces of constant curvature}. New York: McGraw-Hill, 1967.
		
		
		
		\bibitem{woronowicz:su2} S.L. Woronowicz. \textit{Twisted SU(2) Group. An Example of a Non-Commutative Differential Calculus}.	PubL RIMS, Kyoto Univ. 23 (1987), 117-181, 1987.
		\bibitem{woronowicz:unb_affil} S.L. Woronowicz. \textit{Unbounded elements affiliated with C$*$-algebras and non-compact quantum groups
		}. Communications in mathematical physics, 1991 - Springer, 1991.
		
		\bibitem{basic_algebraic_quantum} J. E. Roberts and G. Roepstorff \textit{ Some Basic Concepts of Algebraic Quantum Theory}, Commun. math. Phys. 11, 321—338 (1969), 1969.
		
		\bibitem{isumaru}	Isumaru, A. \textit{ Wave propagation and scattering in random	media} A. Isumaru. – New York: John Wiley \& Sons, 1999.
		
		
		\bibitem{marmat} K.B. Marathe and G. Martucci. \textit{The Mathematical Foundations of Gauge Theories.} North-Holland, 1992.
		
		\bibitem{gauge_princilpal}Matthijs V\'{a}k\'{a}r.
		\textit{ Principal Bundles and Gauge
			Theories}
		Bachelor’s Thesis. Student number 3367428
		Universiteit Utrecht
		Utrecht, June 21, 2011 
		arXiv:2110.06334, 2011.
		
		\bibitem{sheaf_gauge} Grigorios Giotopoulos.
		\textit{ Sheaf Topos Theory:
			A powerful setting for Lagrangian Field Theory} arXiv:2504.08095v1, 2025.
		
		
		\bibitem{nistor:lie_alg}Victor Nistor \textit{ Groupoids and the integration of Lie algebroids}.	J. Math. Soc. Japan	Vol. 52, No. 4, 2000.
		\bibitem{ATKINSON} D.~Atkinson, P.~W.~Johnson, \emph{Quantum Field 
			Theory -- a Self-Contained Introduction},
		Rinton Press, Princeton (2002).
		
		\bibitem{BOGOLIOBOV} N.~N.~Bogolubov, A.~A.~Logunov, I.~T.~Todorov,
		\emph{Introduction to Axiomatic Quantum Field Theory}, Benjamin, Reading,
		Massachusetts (1975).
		
		\bibitem{BALLENTINE} L.~E.~Ballentine, \emph{Quantum Mechanics}, Prentice-Hall
		International, Inc., Englewood Cliffs, New Jersey (1990).
		
		\bibitem{BOHM} A.~Bohm, \emph{Quantum Mechanics: Foundations 
			and Applications}, Springer-Verlag, New York (1994). 
		
		\bibitem{BG} A.~Bohm and M.~Gadella, \textit{ Dirac kets, Gamow 
			Vectors, and Gelfand Triplets}, Springer Lectures Notes in Physics Vol.~348,
		Springer, Berlin (1989).
		
		\bibitem{CAPRI} A.~Z.~Capri, \emph{Nonrelativistic Quantum 
			Mechanics}, Benjamin, Menlo Park, California (1985).
		
		\bibitem{DUBIN} D.~A.~Dubin, M.~A.~Hennings, \emph{Quantum Mechanics, Algebras
			and Distributions}, Longman, Harlow (1990).
		
		\bibitem{GALINDO} A.~Galindo, P.~Pascual, \emph{Quantum Mechanics I},
		Springer-Verlag, Berlin (1990).
		
		\bibitem{KUKULIN} V.~I.~Kukulin, V.~M.~Krasnopol'sky, and J.~Horacek,
		\emph{Theory of resonances}, Kluwer Academic Publishers, Dordrecht (1989).
		
		\bibitem{FOCO} R.~de la Madrid, J.~Phys.~A: Math.~Gen.~{\bf 37}, 8129-8157 (2004); {\sf quant-ph/0407195}.
		
		\bibitem{DIS} R.~de la Madrid, ``Quantum mechanics in rigged
		Hilbert space language,'' Ph.D.~thesis, Universidad de Valladolid 
		(2001). Available at \texttt{http://www.ehu.es/$\sim$wtbdemor/}.
		
		\bibitem{DIRAC} P.~A.~M.~Dirac, \emph{The principles of Quantum Mechanics},
		3rd ed., Clarendon Press, Oxford (1947).
		
		\bibitem{VON} J.~von Neumann, \emph{Mathematische Grundlagen der 
			Quantentheorie}, Springer, Berlin (1931); English translation by R.~T.~Beyer, 
		\textit{ Mathematical Foundations of Quantum Mechanics}, Princeton University 
		Press, Princeton (1955).
		
		\bibitem{QUOTEVONDIRAC} In Ref.~\cite{DIRAC}, page~40, Dirac states 
		that ``\textit{ the bra and ket vectors that we now use form 
			a more general space than a Hilbert space}.'' 
		
		In Ref.~\cite{VON}, page~viii, von Neumann states that ``\textit{ Dirac
			has given a representation of quantum mechanics which is scarcely to be
			surpassed in brevity and elegance,} [...].'' On pages~viii-ix, von Neumann
		says that ``\textit{ The method of Dirac, mentioned above, (and this is overlooked
			today in a great part of quantum mechanical literature, because of the clarity
			and elegance of the theory) in no way satisfies the 
			requirements of mathematical rigor -- not even if these are reduced in a 
			natural and proper fashion to the extent common elsewhere in theoretical 
			physics.}'' On page~ix, von Neumann says that ``[...],\textit{ this requires 
			the introduction of `improper' functions with self-contradictory 
			properties. The insertion of such mathematical `fiction' is frequently
			necessary in Dirac's approach,}[...].'' Thus, essentially, although von
		Neumann recognizes the clarity and beauty of Dirac's formalism, he states
		very clearly that such formalism cannot be implemented within the framework
		of the Hilbert space.
		
		\bibitem{SCHWARTZ} L.~Schwartz, \emph{Th\'eory de Distributions}, Hermann,
		Paris (1950).
		
		\bibitem{GELFAND} I.~M.~Gelfand, N.~Y.~Vilenkin, 
		\emph{Generalized Functions}, Vol.~IV, Academic Press, New York 
		(1964).
		
		\bibitem{MAURIN} K.~Maurin, \emph{Generalized Eigenfunction Expansions and 
			Unitary Representations of Topological Groups}, Polish Scientific 
		Publishers, Warsaw (1968). 
		
		\bibitem{CITEMAURIN} In Ref.~\cite{MAURIN}, page~7, Maurin states that
		``\textit{ It seems to us that this is the formulation 
			which was anticipated by Dirac in his classic 
			monograph.''}
		
		\bibitem{ROBERTS} J.~E.~Roberts, J.~Math.~Phys.~{\bf 7}, 1097--1104 (1966); 
		J.~E.~Roberts, Commun.~Math.~Phys.~{\bf 3}, 98--119
		(1966).
		
		\bibitem{ANTOINE}J.-P.~Antoine, J.~Math.~Phys.~{\bf 10}, 53--69 (1969); 
		J.-P.~Antoine, J.~Math.~Phys.~{\bf 10}, 2276--2290 (1969).
		
		\bibitem{B60} A.~Bohm, ``The Rigged Hilbert Space in Quantum
		Mechanics,'' \emph{Boulder Lectures in Theoretical Physics, 1966}, Vol.~9A
		(Gordon and Breach, New York, 1967). 
		
		\bibitem{QUOTEBALLENTINE} The following quotation, extracted from
		Ref.~\cite{BALLENTINE}, page~19, gives a clear idea of the status
		the RHS is achieving: \textit{ ``...rigged Hilbert space seems to 
			be a more natural mathematical setting for quantum mechanics than Hilbert 
			space.''}
		
		\bibitem{AT93} I.~Antoniou, S.~Tasaki, Int.~J.~Quant.~Chem.~{\bf 44},
		425--474 (1993).
		
		\bibitem{SUCHANECKI} Z.~Suchanecki, I.~Antoniou, S.~Tasaki, 
		O.~F.~Brandtlow, J.~Math.~Phys.~{\bf 37}, 5837--5847 (1996).
		
		\bibitem{DENSE} A subspace $S$ of $\cal H$ is dense in $\cal H$
		if we can approximate any element of $\cal H$ by an element of $S$
		as well as we wish. Thus, for any $f$ of $\cal H$ and for any small
		$\epsilon >0$, we can find a $\varphi$ in $S$ such that 
		$\| f-\varphi \| < \epsilon$. In physical terms, this inequality means that
		we can replace $f$ by $\varphi$ within an accuracy $\epsilon$.
		
		\bibitem{FUNCTIONAL} A function $F: {\mathbf \Phi} \to {\mathbb C}$ is called
		a linear [respectively antilinear] functional over $\mathbf \Phi$ if for any 
		complex numbers $\alpha , \beta$ 
		and for any $\varphi , \psi \in \mathbf \Phi$, it holds that
		$F(\alpha \varphi +\beta \psi)=\alpha F(\varphi) +\beta F(\psi )$ 
		[respectively 
		$F(\alpha \varphi +\beta \psi)=\alpha ^* F(\varphi) +\beta ^*F(\psi )$].
		
		\bibitem{JPA02} R.~de la Madrid, J.~Phys.~A: Math.~Gen.~{\bf 35}, 319--342 
		(2002); {\sf quant-ph/0110165}. 
		
		\bibitem{FP02} R.~de la Madrid, A.~Bohm, and M.~Gadella, Fortsch.~Phys.~{\bf 50}, 185--216 (2002); {\sf quant-ph/0109154}.
		
		\bibitem{IJTP03} R.~de la Madrid, Int.~J.~Theor.~Phys.~{\bf 42}, 2441--2460 
		(2003); {\sf quant-ph/0210167}.
		
		\bibitem{HSDEF} Strictly speaking, a Hilbert space possesses additional
		properties (e.g., it must be complete with respect to the topology induced by
		the scalar product). For a more technical definition of the Hilbert space,
		see for example Ref.~\cite{DIS}.
		
		\bibitem{UNB} An operator $A$ is bounded if there is some finite $K$ such
		that $\| Af \| <K \|f \|$ for all $f\in \cal H$, where $\| \  \|$ denotes 
		the Hilbert space
		norm. When such $K$ does not exist, $A$ is said to be unbounded. For a 
		detailed account of the properties of bounded and unbounded operators, see for
		example Ref.~\cite{DIS}.
		
		\bibitem{RS84} The mathematical reason why quantum mechanical unbounded
		operators cannot be defined on all the vectors of the Hilbert space can be
		found, for example, in Ref.~\cite{RS}, page~84.
		
		\bibitem{RS} M.~Reed, B.~Simon, ``Methods of modern mathematical physics,''
		vol.~I, Academic Press, Inc., New York (1972).
		
		\bibitem{INFENER} If we nevertheless insisted in for example calculating the 
		expectation value~(\ref{exintrodispP}) for elements of $\cal H$ that are not 
		in ${\cal D}(A)$, we would obtain an unphysical infinity value. For instance, 
		if $A$ represents an unbounded Hamiltonian $H$, then the expectation 
		value~(\ref{exintrodispP}) would be infinite for those $\varphi$ of $\cal H$ 
		that lie outside of ${\cal D}(H)$. Because they have infinite 
		energy, those states do not represent physically preparable wave packets.
		
		\bibitem{SNHS} If they were in the Hilbert space, $|a\rangle$ and $\langle a|$
		would be square integrable, and $a$ would belong to the discrete spectrum.
		
		\bibitem{RS274} It is well known that Heisenberg's commutation relation
		necessarily implies that either $P$ or $Q$ is unbounded. See, for example, 
		Ref.~\cite{RS}, page~274.
		
		\bibitem{ZERODERIV} The reason why the derivatives of $\varphi (x)$ must 
		vanish at $x=a,b$ is that we want to be able to apply the Hamiltonian $H$
		as many times as we wish. Since repeated applications of $H$ to $\varphi (x)$
		involve the derivatives of $V(x)\varphi (x)$, and since $V(x)$ is 
		discontinuous at $x=a,b$, the function $V(x)\varphi (x)$ is infinitely 
		differentiable at $x=a,b$ only when the derivatives of $\varphi (x)$ 
		vanish at $x=a,b$. For more details, see Ref.~\cite{ROBERTS}. The vanishing 
		of the derivatives of $\varphi (x)$ at $x=a,b$ must be viewed as a 
		mathematical consequence of the unphysical sharpness of the discontinuities of
		the potential, rather than as a physical consequence of Quantum 
		Mechanics. Note 
		also that in standard numerical simulations, for example, Gaussian wave 
		packets impinging on a rectangular barrier, one never sees that the wave
		packet vanishes at $x=a,b$. This is due to the fact that on a Gaussian wave 
		packet, the Hamiltonian~(\ref{fdoph}) can only be applied once.
		
		
		\bibitem{ONEONEROBERTS} We recall that some authors have erroneously claimed 
		that ``there are more kets than bras''~\cite{ROBERTS}, and that therefore such
		one-to-one correspondence between bras and kets does not hold.
		
		\bibitem{EXTOHS} We can nevertheless  extend 
		Eqs.~(\ref{resonidentPxphi}) and (\ref{resonidentHxphi}) to the 
		whole Hilbert space $L^2$ by a limiting procedure, 
		although the 
		resulting expansions do not involve the Dirac bras and kets any more, but
		simply the eigenfunctions of the differential operators.
		
		\bibitem{COHEN} C.~Cohen-Tannoudji, B.~Diu, and F.~Lalo\"e, 
		\textit{ Quantum Mechanics}, Wiley, New York (1977). 
		
		\bibitem{USEFULNESSPW} This is one of the major reasons why plane waves 
		are so useful in practical calculations.
		
		\bibitem{USEFULNESSBK} This is one of the major reasons why bras and kets
		are so useful in practical calculations.
		
		
		\bibitem{FSHORT} We recall that the direct integral decomposition of the 
		Hilbert space falls short of such factorization, see Ref.~\cite{ANTOINE}.
		
		\bibitem{rhs} Rafael de la Madrid.
		\textit{The role of the rigged Hilbert space in 
			Quantum Mechanics}
		
		
		Departamento de F\'\i sica Te\'orica, Facultad de Ciencias,
		Universidad del Pa\'\i s Vasco, 48080 Bilbao, Spain \\
		%	E-mail: {\texttt{wtbdemor@lg.ehu.es}}}
	
	
	\date{\small{January 4, 2005}} !!!
\end{thebibliography}






















