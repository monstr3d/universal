\documentclass{beamer}
\usepackage{amsmath,amssymb,amsthm,slashed, euscript}



\textwidth=110mm


\title{Noncommutative finite-fold coverings of foliations}
\institute
{
Noncommutative geometry and topology
}

\author{Petr R. Ivankov  }



\theoremstyle{plain}
\newtheorem{defn}{Definition}
\newtheorem{rem}{Remark}
\newtheorem{exm}{Example}
\newtheorem*{claim}{Claim}
\newtheorem{prop}{Proposition}
\newtheorem{empt}[prop]{}%[section]
\newtheorem{lem}{Lemma}%[section]
\newtheorem{thm}{Theorem}%[section]



\newcommand{\A}{\mathcal{A}}
\newcommand{\be}{\begin{equation}}
\newcommand{\ee}{\end{equation}}
\newcommand{\Ga}{\Gamma}
\newcommand{\B}{\mathcal{B}}
\newcommand{\Cc}{\mathcal{C}}
\newcommand{\C}{\mathbb{C}}
\newcommand{\R}{\mathbb{R}}
\newcommand{\D}{\mathcal{D}}
\newcommand{\G}{\mathcal{G}}
\newcommand{\Hc}{\mathcal{H}}
\newcommand{\Lc}{\mathcal{L}}
\newcommand{\Pc}{\mathcal{P}}
\newcommand{\Sc}{\mathcal{S}}
\newcommand{\U}{\mathcal{U}}
\newcommand{\rar}{\rightarrow}
\newcommand{\Ef}{\mathbb{E}}


%Uppercase Gothic characters
\newcommand{\gtA}{\mathfrak{A}}
\newcommand{\gtB}{\mathfrak{B}}
\newcommand{\gtM}{\mathfrak{M}}
\newcommand{\gtN}{\mathfrak{N}}
\newcommand{\gtP}{\mathfrak{P}}
\newcommand{\gtS}{\mathfrak{S}}

%Lowercase Gothic characters
\newcommand{\gtf}{\mathfrak{f}}
\newcommand{\gtg}{\mathfrak{g}}

%Bold Characters
\newcommand{\Cb}{\mathbb{C}}
\newcommand{\Nb}{\mathbb{N}}
\newcommand{\Rb}{\mathbb{R}}
\newcommand{\Zb}{\mathbb{Z}}

%Uppercase Greek characters
\newcommand{\Gm}{\Gamma}
\newcommand{\Te}{\Theta}
\newcommand{\Om}{\Omega}
\newcommand{\s}{ }

%Lowercase Greek characters
\newcommand{\al}{\alpha}
\newcommand{\gm}{\gamma}
\newcommand{\dl}{\delta}
\newcommand{\sg}{\sigma}
\newcommand{\ph}{\varphi}
\newcommand{\te}{\theta}
\newcommand{\ze}{\zeta}

\newcommand{\Id}{\mathrm{Id}}
\newcommand{\Aut}{\mathrm{Aut}}
\newcommand{\Coo}{{\mathrm{C}}^\infty}
\newcommand{\alg}{\mathrm{alg}}
\newcommand{\diag}{\mathrm{diag}}
\newcommand{\spinc}{\textbf{$spin^c$}}
\newcommand{\Hom}{\mathrm{Hom}}
\newcommand{\supp}{\mathrm{supp}}
\newcommand{\Ccl}{\mathbf{C}l}
\newcommand{\xto}{\xrightarrow}

\newcommand{\lto}{\longrightarrow}
\newcommand{\ox}{\otimes}
\newcommand{\nb}{\nabla}
\newcommand{\sS}{\mathcal{S}}
\newcommand{\Dn}{D\!\!\!\!/}
%\newcommand{\ij}{{i,j}}
\newcommand{\aC}{\ensuremath{\underline{\Cb}} }
\newcommand{\scp}[2]{\left\langle{#1},{#2}\right\rangle}
\newcommand{\op}[1]{J{#1}J^\dag}
\newcommand{\sA}{\mathcal{A}} 
\newcommand{\sB}{\mathcal{B}}       %%
\newcommand{\sC}{\mathcal{C}}       %%
\newcommand{\sD}{\mathcal{D}}       %%
\newcommand{\sE}{\mathcal{E}}       %%
\newcommand{\sF}{\mathcal{F}}       %%
\newcommand{\sG}{\mathcal{G}}       %%
\newcommand{\sH}{\mathcal{H}}       %%
\newcommand{\sI}{\mathcal{I}}       %%
\newcommand{\sJ}{\mathcal{J}}       %%
\newcommand{\sK}{\mathcal{K}}       %%
\newcommand{\sL}{\mathcal{L}}       %%
\newcommand{\sM}{\mathcal{M}}       %%
\newcommand{\sN}{\mathcal{N}}       %%
\newcommand{\sO}{\mathcal{O}}       %%
\newcommand{\sP}{\mathcal{P}}       %%
\newcommand{\sQ}{\mathcal{Q}}       %%
\newcommand{\sR}{\mathcal{R}}       %%
\newcommand{\sT}{\mathcal{T}}       %%
\newcommand{\sU}{\mathcal{U}}       %%
\newcommand{\sV}{\mathcal{V}}       %%
\newcommand{\sX}{\mathcal{X}}       %%
\newcommand{\sY}{\mathcal{Y}}       %%
\newcommand{\sZ}{\mathcal{Z}}       %%
\newcommand{\N}{\mathbb{N}}                  %% 

\renewcommand{\a}{\alpha}     
\newcommand{\la}{\lambda}     
\newcommand{\La}{\Lambda}
\newcommand{\bt}{\beta}           %% short for  \beta
 
    
\newcommand{\bydef}{\stackrel{\mathrm{def}}{=}}  
\newcommand{\hookto}{\hookrightarrow}        %% abbreviation
  
\begin{document}
%\titlepage
\begin{frame}
  \titlepage
\end{frame}
	\section{Noncommutative finite-fold coverings}
\begin{frame}
	   \begin{definition}\label{pre_defn} \alert{P. Ivankov}.
		Let $\pi: A \hookto \widetilde{A}$ be an injective *-homomorphism of connected  $C^*$-algebras such that following conditions hold:
		\begin{enumerate}
			\item[(a)] If $\Aut\left(\widetilde{A} \right)$ is a group of *-automorphisms of $\widetilde{A}$ then the group  
			$
			G \bydef \left\{ \left.g \in \Aut\left(\widetilde{A} \right)~\right|~ ga = a;~~\forall a \in A\right\}
			$
			is finite.
			\item[(b)] 	\be\label{cond_b_eqn}
			A \cong \widetilde{A}^G\stackrel{\text{def}}{=}\left\{\left.a\in \widetilde{A}~~\right|~ a = g a;~ \forall g \in G\right\}.\ee
		\end{enumerate}
		We say that the quadruple $\left(A, \widetilde{A}, G, \pi \right)$ and/or *-homomorphism $\pi: A \to \widetilde{A}$   is a \textit{noncommutative finite-fold  pre-covering}. 
	\end{definition}

\end{frame}

\begin{frame}


\begin{definition}
	A topological covering $ p:\widetilde{\mathcal  X}\to \mathcal  X$ is said to be \textit{transitive} if for any $x \in \mathcal  X$ the group
	$$
	G\left(\left.\widetilde{\mathcal  X} \right|\mathcal  X\right)\bydef \left\{ \left.g \in \text{Homeo}\left(\widetilde{\mathcal X} \right)~\right|~ p(g\widetilde{x}) = p(\widetilde{x});~~\forall \widetilde{x} \in \widetilde{\mathcal  X}\right\} 
	$$
	transitively acts on $p^{-1}\left(x\right)$.
\end{definition}
\begin{fact}
	Any finite-fold covering   $ p:\widetilde{\mathcal  X}\to \mathcal  X$ naturally yields an injective *-homomorphism $C\left( p\right) : C_0\left(\sX \right) \hookto C_0\left( \widetilde{\mathcal  X}\right)$.
\end{fact}

\end{frame}

\begin{frame}
	\begin{definition}
		\alert{P. Ivankov}
		Let $\left(A, \widetilde{A}, G, \pi \right)$ be a  noncommutative finite-fold  pre-covering. Suppose both $A$ and  $\widetilde{A}$ are unital. We say that $\left(A, \widetilde{A}, G, \pi \right)$ is an \textit{unital noncommutative finite-fold  covering} if \\ $\pi$ is unital and $\widetilde{A}$ is a finitely generated projective  $A$-module.
	\end{definition}
\begin{lemma}
	\alert{P. Ivankov, A. Pavlov, E. Troicky}
	If $\mathcal  X$ is a connected, compact, Hausdorff space then there is a natural 1-1 correspondence 
	$$
\left(p:\widetilde{\mathcal  X}\to \mathcal  X \right)\leftrightarrow \left(C\left(\mathcal  X\right), C\left(\widetilde\sX\right), G\left(\left.\widetilde{\mathcal  X} \right|\mathcal  X\right), C_0\left(p \right)  \right).  
$$	
	
	between finite-fold transitive coverings of $\mathcal  X$ and unital noncommutative finite-fold  coverings of $C\left(\mathcal  X\right)$.
\end{lemma}

\end{frame}
\begin{frame}


\begin{definition}\label{top_cov_comp_defn}
	\alert{P. Ivankov}. 	A   covering $p: \widetilde{   \mathcal X } \to \mathcal X$ is said to be a \textit{ covering with compactification} if there are compactifications ${   \mathcal X } \hookto {   \mathcal Y }$ and $\widetilde{   \mathcal X } \hookto \widetilde{   \mathcal Y }$ such that:
	\begin{itemize}
		\item There is a finite-fold  covering $\widetilde{p}:\widetilde{   \mathcal Y }\to {   \mathcal Y }$,
		\item The covering $p$ is the restriction of $\widetilde{p}$, i.e. $p = \widetilde{p}|_{\widetilde{   \mathcal X }}$.
	\end{itemize}
\end{definition}


\end{frame}
\begin{frame}
	\begin{definition}\label{fin_comp_defn}\alert{P. Ivankov}
	If $\left(A, \widetilde{A}, G, \pi \right)$ is a noncommutative finite-fold  pre-covering then it is said to be 	a \textit{noncommutative finite-fold covering with unitization} if following conditions hold:
		\begin{enumerate}
			\item[(a)] 
			There are unitizations $A \hookto B$, $\widetilde{A} \hookto \widetilde{B}$.
			%	\item[(b)]$A = B\bigcap \widetilde{A}$,
			\item[(b)] There is an %(strong) 
			unital  noncommutative finite-fold covering	$\left(B ,\widetilde{B}, G, \widetilde{\pi} \right)$ such that $\pi = \widetilde{\pi}|_A$ (or, equivalently $A \cong \widetilde{A}\cap B$) and the action $G \times\widetilde{A} \to \widetilde{A}$ is induced by $G \times\widetilde{B} \to \widetilde{B}$.
		\end{enumerate}
	\end{definition}
\begin{lemma}
	\alert{P. Ivankov}. If $\mathcal X$ is a locally compact, connected, Hausdorff space then there is a one-to-one correspondence 
	$$
\left(p:\widetilde{\mathcal  X}\to \mathcal  X \right)\leftrightarrow \left(C_0\left(\mathcal  X\right), C_0\left(\widetilde\sX\right), G\left(\left.\widetilde{\mathcal  X} \right|\mathcal  X\right), C_0\left(p \right)  \right).  
$$		
	between transitive finite-fold coverings with compactifications and of $\mathcal X$ and noncommutative finite-fold coverings with unitization of $C_0\left(\mathcal X\right)$.

\end{lemma}

\end{frame}

\begin{frame}
	\section{Foliations and their $C^*$-algebras}
\begin{definition}
	
	Let $M$ be a smooth manifold and $TM$ its tangent bundle, so that
	for each $x \in M$, $T_x M$ is the tangent space of $M$ at $x$. A
	smooth subbundle $\mathcal{F}$ of $TM$ is called {\it integrable} if and only if one of
	the following equivalent conditions is satisfied:
	
	\smallskip
	
	\begin{enumerate}
		
		\item[(a)] Every $x \in M$ is contained in a submanifold $W$ of $M$ such that
		$$
		T_y (W) = \mathcal{F}_y \qquad \forall \, y \in W \, ,
		$$
		
		\smallskip
		
		\item[(b)] $C^{\infty} \left( \mathcal{F}\right)  = \{ X \in C^{\infty} \left(TM\right) \, , \ X_x \in
		\mathcal{F}_x \quad \forall \, x \in M \}$ is a Lie algebra.
		
		\smallskip

	\end{enumerate}

\end{definition}


\end{frame}
\begin{frame}
A \textit{foliation} of $M$ is given by an integrable subbundle $\mathcal{F}$ of $TM$.  	A pair $\left(M, \sF\right)$ is said to be a \textit{foliated manifold}.
The \textit{leaves} of the foliation $\left(M , \mathcal F\right)$ are the maximal connected
submanifolds $L$ of $M$ with $T_x (L) = \mathcal{F}_x $, $\forall \, x \in L$,
and the partition of $M$ in leaves $$M = \cup
L_{\alpha}\,,\quad\alpha \in X$$ is characterized geometrically by
its ``local triviality'': every point $x \in M$ has a neighborhood
$\mathcal U$ and a system of local coordinates
$(x^j)_{j = 1 , \ldots , \dim V}$ called
{\it foliation charts}, so
that the partition of $\mathcal U$ in connected components of
leaves corresponds to the partition of 
\begin{equation*}
	{\mathbb
		R}^{\dim M} = {\mathbb R}^{\dim \mathcal F} \times {\mathbb R}^{\text{codim}
		\, \mathcal F}
\end{equation*}
in the parallel affine subspaces 
$
{\mathbb R}^{\dim \mathcal F}
\times {\rm pt}$.
For simplicity of this speech we suppose that $M$ is compact, and all leaves are simply connected. These conditions are not necessary in general.
\end{frame}

\begin{frame}
A \textit{holonomy groupoid} $\G\left(M, \sF\right)$ of a foliated manifold $\left(M, \sF\right)$ is given by
$$
\mathcal G\left(M, \sF\right)\bydef \left\{\left.\gamma = (x, y)\in M\times M~ \right|\text{ $y$ and $x$ are on the same leaf } \right\}.
$$
For any  $\gamma = (x, y)\in \mathcal G\left(M, \sF\right)$ denote by
$$
s\left(\gamma\right)\bydef x, \quad  r\left(\gamma\right)\bydef y, \quad \gamma: x \to y.
$$
$\mathcal G\left(M, \sF\right)$ is a smooth manifold such that
$$
\dim \mathcal G\left(M, \sF\right) = \dim \,M + \dim \,\mathcal F.
$$
 There are the following groupoid operations
$$
\left(x, y \right) \circ \left(y, z \right) \bydef \left(x, z \right), \quad \left(x, y \right)^{-1} \bydef \left(y, z\right). 
$$

\end{frame}
\begin{frame}
For
$x\in M$ one the space of \textit{half-densities} $\Omega_x^{1/2}$ is a one dimensional complex
vector space of maps from the exterior power $\wedge^k \,  \mathcal{F}_x$, $k =
\dim F$, to ${\mathbb C}$ such that
$$
\rho \, (\lambda \, v) =\sqrt{\la}\, \rho \, (v)
\qquad \forall \, v \in \wedge^k \,  \mathcal{F}_x \, , \quad \forall \,
\lambda \in {\mathbb R}_+ \, .
$$
Then, for $\gamma \in\mathcal G\left(M, \sF\right)$, denote by $\Omega_{\gamma}^{1/2}\bydef\Omega_{r\left(\gamma\right)}^{1/2} \otimes_\C
\Omega_{r\left(\gamma\right)}^{1/2}$. One has a one dimensional bundle $
\Omega_{\mathcal G\left(M, \sF\right)}^{1/2}\bydef\, r^*(\Omega_M^{1/2})\otimes s^*(\Omega_M^{1/2})\,
$ over $\mathcal G\left(M, \sF\right)$, The bundle has the natural smooth structure. The  space  $C^{\infty}\left(\mathcal G\left(M, \sF\right) , \Omega^{1/2}_{\mathcal G\left(M, \sF\right)}\right)$ of smooth sections of $
\Omega_{\mathcal G\left(M, \sF\right)}^{1/2}$ is isomorphic to  $C^\infty(\mathcal G)$, however an isomorphism is not natural.
For any $f, g \in C^{\infty}\left(\mathcal G\left(M, \sF\right) , \Omega^{1/2}_{\mathcal G\left(M, \sF\right)}\right)$  the
product $f* g$ and involution $*$ are defined by the following way
\be\nonumber
\begin{split}
(f * g) (\gamma) \bydef \int_{\gamma_1 \circ \gamma_2 = \gamma}
f(\gamma_1) \, g(\gamma_2)\in C^{\infty}\left(\mathcal G\left(M, \sF\right) , \Omega^{1/2}_{\mathcal G\left(M, \sF\right)}\right)  \,,\\
f^*\left(\gamma \right) \bydef \overline f \left( \gamma^{-1}\right) \in C^{\infty}\left(\mathcal G\left(M, \sF\right) , \Omega^{1/2}_{\mathcal G\left(M, \sF\right)}\right).
\end{split}
\ee
\end{frame}
\begin{frame}
For all $x \in M$ denote by
$
\mathcal G_x \bydef \left\{\left.\gamma\in\mathcal G\left(M, \sF\right)~\right| ~s(\gamma) = x \right\}.
$
If $\xi, \eta$ are square integrable half-densities on $\mathcal G_x$ then one can define a scalar $\C$-valued product
$
\left(\xi, \eta \right) = \int_{\gamma\in \mathcal G_x} \overline\xi\left(\gamma^{-1}\right)\eta\left(\gamma\right)
$.
So the space of square integrable half-densities on $\mathcal G_x$ is a Hilbert space, we denote it by $L^2\left( \mathcal G_x\right)$. For any $x \in M$ there is a representation $\rho_x: C^{\infty} (\mathcal G , \Omega^{1/2})\to B\left(L^2\left( \mathcal G_x\right)\right)$ given by
\begin{equation}\nonumber
	(\rho_x (f) \, \xi) \, (\gamma) = \int_{\gamma_1 \circ \gamma_2 =
		\gamma} f(\gamma_1) \, \xi (\gamma_2) \qquad \forall \, \xi \in L^2
	(\mathcal G_x).\
\end{equation}
If $\sH \bydef \bigoplus_{x \in M}L^2\left( \mathcal G_x\right)$ is the direct sum of Hilbert spaces then there is a faithful representation 
$$
\rho \bydef \bigoplus_{x \in M} \rho_ x :C^{\infty} (\mathcal G , \Omega^{1/2})\hookto B\left(\sH \right). 
$$
The $C^*$-norm completion of $C^{\infty} (\mathcal G , \Omega^{1/2})$ in $B\left(\sH \right)$ is said to be the \textit{reduced algebra of the foliation} and denoted by $C^*_r(M, \sF)$.
\end{frame}
\subsection{Coverings of foliations}

\begin{frame}
Let $\left(M, \sF\right)$ be a foliated manifold and let $p: \widetilde M\to M$ be a transitive finite-fold covering. The tangent bundle $T\widetilde M$ of $\widetilde M$ is the inverse image of $TM$. If $\widetilde\sF$ is the inverse image of $\sF$ then $\widetilde\sF$ is integrable so $\left(\widetilde M, \widetilde \sF\right)$ is a foliated manifold. Moreover there is a finite-fold covering $\G\left(\widetilde M, \widetilde \sF\right)\to\G \left( M,  \sF\right)$ of groupoids. Similarly $\Omega_{\mathcal G\left(\widetilde M, \widetilde\sF\right)}^{1/2}$ is the inverse image of $\Omega_{\mathcal G\left( M, \sF\right)}^{1/2}$. Any section of $C^{\infty} \left(\mathcal G\left( M, \sF\right) , \Omega_{\mathcal G\left( M, \sF\right)}^{1/2}\right)$ uniquely  defines its lift in $C^{\infty} \left(\mathcal G\left(\widetilde M, \widetilde\sF\right) , \Omega_{\mathcal G\left(\widetilde M, \widetilde\sF\right)}^{1/2}\right)$, i.e. there is a $\C$-linear map
\be\label{cinf_eqn}
C^{\infty} \left(\mathcal G\left( M, \sF\right) , \Omega_{\mathcal G\left( M, \sF\right)}^{1/2}\right)\hookto C^{\infty} \left(\mathcal G\left(\widetilde M, \widetilde\sF\right) , \Omega_{\mathcal G\left(\widetilde M, \widetilde\sF\right)}^{1/2}\right)
\ee
which is indeed *-homomorphism. The map \eqref{cinf_eqn} yield the \textit{induced by the covering  p $*$-homomorphism of $C^*$-algebras of foliations}
$
C^*_r\left(p \right) :C^*_r \left( M, {\sF}\right)\hookto C^*_r \left(\widetilde M, \widetilde{\sF}\right).
$
\end{frame}
\subsection{Multiplies}
\begin{frame}
	The space $C^{\infty} (\mathcal G , \Omega^{1/2})$ is $C^{\infty} \left(M\right)$-bimodule such that
	\be\nonumber
	\begin{split}
		f\cdot \left(\omega_x \otimes \omega_y \right) \bydef f\left(x \right)\omega_x \otimes \omega_y \quad f \in C^{\infty} \left(M\right),~\omega_x \in \Om^{1/2}_x, ~ \omega_y \in \Om^{1/2}_y, \\
		\left(\omega_x \otimes \omega_y \right) \cdot f \bydef \omega_x \otimes f\left(y \right)\omega_y.
	\end{split}
	\ee
	On the other hand for all $x \in M$ there is a representation $C^{\infty} \left(M\right)\to L^2\left( \mathcal G_x\right)$ given by
	$$
	f \mapsto \left( \xi \mapsto f\left( x\right)\xi \right) \quad  \quad f \in C^{\infty} \left(M\right),~\xi \in L^2\left( \mathcal G_x\right).
	$$
	One has the natural faithful representation
	$
	\phi:  C^{\infty} \left(M\right)\hookto B\left(\sH \right)
	$
	such that
	\be\nonumber
	\begin{split}
		\phi\left( f\right) \rho\left( a\right) = \rho\left(f\cdot a \right),  f \in  C^{\infty} \left(M\right), ~a \in C^{\infty} (\mathcal G , \Omega^{1/2})\\
		\rho\left( a\right) \phi\left( f\right)= \rho\left(a\cdot f \right).
	\end{split}
	\ee
	From the above construction it follows that $C^{\infty} \left(M\right)\subset M\left( C^*_r(M, \sF)\right)$. 
\end{frame}
\section{Application of general theory}
\begin{frame}
Denote by $A^*_r\left(M, \sF \right)  \subset M\left( C^*_r(M, \sF)\right)$ the $C^*$-subalgebra of $M\left( C^*_r(M, \sF)\right)$ generated by a union $C^*_r\left(M, \sF \right)\cup C^{\infty} \left(M\right)$. $C^*_r\left(M, \sF \right)$ is an essential ideal of $A^*_r\left(M, \sF \right)$. If $p: \widetilde M \to M$ is a transitive covering and $\widetilde\sF$ is the $p$-lift of $\sF$ then using a combination of inclusions $C^*_r\left(p \right): C^*_r \left( M, {\sF}\right)\hookto C^*_r \left(\widetilde M, \widetilde{\sF}\right)$ and $C^{\infty} \left(M\right) \hookto C^{\infty} \left(\widetilde M\right)$ one can construct the invective *-homomorphsim $\widetilde \pi: A^*_r \left( M, {\sF}\right)\hookto A^*_r \left(\widetilde M, \widetilde{\sF}\right)$ such that $\left.\widetilde \pi\right|_{  C^*_r \left( M, {\sF}\right)}= C^*_r \left(p\right)$. The action $G\left(\left.\widetilde M\right| M\right) \times \widetilde M\to \widetilde M$ naturally induces the action $G\left(\left.\widetilde M\right| M\right) \times  A^*_r \left(\widetilde M, \widetilde{\sF}\right)\to  A^*_r \left(\widetilde M, \widetilde{\sF}\right)$ such that $G\left(\left.\widetilde M\right| M\right)   C^*_r \left(\widetilde M, \widetilde{\sF}\right)= C^*_r \left(\widetilde M, \widetilde{\sF}\right)$. Since $\widetilde M$ is compact there is a finite set $\left\{\widetilde a_1, ..., \widetilde a_m\right\}\subset C^{\infty} \left(\widetilde M\right)_+$ such that
\be\nonumber
\begin{split}
\widetilde a_1 + ... + \widetilde a_n = 1_{C^{\infty} \left(\widetilde M\right)}= 1_{A^*_r \left(\widetilde M, \widetilde{\sF}\right)},\\
\widetilde a_j \left(g \widetilde a_j \right) = 0; \text{ for any nontrivial }g\in G\left(\left.\widetilde M\right| M\right).
\end{split}
\ee

\end{frame}

\begin{frame}
If $\widetilde e_j \bydef \sqrt{\widetilde a_j}\in C^{\infty} \left(\mathcal G\left( M, \sF\right) , \Omega_{\mathcal G\left( M, \sF\right)}^{1/2}\right)$ then from the above construction it follows that
\be\label{foli_eqn}
\sum_{j = 1}^m \widetilde e_j (g \widetilde e_j)= \begin{cases}
	1_{A^*_r \left(\widetilde M, \widetilde{\sF}\right)} &  g\in G\left(\left.\widetilde M\right| M\right)\text{ is trivial}\\
		0 &  g\in G\left(\left.\widetilde M\right| M\right)\text{ is not  trivial}.
\end{cases}
\ee
From the equation \eqref{foli_eqn} it follows that
$$
\widetilde a \in A^*_r \left(\widetilde M, \widetilde{\sF}\right)\Rightarrow \widetilde a = \sum_{j=1}^m \widetilde e_j a_j; \quad a_j= \sum_{g \in G\left(\left.\widetilde M\right| M\right)} g\left(  \widetilde e_j  \widetilde a\right)\in A^*_r \left( M, {\sF}\right), 
$$
i.e. $A^*_r \left(\widetilde M, \widetilde{\sF}\right)$ is an $A^*_r \left( M, {\sF}\right)$-module generated by the finite set $\left\{\widetilde e_1, ..., \widetilde e_m\right\}\subset A^*_r \left(\widetilde M, \widetilde{\sF}\right)$
\end{frame}
\begin{frame}
\begin{theorem}
If we consider the above situation then the quadruple
$$
\left(C^*_r \left(\widetilde M, \widetilde{\sF}\right), C^*_r \left(\widetilde M, \widetilde{\sF}\right), G\left(\left.\widetilde M\right| M\right), C^*_r\left(p \right) \right)
$$
is a noncommutative finite-fold covering with unitization.
\end{theorem}
Firstly we prove that
	$$
	\left(A^*_r \left(\widetilde M, \widetilde{\sF}\right), A^*_r \left(\widetilde M, \widetilde{\sF}\right), G\left(\left.\widetilde M\right| M\right), \widetilde\pi \right)
	$$
is an unital noncommutative finite-fold covering, i.e.
			\begin{enumerate}
		\item[(a)] The group  
		$$
		G \bydef \left\{ \left.g \in \Aut\left(A^*_r \left(\widetilde M, \widetilde{\sF}\right) \right)~\right|~ ga = a;~~\forall a \in A^*_r \left( M, {\sF}\right)\right\}
		$$
		is finite.
		\item[(b)] 	$
		A^*_r\left( M, {\sF}\right) \cong A^*_r \left(\widetilde M, \widetilde{\sF}\right)^G$
			\item[(c)] $A^*_r \left(\widetilde M, \widetilde{\sF}\right)$ is a finitely generated $A^*_r \left( M, {\sF}\right)$-module.
	\end{enumerate}
\end{frame}
\begin{frame}
	However from our construction it follows that $G \cong G\left(\left.\widetilde M\right| M\right)$.
Condition (b) can be deduced from $C\left(M\right)\cong C\left(\widetilde M\right)^{G\left(\left.\widetilde M\right| M\right)}$and $C^*_r\left( M, {\sF}\right) \cong C^*_r \left(\widetilde M, \widetilde{\sF}\right)^G$. Condition (c) is already proven. Since both $C^*_r\left( M, {\sF}\right)$  and $C^*_r \left(\widetilde M, \widetilde{\sF}\right)$ are essential ideals of $A^*_r\left( M, {\sF}\right)$  and $A^*_r \left(\widetilde M, \widetilde{\sF}\right)$ respectively, thus $$
\left(C^*_r \left(\widetilde M, \widetilde{\sF}\right), C^*_r \left(\widetilde M, \widetilde{\sF}\right), G\left(\left.\widetilde M\right| M\right), C^*_r\left(p \right) \right)
$$
is a noncommutative finite-fold covering with unitization.

\end{frame}

\end{document}























