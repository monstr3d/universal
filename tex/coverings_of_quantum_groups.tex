% !TeX spellcheck = en_US
\documentclass[10]{article}
\usepackage{amsfonts,amssymb,amsmath,amsthm,cite}
\usepackage{graphicx}

%% \usepackage[T1]{fontenc}
%% \usepackage[francais]{babel}
\usepackage[applemac]{inputenc}
\usepackage{amsmath,amssymb,amsthm, hyperref, euscript}
\usepackage[matrix,arrow,curve]{xy}
\usepackage{graphicx}
\usepackage{tabularx}
\usepackage{float}
\usepackage{hyperref}
\usepackage{tikz}
\usepackage{slashed}
\usepackage{mathrsfs}
%\usepackage{mathtools}

\usepackage{hyperref}
\usepackage{amsfonts,amssymb,amsmath,amsthm,cite}
%\usepackage{graphicx}
\usepackage[toc,page]{appendix}
\usepackage{nicefrac}
%% \usepackage[francais]{babel}
\usepackage[applemac]{inputenc}
\usepackage{amssymb, euscript}
\usepackage[matrix,arrow,curve]{xy}
\usepackage{graphicx}
\usepackage{tabularx}
\usepackage{float}
\usepackage{tikz}
\usepackage{slashed}
\usepackage{mathrsfs}
\usepackage{multirow}
\usepackage{rotating}

%\usepackage{mathtools}

\usetikzlibrary{matrix}
\usetikzlibrary{cd}

\usepackage{siunitx}

\usepackage{lmodern}
\usepackage[T1]{fontenc}
\usepackage[babel=true]{microtype}


\usepackage{amsfonts,cite}
\usepackage{graphicx}

%% \usepackage[francais]{babel}
\usepackage[applemac]{inputenc}


\usepackage[sc]{mathpazo}
\usepackage{environ}


\usetikzlibrary{matrix}

\usepackage[T1]{fontenc}
\usepackage{amsfonts,cite}
\usepackage{graphicx}

%% \usepackage[T1]{fontenc}
%% \usepackage[francais]{babel}
\usepackage[applemac]{inputenc}

	\newcommand{\bs}{\begin{split}}
	\newcommand{\es}{\end{split}}

\newcommand{\bea}{\begin{eqnarray}}
	\newcommand{\eea}{\end{eqnarray}}
\newcommand{\bean}{\begin{eqnarray*}}
	\newcommand{\eean}{\end{eqnarray*}}
	
\newcommand{\be}{\begin{equation}}
\newcommand{\ee}{\end{equation}}

\usepackage[sc]{mathpazo}
\linespread{1.05}         % Palatino needs more leading (space between lines)


%\usepackage[usenames]{color}



\DeclareFontFamily{T1}{pzc}{}
\DeclareFontShape{T1}{pzc}{m}{it}{1.8 <-> pzcmi8t}{}
\DeclareMathAlphabet{\mathpzc}{T1}{pzc}{m}{it}
% the command for it is \mathpzc

\textwidth=140mm


% % % % % % % % % % % % % % % % % % % %
\theoremstyle{plain}
\newtheorem{prop}{Proposition}[section]
\newtheorem{prdf}[prop]{Proposition and Definition}
\newtheorem{lem}[prop]{Lemma}%[section]
\newtheorem{cor}[prop]{Corollary}%[section]
\newtheorem{thm}[prop]{Theorem}%[section]
\newtheorem{theorem}[prop]{Theorem}
\newtheorem{lemma}[prop]{Lemma}
\newtheorem{proposition}[prop]{Proposition}
\newtheorem{corollary}[prop]{Corollary}
\newtheorem{statement}[prop]{Statement}

\theoremstyle{definition}
\newtheorem{defn}[prop]{Definition}%[section]
\newtheorem{cordefn}[prop]{Corollary and Definition}%[section]
\newtheorem{empt}[prop]{}%[section]
\newtheorem{exm}[prop]{Example}%[section]
\newtheorem{rem}[prop]{Remark}%[section]
\newtheorem{prob}[prop]{Problem}
\newtheorem{conj}{Conjecture}       %% Hypothesis 1
\newtheorem{cond}{Condition}        %% Condition 1
%\newtheorem{axiom}[thm]{Axiom}           %% Axiom 1 modified
\newtheorem{fact}[prop]{Fact}
\newtheorem{ques}{Question}         %% Question 1
\newtheorem{answ}{Answer}           %% Answer 1
\newtheorem{notn}{Notation}        %% Notations are not numbered

\theoremstyle{definition}
\newtheorem{notation}[prop]{Notation}
\newtheorem{definition}[prop]{Definition}
\newtheorem{example}[prop]{Example}
\newtheorem{exercise}[prop]{Exercise}
\newtheorem{conclusion}[prop]{Conclusion}
\newtheorem{conjecture}[prop]{Conjecture}
\newtheorem{criterion}[prop]{Criterion}
\newtheorem{summary}[prop]{Summary}
\newtheorem{axiom}[prop]{Axiom}
\newtheorem{problem}[prop]{Problem}
%\theoremstyle{remark}
\newtheorem{remark}[prop]{Remark}
\DeclareMathOperator{\Dom}{\mathrm{Dom}}              %% domain of an operator
\newcommand{\Dslash}{{D\mkern-11.5mu/\,}}    %% Dirac operator

\newcommand\myeq{\stackrel{\mathclap{\normalfont\mbox{def}}}{=}}
\newcommand{\nor}[1]{\left\Vert #1\right\Vert}    %\nor{x}=||x||
\newcommand{\vertiii}[1]{{\left\vert\kern-0.25ex\left\vert\kern-0.25ex\left\vert #1
    \right\vert\kern-0.25ex\right\vert\kern-0.25ex\right\vert}}
\newcommand{\Ga}{\Gamma}                     %% short for  \Gamma
\newcommand{\Coo}{C^\infty}                  %% smooth functions
% % % % % % % % % % % % % % % % % % % %


\usepackage[sc]{mathpazo}
\linespread{1.05}         % Palatino needs more leading (space between lines)

\newbox\ncintdbox \newbox\ncinttbox %% noncommutative integral symbols
\setbox0=\hbox{$-$} \setbox2=\hbox{$\displaystyle\int$}
\setbox\ncintdbox=\hbox{\rlap{\hbox
    to \wd2{\hskip-.125em \box2\relax\hfil}}\box0\kern.1em}
\setbox0=\hbox{$\vcenter{\hrule width 4pt}$}
\setbox2=\hbox{$\textstyle\int$} \setbox\ncinttbox=\hbox{\rlap{\hbox
    to \wd2{\hskip-.175em \box2\relax\hfil}}\box0\kern.1em}

\newcommand{\ncint}{\mathop{\mathchoice{\copy\ncintdbox}%
           {\copy\ncinttbox}{\copy\ncinttbox}%
           {\copy\ncinttbox}}\nolimits}  %% NC integral
           
%%% Repeated relations:
\newcommand{\xyx}{\times\cdots\times}      %% repeated product
\newcommand{\opyop}{\oplus\cdots\oplus}    %% repeated direct sum
\newcommand{\oxyox}{\otimes\cdots\otimes}  %% repeated tensor product
\newcommand{\wyw}{\wedge\cdots\wedge}      %% repeated exterior product
\newcommand{\subysub}{\subset\hdots\subset}      %% repeated subset
\newcommand{\supysup}{\supset\hdots\supset}      %% repeated supset

%%% Roman letters:
\newcommand{\id}{\mathrm{id}}                %% identity map
\newcommand{\Id}{\mathrm{Id}}                %% identity map
\newcommand{\pt}{\mathrm{pt}}                %% a point
\newcommand{\const}{\mathrm{const}}          %% a constant
\newcommand{\codim}{\mathrm{codim}}          %% codimension
\newcommand{\cyc}{\mathrm{cyclic}}  %% cyclic sum
\renewcommand{\d}{\mathrm{d}}       %% commutative differential
\newcommand{\dR}{\mathrm{dR}}       %% de~Rham cohomology
\newcommand{\proj}{\mathrm{proj}}                %% a projection
\newcommand{\lb}{\label}

           
\newcommand{\A}{\mathcal{A}}                 %% an algebra
\renewcommand{\a}{\alpha}                    %% short for  \alphapha
\DeclareMathOperator{\ad}{ad}                %% infml adjoint repn
\newcommand{\as}{\quad\mbox{as}\enspace}     %% `as' with spacing
\newcommand{\Aun}{\widetilde{\mathcal{A}}}   %% unital algebra
\newcommand{\B}{\mathcal{B}}                 %% space of distributions
\newcommand{\E}{\mathcal{E}}                 %% space of distributions
\renewcommand{\b}{\beta}                     %% short for \beta
\newcommand{\braCket}[3]{\langle#1\mathbin|#2\mathbin|#3\rangle}
\newcommand{\braket}[2]{\langle#1\mathbin|#2\rangle} %% <w|z>
\newcommand{\C}{\mathbb{C}}                  %% complex numbers
\newcommand{\CC}{\mathcal{C}}                %% space of distributions
\newcommand{\cc}{\mathbf{c}}                 %% Hochschild cycle
\DeclareMathOperator{\Cl}{C\ell}             %% Clifford algebra
\newcommand{\F}{\mathcal{F}}                 %% space of test functions
\newcommand{\G}{\mathcal{G}}                 %% Moyal L^2-filtration
\renewcommand{\H}{\mathcal{H}}               %% Hilbert space
\newcommand{\half}{\tfrac{1}{2}}             %% small fraction  1/2
\newcommand{\hh}{\mathcal{H}}                %% Hilbert space
\newcommand{\hookto}{\hookrightarrow}        %% abbreviation
\newcommand{\Ht}{{\widetilde{\mathcal{H}}}}  %% Hilbert space of forms
\newcommand{\I}{\mathcal{I}}                 %% tracelike functions
\DeclareMathOperator{\Junk}{Junk}            %% the junk DGA ideal
\newcommand{\K}{\mathcal{K}}                 %% compact operators
\newcommand{\ket}[1]{|#1\rangle}             %% ket vector
\newcommand{\ketbra}[2]{|#1\rangle\langle#2|} %% rank one operator
\renewcommand{\L}{\mathcal{L}}               %% operator algebra
\newcommand{\La}{\Lambda}                    %% short for \Lambda
\newcommand{\la}{\lambda}                    %% short for \lambda
\newcommand{\lf}{L_f^\theta}                 %% left mult operator
\newcommand{\M}{\mathcal{M}}                 %% Moyal multplr algebra
\newcommand{\mm}{\mathcal{M}^\theta}
\newcommand{\mop}{{\mathchoice{\mathbin{\;|\;ar_{_\theta}}}
            {\mathbin{\;|\;ar_{_\theta}}}           %% Moyal
            {{\;|\;ar_\theta}}{{\;|\;ar_\theta}}}}    %% product
\newcommand{\N}{\mathbb{N}}                  %% nonnegative integers
\newcommand{\NN}{\mathcal{N}}                %% a Moyal algebra
\newcommand{\nb}{\nabla}                     %% gradient
\newcommand{\Oh}{\mathcal{O}}                %% comm multiplier alg
\newcommand{\om}{\omega}                     %% short for \omega
\newcommand{\opp}{{\mathrm{op}}}             %% opposite algebra
\newcommand{\ox}{\otimes}                    %% tensor product
\newcommand{\eps}{\varepsilon}                    %% tensor product
\newcommand{\otimesyox}{\otimes\cdots\otimes}    %% repeated tensor product
\newcommand{\pa}{\partial}                   %% short for \partial
\newcommand{\pd}[2]{\frac{\partial#1}{\partial#2}}%% partial derivative
\newcommand{\piso}[1]{\lfloor#1\rfloor}      %% integer part
\newcommand{\PsiDO}{\Psi\mathrm{DO}}         %% pseudodiffl operators
\newcommand{\Q}{\mathbb{Q}}                  %% rational numbers
\newcommand{\R}{\mathbb{R}}                  %% real numbers
\newcommand{\rdl}{R_\Dslash(\lambda)}        %% resolvent
\newcommand{\roundbraket}[2]{(#1\mathbin|#2)} %% (w|z)
\newcommand{\row}[3]{{#1}_{#2},\dots,{#1}_{#3}} %% list: a_1,...,a_n
\newcommand{\sepword}[1]{\quad\mbox{#1}\quad} %% well-spaced words
\newcommand{\set}[1]{\{\,#1\,\}}             %% set notation
\newcommand{\Sf}{\mathbb{S}}                 %% sphere
\newcommand{\uhor}[1]{\Omega^1_{hor}#1}
\newcommand{\sco}[1]{{\sp{(#1)}}}
\newcommand{\sw}[1]{{\sb{(#1)}}}
\DeclareMathOperator{\spec}{sp}              %% spectrum
\renewcommand{\SS}{\mathcal{S}}              %% Schwartz space
\newcommand{\sss}{\mathcal{S}}               %% Schwartz space
\DeclareMathOperator{\supp}{\mathfrak{supp}}         %% support
\DeclareMathOperator{\cl}{\mathfrak{cl}}         %% support
\newcommand{\T}{\mathbb{T}}                  %% circle as a group
\renewcommand{\th}{\theta}                   %% short for \theta
\newcommand{\thalf}{\tfrac{1}{2}}            %% small* fraction 1/2
\newcommand{\tihalf}{\tfrac{i}{2}}           %% small* fraction i/2
\newcommand{\tpi}{{\tilde\pi}}               %% extended representation
\DeclareMathOperator{\Tr}{Tr}                %% trace of operator
\DeclareMathOperator{\tr}{tr}                %% trace of matrix
\newcommand{\del}{\partial}                  %% short for  \partial
\DeclareMathOperator{\tsum}{{\textstyle\sum}} %% small sum in display
\newcommand{\V}{\mathcal{V}}                 %% test function space
\newcommand{\W}{\mathcal{W}}                 %% test function space
\newcommand{\vac}{\ket{0}}                   %% vacuum ket vector
\newcommand{\vf}{\varphi}                    %% scalar field
\newcommand{\w}{\wedge}                      %% exterior product
\DeclareMathOperator{\wres}{wres}            %% density of Wresidue
\newcommand{\x}{\times}                      %% cross
\newcommand{\Z}{\mathbb{Z}}                  %% integers
\newcommand{\7}{\dagger}                     %% short for + symbol
\newcommand{\8}{\bullet}                     %% anonymous degree
\renewcommand{\.}{\cdot}                     %% anonymous variable
\renewcommand{\:}{\colon}                    %% colon in  f: A -> B

\newcommand{\sA}{\mathcal{A}}       %%
\newcommand{\sB}{\mathcal{B}}       %%
\newcommand{\sC}{\mathcal{C}}       %%
\newcommand{\sD}{\mathcal{D}}       %%
\newcommand{\sE}{\mathcal{E}}       %%
\newcommand{\sF}{\mathcal{F}}       %%
\newcommand{\sG}{\mathcal{G}}       %%
\newcommand{\sH}{\mathcal{H}}       %%
\newcommand{\sI}{\mathcal{I}}       %%
\newcommand{\sJ}{\mathcal{J}}       %%
\newcommand{\sK}{\mathcal{K}}       %%
\newcommand{\sL}{\mathcal{L}}       %%
\newcommand{\sM}{\mathcal{M}}       %%
\newcommand{\sN}{\mathcal{N}}       %%
\newcommand{\sO}{\mathcal{O}}       %%
\newcommand{\sP}{\mathcal{P}}       %%
\newcommand{\sQ}{\mathcal{Q}}       %%
\newcommand{\sR}{\mathcal{R}}       %%
\newcommand{\sS}{\mathcal{S}}       %%
\newcommand{\sT}{\mathcal{T}}       %%
\newcommand{\sU}{\mathcal{U}}       %%
\newcommand{\sV}{\mathcal{V}}       %%
\newcommand{\sX}{\mathcal{X}}       %%
\newcommand{\sZ}{\mathcal{Z}}       %%

\newcommand{\Om}{\Omega}       %%


\DeclareMathOperator{\ptr}{ptr}     %% Poisson trace
\DeclareMathOperator{\Trw}{Tr_\omega} %% Dixmier trace
\DeclareMathOperator{\vol}{Vol}     %% total volume
\DeclareMathOperator{\Vol}{Vol}     %% total volume
\DeclareMathOperator{\Area}{Area}   %% area of a surface
\DeclareMathOperator{\Wres}{Wres}   %% (Wodzicki) residue

\newcommand{\dd}[1]{\frac{\partial}{\partial#1}}   %% partial derivation
\newcommand{\ddt}[1]{\frac{d}{d#1}}                %% derivative
\newcommand{\inv}[1]{\frac{1}{#1}}                 %% inverse
\newcommand{\sfrac}[2]{{\scriptstyle\frac{#1}{#2}}} %% tiny fraction

\newcommand{\bA}{\mathbb{A}}       %%
\newcommand{\bB}{\mathbb{B}}       %%
\newcommand{\bC}{\mathbb{C}}       %%
\newcommand{\bCP}{\mathbb{C}P}     %%
\newcommand{\bD}{\mathbb{D}}       %%
\newcommand{\bE}{\mathbb{E}}       %%
\newcommand{\bF}{\mathbb{F}}       %%
\newcommand{\bG}{\mathbb{G}}       %%
\newcommand{\bH}{\mathbb{H}}       %%
\newcommand{\bHP}{\mathbb{H}P}     %%
\newcommand{\bI}{\mathbb{I}}       %%
\newcommand{\bJ}{\mathbb{J}}       %%
\newcommand{\bK}{\mathbb{K}}       %%
\newcommand{\bL}{\mathbb{L}}       %%
\newcommand{\bM}{\mathbb{M}}       %%
\newcommand{\bN}{\mathbb{N}}       %%
\newcommand{\bO}{\mathbb{O}}       %%
\newcommand{\bOP}{\mathbb{O}P}     %%
\newcommand{\bP}{\mathbb{P}}       %%
\newcommand{\bQ}{\mathbb{Q}}       %%
\newcommand{\bR}{\mathbb{R}}       %%
\newcommand{\bRP}{\mathbb{R}P}     %%
\newcommand{\bS}{\mathbb{S}}       %%
\newcommand{\bT}{\mathbb{T}}       %%
\newcommand{\bU}{\mathbb{U}}       %%
\newcommand{\bV}{\mathbb{V}}       %%
\newcommand{\bX}{\mathbb{X}}       %%
\newcommand{\bY}{\mathbb{Y}}       %%
\newcommand{\bZ}{\mathbb{Z}}       %%



\newcommand{\al}{\alpha}          %% short for  \alpha
\newcommand{\bt}{\beta}           %% short for  \beta
\newcommand{\Dl}{\Delta}          %% short for  \Delta
\newcommand{\dl}{\delta}          %% short for  \delta
\newcommand{\ga}{\gamma}          %% short for  \gamma
\newcommand{\ka}{\kappa}          %% short for  \kappa
\newcommand{\sg}{\sigma}          %% short for  \sigma
\newcommand{\Sg}{\Sigma}          %% short for  \Sigma
\newcommand{\Th}{\Theta}          %% short for  \Theta
\renewcommand{\th}{\theta}        %% short for  \theta
\newcommand{\vth}{\vartheta}      %% short for  \vartheta
\newcommand{\ze}{\zeta}           %% short for  \zeta

\DeclareMathOperator{\ord}{ord}     %% order of a PsiDO
\DeclareMathOperator{\rank}{rank}   %% rank of a vector bundle
\DeclareMathOperator{\sign}{sign}   %%
\DeclareMathOperator{\sgn}{sgn}   %%
\DeclareMathOperator{\chr}{char}   %%
\DeclareMathOperator{\ev}{ev}       %% evaluation
	\newcommand{\bydef}{\stackrel{\mathrm{def}}{=}}          %% 
\newcommand{\defeq}{\stackrel{\mathrm{def}}{=}}   



\newcommand{\Op}{\mathbf{Op}}
\newcommand{\As}{\mathbf{As}}
\newcommand{\Com}{\mathbf{Com}}
\newcommand{\LLie}{\mathbf{Lie}}
\newcommand{\Leib}{\mathbf{Leib}}
\newcommand{\Zinb}{\mathbf{Zinb}}
\newcommand{\Poiss}{\mathbf{Poiss}}

\newcommand{\gX}{\mathfrak{X}}      %% vector fields
\newcommand{\sol}{\mathfrak{so}}    %% special orthogonal Lie algebra
\newcommand{\gm}{\mathfrak{m}}      %% maximal ideal


\DeclareMathOperator{\Res}{Res}
\DeclareMathOperator{\NCRes}{NCRes}
\DeclareMathOperator{\Ind}{Ind}
%% co/homology theories
\DeclareMathOperator{\rH}{H}        %% any co/homology
\DeclareMathOperator{\rC}{C}        %%  any co/chains
\DeclareMathOperator{\rZ}{Z}        %% cycles
\DeclareMathOperator{\rB}{B}        %% boundaries
\DeclareMathOperator{\rF}{F}        %% filtration
\DeclareMathOperator{\Gr}{gr}        %% associated graded object
\DeclareMathOperator{\rHc}{H_{\mathrm{c}}}   %% co/homology with compact support
\DeclareMathOperator{\drH}{H_{\mathrm{dR}}}  %% de Rham co/homology
\DeclareMathOperator{\cechH}{\check{H}}    %% Cech co/homology
\DeclareMathOperator{\rK}{K}        %% K-groups
\DeclareMathOperator{\rKO}{KO}        %% real K-groups
\DeclareMathOperator{\rKU}{KU}        %% unitary K-groups
\DeclareMathOperator{\rKSp}{KSp}        %% symplectic K-groups
\DeclareMathOperator{\rR}{R}        %% representation ring
\DeclareMathOperator{\rI}{I}        %% augmentation ideal
\DeclareMathOperator{\HH}{HH}       %% Hochschild co/homology
\DeclareMathOperator{\HC}{HC}       %% cyclic co/homology
\DeclareMathOperator{\HP}{HP}       %% periodic cyclic co/homology
\DeclareMathOperator{\HN}{HN}       %% negative cyclic co/homology
\DeclareMathOperator{\HL}{HL}       %% Leibniz co/homology
\DeclareMathOperator{\KK}{KK}       %% KK-theory
\DeclareMathOperator{\KKK}{\mathbf{KK}}       %% KK-theory as a category
\DeclareMathOperator{\Ell}{Ell}       %% Abstract elliptic operators
\DeclareMathOperator{\cd}{cd}       %% cohomological dimension
\DeclareMathOperator{\spn}{span}       %% span
\DeclareMathOperator{\linspan}{span} %% linear span (can't use \span!)



\newcommand{\twobytwo}[4]{\begin{pmatrix} #1 & #2 \\ #3 & #4 \end{pmatrix}}
\newcommand{\CGq}[6]{C_q\!\begin{pmatrix}#1&#2&#3\\#4&#5&#6\end{pmatrix}}
                                    %% q-Clebsch--Gordan coefficients
\newcommand{\cz}{{\bullet}}         %% anonymous degree
\newcommand{\nic}{{\vphantom{\dagger}}} %% invisible dagger
\newcommand{\ep}{{\dagger}}         %% abbreviation for + symbol
\newcommand{\downto}{\downarrow}    %% right hand limit
\newcommand{\isom}{\cong}          %% isomorphism
\newcommand{\lt}{\triangleright}    %% a left action
\newcommand{\otto}{\leftrightarrow} %% bijection
\newcommand{\rt}{\triangleleft}     %% a right action
\newcommand{\semi}{\rtimes}         %% crossed product
\newcommand{\tensor}{\otimes}       %% tensor product
\newcommand{\cotensor}{\square}       %% cotensor product
\newcommand{\trans}{\pitchfork}     %% transverse
\newcommand{\ul}{\underline}        %% for sheaves
\newcommand{\upto}{\uparrow}        %% left hand limit
\renewcommand{\:}{\colon}           %% colon in  f: A -> B
\newcommand{\blt}{\ast}
\newcommand{\Co}{C_{\bullet}}
\newcommand{\cCo}{C^{\bullet}}
\newcommand{\nbs}{\nabla^S}         %% spin connection
\newcommand{\up}{{\mathord{\uparrow}}} %% `up' spinors
\newcommand{\dn}{{\mathord{\downarrow}}} %% `down' spinors
\newcommand{\updn}{{\mathord{\updownarrow}}} %% up or down

%%% Bilinear enclosures:

\newcommand{\bbraket}[2]{\langle\!\langle#1\stroke#2\rangle\!\rangle}
   %% <<w|z>>
\newcommand{\bracket}[2]{\langle#1,\, #2\rangle} %% <w,z>
\newcommand{\scalar}[2]{\langle#1,\,#2\rangle} %% <w,z>
\newcommand{\poiss}[2]{\{#1,\,#2\}} %% {w,z}
\newcommand{\dst}[2]{\langle#1,#2\rangle} %% distributions <u,\phi>
\newcommand{\pairing}[2]{(#1\stroke #2)} %% right-linear pairing
\def\<#1|#2>{\langle#1\stroke#2\rangle} %% \braket (Dirac notation)
\def\?#1|#2?{\{#1\stroke#2\}}        %% left-linear pairing

%%% Accent-like macros:

\renewcommand{\Bar}[1]{\overline{#1}} %% closure operator
\renewcommand{\Hat}[1]{\widehat{#1}}  %% short for \widehat
\renewcommand{\Tilde}[1]{\widetilde{#1}} %% short for \widetilde


\DeclareMathOperator{\bCl}{\bC l}   %% complex Clifford algebra

%%% Small fractions in displays:

\newcommand{\ihalf}{\tfrac{i}{2}}   %% small fraction  i/2
\newcommand{\quarter}{\tfrac{1}{4}} %% small fraction  1/4
\newcommand{\shalf}{{\scriptstyle\frac{1}{2}}}  %% tiny fraction  1/2
\newcommand{\third}{\tfrac{1}{3}}   %% small fraction  1/3
\newcommand{\ssesq}{{\scriptstyle\frac{3}{2}}} %% tiny fraction  3/2
\newcommand{\sesq}{{\mathchoice{\tsesq}{\tsesq}{\ssesq}{\ssesq}}} %% 3/2
\newcommand{\tsesq}{\tfrac{3}{2}}   %% small fraction  3/2


%\newcommand\eqdef{\overset{\mathclap{\normalfont\mbox{def}}}{=}}
\newcommand\eqdef{\overset{\mathrm{def}}{=}}


%+++++++++++++++++++++++++++++++++++

\newcommand{\word}[1]{\quad\text{#1}\enspace} %% well-spaced words
\newcommand{\words}[1]{\quad\text{#1}\quad} %% better-spaced words
\newcommand{\su}[1]{{\sp{[#1]}}}


\newcommand{\opname}[1]{\mathop{\mathrm{#1}}\nolimits}

\newcommand{\hideqed}{\renewcommand{\qed}{}} %% to suppress `\qed'


 

%%%%%%%%%%%%%%%%%%%%%%%%%%%%%
%% 2. Some internal machinery
%%%%%%%%%%%%%%%%%%%%%%%%%%%%%

\newbox\ncintdbox \newbox\ncinttbox %% noncommutative integral symbols
\setbox0=\hbox{$-$}
\setbox2=\hbox{$\displaystyle\int$}
\setbox\ncintdbox=\hbox{\rlap{\hbox
           to \wd2{\box2\relax\hfil}}\box0\kern.1em}
\setbox0=\hbox{$\vcenter{\hrule width 4pt}$}
\setbox2=\hbox{$\textstyle\int$}
\setbox\ncinttbox=\hbox{\rlap{\hbox
           to \wd2{\hskip-.05em\box2\relax\hfil}}\box0\kern.1em}

\newcommand{\disp}{\displaystyle} %% short for  \displaystyle

%\newcommand{\hideqed}{\renewcommand{\qed}{}} %% no `\qed' at end-proof

\newcommand{\stroke}{\mathbin|}   %% (for `\bbraket' and such)
\newcommand{\tribar}{|\mkern-2mu|\mkern-2mu|} %% norm bars: |||

%%% Enclose one argument with delimiters:

\newcommand{\bra}[1]{\langle{#1}\rvert} %% bra vector <w|
\newcommand{\kett}[1]{\lvert#1\rangle\!\rangle} %% ket 2-vector |y>>
\newcommand{\snorm}[1]{\mathopen{\tribar}{#1}%
\mathclose{\tribar}}                 %% norm |||x|||


\newcommand{\End}{\mathrm{End}}       %%
\newcommand{\Hom}{\mathrm{Hom}}       %%
\newcommand{\Mrt}{\mathrm{Mrt}}       %%
\newcommand{\grad}{\mathrm{grad}}       %%
\newcommand{\Spin}{\mathrm{Spin}}       %%
\newcommand{\Ad}{\mathrm{Ad}}       %%
\newcommand{\Pic}{\mathrm{Pic}}       %%
\newcommand{\Aut}{\mathrm{Aut}}       %%
\newcommand{\Inn}{\mathrm{Inn}}       %%
\newcommand{\Out}{\mathrm{Out}}       %%
\newcommand{\Homeo}{\mathrm{Homeo}}       %%
\newcommand{\Diff}{\mathrm{Diff}}       %%
\newcommand{\im}{\mathrm{im}}       %%


\newcommand{\SO}{\mathrm{SO}}       %%
\newcommand{\SU}{SU}       %%
\newcommand{\gso}{\mathfrak{so}}    %% special orthogonal Lie algebra
\newcommand{\gero}{\mathfrak{o}}    %% orthogonal Lie algebra
\newcommand{\gspin}{\mathfrak{spin}} %% spin Lie algebra
\newcommand{\gu}{\mathfrak{u}}      %% unitary Lie algebra
\newcommand{\gsu}{\mathfrak{su}}    %% special unitary Lie algebra
\newcommand{\gsl}{\mathfrak{sl}}    %% special linear Lie algebra
\newcommand{\gsp}{\mathfrak{sp}}    %% symplectic linear Lie algebra



\title{Coverings of Quantum Groups}
\begin{document}
\maketitle  \setlength{\parindent}{0pt}
\begin{center}
\author{
{\textbf{Petr R. Ivankov*}\\
e-mail: * monster.ivankov@gmail.com \\
}
}
\end{center}

%\vspace{1 in}

%\begin{abstract}
%\noindent

\paragraph{}
It is known that any covering space of a topological group has the natural structure of a topological group. This article discusses a noncommutative generalization of this fact. A noncommutative generalization of the topological group is a quantum group. Also there is a noncommutative generalization of a covering. The combination of these algebraic constructions yields a motive to research the generalization of coverings of topological groups. In contrary to a topological group a covering space of a quantum group does not have the natural structure of the quantum group. However a covering space of a quantum group satisfies to a condition which is weaker than the condition of a covering space of a topological group.

%\end{abstract}
%\tableofcontents
\section{Motivation. Preliminaries}


\paragraph{}  
In this article we discuss a noncommutative analog of the following proposition.
\begin{prop}\label{cov_top_prop}\cite{mimuta_toda_lie} If $G$ is a topological group and $\pi:\widetilde{G} \to G$ is a covering, then for a covering space $\widetilde{G}$ one can introduce uniquely the structure of a topological group on $\widetilde{G}$ such that $\pi$ is a homomorphism and an arbitrary point $\widetilde{e}$ of the fibre over the unit $e$ of $G$ is the unit.
\end{prop}
For this purpose we need noncommutative generalizations of following objects:
\begin{itemize}
	\item Topological spaces,
	\item Coverings,
	\item Topological groups. 
\end{itemize}
\subsection{Generalization of topological objects}
\subsubsection{Noncommutative topological spaces}
  \paragraph*{} Gelfand-Na\u{\i}mark theorem \cite{arveson:c_alg_invt} states the correspondence between  locally compact Hausdorff topological spaces and commutative $C^*$-algebras. 


\begin{thm}\label{gelfand-naimark}\cite{arveson:c_alg_invt} (Gelfand-Na\u{\i}mark). 
Let $A$ be a commutative $C^*$-algebra and let $\mathcal{X}$ be the spectrum of A. There is the natural $*$-isomorphism $\gamma:A \to C_0(\mathcal{X})$.
\end{thm}

\paragraph*{}So any (noncommutative) $C^*$-algebra may be regarded as a generalized (noncommutative)  locally compact Hausdorff topological space. 
\subsubsection{Generalization of coverings}
\paragraph*{} Following theorem gives a pure algebraic description of finite-fold coverings of compact spaces.
\begin{thm}\label{pavlov_troisky_thm}\cite{pavlov_troisky:cov}
	Suppose $\mathcal X$ and $\mathcal Y$ are compact Hausdorff connected spaces and $p :\mathcal  Y \to \mathcal X$
	is a continuous surjection. If $C(\mathcal Y )$ is a projective finitely generated Hilbert module over
	$C(\mathcal X)$ with respect to the action
	\begin{equation*}
	(f\xi)(y) = f(y)\xi(p(y)), ~ f \in  C(\mathcal Y ), ~ \xi \in  C(\mathcal X),
	\end{equation*}
	then $p$ is a finite-fold  covering.
\end{thm}
\begin{defn}
	If $A$ is a $C^*$- algebra then an action of a group $H$ is said to be {\it involutive } if $ga^* = \left(ga\right)^*$ for any $a \in A$ and $g\in H$. Action is said to be \textit{non-degenerated} if for any nontrivial $g \in H$ there is $a \in A$ such that $ga\neq a$. 
\end{defn}
Following definition is motivated by the Theorem \ref{pavlov_troisky_thm}.
\begin{defn}\cite{ivankov:qnc}\label{fin_def_uni}
	Let $A \hookto \widetilde{A}$ be an injective *-homomorphism of unital $C^*$-algebras. Suppose that there is a non-degenerated involutive action $H \times \widetilde{A} \to \widetilde{A}$ of finite group, such that $A = \widetilde{A}^H\stackrel{\text{def}}{=}\left\{a\in \widetilde{A}~|~ a = g a;~ \forall g \in H\right\}$. There is an $A$-valued product on $\widetilde{A}$ given by
	\begin{equation*}\label{finite_hilb_mod_prod_eqn}
	\left\langle a, b \right\rangle_{\widetilde{A}}=\sum_{g \in H} g\left( a^*, b\right) 
	\end{equation*}
	and $\widetilde{A}$ is an $A$-Hilbert module. We say that $\left(A, \widetilde{A}, H \right)$ is an \textit{unital noncommutative finite-fold  covering} if $\widetilde{A}$ is a finitely generated projective $A$-Hilbert module.
\end{defn}
\subsubsection{Generalization of topological groups}
\paragraph*{} A compact quantum group can be regarded as a noncommutative analog of a compact topological group.
\begin{defn}\cite{neshveyev_tuset_qg}
(Woronowicz) A \textit{compact quantum group} is a pair $\left( A, \Delta\right) $, where $A$
is an unital $C^*$ -algebra and $\Delta: A \to A\otimes A $  is an unital *-homomorphism, called \textit{comultiplication}, such that
\begin{enumerate}
	\item [(a)] $\left( \Delta \otimes \Id_A\right) \Delta = \left( \Id_A \otimes \Delta \right) \Delta$ as homomorphisms $A \to A \otimes A  \otimes A$, (\textit{coassociativity});
\item[(b)] The spaces $\left(A \otimes 1 \right) \Delta A = \mathrm{span}\left\{\left(a \otimes 1\right)  \Delta \left( b \right)~|~a,b \in A \right\}$   and $\left(1 \otimes A\right) \Delta A$  are dense in $A\otimes A$  (\textit{cancellation property}).
\end{enumerate}
In this definition by the tensor product of $C^*$ -algebras we  mean the minimal tensor product.
\end{defn}
Following example shows that a compact topological group is a special case of a quantum group.

\begin{exm}\label{comm_exm}\cite{neshveyev_tuset_qg}
	Let $G$ be a compact group. Take $A$ to be the $C^*$-algebra $C\left(G\right)$ of continuous functions on $G$. Then $A \otimes A = C\left(G \times G\right)$, so we can define $\Delta$ by
	$$
	\Delta\left(f\right)\left(g,h\right) = f\left(gh\right) \text{ for all } g, h \in G
	$$
	Coassociativity of $\Delta$ follows from associativity of the product in $G$. To see that the cancellation property holds, note that $\left(A \otimes 1 \right) \Delta A$ is the unital $C^*$-subalgebra of $C\left(G\times G\right)$	spanned by all functions of the form $\left(g, h\right) \mapsto  f_1\left(g\right)f_2\left(gh\right)$. Since such functions separate points of $G\times G$, the $C^*$-algebra $\left(A \otimes 1 \right) \Delta A$ is dense in $C\left(G\times G\right)$ by the Stone-Weierstrass
	theorem.
	Any compact quantum group $\left(A, \Delta\right)$ with abelian $A$ is of this form. Indeed, by the Gelfand theorem, $A = C\left(G\right)$ for a compact space $G$. Then, since $A\otimes A = C\left(G\times G\right)$, the
	unital *-homomorphism $\Delta$  is defined by a continuous map $G\times G \to G$. Coassociativity
	means that
	$$
	f\left(\left(gh\right)k\right) = f\left(g\left(hk\right)\right) \text{ for all } f \in C\left(G\right),
	$$
	whence $\left(gh\right)k = g\left(hk\right)$, so $G$ is a compact semigroup. If $gh = gk$, then $f_1\left(g\right)f_2\left(gh\right) =
	f_1\left(g\right)f_2\left(gk\right)$ for all $f_1; f_2 \in C\left(G\right)$. By the cancellation property the functions of the
	form $\left(g',h'\right) \mapsto f_1\left(g'\right)f_2\left(g'h'\right)$ span a dense subspace of $C\left(G \times G\right)$. It follows that
	$f\left(g, h\right) = f\left(g, k \right)$ for all $f \in   C\left(G\times G\right)$, whence $h = k$. Similarly, if $hg = kg$, then $h = k$.
	Thus $G$ is a semigroup with cancellation.
	In \cite{neshveyev_tuset_qg} it is proven that that any compact semigroup with cancellation is a group.
\end{exm}


\subsection{Finite Galois coverings}\label{fin_gal_cov_sec}
\paragraph*{} Here I follow to \cite{auslander:galois}. Let $A \hookto \widetilde{A}$ be an injective homomorphism of unital algebras, such that
\begin{itemize}
	\item $\widetilde{A}$ is a projective finitely generated $A$-module,
	\item There is an action $G \times \widetilde{A} \to \widetilde{A}$ of a finite group $G$ such that $$A = \widetilde{A}^G=\left\{\widetilde{a}\in \widetilde{A}~|~g\widetilde{a}=\widetilde{a}; ~\forall g \in G\right\}.$$
\end{itemize}
Let us consider the category $\mathscr{M}^G_{\widetilde{A}}$ of $G-\widetilde{A}$ modules, i.e.  any object $M \in \mathscr{M}^G_{\widetilde{A}}$ is a $\widetilde{A}$-module with equivariant action of $G$, i.e. for any $m \in M$ a following condition holds
$$
g\left(\widetilde{a}m \right)=  \left(g\widetilde{a} \right) \left(gm \right) \text{ for any } \widetilde{a} \in \widetilde{A}, ~ g \in G.
$$
Any morphism $\varphi: M \to N$ in the category $\mathscr{M}^G_{\widetilde{A}}$ is $G$- equivariant, i.e.
$$
\varphi\left( g m\right)= g \varphi\left( m\right)   \text{ for any } m \in M, ~ g \in G.
$$
Let $\widetilde{A}\left[ G\right]$ be an algebra such that $\widetilde{A}\left[ G\right] \approx \widetilde{A}\times G$ as an Abelian group and a multiplication law is given by
$$
\left( a, g\right)\left( b, h\right) =\left(a\left(gb \right), gh  \right).
$$
The category $\mathscr{M}^G_{\widetilde{A}}$ is equivalent to the category $\mathscr{M}_{\widetilde{A}\left[ G\right]}$ of $\widetilde{A}\left[ G\right]$ modules. Otherwise in \cite{auslander:galois} it is proven the equivalence between a category $\mathscr{M}_{A}$ of $A$-modules and the category $\mathscr{M}_{\widetilde{A}\left[ G\right]}$. It turns out that the category $\mathscr{M}^G_{\widetilde{A}}$ is equivalent to the category $\mathscr{M}_{A}$.

\section{Main result}
\paragraph*{} 
From the Proposition \ref{cov_top_prop}, Theorem \ref{pavlov_troisky_thm} and Example \ref{comm_exm} it turns out the following lemma

\begin{lem}\label{comm_lem}
	Let $\left(A, \Delta \right)$ be a commutative compact quantum group, and let   $\left(A, \widetilde{A},  H \right)$ be a noncommutative finite-fold covering such that $\widetilde{A}$ is a commutative algebra. Following condition holds:
	\begin{enumerate}
		\item[(i)] There is the natural structure   $\left(\widetilde{A}, \widetilde{\Delta} \right)$ of the compact quantum group, such that
		$$
		\widetilde{\Delta}\left( a\right) = \Delta\left(a \right) \text{ for any } a \in A.
		$$
		\item[(ii)] Operation $\Delta$ is $H$-equivariant, i.e. from
		\begin{equation*}
		\widetilde{\Delta}\left(\widetilde{a} \right)= \sum_{\iota \in I}\widetilde{b}_\iota \otimes \widetilde{c}_\iota
		\end{equation*}
		it turns out that for any $g \in H$ following condition holds
		\begin{equation*}
		\begin{split}
		\widetilde{\Delta}\left(g\widetilde{a} \right) =\sum_{\iota \in I}g\widetilde{b}_\iota \otimes g\widetilde{c}_\iota.
		\end{split}
		\end{equation*}
	\end{enumerate}
\end{lem}
\begin{proof}
Indeed this lemma is an algebraic interpretation of the topological Proposition \ref{cov_top_prop}. 
\end{proof}
\paragraph*{}
The Lemma \ref{comm_lem} is not true in general, there is a counterexample described in the Section \ref{counter_sec}. However any quantum group satisfies to a following theorem.

\begin{thm}\label{main_thm}
Let $\left(A, \Delta \right)$ be a  quantum group. Let
$\left(A, \widetilde{A},  H \right)$ be a noncommutative finite-fold covering projection. There are natural $A$-bimodule morphisms
\begin{equation*}
\begin{split}
\Delta_L : \widetilde{A} \to \widetilde{A} \otimes A, \\
\Delta_R : \widetilde{A} \to A \otimes \widetilde{A}.
\end{split}
\end{equation*}
such that following conditions hold:
\begin{enumerate}


\item[(i)] Above morphisms are $H$-equivariant, i.e. for any $g \in H$ from
\begin{equation*}
\begin{split}
\Delta_L \left( \widetilde{a}\right) = \sum_{\iota \in I}\widetilde{b}_\iota \otimes c_\iota,\\
\Delta_R \left( \widetilde{a}\right) = \sum_{\iota \in I}c_\iota \otimes\widetilde{d}_\iota
\end{split}
\end{equation*}
it turns out that
\begin{equation*}
\begin{split}
\Delta_L \left(g \widetilde{a}\right) = \sum_{\iota \in I}g\widetilde{b}_\iota \otimes c_\iota,\\
\Delta_R \left(g \widetilde{a}\right) = \sum_{\iota \in I}c_\iota \otimes g\widetilde{d}_\iota.
\end{split}
\end{equation*}
	\item [(ii)] If $a \in A$ then
\begin{equation*}
\begin{split}
\Delta_L \left(a\right) = \Delta\left( a\right) ,\\
\Delta_R \left(a\right) = \Delta\left( a\right).
\end{split}
\end{equation*}
\end{enumerate}
\end{thm}
\begin{proof}
(i) If we apply to $\Delta: A \to A\otimes A$ a functor $ \widetilde{A}\otimes_A-$ then we have
$$
\Delta_L: \widetilde{A} \to \widetilde{A} \otimes A
$$
From the Section \ref{fin_gal_cov_sec} it follows that $\Delta_L$  is left $H$-equivariant. Similarly one can construct $\Delta_R$.
\newline
(ii) Follows from the definition of functors $ \widetilde{A}\otimes_A-$ and $ -\otimes_A \widetilde{A}$ and from that the *-homomorphism $A \hookto \widetilde{A}$ is injective.
\end{proof} 
\begin{rem}
	The statement of Theorem \ref{main_thm} is weaker than the statement of the Lemma \ref{comm_lem}. In fact the Theorem \ref{main_thm} describes a left and right action of the group $G$ on the quotient group $\widetilde{G}/H \approx G$.
\end{rem}
%\begin{rem}
%From \ref{fin_gal_cov_sec} it follows that both functors $\widetilde{A}\otimes_A-$ and $ -\otimes_A \widetilde{A}$ are equivalences between $\mathscr{M}_{A}$ and $\mathscr{M}^G_{\widetilde{A}}$.
%\end{rem}

\section{Hopf Algebras}\label{hopf}

%This section of the review is mostly based on the Refs. \cite{11}-\cite{17}.

\subsection{\bf \em Coalgebras\label{hopf1}}
\setcounter{equation}0

We consider an associative unital algebra ${\cal A}$ (over the field of complex numbers
$\mathbb{C}$;
in what follows, all algebras that are introduced will also be understood to be over the
field of complex numbers). Each element of ${\cal A}$
can be expressed as a linear combination of
basis elements $e_{i} \in {\cal A}$, where $i = 1,2,3, \dots$, and
the identity element $I$ is given by the formula
$$
I = E^{i}\, e_{i} \;\;\;\;\;\; (E^{i} \in \mathbb{C}) \; ,
$$
(we imply summation over repeated indices).
Then for any two elements
$e_{i}$ and $e_{j}$ we define their multiplication in the form
\be
\lb{2.1}
{\cal A} \otimes {\cal A} \stackrel{m}{\longrightarrow} {\cal A}
\;\;\;\; \Rightarrow \;\;\;\;
e_{i} \cdot e_{j} = m^{k}_{ij} e_{k} \, ,
\ee
where $m^{k}_{ij}$ is a certain set of complex numbers that satisfy the condition
\be
\lb{2.2}
E^{i}m^{k}_{ij} = m^{k}_{ji}E^{i} =  \delta^{k}_{j}
\ee
for the identity element, and also the condition
\be
\lb{2.3}
m^{l}_{ij} m^{n}_{lk} = m^{n}_{il} m^{l}_{jk} \equiv m^{n}_{ijk} \; ,
\ee
that is equivalent to the condition of associativity
for the algebra ${\cal A}$:
\be
\lb{2.4}
(e_{i} e_{j}) e_{k} = e_{i} (e_{j} e_{k}).
\ee
The condition of associativity (\ref{2.4}) for the multiplication (\ref{2.1}) can obviously be
represented in the form of the commutativity of the diagram:

\unitlength=0.7cm
\begin{picture}(15,4.5)(-3,-1)
	\put(3.3,2.5){${\cal A} \otimes {\cal A} \otimes {\cal A}$}
	\put(6.5,2.7){\vector(1,0){2.2}}
	\put(6.8,3){\footnotesize $id \otimes m$}
	\put(4.2,0.5){${\cal A} \otimes {\cal A}$}
	\put(5,2.2){\vector(0,-1){1.2}}
	\put(3,1.5){\footnotesize $m \otimes id$}
	\put(6,0.7){\vector(1,0){3.2}}
	\put(7.3,1){\footnotesize $m$}
	\put(9,2.5){${\cal A} \otimes {\cal A}$}
	\put(9.8,2.2){\vector(0,-1){1.2}}
	\put(10,1.5){\footnotesize $m$}
	\put(9.5,0.5){${\cal A}$}
	
	\put(4,-0.7){Fig. 1. Associativity axiom.}
	
\end{picture}
%\begin{center}
%Fig. 1. Associativity axiom.
%\end{center}


In Fig.1 the map $m$ represents multiplication:
${\cal A} \otimes {\cal A} \stackrel{m}{\longrightarrow} {\cal A}$, and $id$ denotes the
identity mapping. The existence of the unit element $I$ means that one can define a mapping i:
$\mathbb{C} \to {\cal A}$ (embedding of $\mathbb{C}$ in ${\cal A}$)
\be
\lb{2.5}
k \stackrel{\bf i}{\longrightarrow} k \cdot I \; , \;\;
k \in \mathbb{C}
\ee
For $I$ we have the condition (\ref{2.2}), which is visualized
as the diagram in Fig.2:

\unitlength=0.8cm
\begin{picture}(15,5)(-3,0)
	\put(4.2,2.5){$\mathbb{C} \otimes {\cal A}$}
	\put(5,2.2){\vector(2,-1){1.5}}
	\put(5,2.2){\vector(-2,1){0.1}}
	\put(5,3){\vector(1,1){1}}
	\put(4.3,3.5){{\bf i}$ \otimes id$}
	\put(6.5,4.2){${\cal A} \otimes {\cal A}$}
	\put(9,3){\vector(-1,1){1}}
	\put(8.7,3.5){$id \otimes ${\bf i}}
	\put(8.5,2.5){${\cal A} \otimes \mathbb{C}$}
	\put(9,2.2){\vector(-2,-1){1.5}}
	\put(9,2.2){\vector(2,1){0.1}}
	\put(6.8,1.2){${\cal A}$}
	\put(7,4){\vector(0,-1){2.2}}
	\put(7.2,3){$m$}
	\put(3.5,0.5){Fig. 2. \it Axioms for the identity.}
\end{picture}

\noindent
where the mappings
\be
\lb{natis}
\mathbb{C} \otimes {\cal A} \leftrightarrow {\cal A} \;\;
{\rm  and} \;\;
{\cal A} \otimes \mathbb{C} \leftrightarrow {\cal A}
\ee
are natural isomorphisms.
One of the advantages of the diagrammatic language used here is that it leads
directly to the definition of a new fundamental object -- the coalgebra -- if we
reverse all the arrows in the diagrams of Fig.1 and Fig.2.

%{\bf Definition 1.}

\newtheorem{def1}{Definition}%[section]
\begin{def1} \label{def1}
	{\it A coalgebra ${\cal C}$ is a vector space
		(with the basis $\{ e_{i} \}$) equipped with the
		mapping
		$\Delta : {\cal C} \rightarrow {\cal C} \otimes {\cal C}$
		\be
		\lb{2.6}
		\Delta (e_{i}) = \Delta^{kj}_{i} e_{k} \otimes e_{j} \; ,
		\ee
		which is called comultiplication, and also equipped with the mapping
		$\epsilon : {\cal C} \rightarrow \mathbb{C}$,
		which is called the coidentity. The coalgebra ${\cal C}$ is called coassociative if the mapping
		$\Delta$ satisfies the condition of coassociativity
		(cf. the diagram in Fig.1 with the arrows reversed and
		the symbol $m$ changed to $\Delta$)
		\be
		\lb{2.7}
		(id \otimes \Delta) \Delta = (\Delta \otimes id) \Delta
		\;\;\;\;  \Rightarrow  \;\;\;\;
		\Delta^{nl}_{i}\Delta^{kj}_{l} =
		\Delta^{lj}_{i}\Delta^{nk}_{l} \equiv \Delta^{nkj}_{i} \; .
		\ee
		The coidentity $\epsilon$ must satisfy the following conditions (cf. the diagram in Fig.2 with arrows reversed and
		symbols $m,{\bf i}$ changed to $\Delta,\epsilon$)
		\be
		\lb{2.8}
		m \bigl((\epsilon \otimes id) \Delta ({\cal C})\bigr) =
		m \bigl((id \otimes \epsilon) \Delta ({\cal C})\bigr) = {\cal C}
		\;\;\;\; \Rightarrow \;\;\;\;
		\epsilon_{i}\Delta^{ij}_{k} = \Delta^{ji}_{k} \epsilon_{i} =
		\delta^{j}_{k} \; .
		\ee
		Here $m$ realizes the natural isomorphisms (\ref{natis})
		as a multiplication map:
		$m(c \otimes e_{i}) =  m(e_{i} \otimes c) = c \cdot e_{i}$
		($\forall c \in \mathbb{C}$),
		and the complex numbers $\epsilon_{i}$ are determined from the relations $\epsilon(e_{i}) = \epsilon_{i}$.}
	%Below to simplify formulas we often omit the symbol
	%${\bf i}$, where it will not lead to a misunderstanding.}
	\end{def1}
	
	
	
	For algebras and coalgebras, the concepts of modules and comodules can be introduced.
	Thus, if ${\cal A}$ is an algebra, the left ${\cal A}$-module
	can be defined as a vector space $N$
	and a mapping $\psi: \;\; {\cal A} \otimes N \rightarrow N$
	(action of ${\cal A}$ on $N$) such that the diagrams on Fig.3
	are commutative.
	
	\unitlength=0.7cm
	\begin{picture}(15,5)(-4,-1)
\put(0.4,2.5){${\cal A} \otimes {\cal A} \otimes N$}
\put(3.5,2.7){\vector(1,0){2.2}}
\put(4,3){\footnotesize  $id \otimes \psi$}
\put(1.1,0.5){${\cal A} \otimes  N$}
\put(2,2.2){\vector(0,-1){1.2}}
\put(0.3,1.5){\footnotesize $m \otimes id$}
\put(3,0.7){\vector(1,0){3.2}}
\put(4.3,1){\footnotesize $\psi$}
\put(6,2.5){${\cal A} \otimes N$}
\put(6.8,2.2){\vector(0,-1){1.2}}
\put(7,1.5){\footnotesize $\psi$}
\put(6.5,0.5){$N$}
%\end{picture}
\put(9,1.5){$\mathbb{C} \otimes N$}
\put(10,1.2){\vector(2,-1){1.5}}
\put(10,1.2){\vector(-2,1){0.1}}
\put(10,2){\vector(1,1){1}}
\put(9.3,2.5){\footnotesize {\bf i}$ \otimes id$}
\put(11.1,3.2){${\cal A} \otimes N$}
%\put(14,2){\vector(-1,1){1}}
%\put(13.7,2.5){$id \otimes ${\bf i}}
%\put(13.5,1.5){${\cal A} \otimes \mathbb{C}$}
%\put(14,1.2){\vector(-2,-1){1.5}}
%\put(14,1.2){\vector(2,1){0.1}}
\put(11.7,0.2){$N$}
\put(12,3){\vector(0,-1){2.2}}
\put(12.2,2){\footnotesize $\psi$}
\put(1.2,-0.8){Fig. 3. Axioms for left ${\cal A}$-module.}
\end{picture}

\noindent
In other words, the space $N$ is the representation
space of the algebra ${\cal A}$.

If $N$ is a (co)algebra and the mapping $\psi$ preserves the (co)algebraic structure
of $N$ (see below), then $N$ is called the
{\it left ${\cal A}$-module (co)algebra}. The concepts of
{\it right module (co)algebra} is introduced similarly. If $N$ is
simultaneously the left and the right ${\cal A}$-module,
then $N$ is called the
{\it two-sided ${\cal A}$-module}.
It is obvious that the algebra ${\cal A}$ itself
is a two-sided ${\cal A}$-module for which the
left and right actions are given by the left and
right multiplications in the algebra.

Now suppose that ${\cal C}$ is a coalgebra; then a left
${\cal C}$-comodule can be defined as a
space $M$ together with a mapping
$\Delta_{L}$: $M \rightarrow {\cal C} \otimes M$
(coaction of ${\cal C}$ on $M$)
satisfying the axioms in Fig.4 (in the diagrams of Fig.3, where the modules
were defined, it is necessary to reverse all the arrows):

\unitlength=0.7cm
\begin{picture}(15,4.5)(-4,-1)
\put(0.5,2.5){${\cal C} \otimes {\cal C} \otimes M$}
\put(5.5,2.7){\vector(-1,0){2}}
\put(4,3){\footnotesize $id \otimes \Delta_{L}$}
\put(1.3,0.5){${\cal C} \otimes  M$}
\put(2,1){\vector(0,1){1.2}}
\put(0.3,1.5){\footnotesize $\Delta \otimes id$}
\put(6.2,0.7){\vector(-1,0){3}}
\put(4.3,0.9){\footnotesize $\Delta_{L}$}
\put(6,2.5){${\cal C} \otimes M$}
\put(6.8,1){\vector(0,1){1.2}}
\put(7,1.5){\footnotesize $\Delta_{L}$}
\put(6.5,0.5){$M$}
%\end{picture}

\put(10,1.5){$\mathbb{C} \otimes M$}
\put(11,1.2){\vector(2,-1){1.5}}
\put(11,1.2){\vector(-2,1){0.1}}
\put(12,3){\vector(-1,-1){1}}
\put(10,2.5){\footnotesize $\epsilon \otimes id$}
\put(12,3.2){${\cal C} \otimes M$}
%\put(14,2){\vector(-1,1){1}}
%\put(13.7,2.5){$id \otimes i$}
%\put(13.5,1.5){${\cal C} \otimes \mathbb{C}$}
%\put(14,1.2){\vector(-2,-1){1.5}}
%\put(14,1.2){\vector(2,1){0.1}}
\put(12.8,0.2){$M$}
\put(13,0.8){\vector(0,1){2.2}}
\put(13.2,1.9){\footnotesize $\Delta_{L}$}

\put(3,-0.7){Fig. 4. Axioms for left ${\cal A}$-comodule.}
\end{picture}
%\begin{center}
%Fig. 4. Axioms for left ${\cal A}$-comodule.
%\end{center}

If $M$ is a (co)algebra and the mapping $\Delta_L$ preserves the (co)algebraic
structure (for example, is a homomorphism; see below), then $M$ is called
a left ${\cal C}$-comodule (co)algebra. Right comodules are introduced
similarly, after which two-sided comodules are defined in the natural manner.
It is obvious that the coalgebra ${\cal C}$ is a two-sided ${\cal C}$-comodule.

Let ${\cal V}, \; \tilde{{\cal V}}$ be two vector spaces with bases
$\{ e_{i} \}, \; \{ \tilde{e}_{i} \}$.
We denote by ${\cal V}^{*}, \; \tilde{{\cal V}}^{*}$
the corresponding dual linear spaces whose
basis elements are linear functionals
$\{ e^{i} \}: \; {\cal V} \rightarrow \mathbb{C}$,
$\{ \tilde{e}^{i} \}: \tilde{{\cal V}} \rightarrow \mathbb{C}$.
For the values of these functionals, we use the expressions
$ \langle e^{i}\,|e_{j}\rangle $ and
$\langle \tilde{e}^{i}\,|\tilde{e}_{j}\rangle $.
For every mapping $L: \;\; {\cal V} \rightarrow \tilde{\cal V}$ it is possible to define a
unique mapping $L^{*}: \;\; \tilde{\cal V}^{*} \rightarrow {\cal V}^{*}$
induced by the equations
\be
\lb{2.9}
\langle \tilde{e}^{i} \, | L(e_{j}) \rangle  =
\langle  L^{*}(\tilde{e}^{i}) \, | e_{j} \rangle  ,
\ee
if the matrix $\langle e^{i}\, |e_{j}\rangle $ is invertible.
In addition, for the dual objects there exists the linear injection
$$
\rho: \;\; {\cal V}^{*} \otimes \tilde{\cal V}^{*} \rightarrow
( {\cal V} \otimes \tilde{\cal V} )^{*} \; ,
$$
which is given by the equations
$$
\langle \rho(e^{i} \otimes \tilde{e}^{j}) \,
| e_{k} \otimes \tilde{e}_{l} \rangle  =
\langle e^{i} \,| e_{k}\rangle \; \langle \tilde{e}^{j}\,
| \tilde{e}_{l} \rangle  \; .
$$
A consequence of these facts is that for every coalgebra
$({\cal C}, \; \Delta, \; \epsilon)$
it is possible to define an algebra ${\cal C}^{*}=
{\cal A}$ (as dual object to ${\cal C}$) with
multiplication $m =  \Delta^{*} \cdot \rho$ and
the unit element $I$ that satisfy the relations
($\forall a,a' \in {\cal A}$, $\forall c \in {\cal C}$):
$$
\langle a|c_{(1)}\rangle \langle a'|c_{(2)}\rangle  =
\langle \rho( a \otimes a')| \Delta(c)\rangle
= \langle \Delta^* \cdot \rho( a \otimes a')| c \rangle  =
\langle a \cdot a' \, | \, c \rangle \; ,
\;\;\;\;\;\;\; \langle  I | c \rangle  = \epsilon(c) \; .
$$
Here we denote $a \cdot a' := \Delta^* \cdot \rho( a \otimes a')$
and use the convenient Sweedler notation of Ref. [11] for comultiplication
in ${\cal C}$ (cf. eq. (\ref{2.6})):
\be
\lb{sweed}
\Delta(c) = \sum_{c} c_{(1)} \otimes c_{(2)} \; .
\ee
The summation symbol $\sum_{c}$ will usually be
omitted in the equations. We also use the Sweedler notation
for left and right coactions
$\Delta_L(v) = \bar{v}^{(-1)} \otimes v^{(0)}$ and
$\Delta_R(v) = v^{(0)} \otimes \bar{v}^{(1)}$,
where index $(0)$ is
reserved for the comodule elements and summation symbols $\sum_{v}$ are also omitted.

Thus, duality in the diagrammatic definitions of the algebras and coalgebras
(reversal of the arrows) has in particular the consequence that the algebras
and coalgebras are indeed duals to each other.

It is natural to expect that an analogous duality can also be traced for
modules and comodules. Let ${\cal V}$ be a left comodule
for ${\cal C}$. Then the left
coaction of ${\cal C}$ on ${\cal V}$:
$v \mapsto \sum_{v} \; \bar{v}^{(-1)} \otimes v^{(0)} \;\;
(\bar{v}^{(-1)}\in {\cal C}, \;\; v^{(0)} \in {\cal V})$
induces a right action of ${\cal C}^{*}={\cal A}$ on
${\cal V}$:
$$
(v,a) \;\;\; \mapsto \;\;\;
v \triangleleft a =
\langle a\, |\bar{v}^{(-1)}\rangle  \; v^{(0)} \; ,
\;\;\;\;\;\; a \in {\cal A} \; ,
$$
%(here and in what follows, we omit the
%summation sign $\sum_{v}$)
and therefore ${\cal V}$ is a right module for ${\cal A}$.
Conversely, the right coaction of
${\cal C}$ on ${\cal V}$:
$v \mapsto v^{(0)} \otimes \bar{v}^{(1)}$ induces the left action of
${\cal A} = {\cal C}^{*}$ on ${\cal V}$:
$$
(a,v) \;\;\; \mapsto \;\;\; a \triangleright v
= v^{(0)} \langle a|\bar{v}^{(1)}\rangle   \; .
$$
From this we immediately conclude that the coassociative
coalgebra ${\cal C}$ (which coacts on itself by the coproduct) is
a natural module for its dual algebra ${\cal A}={\cal C}^{*}$. Indeed, the right
action ${\cal C} \otimes {\cal A} \rightarrow {\cal C}$ is determined by the equations
\be
\lb{2.10}
(c,a) \;\;\; \mapsto \;\;\; c \triangleleft a = \langle a|c_{(1)}\rangle  c_{(2)}
\ee
whereas for the left action ${\cal A} \otimes {\cal C} \rightarrow {\cal C}$ we have
\be
\lb{2.11}
(a,c) \;\;\; \mapsto \;\;\; a \triangleright c = c_{(1)} \langle a|c_{(2)}\rangle  \; .
\ee
Here $a \in {\cal A} \;\; c \in {\cal C}$.
The module axioms (shown as the diagrams in Fig. 3)
hold by virtue of the coassociativity of ${\cal C}$.

Finally, we note that the action of a certain algebra
$H$ on ${\cal C}$ from the
left (from the right) induces an action of
$H$ on ${\cal A} = {\cal C}^{*}$ from the right
(from the left). This obviously follows from relations of the type (\ref{2.9}).

\subsection{\bf \em Bialgebras\label{hopf2}}
\setcounter{equation}0

So-called bialgebras are the next important objects that are used in the theory of quantum groups.

%{\bf Definition 2.}

\newtheorem{def2}[def1]{Definition}%[section]
\begin{def2} \label{def2}
{\it An associative algebra
	${\cal A}$ with identity that is
	simultaneously a coassociative coalgebra with coidentity is called
	a bialgebra if the algebraic and coalgebraic structures are self-consistent.
	Namely, the comultiplication and coidentity must be homomorphisms of the algebras:
	\be
	\lb{2.12}
	\Delta(e_{i})\, \Delta(e_{j}) = m_{ij}^{k} \Delta(e_{k}) \Rightarrow
	\Delta^{i'i''}_{i} \Delta^{j'j''}_{j} m^{k'}_{i'j'} m^{k''}_{i''j''} =
	m_{ij}^{k}\, \Delta^{k'k''}_{k} \; , \;\;
	\ee
	$$
	\Delta(I) = I \otimes I \; , \;\;
	\epsilon(e_{i}\cdot e_{j}) = \epsilon(e_{i})\,
	\epsilon(e_{j}) \; , \;\;
	\epsilon(I) = E^i \, \epsilon_i = 1 \; .
	$$
}
\end{def2}
Note that for every bialgebra we have a certain freedom in
the definition of the multiplication (\ref{2.1}) and the comultiplication (\ref{2.6}).
Indeed, all the axioms (\ref{2.3}), (\ref{2.7}), and
(\ref{2.12}) are satisfied if instead of (\ref{2.1})  we take
$$
e_{i} \cdot e_{j} = m^{k}_{ji} \; e_{k},
$$
or instead of (\ref{2.6}) choose
\be
\lb{2.13}
%\Delta' (e_{i}) =
\Delta^{\sf op}(e_{i}) =
\Delta^{jk}_{i} \, e_{k} \otimes e_{j} \; ,
\ee
(such algebras are denoted as ${\cal A}^{op}$ and ${\cal A}^{cop}$, respectively).
Then the algebra ${\cal A}$ is called noncommutative if
$m^{k}_{ij} \neq m^{k}_{ji}$, and noncocommutative if $\Delta^{ij}_{k} \neq
\Delta^{ji}_{k}$.

In quantum physics, it is usually assumed that all algebras
of observables are bialgebras. Indeed, a coalgebraic structure is
needed to define the action of the algebra ${\cal A}$ of observables on the
state $|\psi_{1}\rangle \otimes |\psi_{2}\rangle$ of the system that is the
composite system formed
from two independent systems with wave functions
$|\psi_{1} \rangle$ and $|\psi_{2} \rangle$
\be
\lb{phys}
a \triangleright (|\psi_{1}\rangle \otimes |\psi_{2}\rangle) =
\Delta(a) \, (|\psi_{1}\rangle \otimes |\psi_{2}\rangle) =
a_{(1)} \, |\psi_{1}\rangle \otimes a_{(2)} \, |\psi_{2}\rangle \;\;
(\forall a \in {\cal A}) \; .
\ee
In other words, for bialgebras it is possible to
formulate a theory of representations in which new representations
can be constructed by direct multiplication of old ones.

A classical example of a bialgebra is the universal enveloping algebra of a Lie algebra $\mathfrak{g}$,
in particular, the spin algebra $\mathfrak{su}(2)$
in three-dimensional space.
To demonstrate this, we consider the Lie algebra
$\mathfrak{g}$
with generators $J_{\alpha} \;\;
(\alpha = 1,2,3, \dots)$, that satisfy the antisymmetric multiplication rule
(defining relations)
\be
\lb{lie}
[J_{\alpha}, \, J_{\beta} ] =
t_{\alpha\beta}^{\gamma} J_{\gamma} .
\ee
Here $t^{\gamma}_{\alpha\beta} = - t^{\gamma}_{\beta\alpha}$ are structure
constants which satisfy Jacoby identity. The enveloping algebra of this
algebra is the algebra $U(\mathfrak{g})$ with basis elements consisting of the
identity $I$ and the elements
$e_{i} = J_{\alpha_{1}} \cdots J_{\alpha_{n}}
\; \forall n \geq 1$, where the products of the
generators $J$ are ordered lexicographically, i.e.,
$\alpha_1 \leq \alpha_2 \leq \dots \leq \alpha_n$.
The coalgebraic structure for the algebra $U(\mathfrak{g})$ is specified by means of the mappings
\be
\lb{2.14}
\Delta(J_{\alpha})
=J_{\alpha} \otimes I  + I \otimes J_{\alpha} \; , \;\;
\epsilon(J_{\alpha}) = 0 \; , \;\;
\epsilon(I) = 1 \; ,
\ee
which satisfy all the axioms of a bialgebra. The mapping $\Delta$ in (\ref{2.14})
is none other than the rule for addition of spins. In fact one can
quantize the coalgebraic structure (\ref{2.14}) for universal enveloping algebra $U(\mathfrak{g})$
and consider the noncocommutative comultiplications $\Delta$. Such quantizations
will be considered below in Section {\bf \ref{semicl}}
and leads to the definition of
Lie bialgebras.

Considering exponentials of elements of a Lie algebra, one can arrive
at the definition of a group bialgebra of the group $G$ with structure mappings
\be
\lb{2.15}
\Delta(h) = h \otimes h \; , \;\; \epsilon(h) =1 \;\;\;\;
(\forall \; h \in G),
\ee
which obviously follow from (\ref{2.14}). The next important example
of a bialgebra is the algebra ${\cal A}(G)$ of functions
$f$ on a group
($f: G \rightarrow \mathbb{C}$).
This algebra is dual to the group algebra of the group $G$, and its structure
mappings have the form ($f,f' \in {\cal A}(G); \;\; h,h' \in G$):
\be
\lb{2.15b}
(f \cdot f')(h) = f(h)f'(h) \, , \;\;\;\;
f(h \cdot h') = (\Delta(f)) (h,h') = f_{(1)}(h)\; f_{(2)}(h') \, , \;\;\;\;
\epsilon(f) = f(I) \; ,
\ee
where $I_G$ is the identity element in the group $G$.
In particular, if the functions $T^{i}_{j}$ realize a matrix representation of the
group $G$, then we have
\be
\lb{2.15a}
T^{i}_{j}(hh') = T^{i}_{k}(h)T^{k}_{j}(h') \;\; \Rightarrow
\;\; \Delta(T^{i}_{j}) =  T^{i}_{k}  \otimes T^{k}_{j} \; ,
\ee
(the functions $T^{i}_{j}$ can be regarded as generators
of a subalgebra in the algebra ${\cal A}(G)$).
Note that if $\mathfrak{g}$ is non-Abelian, then
$U(\mathfrak{g})$ and $G$ are noncommutative but
cocommutative bialgebras, whereas ${\cal A}(G)$ is a commutative but noncocommutative bialgebra.
Anticipating, we mention that the most
interesting quantum groups are
associated with noncommutative and noncocommutative bialgebras.

It is obvious that for a bialgebra ${\cal H}$ it is also possible to introduce
the concepts of left (co)modules and (co)module (co)algebras
(right (co)modules and (co)module (co)algebras are
introduced in exactly
the same way). Moreover, for the bialgebra ${\cal H}$ it is possible to introduce the
concept of a left (right) bimodule $B$, i.e., a left (right) ${\cal H}$-module that is
simultaneously a left (right) ${\cal H}$-comodule; at the same time, the module and
comodule structures must be self-consistent:
$$
\Delta_{L} ({\cal H} \triangleright B) =
\Delta({\cal H}) \triangleright \Delta_{L}(B) \; ,
$$
$$
(\epsilon \otimes id) \Delta_{L} (b) = b \; , \;\; b \in B \; ,
$$
where $\Delta_{L} (b) = \bar{b}^{(-1)} \otimes b^{(0)}$
and $\bar{b}^{(-1)} \in {\cal H}$, $b^{(0)} \in B$.
On the other hand, in the case of bialgebras the conditions of conserving
of the (co)algebraic structure of (co)modules can be represented in a
more explicit form. For example, for the left ${\cal H}$-module algebra
${\cal A}$ we have
$(a,b \in {\cal A}; \;\; h \in {\cal H})$:
$$
h \triangleright (ab) =
(h_{(1)} \triangleright a)
(h_{(2)} \triangleright b) \; , \;\; h \triangleright I_{A} =
\epsilon(h) I_{A} \; .
$$
In addition, for the left ${\cal H}$-module coalgebra ${\cal A}$ we must have
$$
\Delta(h \triangleright a) = \Delta(h) \triangleright \Delta(a) =
(h_{(1)} \triangleright a_{(1)}) \otimes
(h_{(2)} \triangleright a_{(2)}) \; , \;\;
\epsilon(h \triangleright a) = \epsilon(h)\epsilon(a) \; .
$$
Similarly, the algebra ${\cal A}$ is a left ${\cal H}$-comodule algebra if
$$
\Delta_{L}(ab) = \Delta_{L}(a)\, \Delta_{L}(b) \; , \;\;
\Delta_{L}(I_{A}) = I_{\cal H} \otimes I_{A} \; ,
$$
and, finally, the coalgebra ${\cal A}$ is a left  ${\cal H}$-comodule coalgebra if
\be
\lb{2.16}
(id \otimes \Delta)\Delta_{L}(a) =
m_{\cal H}(\Delta_{L} \otimes \Delta_{L})\Delta(a) \; , \;\;
(id \otimes \epsilon_{A})\Delta_{L}(a) =
I_{\cal H}\epsilon_{A}(a) \; ,
\ee
where
$$
m_{\cal H}(\Delta_{L}\otimes\Delta_{L})(a \otimes b) =
\bar{a}^{(-1)} \bar{b}^{(-1)}\otimes a^{(0)} \otimes b^{(0)} \; .
$$

We now consider the bialgebra ${\cal H}$, which acts on a certain module algebra ${\cal A}$. One
further important property of bialgebras is that we can define a new associative algebra ${\cal A}
\sharp {\cal H}$ as the cross product (smash product) of ${\cal A}$ and ${\cal H}$. Namely:

%{\bf Definition 3.}

\newtheorem{def3}[def1]{Definition}%[section]
\begin{def3} \label{def3}
{\it The left smash product ${\cal A} \sharp {\cal H}$ of the bialgebra ${\cal H}$ and its left
	module algebra ${\cal A}$ is an associative
	algebra such that: \\
	1) As a vector space,
	${\cal A} \sharp {\cal H}$ is identical to ${\cal A} \otimes {\cal H}$ \\
	2) The product is defined in the sense ($h,g \in {\cal H}$; $a,b \in {\cal A}$)
	\be
	\lb{2.17}
	(a \sharp g)\, (b \sharp h) = \sum_{g} a(g_{(1)} \triangleright b) \sharp
	(g_{(2)} h) \equiv (a \sharp I)\;
	(\Delta(g) \triangleright ( b \sharp h)) \; ;
	\ee
	3) The identity element is $I \sharp I$.}
	\end{def3}
	If the algebra ${\cal A}$ is the bialgebra dual to the bialgebra ${\cal H}$, then the relations
	(\ref{2.17}) and (\ref{2.11}) define the rules for interchanging the elements $(I \sharp g)$ and
	$(a \sharp I)$:
	\be
	\lb{2.18}
	(I \sharp g) \, (a \sharp I) =
	(a_{(1)} \sharp I) \, \langle g_{(1)}|a_{(2)}\rangle \,
	(I \sharp g_{(2)}) \; .
	\ee
	Thus, the subalgebras ${\cal A}$ and ${\cal H}$ in
	${\cal A} \, \sharp \, {\cal H}$ do not commute with each other.
	The smash product depends on which action
	(left or right) of the algebra ${\cal H}$ on ${\cal A}$ we choose.
	In addition, the smash product generalizes the concept of the semidirect product.
	In particular, if we take as bialgebra ${\cal H}$
	the Lorentz group algebra
	(see (\ref{2.15})),
	and as module ${\cal A}$ the group of translations in Minkowski space,
	then the smash product ${\cal A} \, \sharp \, {\cal H}$ defines the structure of the Poincare group.
	
	The coanalog of the smash product, the smash coproduct
	${\cal A} \, \underline{\sharp} \, {\cal H}$, can also be defined.
	For this, we consider the bialgebra
	${\cal H}$ and its comodule coalgebra ${\cal A}$.
	Then on the space ${\cal A} \otimes {\cal H}$ it is
	possible to define the structure of a coassociative coalgebra:
	\be
	\lb{2.19}
	\Delta (a \, \underline{\sharp} \, h) =
	(a_{(1)} \, \underline{\sharp} \,
	{ \bar{a}_{(2)} }^{\;\;(-1)} h_{(1)}) \otimes
	(a_{(2)}^{\;\;(0)} \, \underline{\sharp} \,  h_{(2)}) \; , \;\;
	\epsilon(a \, \underline{\sharp} \,  h) =
	\epsilon(a)\epsilon(h) \; .
	\ee
	The proof of the coassociativity reduces to verification of the identity
	$$
	(m_{\cal H}( \Delta_{L}\otimes \Delta_{\cal H}) \otimes id)
	(id \otimes \Delta_{L})\Delta_{\cal A}(a) =
	(id \otimes id \otimes \Delta_{L})
	(id \otimes \Delta_{\cal A} )\Delta_{L}(a) \; ,
	$$
	which is satisfied if we take into account the axiom (\ref{2.16}) and the comodule axiom
	\be
	\lb{2.20}
	(id \otimes \Delta_{L})\Delta_{L}(a) =
	(\Delta_{\cal H} \otimes id)\Delta_{L}(a) \; .
	\ee
	
	Note that from the two bialgebras ${\cal A}$ and ${\cal H}$, which act and coact
	on each other in a special manner, it is possible to organize a
	new bialgebra that is simultaneously the smash product and smash
	coproduct of ${\cal A}$ and ${\cal H}$ (bicross product; see Ref. \cite{14}).
	
	\subsection{\bf \em Hopf algebras. Universal ${\cal R}$-matrices\label{hopf3}}
	\setcounter{equation}0
	
	We can now introduce the main concept in the theory of quantum groups,
	namely, the concept of the Hopf algebra.
	
	%{\bf Definition 4.}
	
	\newtheorem{def4}[def1]{Definition}
	\begin{def4} \label{def4}
{\it A bialgebra ${\cal A}$ equipped with an additional mapping $S: \;\; {\cal A} \rightarrow {\cal
		A}$ such that
	\be
	\lb{2.21}
	\begin{array}{c}
		m(S \otimes id) \Delta = m(id \otimes S) \Delta =
		\hbox{\bf i} \cdot \epsilon  \Rightarrow  \\
		S(a_{(1)}) \, a_{(2)}  =
		a_{(1)} \, S(a_{(2)}) = \epsilon(a) \cdot I \;\; (\forall a \in {\cal A})
	\end{array}
	\ee
	is called a Hopf algebra. The mapping $S$ is called the antipode and is an
	antihomomorphism with respect to both multiplication and comultiplication:
	\be
	\lb{2.22}
	S(ab) = S(b)S(a) \; , \;\; (S \otimes S)\Delta(a) = \sigma \cdot
	\Delta(S(a)) \; ,
	\ee
	where $a,b \in {\cal A}$ and $\sigma$ denotes the operator of transposition, $\sigma (a \otimes b)
	= (b \otimes a)$.}
	\end{def4}
	If we set
	\be
	\lb{2.23}
	S(e_{i}) = S^{j}_{i}e_{j} \; ,
	\ee
	then the axiom (\ref{2.21}) can be rewritten in the form
	\be
	\lb{2.24}
	\Delta^{ij}_{k}S^{n}_{i}m^{l}_{nj} =
	\Delta^{ij}_{k}S^{n}_{j}m^{l}_{in} = \epsilon_{k} E^{l} \; .
	\ee
	From the axioms for the structure mappings of a Hopf algebra,
	it is possible to obtain the useful equations
	\be
	\lb{2.25}
	\begin{array}{c}
S^{i}_{j}\epsilon_{i} = \epsilon_{j} \; , \;\;
S^{i}_{j}E^{j} = E^{i} \; , \\  \\
\Delta^{ji}_{k}(S^{-1})^{n}_{i}m^{l}_{nj} =
\Delta^{ji}_{k}(S^{-1})^{n}_{j}m^{l}_{in} = \epsilon_{k} E^{l} \; ,
\end{array}
\ee
which we shall use in what follows. Note that, in general, the antipode $S$
is not necessarily invertible. An invertible antipode is called bijective.

In quantum physics the existence of the antipode $S$ is needed to define a space of
contragredient states $\langle \psi |$ (contragredient module of ${\cal A}$)
with pairing $\langle \psi | \phi \rangle$:
$\langle \psi | \otimes | \phi \rangle \to \mathbb{C}$.
Left actions of the Hopf algebra ${\cal A}$
of observables to the contragredient states are (cf. the actions (\ref{phys}) of ${\cal A}$ to the
states $| \psi_1 \rangle \otimes | \psi_2 \rangle$):
\be
\lb{contg}
a \triangleright \langle \psi | : =
\langle \psi | \, S(a) \;\; (a \in {\cal A}) \; ,
\ee
$$
a \triangleright (\langle \psi_1 | \otimes \langle \psi_2 |) : =
(\langle \psi_1 | \otimes \langle \psi_2 |) \Delta(S(a)) =
\langle \psi_1 | \, S(a_{(2)}) \otimes \langle \psi_2 | \, S(a_{(1)}) \; .
$$
The states $\langle \psi |$ are called left dual to the states
$| \phi \rangle$; the right dual ones are introduced with the help of the inverse
antipode $S^{-1}$ (see e.g. \cite{17}).
Then, the covariance of the pairing $\langle \psi | \phi \rangle$
under the left action of ${\cal A}$ can be established:
$$
\begin{array}{c}
a \triangleright \langle \psi | \phi \rangle \equiv
(a_{(1)} \triangleright \langle \psi |)\; (a_{(2)}
\triangleright |\phi \rangle) =
\langle \psi | S(a_{(1)}) \, a_{(2)} | \phi \rangle = \epsilon(a) \langle \psi | \phi \rangle
\; , \\ [0.2cm]
a \triangleright \langle \psi_1 | \phi_1 \rangle \;
\langle \psi_2 | \phi_2 \rangle = a \triangleright
(\langle \psi_1 | \otimes \langle \psi_2 |)
(| \phi_1 \rangle \otimes | \phi_2 \rangle) = \\ [0.2cm]
= \langle \psi_1 | S(a_{(2)})a_{(3)}| \phi_1 \rangle
\langle \psi_1 | S(a_{(1)})a_{(4)}| \phi_1 \rangle =
\epsilon(a) \, \langle \psi_1 | \phi_1 \rangle \;
\langle \psi_2 | \phi_2 \rangle \; .
\end{array}
$$

The universal enveloping algebra $U(\mathfrak{g})$
and the group bialgebra of the group $G$
that we considered above can again serve as examples of cocommutative
Hopf algebras. An example of a commutative
(but noncocommutative)
Hopf algebra is the bialgebra ${\cal A}(G)$,
which we also considered above. The antipodes for these algebras have the form
\begin{equation}
\label{2.25a}
\begin{array}{c}
	U(\mathfrak{g}): \;\; S(J_{\alpha}) = - J_{\alpha} \; , \;\;
	S(I) = I \; ,
	\\ \\
	G: \;\; S(h) = h^{-1} \; ,
	\\ \\
	{\cal A}(G): \;\; S(f)(h) = f(h^{-1}) \; ,
\end{array}
\end{equation}
and satisfy the relation $S^{2} =id$,
which holds for all commutative or cocommutative Hopf algebras.

From the point of view of the axiom (\ref{2.21}), $S(a)$ looks like the
inverse of the element $a$, although in the general case $S^{2} \neq id$.
We recall that if a set ${\cal G}$
of elements with associative
multiplication ${\cal G} \otimes {\cal G} \rightarrow {\cal G}$
and with identity (semigroup) also contains all the
inverse elements, then such a set ${\cal G}$ becomes a group. Thus, from the point of
view of the presence of the mapping $S$, a Hopf algebra  generalizes the notion
of the group algebra (for which $S(h) = h^{-1}$), although by itself it obviously need
not be a group algebra. In accordance with Drinfeld's definition [13] the concepts
of a Hopf algebra and a quantum group are more or less equivalent. Of course, the most
interesting examples of quantum groups arise when one considers
noncommutative and noncocommutative Hopf algebras.
\vspace{0.2cm}

Consider a noncommutative Hopf algebra ${\cal A}$ which is also noncocommutative
$\Delta \neq \Delta^{\sf op} \equiv \sigma \Delta$,
where $\sigma$ is the transposition operator
$\sigma(a \otimes b) = b \otimes a$ ($\forall a,b \in {\cal A}$).
%{\bf Definition 5.}
\newtheorem{def5}[def1]{Definition}
\begin{def5} \label{def5}
{\it A Hopf algebra ${\cal A}$ for which there exists an invertible element
	${\cal R} \in {\cal A} \otimes
	{\cal A}$ such that
	\be
	\lb{2.26}
	\Delta^{\sf op}(a) = {\cal R} \Delta(a) {\cal R}^{-1} \; ,
	\;\;\;\;\; \forall a \in {\cal A} \; ,
	\ee
	\be
	\lb{2.27}
	(\Delta \otimes id) ({\cal R}) = {\cal R}_{13} {\cal R}_{23} \; , \; \;
	(id \otimes \Delta) ({\cal R}) = {\cal R}_{13} {\cal R}_{12}
	\ee
	is called quasitriangular. Here the element
	\be
	\lb{2.28}
	{\cal R}=\sum_{ij} R^{(ij)} e_{i} \otimes e_{j}
	\ee
	is called the universal ${\cal R}$ matrix,
	$R^{(ij)} \in \mathbb{C}$ are
	the constants and the symbols ${\cal R}_{12}, {\cal R}_{13},
	{\cal R}_{23}$ have the meaning
	\be
	\lb{2.30a}
	{\cal R}_{12} = \sum_{ij} R^{(ij)} e_{i} \otimes e_{j}
	\otimes I \; , \;\;
	{\cal R}_{13} = \sum_{ij} R^{(ij)} e_{i} \otimes I \otimes e_{j}  \; , \;\;
	{\cal R}_{23} =
	\sum_{ij} R^{(ij)} I \otimes e_{i} \otimes e_{j} \; .
	\ee}
	\end{def5}
	The relation (\ref{2.26}) shows that the noncocommutativity in a quasitriangular
	Hopf algebra is kept "under control." It can be shown \cite{13'} that for such
	a Hopf algebra the universal ${\cal R}$ matrix (\ref{2.28}) satisfies
	the Yang-Baxter equation
	\be
	\lb{2.30}
	{\cal R}_{12}{\cal R}_{13}{\cal R}_{23} = {\cal R}_{23}{\cal R}_{13}{\cal R}_{12} \; ,
	\ee
	(to which a considerable part of the review will be devoted)
	and the relations
	\be
	\lb{2.29}
	(id \otimes \epsilon){\cal R} = (\epsilon \otimes id){\cal R} = I \; ,\;\;
	\ee
	\be
	\lb{2.29a}
	\begin{array}{c}
(S \otimes id){\cal R} = {\cal R}^{-1} \;\; \Leftrightarrow \;\;
(S^{-1} \otimes id){\cal R}^{-1} = {\cal R} \; , \\ [0.2cm]
(id \otimes S){\cal R}^{-1} = {\cal R}
\;\; \Leftrightarrow \;\;
(id \otimes S^{-1}){\cal R} = {\cal R}^{-1}  \; .
\end{array}
\ee
The Yang-Baxter eq. (\ref{2.30}) follows from
(\ref{2.26}) and (\ref{2.27}):
\be
\lb{2.31}
{\cal R}_{12} \, {\cal R}_{13} \, {\cal R}_{23} =
{\cal R}_{12} \; (\Delta \otimes id)({\cal R}) =
(\Delta^{\sf op} \otimes id)({\cal R})\; {\cal R}_{12} =
{\cal R}_{23} \, {\cal R}_{13} \, {\cal R}_{12} \; .
\ee
It is easy to derive the relations (\ref{2.29}) by applying $(\epsilon \otimes id \otimes id)$
and
$(id \otimes id \otimes \epsilon)$ respectively to
the first and second relation in (\ref{2.27}),
and then taking into account (\ref{2.8}).
Next, we prove the equalities in (\ref{2.29a}).
We consider expressions
${\cal R} \cdot (S \otimes id) \, {\cal R}$ and
${\cal R} \cdot (id \otimes S^{-1}) \, {\cal R}$
and make use of the
Hopf algebra axioms (\ref{2.21})
and equations (\ref{2.27}), (\ref{2.29}):
$$
\begin{array}{c}
{\cal R}_{23} \cdot (id \otimes S \otimes id) \, {\cal R}_{23} =
(m_{12} \otimes id_3) \bigl( {\cal R}_{13} \,
(id \otimes S \otimes id) {\cal R}_{23} \bigr) = \\ [0.1cm]
= (m_{12} \otimes id_3)\,
(id \otimes S \otimes id) {\cal R}_{13} \, {\cal R}_{23} =
(m_{12} \otimes id_3)\,
\bigl( (id \otimes S)\Delta \otimes id \bigr){\cal R} =
(\hbox{\bf i} \cdot \epsilon \otimes id) \, {\cal R} = I \; ,
\end{array}
$$
$$
\begin{array}{c}
{\cal R}_{12} \, (id \otimes S^{-1}) \, {\cal R}_{12} =
(id_1  \otimes   m_{23})\,
(id \otimes id \otimes S^{-1} ) {\cal R}_{12} \, {\cal R}_{13} =
\\ [0.2cm]
= (id_1  \otimes   m_{23})\,
\bigl(id \otimes  (id \otimes S^{-1})\Delta^{\sf op} \bigr){\cal R} =
(id \otimes \hbox{\bf i} \cdot \epsilon) \, {\cal R} = I \; ,
\end{array}
$$
where the ultimate equality follows from (\ref{2.21})
which is written in the form
$a_{(2)} S^{-1}(a_{(1)}) = \epsilon(a) I$.


The next important concept that we shall need in what follows
is the concept of the Hopf algebra ${\cal A}^{*}$ that is the dual of the
Hopf algebra ${\cal A}$.
We choose in ${\cal A}^{*}$ basis elements $\{ e^{i} \}$ and define multiplication, the identity,
comultiplication, the coidentity, and the antipode for ${\cal A}^{*}$ in the form
\be
\lb{2.33}
e^{i} e^{j}  = m^{ij}_{k} e^{k} \; , \;\; I = \bar{E}_{i}e^{i} \; , \;\;
\Delta(e^{i}) = \Delta^{i}_{jk} e^{j} \otimes e^{k} \; , \;\;
\epsilon(e^{i}) = \bar{\epsilon}^{i} \; , \;\;
S(e^{i}) = \bar{S}^{i}_{j} e^{j} \; .
\ee

%{\bf Definition 6.}

\newtheorem{def6}[def1]{Definition}
\begin{def6} \label{def6}
{\it Two Hopf algebras ${\cal A}$ and ${\cal A}^{*}$ with corresponding bases
	$\{e_{i} \}$ and $\{e^{i} \}$ are
	said to be dual to each other if there exists a
	nondegenerate pairing $\langle  . | . \rangle $:
	${\cal A}^{*} \otimes {\cal A} \rightarrow \mathbb{C}$ such that
	\be
	\lb{2.34}
	\begin{array}{c}
		\langle  e^{i}e^{j} | e_{k}\rangle  \equiv \langle  e^{i} \otimes e^{j} | \Delta(e_{k})\rangle  =
		\langle  e^{i} | e_{k'}\rangle  \Delta_{k}^{k'k''} \langle  e^{j} | e_{k''}\rangle  \\ \\
		\langle  e^{i} | e_{j} e_{k}\rangle  \equiv \langle  \Delta(e^{i})| e_{j} \otimes e_{k}\rangle  =
		\langle  e^{i'} | e_{j}\rangle  \Delta^{i}_{i'i''} \langle  e^{i''} | e_{k}\rangle  \\ \\
		\langle S(e^{i}) | e_{j} \rangle  = \langle e^{i} | S(e_{j}) \rangle  \; , \;\;
		\langle  e^{i} | I \rangle  = \epsilon (e^{i}) \; , \;\;
		\langle  I | e_{i} \rangle  = \epsilon (e_{i}) \; .
	\end{array}
	\ee}
	\end{def6}
	Since the pairing $\langle .|.\rangle $ (\ref{2.34}) is nondegenerate,
	we can always choose basis elements $\{ e^{i} \}$ such that
	\be
	\lb{2.35}
	\langle  e^{i} | e_{j} \rangle  = \delta^{i}_{j} \; .
	\ee
	Then from the axioms for the pairing (\ref{2.34}) and from the
	definitions of the structure maps (\ref{2.1}), (\ref{2.23}), and (\ref{2.33}) in
	${\cal A}$ and ${\cal A}^{*}$ we readily deduce
	\be
	\lb{2.36}
	m^{ij}_{k} = \Delta^{ij}_{k} \; , \;\;
	m_{ij}^{k} = \Delta_{ij}^{k} \; , \;\;
	\bar{S}^{i}_{j} = S^{i}_{j} \; , \;\;
	\bar{\epsilon}^{i} = E^{i} \; , \;\;
	\bar{E}_{i} = \epsilon_{i} \; .
	\ee
	Thus, the multiplication, identity, comultiplication, coidentity,
	and antipode in a Hopf algebra define, respectively, comultiplication,
	coidentity, multiplication, identity, and antipode in the dual Hopf algebra.
	
	%\newpage
	\vspace{0.3cm}
	
	%\subsection{Example: the group algebra of finite
%group and its algebra of functions}\setcounter{equation}0

\noindent
{\bf Remark.} In \cite{Pontr} L.S. Pontryagin showed that the set of characters of an abelian locally compact
group $G$ is an abelian group, called the dual group $G^*$ of $G$. The group $G^*$ is also locally
compact. Moreover, the dual group of $G^*$ is isomorphic to $G$. This beautiful theory becomes
wrong if $G$ is a noncommutative group, even if it is finite. To restore the duality principle one
can replace the set of characters for a finite noncommutative group $G$ by the category of its
irreducible representations (irreducible representations for the commutative groups are exactly
characters). Indeed, T. Tannaka and M. Krein showed that the compact group $G$ can be recovered
from the set of its irreducible unitary representations. They proved a duality theorem for compact
groups, involving irreducible representations of $G$ (although no group-like structure is to be put
on that class, since the tensor product of two irreducible representation may no longer be
irreducible). However, the tensor product of two irreducible representations of the compact group
$G$ can be expanded as a
sum of irreducible representations and, thus, the dual object has the structure of an algebra. Recall (see (\ref{2.15a})) that
matrix representations of group $G$ are realized by the sets
of special functions $T^i_{\; k}$.
One can consider the group algebra ${\cal G}$ of finite group $G$
and the algebra ${\cal A}(G) \equiv {\cal G}^\star$ of functions on the
group $G$ as simplest
examples of the Hopf algebras. The structure mappings for these algebras
have been defined in (\ref{2.15}), (\ref{2.15b}) and (\ref{2.25a}).
Note that the algebras ${\cal G}$ and ${\cal G}^\star$
are Hopf dual to each other.
The detailed structure of ${\cal G}^\star$ follows
from the representation theory of finite groups
(see e.g. \cite{Sierre}).



\subsection{\bf \em Heisenberg and Quantum doubles.
Yetter--Drinfeld modules\label{hopf4}}
\setcounter{equation}0

In Subsection 2.2 we have defined (see Definition \ref{def3}) the notion of the smash
(cross) product of the bialgebra and its module algebra.
Since the Hopf dual algebra ${\cal A}^*$ is the natural right and left module
algebra for the Hopf algebra ${\cal A}$ (\ref{2.10}), (\ref{2.11}),
one can immediately define the right
${\cal A}^* \sharp {\cal A}$ and the left ${\cal A} \sharp {\cal A}^*$
cross products of the algebra ${\cal A}$ on ${\cal A}^*$.
These cross-product algebras are called
Heisenberg doubles of ${\cal A}$
and they are the associative algebras with
nontrivial cross-multiplication rules (cf. eq. (\ref{2.18})):
\begin{equation}
\label{25d}
a \, \bar{a} = (a_{(1)} \triangleright \bar{a}) \,
a_{(2)} = \bar{a}_{(1)} \, \langle a_{(1)} \, | \, \bar{a}_{(2)} \rangle \, a_{(2)}  \; ,
\end{equation}
\begin{equation}
\label{25dd}
\bar{a} \, a = a_{(1)} (\bar{a} \triangleleft a_{(2)}) \,
= a_{(1)} \, \langle \bar{a}_{(1)} \, | \, a_{(2)} \rangle \, \bar{a}_{(2)}  \; ,
\end{equation}
where $a \in {\cal A}$ and $\bar{a} \in {\cal A}^*$.
Here we discuss only the left cross product algebra ${\cal A} \sharp {\cal A}^*$
(\ref{25d}) (the other one (\ref{25dd})
is considered analogously).

As in the previous subsection, we denote $\{ e^{i} \}$ and $\{ e_{i} \}$
the dual basis elements of ${\cal A}^*$ and ${\cal A}$, respectively.
In terms of this basis we rewrite (\ref{25d}) in the form
\begin{equation}
\label{25f}
e_r \, e^n =  e^i \, \Delta^n_{if} \langle e_j | e^f \rangle  \, \Delta^{jk}_r  \, e_k
= m^n_{ij} \,  e^i \, e_k \, \Delta^{jk}_r  \; .
\end{equation}
Let us define a right ${\cal A}^*$ - coaction and
a left ${\cal A}$ - coaction
on the algebra ${\cal A} \sharp {\cal A}^*$, such that these coactions
respect the algebra structure of ${\cal A} \sharp {\cal A}^*$:
\begin{equation}
\label{26} \Delta_R( z) = C \, (z \otimes 1) \, C^{-1} \; , \;\;\;
\Delta_L( z) = C^{-1} \, (1 \otimes z) \, C \; , \;\;\;
C \equiv e_{i} \otimes e^{i} \; .
\end{equation}
The inverse of the canonical element $C$ is
$$
C^{-1} = S(e_{i}) \otimes e^{i} = e_{i} \otimes S(e^{i}) \; ,
$$
and $\Delta_R$, $\Delta_L$ (\ref{26}) are represented in the form
\begin{equation}
\label{27}
\Delta_R( z) =
(e_{k(1)} \, z \, S(e_{k(2)})) \otimes e^{k} \; , \;\;\;
\Delta_L( z) =
e_{k} \otimes  S(e^k_{(1)}) \, z \, e^k_{(2)}
\; .
\end{equation}
Note that $\Delta_R(\bar{z}) = \Delta(\bar{z})$ $\forall \bar{z} \in {\cal A}^*$
and $\Delta_L(z) = \Delta(z)$ $\forall z \in {\cal A}$
(here ${\cal A}$ and ${\cal A}^*$ are understood as the Hopf subalgebras
in ${\cal A} \sharp {\cal A}^*$ and $\Delta$ are corresponding
comultiplications). Indeed, for $z \in {\cal A}$ we have
$$
\Delta_L(z) = e_{k} \otimes  S(e^k_{(1)}) \, z \, e^k_{(2)} =
e_{k} \otimes  S(e^k_{(1)}) \, e^k_{(2)} \, \langle z_{(1)} \, | \, e^k_{(3)} \rangle
z_{(2)} =
$$
$$
= e_{k} \,  \langle z_{(1)} \, | \, e^k \rangle \otimes
z_{(2)} = z_{(1)} \otimes  z_{(2)} \; ,
$$
(the proof of $\Delta_R(\bar{z}) = \Delta(\bar{z})$ is similar).
The axioms
$$
(id \otimes \Delta)\Delta_R = (\Delta_R \otimes id)\Delta_R
\;\; , \;\;\;
(id \otimes \Delta_L)\Delta_L = (\Delta \otimes id)\Delta_L \; ,
$$
$$
(id \otimes \Delta_R)\Delta_L(z) = C^{-1}_{13} \,
(\Delta_L \otimes id)\Delta_R(z) \, C_{13} \; ,
$$
can be verified directly by using relations (cf. (\ref{2.27}))
$$
(id \otimes \Delta) C_{12} = C_{13} \, C_{23} \; , \;\;\;
(\Delta \otimes id) C_{12} = C_{13} \, C_{23} \; ,
$$
and the pentagon identity \cite{BSk} for $C$
\be
\lb{c5}
C_{12} \, C_{13} \, C_{23} = C_{23} \, C_{12} \; .
\ee
The proof of (\ref{c5}) is straightforward (see (\ref{25f})):
$$
C_{12} \, C_{13} \, C_{23} = e_i \, e_j \otimes e^i \, e_k \otimes e^j \, e^k =
e_n \otimes  m^n_{ij} \,  e^i \, e_k \, \Delta^{jk}_r \otimes e^r =
$$
$$
= e_n \otimes  e_r \, e^n  \otimes e^r = C_{23} \, C_{12} \; .
$$
The pentagon identity (\ref{c5}) is used for the construction
of the explicit solutions of the tetrahedron equations
(3D generalizations of the Yang-Baxter equations).

Although ${\cal A}$ and ${\cal A}^*$ are Hopf algebras, their Heisenberg
doubles  ${\cal A} \sharp {\cal A}^*$, ${\cal A}^* \sharp {\cal A}$ are not Hopf algebras. But
as we have seen just before the algebra ${\cal A} \sharp {\cal A}^*$
(as well as ${\cal A}^* \sharp {\cal A}$) still possesses some
covariance properties, since the coactions (\ref{26}) are
covariant transformations (homomorphisms) of the algebra
${\cal A} \sharp {\cal A}^*$.

The natural question is the following: is it possible to invent such
a cross-product of the Hopf algebra and its dual Hopf algebra to
obtain a new Hopf algebra?
Drinfeld \cite{13} showed that there exists a quasitriangular
Hopf algebra ${\cal D}({\cal A})$ that is a special smash product of the Hopf
algebras ${\cal A}$ and ${\cal A}^{o}$:
${\cal D}({\cal A}) = {\cal A} \Join {\cal A}^{o}$, which is called the quantum double.
Here we denote by ${\cal A}^{o}$ the algebra ${\cal A}^{*}$ with opposite
comultiplication: $\Delta(e^{i}) = m^{i}_{kj} e^{j} \otimes e^{k}$,
${\cal A}^o = ({\cal A}^*)^{cop}$.
It follows
from (\ref{2.25}) that the antipode for ${\cal A}^{o}$ will be not $S$ but
the skew antipode $S^{-1}$. Thus, the structure mappings for ${\cal A}^{o}$ have the form
\be
\lb{2.37}
e^{i}e^{j} = \Delta^{ij}_{k} e^{k} \; , \; \;
\Delta(e^{i}) = m^{i}_{kj} e^{j} \otimes e^{k} \; , \; \;
S(e^{i}) = (S^{-1})^{i}_{j} e^{j} \; .
\ee
The algebras ${\cal A}$ and ${\cal A}^{o}$ are said to be antidual,
and for them we can introduce the antidual
pairing $\langle \langle .|.\rangle \rangle $:
${\cal A}^{o} \otimes {\cal A} \rightarrow \mathbb{C}$, which satisfies the conditions
\be
\lb{2.38}
\begin{array}{c}
\langle \langle  e^{i}e^{j} | e_{k}\rangle \rangle  \equiv \langle \langle  e^{i} \otimes e^{j} | \Delta(e_{k})\rangle \rangle  =
\Delta_{k}^{ij}  \; , \\ \\
\langle \langle  e^{i} | e_{k} e_{j}\rangle \rangle  \equiv \langle \langle  \Delta(e^{i})| e_{j} \otimes e_{k}\rangle \rangle  =
m^{i}_{kj}  \; , \\ \\
\langle \langle S(e^{i}) | e_{j} \rangle \rangle  = \langle \langle e^{i} | S^{-1}(e_{j}) \rangle \rangle
= (S^{-1})^{i}_{j} \; , \;\; \\ \\
\langle \langle e^{i} | S(e_{j}) \rangle \rangle  = \langle \langle S^{-1}(e^{i}) | e_{j}\rangle \rangle  = S^{i}_{j}
\; , \;\; \\ \\
\langle \langle  e^{i} | I \rangle \rangle  = E^{i} \; , \;\;
\langle \langle  I | e_{i} \rangle \rangle  = \epsilon_{i} \; .
\end{array}
\ee
The universal $R$-matrix can be expressed in the form
of the canonical element
\be
\lb{2.39}
{\cal R} = (e_{i} \Join I) \otimes (I \Join e^{i}) \; ,
\ee
and the multiplication in ${\cal D}({\cal A})$
is defined in accordance with (the summation signs are omitted)
\be
\lb{2.40}
(a \Join \alpha)(b \Join \beta) =
a \left( (\alpha_{(3)} \triangleright b) \triangleleft
S(\alpha_{(1)}) \right) \Join \alpha_{(2)} \beta \; ,
\ee
where $\alpha, \beta \in {\cal A}^{o}$,
$a,b \in {\cal A}$, $\Delta^{2}(\alpha) =
\alpha_{(1)} \otimes \alpha_{(2)} \otimes \alpha_{(3)}$ and
\be
\lb{2.41}
\alpha \triangleright b = b_{(1)} \langle \langle \alpha | b_{(2)} \rangle \rangle  \; , \;\;
b \triangleleft \alpha = \langle \langle \alpha | b_{(1)} \rangle \rangle  b_{(2)}\; .
\ee

The coalgebraic structure on the quantum double is defined by
the direct product of the coalgebraic structures on the Hopf algebras
${\cal A}$ and ${\cal A}^{o}$:
\be
\lb{2.42}
\Delta(e_{i} \Join e^{j}) =
\Delta(e_{i} \Join I) \Delta(I \Join e^{j}) =
\Delta^{nk}_{i} m^{j}_{lp}(e_{n} \Join e^{p})
\otimes (e_{k} \Join e^{l}) \; .
\ee
Finally, the antipode and coidentity for ${\cal D}({\cal A})$ have the form
\be
\lb{2.43}
S(a \Join \alpha) = S(a) \Join S(\alpha) \; , \;\;
\epsilon(a \Join \alpha) = \epsilon(a) \epsilon(\alpha)  \; .
\ee

All the axioms of a Hopf algebra can be verified for ${\cal D}({\cal A})$
by direct calculation. A simple proof of the associativity of the
multiplication (\ref{2.40}) and the coassociativity of the
comultiplication (\ref{2.42}) can be found in Ref. \cite{15}.

Taking into account (\ref{2.41}), we can rewrite (\ref{2.40}) as the
commutator for the elements $(I \Join \alpha)$ and $(b \Join I)$:
\be
\lb{2.44a}
(I \Join \alpha)(b \Join I) = \langle \langle  S( \alpha_{(1)}) | b_{(1)} \rangle \rangle
(b_{(2)} \Join I) (I \Join \alpha_{(2)})
\langle \langle   \alpha_{(3)} | b_{(3)} \rangle \rangle
\ee
or, in terms of the basis elements
$\alpha = e^{t}$ and
$b = e_{s}$ we have \cite{13}
\be
\lb{2.44}
\begin{array}{c}
(I \Join e^{t})(e_{s} \Join I) =  m^{t}_{klp} \Delta_{s}^{njk}
(S^{-1})^{p}_{n} (e_{j} \Join I)(I \Join e^{l}) \equiv \\ \\
\left( m^{t}_{ip}(S^{-1})^{p}_{n} \Delta^{nr}_{s} \right)
\left( m^{i}_{kl} \Delta^{jk}_{r} \right)
(e_{j} \Join I)(I \Join e^{l}) \; ,
\end{array}
\ee
where $m^{t}_{klp}$ and $\Delta^{njk}_{s}$ are defined in (\ref{2.3}) and (\ref{2.7}),
and $(S^{-1})^{p}_{n}$  is the matrix of the skew antipode.

The consistence of definitions of left and right bimodules over the quantum double ${\cal D}({\cal A})$
should be clarified in view of the nontrivial
structure of the cross-multiplication rule (\ref{2.44a}), (\ref{2.44})
for subalgebras ${\cal A}$ and ${\cal A}^o$. It can be done (see, e.g. \cite{Rosso2}) if one considers
left or right coinvariant bimodules (Hopf modules): $M^L =  \{ m: \;\; \Delta_L(m) = 1 \otimes m  \}$ or
$M^R =  \{ m: \;\; \Delta_R(m) = m \otimes 1 \}$. E.g., for $M^R$ one can define the left ${\cal A}$
and left ${\cal A}^o$-module actions as
\be
\lb{2.44b}
a \triangleright m = a_{(1)} \, m \, S(a_{(2)}) \; ,
\ee
\be
\lb{2.44bb}
\alpha \triangleright m = \langle \langle S(\alpha) , m_{(-1)} \rangle \rangle \, m_{(0)} \; ,
\ee
where  $\Delta_L(m) = m_{(-1)} \otimes m_{(0)}$ is the left  ${\cal A}$-coaction on $M^R$
and $a \in {\cal A}$, $\alpha \in {\cal A}^o$.
Note that  left ${\cal A}$-module action (\ref{2.44b}) respects the right coinvariance of $M^R$.
The compatibility condition for the left ${\cal A}$-action (\ref{2.44b})
and left ${\cal A}$-coaction  $\Delta_L$
is written in the form (we represent $\Delta_L(a \triangleright m)$ in two different ways):
\be
\lb{2.44c}
(a \triangleright m)_{(-1)} \otimes (a \triangleright m)_{(0)} =
a_{(1)} \, m_{(-1)} \, S(a_{(3)})
\otimes a_{(2)} \triangleright m_{(0)} \; .
\ee
A module with the property (\ref{2.44c}) is called Yetter-Drinfeld module. Then,
using (\ref{2.44b}), (\ref{2.44c}) and opposite coproduct for ${\cal A}^o$, we obtain
\be
\lb{2.44d}
\begin{array}{c}
\alpha \triangleright (a \triangleright m) = \alpha \triangleright (a_{(1)} \, m \, S(a_{(2)})) =
\langle \langle S(\alpha) , a_{(1)} \, m_{(-1)} \, S(a_{(3)})  \rangle \rangle \,
a_{(2)} \triangleright  m_{(0)} = \\ \\
=  \langle \langle S(\alpha_{(1)}) , \, a_{(1)} \rangle \rangle
\,\langle \langle \alpha_{(3)}, \,   a_{(3)}  \rangle \rangle \,
a_{(2)} \triangleright  (\alpha_{(2)} \triangleright m ) \; ,
\end{array}
\ee
and one can recognize in eq. (\ref{2.44d}) the quantum double multiplication
formula (\ref{2.44a}).

It follows from Eqs. (\ref{2.3}), (\ref{2.7}) and from the identities for the skew antipode
(\ref{2.25}) that
\be
\lb{2.45}
\left( m^{q}_{tk} \Delta^{ks}_{m} \right)
\left( m^{t}_{ip}(S^{-1})^{p}_{n} \Delta^{nr}_{s} \right) =
\delta^{q}_{i}\delta^{r}_{m} \; ,
\ee
and this enables us to rewrite (\ref{2.44}) in the form
$$
\left( m^{q}_{tk} \Delta^{ks}_{m} \right)
(I \Join e^{t})(e_{s} \Join I) =  ( m^{q}_{kl} \Delta_{m}^{jk} )
(e_{j} \Join I)(I \Join e^{l}) \; .
$$
This equation is equivalent to the axiom (\ref{2.26}) for the universal matrix ${\cal R}$ (\ref{2.39}).
The relations (\ref{2.27}) for ${\cal R}$ (\ref{2.39}) are readily verified.
Thus, ${\cal D}({\cal A})$ is indeed a quasitriangular Hopf algebra with universal
${\cal R}$ matrix represented by (\ref{2.39}).

In conclusion, we note that many relations for the structure constants
of Hopf algebras [for example, the relation (2.45)] can be obtained
and represented in a transparent form by means of the following diagrammatic technique: \\
\unitlength=1cm
\begin{picture}(15,4)
\put(0,2){$\Delta_{k}^{ij}=$}
\put(2,2.1){\line(-1,-1){0.6}}
\put(1.7,1.8){\vector(-1,-1){0.1}}
\put(1,1.2){$i$}
\put(2,2.1){\line(0,1){0.8}}
\put(2,2.5){\vector(0,-1){0.1}}
\put(2,3){$k$}
\put(2,2.1){\line(1,-1){0.6}}
\put(2.3,1.8){\vector(1,-1){0.1}}
\put(2.7,1.2){$j$}
\put(3.5,2){$m^{k}_{ij}=$}
\put(5.5,2.2){\line(1,1){0.6}}
\put(5.8,2.5){\vector(-1,-1){0.1}}
\put(6.2,2.9){$j$}
\put(5.5,2.2){\line(0,-1){0.8}}
\put(5.5,1.8){\vector(0,-1){0.1}}
\put(5.45,1){$k$}
\put(5.5,2.2){\line(-1,1){0.6}}
\put(5.2,2.5){\vector(1,-1){0.1}}
\put(4.9,2.9){$i$}
\put(7.5,2){$\epsilon_{i} =$}
\put(8.8,1.9){\line(0,1){0.7}}
\put(8.8,2.3){\vector(0,-1){0.1}}
\put(8.8,1.7){\circle{0.4}}
\put(8.75,1.6){$\epsilon$}
\put(8.8,2.7){$i$}
\put(10,2){$E^{i} =$}
\put(11.3,2.3){\line(0,-1){0.7}}
\put(11.3,2){\vector(0,-1){0.1}}
\put(11.3,2.5){\circle{0.4}}
\put(11.3,1.2){$i$}
\put(12.5,2){$S^{i}_{j} =$}
\put(13.8,2.4){\line(0,1){0.6}}
\put(13.8,2.75){\vector(0,-1){0.1}}
\put(13.8,3.2){$j$}
\put(13.8,2.2){\circle{0.4}}
\put(13.8,2){\line(0,-1){0.6}}
\put(13.8,1.75){\vector(0,-1){0.1}}
\put(13.7,2.1){$s$}
\put(13.8,1){$i$}
\end{picture}

For example, the axioms of associativity (2.3) and coassociativity
(\ref{2.7}) and the axioms for the antipode (\ref{2.24}) can be
represented in the form

\unitlength=0.7cm
\begin{picture}(20,5)
\put(1,2.1){\line(-1,-1){0.6}}
\put(0.7,1.8){\vector(-1,-1){0.1}}
\put(0,1.2){$n$}
\put(1,2.1){\line(0,1){0.8}}
\put(1,2.5){\vector(0,-1){0.1}}
\put(1.2,2.3){$l$}
\put(1,2.1){\line(1,-1){0.6}}
\put(1.3,1.8){\vector(-1,1){0.1}}
\put(1.7,1.2){$k$}
\put(1,2.9){\line(1,1){0.6}}
\put(1.3,3.2){\vector(-1,-1){0.1}}
\put(1.7,3.6){$j$}
\put(1,2.9){\line(-1,1){0.6}}
\put(0.7,3.2){\vector(1,-1){0.1}}
\put(0.4,3.6){$i$}
\put(2.2,2.3){$=$}
\put(3.7,2.4){\line(-1,-1){0.6}}
\put(3.4,2.1){\vector(-1,-1){0.1}}
\put(2.8,1.5){$n$}
\put(3.7,2.4){\line(1,0){0.8}}
\put(4.1,2.4){\vector(-1,0){0.1}}
\put(4,2.6){$l$}
\put(4.5,2.4){\line(1,-1){0.6}}
\put(4.8,2.1){\vector(-1,1){0.1}}
\put(5.1,1.5){$k$}
\put(4.5,2.4){\line(1,1){0.6}}
\put(4.8,2.7){\vector(-1,-1){0.1}}
\put(5.1,3.3){$j$}
\put(3.7,2.4){\line(-1,1){0.6}}
\put(3.4,2.7){\vector(1,-1){0.1}}
\put(2.8,3.3){$i$}

\put(7.5,2.1){\line(-1,-1){0.6}}
\put(7.2,1.8){\vector(-1,-1){0.1}}
\put(6.5,1.2){$n$}
\put(7.5,2.1){\line(0,1){0.8}}
\put(7.5,2.5){\vector(0,-1){0.1}}
\put(7.7,2.3){$l$}
\put(7.5,2.1){\line(1,-1){0.6}}
\put(7.8,1.8){\vector(1,-1){0.1}}
\put(8.2,1.2){$k$}
\put(7.5,2.9){\line(1,1){0.6}}
\put(7.8,3.2){\vector(1,1){0.1}}
\put(8.2,3.6){$j$}
\put(7.5,2.9){\line(-1,1){0.6}}
\put(7.2,3.2){\vector(1,-1){0.1}}
\put(6.9,3.6){$i$}
\put(8.7,2.3){$=$}
\put(10.2,2.4){\line(-1,-1){0.6}}
\put(9.9,2.1){\vector(-1,-1){0.1}}
\put(9.3,1.5){$n$}
\put(10.2,2.4){\line(1,0){0.8}}
\put(10.6,2.4){\vector(1,0){0.1}}
\put(10.5,2.6){$l$}
\put(11,2.4){\line(1,-1){0.6}}
\put(11.3,2.1){\vector(1,-1){0.1}}
\put(11.6,1.5){$k$}
\put(11,2.4){\line(1,1){0.6}}
\put(11.3,2.7){\vector(1,1){0.1}}
\put(11.6,3.3){$j$}
\put(10.2,2.4){\line(-1,1){0.6}}
\put(9.9,2.7){\vector(1,-1){0.1}}
\put(9.3,3.3){$i$}

\put(13.8,2.5){\line(0,1){1.3}}
\put(13.8,2.8){\vector(0,-1){0.1}}
\put(13.8,3.5){\vector(0,-1){0.1}}
\put(13.8,4){$k$}
\put(13.8,2.2){\circle{0.6}}
\put(13.8,1.9){\line(0,-1){1.3}}
\put(13.7,2.1){$s$}
\put(13.8,0.1){$l$}
\put(13.8,2.2){\oval(1.5,2)[l]}
\put(13.05,2.2){\vector(0,-1){0.1}}
\put(13.8,1.5){\vector(0,-1){0.1}}
\put(13.8,0.9){\vector(0,-1){0.1}}

\put(14.5,2.1){$=$}
\put(15.7,2.5){\line(0,1){1.3}}
\put(15.7,2.8){\vector(0,-1){0.1}}
\put(15.7,3.5){\vector(0,-1){0.1}}
\put(15.7,4){$k$}
\put(15.7,2.2){\circle{0.6}}
\put(15.7,1.9){\line(0,-1){1.3}}
\put(15.6,2.1){$s$}
\put(15.7,0.1){$l$}
\put(15.7,2.2){\oval(1.5,2)[r]}
\put(16.45,2.2){\vector(0,-1){0.1}}
\put(15.7,1.5){\vector(0,-1){0.1}}
\put(15.7,0.9){\vector(0,-1){0.1}}

\put(17,2.1){$=$}
\put(18.2,2.95){\line(0,1){0.9}}
\put(18.2,3.4){\vector(0,-1){0.1}}
\put(18.2,2.7){\circle{0.5}}
\put(18.1,2.6){$\epsilon$}
\put(18.2,4){$k$}

\put(18.2,1.65){\line(0,-1){1}}
\put(18.2,1.15){\vector(0,-1){0.1}}
\put(18.2,1.9){\circle{0.5}}
\put(18.2,0.2){$l$}
\end{picture}


Now we make three important remarks relating to the further
development of the theory of Hopf algebras.

\subsection{\bf \em Twisted, ribbon and quasi-Hopf
algebras\label{trqH}}
\setcounter{equation}0

\noindent
{\bf Remark 1.}{\it Twisted Hopf algebras.} \\
Consider a Hopf algebra
${\cal A}$ $(\Delta, \, \epsilon, \, S)$.
Let ${\cal F}$
be an invertible element of ${\cal A} \otimes {\cal A}$ such that:
\be
\lb{ef}
(\epsilon \otimes id) {\cal F} = 1 = (id \otimes \epsilon ) {\cal F} \;  ,
\ee
and we denote ${\cal F} = \sum_i \alpha_i \otimes \beta_i$,
${\cal F}^{-1} = \sum_i \, \gamma_i \otimes \delta_i$,
$I \equiv 1$.
Following the twisting procedure \cite{17}
one can define a new Hopf algebra
${\cal A}^{(F)}$
$(\Delta^{(F)}, \, \epsilon^{(F)}, \, S^{(F)})$ (twisted Hopf algebra)
with the new structure mappings
\be
\lb{2}
\Delta^{(F)}(a) = {\cal F} \, \Delta(a) \, {\cal F}^{-1} \; ,
\ee
\be
\lb{222}
\epsilon^{(F)}(a) = \epsilon (a) \; , \;\;\;  S^{(F)}(a) = U \, S(a) \, U^{-1}  \;\;\;
(\forall a \in {\cal A}) \; ,
\ee
where the twisting element ${\cal F}$ satisfies the cocycle equation
\be
\lb{cocycl}
{\cal F}_{12} \, (\Delta \otimes id) {\cal F} =
{\cal F}_{23} \, (id \otimes \Delta) {\cal F} \; ,
\ee
and the element $U =  \alpha_i \, S(\beta_i)$ is invertible and obeys
\be
\lb{U}
U^{-1} =  S(\gamma_i) \, \delta_i
\; , \;\;\;
S(\alpha_i) \, U^{-1} \, \beta_i = 1 \; ,
\ee
(the summation over $i$ is assumed). First of all we show
that the algebra ${\cal A}^{(F)}$
$(\Delta^{(F)}, \, \epsilon)$ is a bialgebra. Indeed,
the cocycle equation (\ref{cocycl})
guarantees the coassociativity condition (\ref{2.7})
for the new coproduct $\Delta^{(F)}$ (\ref{2}).
Then the axioms for counit $\epsilon$ (\ref{2.8}) are easily deduced from
(\ref{ef}). Considering the identity
$$
m(id \otimes S \otimes id) \left( {\cal F}^{-1}_{23} \, {\cal F}_{12} \,
(\Delta \otimes id) {\cal F} \right) = m(id \otimes S \otimes id)
(id \otimes \Delta ) {\cal F}
$$
we obtain the form for $U^{-1}$ (\ref{U}).
The second relation in (\ref{U}) is obtained from the identity:
$m(S \otimes id) {\cal F}^{-1}{\cal F} = 1$.

Now the new antipode $S^{(F)}$ (\ref{222}) follows from equation
$$
m (id \otimes S) \, (\Delta^{(F)}(a) \, {\cal F}) =
m (id \otimes S) \, ({\cal F} \, \Delta(a) ) \; ,
$$
which is rewritten in the form $\tilde{a}_{(1)} U S(\tilde{a}_{(2)}) = \epsilon(a) \, U$,
where $\Delta^{(F)}(a) = \tilde{a}_{(1)} \otimes \tilde{a}_{(2)}$.

If the algebra ${\cal A}$ is a quasitriangular Hopf algebra with the universal
${\cal R}$ matrix
(\ref{2.26}), then the new Hopf algebra ${\cal A}^{(F)}$ is also quasitriangular
and a new universal ${\cal R}$-matrix is
\be
\lb{2a}
{\cal R}^{(F)} = {\cal F}_{21} \, {\cal R} \, {\cal F}^{-1} \; ,
\ee
since we have
$$
{\Delta^{(F)}}' = {\cal F}_{21} \, \Delta^{\sf op} \, {\cal F}_{21}^{-1} =
{\cal F}_{21} \, {\cal R} \, \Delta \, {\cal R}^{-1} \, {\cal F}_{21}^{-1} =
\left( {\cal F}_{21} \, {\cal R} \, {\cal F}^{-1} \right) \Delta^{(F)} \left( {\cal F} \,
{\cal R}^{-1} \, {\cal F}_{21}^{-1} \right) \; .
$$
The Yang - Baxter eq. (\ref{2.30}) for $R$-matrix (\ref{2a}) can be
directly checked with the help of (\ref{2.26}) and (\ref{cocycl}).

Impose additional relations on ${\cal F}$
\be
\lb{4}
(\Delta \otimes id) {\cal F} =
{\cal F}_{13} \, {\cal F}_{23} \;\; , \;\;\;
(id \otimes \Delta) {\cal F} =
{\cal F}_{13} \, {\cal F}_{12} \; ,
\ee
which, together with (\ref{cocycl}), implies the Yang-Baxter equation for
${\cal F}$.
Using (\ref{2.26}) one deduces from (\ref{4}) equations
\be
\lb{5aa}
{\cal R}_{12} \, {\cal F}_{13} \, {\cal F}_{23} =
{\cal F}_{23} \, {\cal F}_{13} \, {\cal R}_{12} \;\; , \;\;\;
{\cal F}_{12} \, {\cal F}_{13} \, {\cal R}_{23} =
{\cal R}_{23} \, {\cal F}_{13} \, {\cal F}_{12} \; .
\ee
Eqs. (\ref{5aa}) and the Yang-Baxter relations for universal elements ${\cal R}$, ${\cal F}$
define the twist which is proposed in
\cite{16} (the additional condition
${\cal F}^{21}{\cal F} = 1 \otimes 1$ is assumed in \cite{16}).

Note that if ${\cal A}$ is the Hopf algebra
of functions on the group algebra of group $G$ (\ref{2.15b}),
then eq. (\ref{cocycl})
can be written in the form of 2-cocycle equation:
$$
{\cal F}(a,b) \, {\cal F}(ab,c) = {\cal F}(b,c){\cal F}(a,bc) \; ,
\;\;\; (\forall a,b,c \in G) \; ,
$$
for the projective representation $\rho$ of $G$:
$\rho(a) \rho(b) = {\cal F}(a,b) \, \rho(a \, b)$.
That is why  eq. (\ref{cocycl}) is called cocycle equation.

Many explicit solutions
of the cocycle equation (\ref{cocycl}) are known (see e.g.
\cite{tw1}--\cite{tw3} and references therein).

\noindent
{\bf Remark 2.}{\it Ribbon Hopf algebras.} \\
Here we explain the notion of the ribbon Hopf algebras \cite{16'}.
Consider quasitriangular Hopf algebra ${\cal A}$ and represent the universal
${\cal R}$-matrix in the form
\be
\lb{unrab}
{\cal R}= \sum_{\mu} \alpha_{\mu} \otimes \beta_{\mu} \; ,
\;\;\;\; {\cal R}^{-1} =
\sum_{\mu} \, \gamma_{\mu} \otimes \delta_{\mu} \; ,
\ee
where $\alpha_{\mu}, \, \beta_{\mu}, \, \gamma_{\mu}, \, \delta_{\mu} \in {\cal A}$.
By using the right equalities in
(\ref{2.29a}) we represent the identities
$(id \otimes S)\bigl( {\cal R}{\cal R}^{-1} \bigr)= I
= (id \otimes S)\bigl({\cal R}^{-1}{\cal R}\bigr)$ as
\be
\lb{close}
\alpha_{\mu} \, \alpha_\nu \otimes \beta_{\nu} \, S(\beta_\mu)
=I = \alpha_{\mu} \, \alpha_\nu \otimes
S(\beta_{\nu}) \, \beta_\mu \; .
\ee
(the summation over repeated
indices $\mu$ and $\nu$ is assumed
and we write $I$ instead of $(I \otimes I)$)
while for $(S \otimes id){\cal R}{\cal R}^{-1} = I =
(S \otimes id){\cal R}^{-1}{\cal R}$
we have
\be
\lb{close1}
S(\gamma_{\mu}) \, \gamma_\nu \otimes \delta_{\nu} \, \delta_\mu
= I =
\gamma_{\mu} \, S(\gamma_\nu) \otimes \delta_{\nu} \, \delta_\mu \; .
\ee
We use identities (\ref{close}) and (\ref{close1})
below in Subsection {\bf \ref{qtrace}}
(Remark 1).

Consider the element
$u=\sum_{\mu} \, S(\beta_{\mu}) \, \alpha_{\mu}$ for which
the following proposition holds


%\\{\bf Proposition 1.}
\begin{proposition}\label{prop1}
{\it (see \cite{13'}). \\
	1.) For any $a \in {\cal A}$ we have
	\be
	\lb{0.2}
	S^{2}(a) \, u = u \, a \; .
	\ee
	2.) the element $u$ is invertible, with
	\be
	\lb{u-inv}
	u^{-1} =  S^{-1}(\delta_{\mu}) \, \gamma_{\mu} \; .
	\ee
}
\end{proposition}
{\bf Proof.}
1.) From the relation (\ref{2.26}) it follows that
$\forall a \in {\cal A}$ (the summation signs are omitted):
$$
\alpha_{\mu} \, a_{(1)} \otimes \beta_{\mu} \, a_{(2)} \otimes a_{(3)} =
a_{(2)} \, \alpha_{\mu} \otimes a_{(1)} \, \beta_{\mu} \otimes a_{(3)} \; ,
$$
where $a_{(1)} \otimes  a_{(2)} \otimes a_{(3)} =
(\Delta \otimes id)\Delta(a)$.
From this we obtain
$$
S^{2}( a_{(3)} ) \, S( \beta_{\mu} \, a_{(2)}) \, \alpha_{\mu} a_{(1)} =
S^{2}( a_{(3)} ) \, S( a_{(1)} \, \beta_{\mu} ) \, a_{(2)} \alpha_{\mu} \; ,
$$
or
$$
S^{2}( a_{(3)} ) \, S( a_{(2)}) \, u \, a_{(1)} =
S^{2}( a_{(3)} ) \, S(\beta_{\mu}) \, S( a_{(1)} ) \, a_{(2)} \alpha_{\mu} \; .
$$
Applying to this equation the axioms (\ref{2.21}),
%and taking into account the fact that
%$\epsilon(a_{(2)}) a_{(1)} = \epsilon(a)I$,
we obtain (\ref{0.2}). \\
2.) Putting $w =  S^{-1}(\delta_{\mu}) \, \gamma_{\mu}$, we have
$$
u \, w =  u \, S^{-1}(\delta_{\mu}) \, \gamma_{\mu} =
S(\delta_{\mu})\, u  \, \gamma_{\mu} =
S(\beta_{\nu} \, \delta_{\mu})\, \alpha_{\nu}
\, \gamma_{\mu} \; .
$$
Since ${\cal R} \cdot {\cal R}^{-1}
= \alpha_{\nu} \gamma_{\mu} \otimes
\beta_{\nu} \delta_{\mu} = I$, we have $u \, w = I$.
It follows from last equation and from (\ref{0.2}) that $S^{2}(w) \, u =1$,
and therefore the element $u$ has both a right and left inverse
(\ref{u-inv}). \hfill \qed

\noindent
Thus, the element $u$ is invertible and we can rewrite (\ref{0.2}) in the form
\be
\lb{0.3}
S^{2}(a) = u \, a \, u^{-1} \; .
\ee
This relation shows, in particular, that the operation of taking
the antipode is not involutive.

\begin{proposition}\label{prop1b}
{\it (see \cite{13'}). \\
	Define the following elements:
	\be
	\lb{u1234}
	u_{1} \equiv u =  S(\beta_{\mu}) \, \alpha_{\mu}
	\; , \;\;\;
	u_{2} =  S(\gamma_{\mu}) \, \delta_{\mu} \; , \;\;\;
	u_{3} =  \beta_{\mu} \, S^{-1}(\alpha_{\mu})  \; , \;\;\;
	u_{4} = \gamma_{\mu} \, S^{-1}(\delta_{\mu})  \; .
	\ee
	The relations (\ref{0.3})
	are satisfied if we take any of the elements
	$u_i$ from (\ref{u1234}):
	\be
	\lb{0.3b}
	S^{2}(a) = u_i \, a \, u_i^{-1} \; , \;\;\;\;\;
	\forall a \in {\cal A} \; .
	\ee
	In addition we have
	$S(u_1)^{-1} = u_{2}$, $S(u_{3})^{-1} = u_{4}$,
	and it turns out
	that all $u_{i}$ commute with each other,
	while the elements
	$u_{i}u_{j}^{-1} = u_{j}^{-1}u_{i}$ are central in ${\cal A}$.
	Consequently, the element $u \, S(u)= u_{1} \, u^{-1}_{2}$ is also central.} \end{proposition}
	{\bf Proof.} In view of relation (\ref{u-inv}) we have
	$S(u_1)^{-1} = S(u^{-1}) = S(\gamma_\mu) \delta_\mu = u_2$
	and $u_2^{-1} = S(u) = S^{-1}(u)=
	S^{-1}(\alpha_{\mu}) \, \beta_{\mu}$, where we use
	the identity $S^{2}(u)=u$ which follows from (\ref{0.3}).
	Applying the map $S$ to both parts of (\ref{0.3}), we deduce
	$S^3(a) = u_2 S(a) u_2^{-1}$ which is equivalent to
	(\ref{0.3b}) for $i=2$.
	Note that from (\ref{2.29a}) we have
	${\cal R}^{\pm} = (S^{-1} \otimes S^{-1}){\cal R}^{\pm}$.
	Thus,
	one can make in all formulas above the
	substitution $\alpha_\mu \to S^{-1}(\alpha_\mu )$,
	$\beta_\mu \to S^{-1}(\beta_\mu )$ and
	$\gamma_\mu \to S^{-1}(\gamma_\mu )$,
	$\delta_\mu \to S^{-1}(\delta_\mu )$
	to exchange the elements $u_1$ and $u_2$
	respectively to the elements $u_3$ and $u_4$. It means that
	equations (\ref{0.3b}) are valid for $i=3,4$ and we have
	$u_4 = S(u_3)^{-1}$. Relations (\ref{0.3b})
	yields $S^2(u_j)=u_j$ $(\forall j)$ and substitution
	$a=u_j$ to (\ref{0.3b})
	gives $u_i u_j = u_j u_i$ $(\forall i,j=1,...,4)$.
	Finally, for any $a \in {\cal A}$, we have
	$u_{j}^{-1}u_{i} \, a \, u_{i}^{-1}u_{j} =
	u_{j}^{-1}\, S^2(a) \, u_{j} = a$, which means that
	elements $u_{j}^{-1}u_{i} = u_{i} u_{j}^{-1}$
	are central. \hfill \qed
	
	\vspace{0.2cm}
	\noindent
	In Ref. \cite{13'} it was noted that
	$$
	\Delta(u) = ({\cal R}_{21}{\cal R}_{12})^{-1}(u \otimes u) =
	(u \otimes u) ({\cal R}_{21}{\cal R}_{12})^{-1} \; .
	$$
	\vspace{0.2cm}
	
	On the basis of all these propositions, we introduce the important concept of a ribbon Hopf algebra
	(see \cite{16'}):
	
	%{\bf Definition 7.}
	
	\newtheorem{def7}[def1]{Definition}
	\begin{def7} \label{def7}
{\it Consider a quasitriangular Hopf algebra $({\cal A}, \; {\cal R})$. Then the triplet
	$({\cal A}, \; {\cal R}, \; v)$ is called a ribbon Hopf algebra if
	$v$ is a central element in ${\cal A}$ and
	$$
	v^{2} = u \, S(u) \; , \;\; S(v) = v \; , \;\; \epsilon(v) =1 \; ,
	$$
	$$
	\Delta(v) = ({\cal R}_{21} \, {\cal R}_{12})^{-1} \, (v \otimes v) \; .
	$$}
	\end{def7}
	For each quasitriangular Hopf algebra ${\cal A}$, we can define
	${\cal A}$-colored ribbon graphs \cite{16'}. If, moreover, ${\cal A}$
	is a ribbon Hopf algebra, then for each
	${\cal A}$-colored ribbon graph we can associate the central element of
	${\cal A}$ that generalizes the
	Jones polynomial being an invariant of a knot in
	$\mathbb{R}^3$
	(see \cite{16'}, \cite{18}).
	
	\noindent
	{\bf Remark 3.}{\it Quasi-Hopf algebras.} \\
	One can introduce a generalization of a Hopf algebra, called
	a quasi-Hopf algebra, \cite{17} which is defined as an associative unital algebra ${\cal A}$
	with homomorphism $\Delta: \;\;
	{\cal A} \rightarrow {\cal A} \otimes {\cal A}$, homomorphism
	$\epsilon: \;\; {\cal A} \rightarrow \mathbb{C}$,
	antiautomorphism $S: \;\; {\cal A} \rightarrow {\cal A}$
	and invertible element $\Phi \in
	{\cal A}\otimes {\cal A}\otimes {\cal A}$. At the same time $\Delta$, $\epsilon$, $\Phi$
	and $S$ satisfy the axioms
	\be
	\lb{2.47}
	(id \otimes \Delta) \Delta(a) =
	\Phi \cdot (\Delta \otimes id) \Delta(a) \cdot \Phi^{-1} , \;\;
	a \in {\cal A},
	\ee
	\be
	\lb{2.48}
	(id \otimes id \otimes \Delta)(\Phi )
	\cdot (\Delta \otimes id\otimes id) (\Phi) =
	(I \otimes \Phi) \cdot (id \otimes \Delta \otimes id)(\Phi ) \cdot
	(\Phi \otimes I),
	\ee
	\be
	\lb{2.48a}
	(\epsilon \otimes id) \Delta = id = (id \otimes \epsilon) \Delta \; ,\;\;
	(id \otimes \epsilon \otimes id)\Phi = I \otimes I
	\ee
	\be
	\lb{2.49}
	S(a_{(1)}) \, \alpha \,  a_{(2)} = \epsilon(a) \, \alpha \; , \;\;
	a_{(1)} \, \beta \, S(a_{(2)}) = \epsilon(a) \, \beta \; ,
	\ee
	$$
	\phi_i \, \beta \,  S(\phi'_i) \, \alpha \, \phi''_i = I \; , \;\;
	S(\bar{\phi}_i) \, \alpha \,  \bar{\phi}'_i \, \beta \, S(\bar{\phi}''_i) = I \; ,
	$$
	where $\alpha$   and  $\beta$   are   certain   fixed   elements   of
	${\cal A}$, $\Delta(a) = a_{(1)} \otimes a_{(2)}$, and
	$$
	\Phi : = \phi_i \otimes \phi'_i \otimes \phi''_i \; , \;\;
	\Phi^{-1} : = \bar{\phi}_i \otimes \bar{\phi}'_i
	\otimes \bar{\phi}''_i \; ,
	$$
	(summation over $i$ is assumed).
	Thus, a quasi-Hopf algebra differs from an ordinary Hopf algebra in
	that the axiom of coassociativity is replaced by the weaker condition (\ref{2.47}).
	In other words, a quasi-Hopf algebra is noncoassociative, but this
	noncoassociativity is kept under control by means of the element $\Phi$.
	The axioms (\ref{2.49}) (which looks like different definitions
	of the left and right antipodes) generalize the axioms (\ref{2.21}) for usual Hopf algebras
	and consequently the elements $\alpha$ and $\beta$ involve into the play with
	the contragredient representations of the quasi-Hopf algebras.
	
	To make the pentagonal condition (\ref{2.48}) more transparent, let us consider
	(following Ref. \cite{17}) the algebra ${\cal A}$ as the algebra of functions on
	a "noncommutative" space $X$ equipped with a $*$ product: $X \times X \rightarrow X$.
	Then, elements $a \in {\cal A}$, $b \in {\cal A} \otimes {\cal A}, \dots$
	are written in the form $a(x)$, $b(x,y)$ $\dots$ and $\Delta(a)$ is represented as
	$a(x * y)$. The homomorphism $\epsilon$ defines the point in $X$, which we denote $1$ and
	instead of $\epsilon(a)$ we write $a(1)$. Then, equations (\ref{2.47}) -- (\ref{2.48a})
	are represented in the form \cite{17}:
	$$
	a(x *(y*z)) = \Phi(x,y,z) \, a((x *y )*z) \, \Phi(x,y,z)^{-1} \; ,
	$$
	\be
	\label{5phi}
	\Phi(x,y,z*u) \, \Phi(x*y,z,u) = \Phi(y,z,u)  \, \Phi(x,y*z,u) \, \Phi(x,y,z) \; ,
	\ee
	$$
	a(1 * x) = a(x) = a(x *1) \; , \;\;\; \Phi(x,1,z) = 1 \; .
	$$
	Now it is clear that (\ref{5phi}) (and respectively (\ref{2.48}))
	is the sufficient condition for the commutativity of the diagram:
	
	\unitlength=0.7cm
	\begin{picture}(20,5)
\put(1,3){$a\left(((x *y)*z)*u \right)$}
\put(6.5,3){$\longrightarrow$}
\put(8,3){$a((x *y)*(z*u))$}
\put(13.5,3){$\longrightarrow$}
\put(15,3){$a(x *(y*(z*u)))$}
\put(4,2.5){\vector(0,-1){1}}
\put(1,0.8){$a((x *(y*z))*u)$}
\put(9,0.955){\line(1,0){3}}
\put(11.5,0.8){$\rightarrow$}
\put(17,2.5){\vector(0,-1){1}}
\put(15,0.8){$a(x *((y*z)*u))$}

\end{picture}

\noindent
{\bf Remark.} Applications of the theory of quasi-Hopf algebras to the solutions of the
Knizhnik-Zamolodchikov equations are discussed in Ref. \cite{17}. On the other hand, one can
suppose that, by virtue of the occurrence of the pentagonal relation (\ref{2.48}) for the element
$\Phi$, quasi-Hopf algebras will be associated with multidimensional generalizations of Yang-Baxter
equations.

\

\section{The double covering of the quantum group $SO_q(3)$}
\paragraph*{} 

The covering of the quantum $SO\left(3 \right)$ is described in  \cite{dijkhuizen:so_doublecov,podles:so_su}. Here we would like to prove that the covering complies with the general theory of noncommutative coverings described in \ref{cov_fin_bas_sec}. Moreover  we prove that the covering gives an unoriented unoriented spectral triple (cf. Definition \ref{unoriented_defn}  and \ref{unoriented_empt}).

\subsection{Basic constructions}
\paragraph*{}
Denote by 
\be\label{su_q_2_G_eqn}
G = \left\{\left. g \in \Aut\left(C\left( SU_q\left(2 \right)\right)  \right)~\right|~ ga = a;~~\forall a \in C\left(SO_q\left(3 \right)\right) \right\}
\ee
Denote by
\be\label{su_q_2_gr_eqn}
C\left(SU_q\left(2 \right)\right)_N \bydef \left\{ \left.\widetilde{a} \in C\left(SU_q\left(2 \right)\right)~\right|~ \widetilde{a}\bt\bt^* = q^{2N}\bt\bt^*\widetilde{a} \right\}; \quad \forall N \in \Z
\ee
and let us prove that the equation \eqref{su_q_2_gr_eqn} yields the $\Z$-grading of $C\left(SU_q\left(2 \right)\right)$. Really following conditions hold:
\begin{itemize}
	\item 
	\be\label{su_q_2_N_eqn}
	\begin{split} 
		\widetilde{a} \in C\left(SU_q\left(2 \right)\right)_N \Leftrightarrow	 \widetilde{a} =
		\left\{
		\begin{array}{c l}
			\al^N	\sum_{\substack{ j = 0\\ k = 0}}^\infty  b_{jk}\bt^j\bt^{*k} & N \ge 0 \\
			& \\
			\left( \al^{*}\right)^{-N}  	\sum_{\substack{ j = 0\\ k = 0}}^\infty  b_{jk}\bt^j\bt^{*k} & N < 0
		\end{array}\right.
	\end{split}	
	\ee
	\item From the Theorem \ref{su_q_2_bas_thm} it follows that the linear span of given by the equation \eqref{su_q_2_fin_eqn} elements is dense in $C\left( SU_q(2)\right)$, hence $C\left( SU_q(2)\right)$ is the $C^*$-norm completion of the following direct sum
	\be\label{su_q_2_nsum_eqn}
	\bigoplus_{N \in \Z}  C\left(SU_q\left(2 \right)\right)_N.
	\ee
\end{itemize}
From  $\bt\bt^*\in C\left(SO_q\left(3 \right)\right)$ it follows that the grading is $G$-invariant, i.e.
\be\label{su_q_2_ginv_eqn}
\begin{split}
	g C\left(SU_q\left(2 \right)\right)_N = C\left(SU_q\left(2 \right)\right)_N; \quad \forall g \in G, \quad \forall N \in \N.
\end{split}	
\ee	
$C\left(SU_q\left(2 \right)\right)_0$ is a commutative $C^*$-algebra generated generated by $\bt$ and $\bt^*$ i.e.
$$
\widetilde{a} \in C\left(SU_q\left(2 \right)\right)_0 \Rightarrow \widetilde{a} = \sum_{\substack{j = 0\\ k = 0}}^\infty c_{jk}\bt^j\bt^{*k}; \quad c_{jk}\in \C.
$$
It follows that $g\bt \in  C\left(SU_q\left(2 \right)\right)_0$, i.e.  
$
g\bt = \sum_{\substack{j = 0\\ k = 0}}^\infty c_{jk}\bt^j\bt^{*k},
$
and taking into account $\bt^2\in C\left( SO_q(3)\right)\Rightarrow\left(g\bt \right)^2 = \bt^2$ one concludes
\be\label{su_q_2_bpm_eqn}
g\bt = \pm \bt.
\ee
From $\al \in C\left(SU_q\left(2 \right)\right)_1$ and the equations \eqref{su_q_2_N_eqn}, \eqref{su_q_2_ginv_eqn}  it turns out that for any $g \in G$ one has
\be\label{su_q_2_gal_eqn}
g \al\in  C\left(SU_q\left(2 \right)\right)_1 \Rightarrow g\al=\al\sum_{\substack{ j = 0\\ k = 0}}^\infty b_{jk}\bt^j\bt^{*k};\quad b_{jk}\in \C
\ee
Otherwise from $\al^2 \in SO_q\left(3 \right)$ it turns out 
\be\label{su_q_2_al_eqn}
\begin{split}
	\al^2 = g\al^2 = (g\al)^2 = \left(\al\sum_{\substack{ j = 0\\ k = 0}}^\infty b_{jk}\bt^j\bt^{*k} \right)^2 =
	\\
	= \al^2\sum_{\substack{j = 0\\ k = 0}}^\infty\sum_{\substack{l = 0\\m = 0}}^\infty q^{k+m}b_{jk}b_{lm} \bt^{j+l}\left( \bt^*\right)^{k+m}= \al^2\sum_{\substack{ j = 0\\ k = 0}}^\infty c_{rs}\bt^r\bt^{*s}; \\
	\text{where} \quad c_{rs} = \sum_{l = 0}^r \sum_{m = 0}^s q^{k+m}b_{r-l, s-m}b_{lm}
\end{split}
\ee
It turns out that $c_{0,0}= 1$ and $c_{rs}=0$ for each $\left(r,s \right) \neq \left( 0,0\right)$. Otherwise $c_{0,0}=b_{0,0}^2$ it turns out that $b_{0,0}= \epsilon = \pm1$. Suppose that there are $j, k \in \N^0$ such that $b_{jk}\neq 0$ and $j + k > 0$. If $j$ and $k$ are such that 
\bean
b_{jk}\neq 0,\\
j+k =\min_{\substack{b_{mn}\neq 0\\ m+n>0}} m+n
\eean
then $c_{jk} = \epsilon b_{jk}\left(1+q^{j+k} \right) \neq 0$. There is a contradiction with $c_{rs}=0$ for each $r+s > 0$. It follows that if $j+k > 0$ then $b_{jk}= 0$, hence one has
$$
g \al = \eps\al = \pm\al.
$$
In result we have 
\bean
g \al = \pm \al,
g \bt = \pm \bt
\eean
If $g\al = \al$ and $g \bt = -\bt$ then $g\left(\al\bt \right) = - \al\bt$, it is impossible because $\al\bt \in SO_q\left(3 \right)$. It turns out that $G=\Z_2$ and if $g \in G$ is not trivial then $g\al = -\al$ and $g\bt = -\bt$.
So we proved the following lemma
\begin{lem}\label{su_q_2_group_lem}
	If $G = \left\{ g \in \Aut\left(C\left( SU_q\left(2 \right)\right)  \right)~|~ ga = a;~~\forall a \in C\left(SO_q\left(3 \right)\right) \right\}$ then $G \approx \Z_2$. Moreover  if $g \in G$ is the nontrivial element then
	\bean
	g\left( \al^k\bt^n\bt^{*m}\right) = \left( -1\right)^{k+m+n}  \al^k\bt^n\bt^{*m}, \\ g\left( \al^{*k}\bt^n\bt^{*m}\right) = \left( -1\right)^{k+m+n}  \al^{*k}\bt^n\bt^{*m}.
	\eean 
	
\end{lem}
\subsection{Covering of $C^*$-algebra}

\begin{lem}\label{su_q_2_fin_lem}
	$C\left( SU_q\left(2\right)\right)$ is a finitely generated projective $C\left( 	SO_q\left(3 \right) \right)$ module.
\end{lem}
\begin{proof}
	Let $A_f$ be given by the Theorem \ref{su_q_2_bas_thm}. 
	If $A_f^{\Z_2}= A_f\bigcap C\left( 	SO_q\left(3 \right) \right) $ then from \eqref{su_q_2_z2_ncomm_eqn} and the Theorem \ref{su_q_2_bas_thm} it turns out that given by \eqref{su_q_2_fin_eqn} elements
	\be\nonumber
	\al^k\bt^n\bt^{*m}~~ \text{ and }~~ \al^{*k'}\bt^n\bt^{*m}
	\ee 
	with even $k+m+n$ or $k'+m+n$ is the basis of $A_f^{\Z_2}$.	If 
	$$
	\widetilde{a} = \al^k\bt^n\bt^{*m} \notin A_f^{\Z_2}
	$$
	then $k+m+n$ is odd. If $m > 0$ then
	$$
	\widetilde{a} = \al^k\bt^n\bt^{*m-1} \bt^* = a \bt^* \text{ where } a \in A_f^{\Z_2}.
	$$
	If $m = 0$ and $n > 0$ then
	$$
	\widetilde{a} = \al^k\bt^{n-1} \bt = a \bt \text{ where } a \in A_f^{\Z_2}
	$$
	If $m = 0$ and $n= 0$ then $k > 0$ and
	$$
	\widetilde{a} = \al^{k-1}\al = a \al \text{ where } a \in A_f^{\Z_2}.
	$$
	From 
	$$
	\widetilde{a} = \al^{*k'}\bt^n\bt^{*m} \notin A_f^{\Z_2}
	$$
	it follows that  $k'+m+n$ is odd. Similarly to the above proof one has
	$$
	\widetilde{a} = a \al\text{ or } \widetilde{a} = a \al^* \text{ or } \widetilde{a} = a \bt \text{ or } \widetilde{a} = a \bt^*   \text{ where } a \in A_f^{\Z_2}.
	$$
	From the above equations it turns out that $A_f$ is a left  $A_f^{\Z_2}$-module generated by $\al, \al^*, \bt, \bt^*$. Algebra $A_f^{\Z_2}$ (resp. $A_f$) is dense in $ C\left( 	SO_q\left(3 \right) \right) $ (resp.  $C\left( 	SU_q\left(2 \right) \right)$ ) it follows that  $C\left( 	SU_q\left(2 \right) \right)$ is a left  $C\left( 	SO_q\left(3 \right) \right)$-module generated by $\al, \al^*, \bt, \bt^*$. From the Corollary \ref{fin_hpro_cor} it turns out that $C\left( SU_q\left(2\right)\right)$ is a finitely generated projective $C\left( 	SO_q\left(3 \right) \right)$ module.
	
\end{proof}
\begin{corollary}
	The	triple $\left(C\left( 	SO_q\left(3 \right) \right), C\left( 	SU_q\left(2 \right) \right), \Z_2 \right)$ is an unital noncommutative finite-fold  covering.
\end{corollary}
\begin{proof}
	Follows from $C\left( 	SO_q\left(3 \right) \right)=C\left( SU_q\left(2 \right) \right)^{\Z_2}$ and Lemmas \ref{su_q_2_group_lem}, \ref{su_q_2_fin_lem}.
\end{proof}

\begin{corollary}
	The	triple $\left(C\left( 	SO_q\left(3 \right) \right), C\left( 	SU_q\left(2 \right) \right), \Z_2\times \Z_2, \pi\right)$ is an unital noncommutative finite-fold  covering.
\end{corollary}

\begin{problem}
	%Example 6.3.2.
	It is known that there is an unitary element $u \in C\left(  SU_q\left(2 \right)\right)$ (cf. \cite{chakraborty_pal:quantum_su_2}) such that $\left[u\right]\in \K_1\left(C\left( \left( SU_q\left(2 \right)\right) \right)\right) $ is not trivial and has infinite period.
	\\ Question. Does an $\left(u, n\right)$-covering   $\left(C\left(  SU_q\left(2 \right)\right)\otimes\K, \widetilde A, \Z_{n}\right)$ (cf. Definition \ref{hurewicz_u_n_defn}) exist?  
\end{problem}


\subsection{$SO_q\left( 3\right)$ as an unoriented  spectral triple }
Let  $h: C\left( SU_q\left(2\right)\right)$ be a given by the equation \eqref{su_q_2_haar_eqn} state, and $L^2\left( C\left( SU_q\left(2\right)\right), h\right)$ is the Hilbert space of the corresponding GNS representation  (cf. Section \ref{gns_constr_sec}). The state $h$ is  $\Z_2$-invariant, hence there is an action of $\Z_2$ on $L^2\left( C\left( SU_q\left(2\right)\right), h\right)$ which is naturally induced by the action  $\Z_2$ on $\Coo\left( SU_q\left(2\right)\right) $. From the above construction it follows that the unital orientable spectral triple
\be\nonumber
\left(\Coo\left( SU_q\left(2\right)\right), L^2\left( C\left( SU_q\left(2\right)\right), h\right), \widetilde{D} \right).
\ee
\begin{exercise}
	Using an unital noncommutative finite-fold covering  $$\left(C\left( 	SO_q\left(3 \right) \right), C\left( 	SU_q\left(2 \right) \right), \Z_2 \right)$$
	(cf. Definition \ref{fin_unital_defn})
	rove that there is is an unoriented  spectral triple (cf. Definition \ref{unoriented_defn}) given by
	$$
	\left(\Coo\left( SO_q\left(3\right)\right), L^2\left( C\left( SU_q\left(2\right)\right), h\right)^{\Z_2},D \right) 
	$$ where $\Coo\left( SO_q\left(3\right)\right)\bydef C\left( SO_q\left(3\right)\right)\bigcap \Coo\left( SU_q\left(2\right)\right)$ and $D\bydef \widetilde{ D}|_{L^2\left( C\left( SU_q\left(2\right)\right), h\right)^{\Z_2}}$.
	
\end{exercise}

\section{Counterexample}\label{counter_sec}
\paragraph*{}
The counterexample of the Lemma \ref{comm_lem} is discussed here. 
\subsection{Noncommutative quantum $SU(2)$ group}

\paragraph{} Let $q$ be a real number such that $0<q<1$. 
A quantum group $C\left( \SU_q(2)\right) $ is an universal $C^*$-algebra algebra generated by two elements $\al$ and $\beta$ satisfying following relations:
\begin{equation}\label{su_q_2_rel_eqn}
\begin{split}
\al^*\al + \beta^*\beta = 1, ~~ \al\al^* + q^2\beta\beta^* =1,
\\
\al\bt - q \bt\al = 0, ~~\al\bt^*-q\bt^*\al = 0,
\\
\bt^*\bt = \bt\bt^*.
\end{split}
\end{equation}
\paragraph*{} The structure of the quantum group on $C\left( SU_q\left( 2\right)\right) $ is given by
\begin{equation}\label{su_2_qgr_eqn}
\begin{split}
\Delta(\alpha)=\alpha\otimes\alpha-q\beta^*\otimes\beta,\\
\Delta(\beta)=\beta\otimes\alpha+\alpha^*\otimes\beta.
\end{split}
\end{equation}

From  $C\left( SU_1\left(2 \right)\right) \approx C\left(SU\left(2 \right)  \right)$ it follows that  $C\left( \SU_q(2)\right) $ can be regarded as a noncommutative deformation of $SU(2)$. It is proven in \cite{woronowicz:su2} that the spectrum of $\bt\bt^*$ is the discrete set
$$
\left\{1, q^2, q^4, q^6, ..., 0\right\} \subset \C.
$$
If $n \in \N^0$ and  $f_n: \R \to \R$ is a continuous function such that 
$$
f_n(t)=\left\{
\begin{array}{c l}
0 & t \le q^{2n + 1} \\
0 & t \ge q^{2n - 1} \\
1 & t = q^{2n}
\end{array}\right..
$$
then $p^\al_n = f_n\left(\bt\bt^* \right) \in C\left( \SU_q(2)\right)$ is a projection.  
Let $Q, S \in B\left( \ell_2\left(\N^0 \right)\right) $ be given by 
\begin{equation*}
\begin{split}
Qe_k= q^ke_k, \\
Se_k = \left\{
\begin{array}{c l}
e_{k-1} & k > 0 \\
0 & k = 0
\end{array}\right.,
\end{split}
\end{equation*}
and let $R \in B\left( \ell_2\left(\Z \right)\right) $ be given by $e_k \mapsto e_{k+1}$.
There is  a faithful representation   $C\left(\SU_q\left( 2\right) \right) \to B\left(\ell_2\left(\N^0 \right) \otimes \ell_2\left(\Z \right) \right)  $ \cite{woronowicz:su2} given by

\begin{equation}\label{su_2_q_repr_eqn}
\begin{split}
\al \mapsto S\sqrt{1 - Q^2} \otimes 1_{B\left(\ell_2\left(\Z \right) \right) }, \\
\bt \mapsto Q \otimes R.
\end{split}
\end{equation}

If $R_\R \in B\left(  L^2\left(\R \right)\right) $ is given by
$$
R_\R\left(  \xi\right)  = e^{2\pi ix}\xi; \text{ where } e^{2 \pi ix} \in C_b\left(\R \right) 
$$
then similarly to \eqref{su_2_q_repr_eqn} one has a representation $C\left(\SU_q\left( 2\right) \right) \to B\left(\ell^2\left(\N^0 \right) \otimes L^2\left(\R\right) \right)$ given by

\begin{equation}\label{s_2_q_repr_eqn}
\begin{split}
\al \mapsto S\sqrt{1 - Q^2} \otimes 1_{B\left(L^2\left(\R\right) \right) }, \\
\bt \mapsto Q \otimes R_\R.
\end{split}
\end{equation}

\subsection{Finite-fold coverings}

\paragraph*{}
If $R^{\frac{1}{n}}_\R \in B\left(  L^2\left(\R \right)\right) $ is given by
$$
R^{\frac{1}{n}}_\R\left(  \xi\right)  = e^{\frac{2\pi ix}{n}}\xi.
$$ 
then $\left(R^{\frac{1}{n}}_\R \right)^n =  R_\R$. If $\widetilde{q} = \sqrt[n]{q}$ and 
$$
\widetilde{\bt} = \sum_{k = 0}^{\infty} \widetilde{q}^k p^\al_k\otimes R^{\frac{1}{n}}_\R\in B\left(\ell^2\left(\N^0 \right) \otimes L^2\left(\R\right) \right)  
$$
then $\widetilde{\bt}^n = \bt$. Denote by $C\left(\SU_q\left( 2\right) \right) \left[\widetilde{\bt} \right]$ a $C^*$-subalgebra of  $B\left(\ell^2\left(\N^0 \right) \otimes L^2\left(\R\right) \right)$ generated by $C\left(\SU_q\left( 2\right) \right) \bigsqcup \left\{\widetilde{\bt}\right\}$. Denote by $M\left[\widetilde{\bt} \right] \subset C\left(\SU_q\left( 2\right) \right) \left[\widetilde{\bt} \right]$ a free module left  $C\left(\SU_q\left( 2\right) \right)$ module given by
$$
M\left[\widetilde{\bt} \right] = \bigoplus_{j = 0}^{n-1} C\left(\SU_q\left( 2\right) \right) \widetilde{\bt}^j.
$$
If $j \in \{0,..., n-1\}$, $j \in \N^0$ then from $p^\al_k \in  C\left(\SU_q\left( 2\right) \right)$ it follows that $p^\al_k \widetilde{\bt}^j= p^\al_k \widetilde{q}^{-k}\left(R^{\frac{1}{n}}_\R \right)^j \in M\left[\widetilde{\bt} \right]$, hence $p^\al_k\left(R^{\frac{1}{n}}_\R \right)^j \in M\left[\widetilde{\bt} \right]$. Moreover if $\left\{z_k \in \C\right\}_{k \in \N^0}$ then from $\lim_{k \to \infty}z_k = 0$ it turns out
$$
\sum_{k = 0}^{\infty} z_kp^\al_k\left(R^{\frac{1}{n}}_\R \right)^k \in M\left[\widetilde{\bt} \right].
$$



Following conditions hold:


\begin{equation*}
\begin{split}
\widetilde{\bt}^j \al = \left( \sum_{k=0}^{\infty}\widetilde{q}^{jk} p^\al_k\otimes \left(R^{\frac{1}{n}}_\R \right)^j\right)\left(  S \sqrt{1 - Q^2} \otimes 1 \right) = \left(  S \sqrt{1 - Q^2} \otimes 1 \right) \left( \sum_{k=0}^{\infty}\widetilde{q}^{j\left( k+1\right) } p^\al_k\otimes \left(R^{\frac{1}{n}}_\R \right)^j\right)
\end{split}
\end{equation*}
From $\lim_{k \to \infty}\widetilde{q}^{j\left( k+1\right) } = 0$ it turns out $\widetilde{\bt}^j \al$ lies in $M\left[\widetilde{\bt} \right]$. Similarly we have $\widetilde{\bt}^j \al^*\in \left[\widetilde{\bt} \right]$ it follows that
$$
M\left[\widetilde{\bt} \right] =C\left(\SU_q\left( 2\right) \right) \left[\widetilde{\bt} \right],
$$
i.e. $C\left(\SU_q\left( 2\right) \right) \left[\widetilde{\bt} \right]$ is a  finitely generated free  $C\left(\SU_q\left( 2\right) \right)$-module 
\begin{equation}\label{su_q_2_dir_sum_eqn}
C\left( \SU_q\left(2\right) \right) \left[\widetilde{\bt} \right] = \bigoplus_{j = 0}^{n-1} C\left(\SU_q\left( 2\right) \right) \widetilde{\bt}^j= C\left(\SU_q\left( 2\right) \right)^n
\end{equation}
There is the action of $\Z_n$ on $C\left(\SU_q\left( 2\right) \right) \left[\widetilde{\bt} \right]$ given by
\begin{equation*}
\begin{split}
\overline{m} a \widetilde{\bt}^k = e^{\frac{2 \pi i mk}{n}} a \widetilde{\bt}^k; \text{ where } a \in C\left(\SU_q\left( 2\right) \right), ~\overline{m} \in \Z_n, ~m \in \Z \text{ is representative of } \overline{m}.
\end{split}
\end{equation*}




The above construction gives a following result.
\begin{thm}\cite{ivankov:qnc}
	The triple $\left( C\left( \SU_q\left( 2\right)\right), C\left( \SU_q\left( 2\right)\right)\left[\widetilde{\bt}\right], \Z_n\right)$ is an unital noncommutative finite-fold covering.
\end{thm}




\subsection{The structure of the covering algebra}\label{cov_alg_str_sec}
\paragraph*{}
From the above construction it follows that
$$
\widetilde{\widetilde{\bt}}= \sum_{j = 0}^{\infty} q^j p^\al_j \otimes R^{\frac{1}{n}}_\R \in C\left( \SU_q\left( 2\right)\right)\left[\widetilde{\bt}\right].
$$
Direct calculations shows that
\begin{equation*}
\begin{split}
\al^*\al + \widetilde{\widetilde{\bt}}^*\widetilde{\widetilde{\bt}} = 1, ~~ \al\al^* + q^2\widetilde{\widetilde{\bt}}\widetilde{\widetilde{\bt}}^* =1,
\\
\al\widetilde{\widetilde{\bt}} - q \widetilde{\widetilde{\bt}}\al = 0, ~~\al\widetilde{\widetilde{\bt}}^*-q\widetilde{\widetilde{\bt}}^*\al = 0,
\\
\widetilde{\widetilde{\bt}}^*\widetilde{\widetilde{\bt}} = \bt\widetilde{\widetilde{\bt}}^*.
\end{split}
\end{equation*}
Above relations coincide with \eqref{s_2_q_repr_eqn} it follows that there is a $*$-isomorphism given by
\begin{equation*}
\begin{split}
C\left( \SU_q\left( 2\right)\right)\xrightarrow{\approx}C\left( \SU_q\left( 2\right)\right)\left[\widetilde{\bt}\right],\\
\al \mapsto \al,~~
\bt\mapsto \widetilde{\widetilde{\bt}},
\end{split}
\end{equation*}
i.e. the covering algebra $C\left( \SU_q\left( 2\right)\right)\left[\widetilde{\bt}\right]$ is *-isomorphic to the base algebra $C\left( \SU_q\left( 2\right)\right)$.


\subsection{Symmetry and grading}
\paragraph*{}
Let $\A\subset SU_q\left( 2\right)$ is a dense subalgebra which is generated by $\al, \al^*, \bt, \bt^*$ as an abstract algebra.
\begin{thm}\cite{woronowicz:su2}
The set of all elements of the form
\begin{equation}\label{su_q_2_basis_eqn}
\al^k\bt^n\bt^{*m} \text{ and } \al^{*k'}\bt^n\bt^{*m}
\end{equation}
where $k, m, n = 0, 1, 2, \dots,~k'=1,2, \dots$ forms a basis in $\A$: any element of $\A$ can be written in the unique way as a finite linear combination of elements \eqref{su_q_2_basis_eqn}.
\end{thm}
From the above theorem there is an action  of $U\left( 1\right)$ on $\A$ given by
$$
g \left( \al^k\bt^n\bt^{*m} \right)  = \varphi(g)_{\C^\times}^{n - m}\al^{*k'}\bt^n\bt^{*m} \text{ and } g \left( \al^{*k'}\bt^n\bt^{*m}\right)   = \varphi_{\C^\times}\left(g \right) ^{n - m}\al^{*k'} \al^{*k'}\bt^n\bt^{*m}
$$
where $g \in U\left(1 \right)$ and $ \varphi_{\C^\times}: U\left( 1\right) \to \C^\times$ the natural homomorphism from $U\left(1 \right)$ to the multiplicative group of complex numbers. There is a $\Z$-grading 
$$
\A = \bigoplus_{j \in \Z} \A_j
$$
such that $a \in \A_j$ is equivalent to
$$
ga = \varphi(g)^j_{\C^\times}a \text { for any } g \in U\left( 1\right).
$$  
It turns out
$$
\al^k\bt^n\bt^{*m} \text{ and } \al^{*k'}\bt^n\bt^{*m} \text{ lie in }  \A_{n - m}.
$$
Let $\left( C\left( \SU_q\left( 2\right)\right), C\left( \SU_q\left( 2\right)\right)\left[\widetilde{\bt}\right], \Z_n\right)$ be a covering projection. From $\widetilde{\beta}^n = \beta$ and \eqref{su_q_2_dir_sum_eqn} it follows that there is the natural $\Z$-grading on $C\left( \SU_q\left( 2\right)\right)\left[\widetilde{\bt}\right]$
given by
$$
a \beta^j \in \left( C\left( \SU_q\left( 2\right)\right)\left[\widetilde{\bt}\right]\right)_{nk + j} \text{ where } a \in C\left( \SU_q\left( 2\right)\right)_k
$$ 
where subscripts mean the grading.

\subsection{Contradiction}
\paragraph*{}
Suppose that there is a structure of quantum group $\left( C\left(\SU_q\left( 2\right)\right)\left[\widetilde{\bt}\right], \widetilde{\Delta} \right)$ which satisfies to the Lemma \ref{comm_lem}.  From $\bt = \widetilde{\bt}^n$,  \eqref{su_2_qgr_eqn}, and the condition (i) of the Lemma \ref{comm_lem} it turns out 
\begin{equation}\label{bt_n_eqn}
\left(\widetilde{\Delta}\left(\widetilde{\bt} \right)  \right)^n = \Delta\left( \bt\right)  = \beta\otimes\alpha+\alpha^*\otimes\beta = \widetilde{\bt}^n\otimes\alpha+\alpha^*\otimes\widetilde{\bt}^n. 
\end{equation}
Denote by $$
D \stackrel{\text{def}}{=} C\left( \SU_q\left( 2\right)\right)\left[\widetilde{\bt}\right] \otimes C\left( \SU_q\left( 2\right)\right)\left[\widetilde{\bt}\right].
$$
The $\Z$-grading on $C\left( \SU_q\left( 2\right)\right)\left[\widetilde{\bt}\right]$ induces  the natural $\Z\times\Z$ grading on  $D = C\left( \SU_q\left( 2\right)\right)\left[\widetilde{\bt}\right] \otimes C\left( \SU_q\left( 2\right)\right)\left[\widetilde{\bt}\right]$.
Clearly 
\begin{equation*}
\begin{split}
\widetilde{\bt}^n\otimes\alpha \in D_{(n,0)},\\
\alpha^*\otimes\widetilde{\bt}^n \in D_{(0,n)}
\end{split}
\end{equation*}
where subscripts $(n, 0)$ and $(0, n)$ mean grading.
Suppose that
$$
\widetilde{\Delta}\left(\widetilde{\bt} \right) = \sum_{\left( j, k\right)  \in \Z\times \Z} a_{jk}
$$
where
$$
a_{jk} \in D_{(j,k)}.
$$
Let $j_{\text{max}} \in \Z$ be a maximal number such that there is $k \in \Z$ which satisfy to the condition $a_{j_{\text{max}}, k} \neq 0$. The inequality $j_{\text{max}} > 1$ contradicts with \eqref{bt_n_eqn} because right part of \eqref{bt_n_eqn} does not contain summands in
	$
	D_{\left( nj_{\text{max}},k\right) }
	$.
	Similarly one can prove that the minimal value $j_{\text{min}}$ of $j$ such that $a_{j_{\text{min}}, k} \neq 0$ satisfies to an inequality $j_{\text{min}} \ge 0$. Using the same arguments one can prove that if $a_{jk} \neq 0$ then $0 \ge k \ge 1$. In result one has
		$$
	\widetilde{\Delta}\left(\widetilde{\bt} \right) = a_{00} + a_{01} + a_{10} + a_{11}.
		$$
		If $a_{00} \neq 0$ then $\left(	\widetilde{\Delta}\left(\widetilde{\bt} \right) \right)^n \bigcap D^{(0,0)} \neq 0$ and from this contradiction it turns out
		$a_{00}=0$. Similarly $a_{11} = 0$.
		Following condition holds
\begin{equation*}
\begin{split}
	\widetilde{\Delta}\left(\widetilde{\bt} \right)^n=\left(a_{01}+a_{10} \right)^n = a_{01}^n + a_{10}^n + r,\\
 a_{01}^n \in D^{0,n},\\
 a_{10}^n \in D^{n, 0},\\
 r \notin   D^{0,n} \bigoplus  D^{n,0},
\end{split}
\end{equation*}
hence $a_{01}^n= \alpha^*\otimes\widetilde{\bt}^n$, $a_{01}^n= \alpha^*\otimes\widetilde{\bt}^n$. Otherwise
\begin{equation*}
\begin{split}
r = n a_{10} a_{01}^{n-1} + r'
\end{split}
\end{equation*}
where $n a_{10} a_{01}^{n-1} \in D^{(1,n-1)}$, $r' \notin D^{(1,n-1)}$. From $a_{10}^n a_{01}^{n}\neq 0$ it turns out $a_{10} a_{01}^{n-1}\neq 0$ hence $r \neq 0$.
It follows that
$$
\widetilde{\Delta}\left(\widetilde{\bt} \right)^n=\left(a_{01}+a_{10} \right)^n  \neq \widetilde{\bt}^n\otimes\alpha+\alpha^*\otimes\widetilde{\bt}^n.
$$
This contradiction proves that the quantum group $\left( C\left( \SU_q\left( 2\right)\right), \Delta\right) $ and the finite-fold noncommutative covering $\left( C\left( \SU_q\left( 2\right)\right), C\left( \SU_q\left( 2\right)\right)\left[\widetilde{\bt}\right], \Z_n\right)$ projection do not satisfy to the Lemma \ref{comm_lem}.
\begin{rem}
	From \ref{cov_alg_str_sec} it follows the *-isomorphism $C\left( \SU_q\left( 2\right)\right) \xrightarrow{\approx}C\left( \SU_q\left( 2\right)\right)\left[\widetilde{\bt}\right]$, hence there is a structure of quantum group on $C\left( \SU_q\left( 2\right)\right)\left[\widetilde{\bt}\right]$. However in contrary to the commutative case this structure does not naturally follow from the structure of the quantum group $\left( C\left( \SU_q\left( 2\right)\right), \Delta\right)$ and the noncommutative finite-fold covering projection $$\left( C\left( \SU_q\left( 2\right)\right), C\left( \SU_q\left( 2\right)\right)\left[\widetilde{\bt}\right], \Z_n\right).$$
\end{rem}


 
\section{Conclusion}
\paragraph{} There is a set of geometrical statements which have noncommutaive generalizations, e.g. in \cite{ivankov:qncstr} it is proven a noncommutative analog of the theorem about a covering projection of a Riemannian manifold.
The  described in the Section  \ref{counter_sec} counterexample proves that the  analogy between coverings of topological groups and quantum groups is not full. However coverings of quantum groups satisfy to the Theorem \ref{main_thm} which is weaker than the Lemma \ref{comm_lem} about coverings of commutative quantum groups. 


\begin{thebibliography}{10}
	
	
\bibitem{auslander:galois} M. Auslander; I. Reiten; S.O. Smal\o{}. \textit{Galois actions on rings and finite Galois coverings}. Mathematica Scandinavica (1989), Volume: 65, Issue: 1, page 5-32, ISSN: 0025-5521; 1903-1807/e , 1989.

%\bibitem{ant_azz_scan:flat_k}Paolo Antonini, Sara Azzali, Georges Skandalis {\it Flat bundles, von Neumann algebras and $K$-theory with $\mathbb{R}/\mathbb{Z}$-coefficients}, arXiv:1308.0218, 2013.

\bibitem{arveson:c_alg_invt} W. Arveson. {\it An Invitation to $C^*$-Algebras}, Springer-Verlag. ISBN 0-387-90176-0, 1981.

%\bibitem{bezandry_diagana:bound_unbound}Paul H. Bezandry, Toka Diagana {\it Bounded and Unbounded Linear Operators}, in {\it Almost Periodic Stochastic Processes}, Springer, 2011.

%\bibitem{ballentine:qm} Leslie E Ballentine. {\it Quantum Mechanics: A Modern Development.} World Scientific Publishing Co. Pte. Ltd. 2000.

%\bibitem{blackadar:ko} B. Blackadar. {\it K-theory for Operator Algebras}, Second edition. Cambridge University Press 1998.

%\bibitem{blackadar:kocalg_neumann} B. Blackadar {\it Operator Algebras Theory of C* - Algebras and von Neumann Algebras}. Springer-Verlag Berlin Heidelberg 2006.

%\bibitem{blackadar:shape_theory} B. Blackadar, {\it Shape theory for $C^*$-algebras}, Math. Scand. 56 , 249-275, 1985.

%\bibitem{blecher:hilb_gen} D.P. Blecher. {\it A generalization of Hilbert modules}, J.Funct. An. 136, 365-421 1996.

%\bibitem{bost:oka}J.-B. Bost, {\it Principe d�Oka, $K$-th\'eorie et syst$\grave{e}$mes dynamiques non commutatifs}. Invent. Math. 101 (1990), 261�333. 1990.

%\bibitem{bogachev_measure_v1}V. I. Bogachev. {\it Measure Theory} (volume 1). Springer-Verlag, Berlin, 2007.
%\bibitem{bogachev_measure_v2}V. I. Bogachev. {\it Measure Theory}. (volume 2). Springer-Verlag, Berlin, 2007.



%\bibitem{bourbaki_sp:gt} N. Bourbaki, {\it General Topology}, Chapters 1-4, Springer, Sep 18, 1998.

%\bibitem{bratteli_robinson:oa_qsm}O. Bratteli and D. W. Robinson, {\it Operator Algebras and Quantum Statistical Mechanics 1}. Springer, New York. 1987.




%\bibitem{bruckler:tensor} Franka Miriam Br\"uckler. {\it Tensor products of $C^*$-algebras, operator spaces and Hilbert $C^*$-modules}. Mathematical Communications 4(1999), 1999.
\bibitem{chakraborty_pal:quantum_su_2} Partha Sarathi Chakraborty,  Arupkumar Pal. \textit{Equivariant spectral triples on the quantum $SU(2)$ group}. arXiv:math/0201004v3, 2002.


%\bibitem{chang:fermionic} Ee Chang-Young, Hiroaki Nakajima, Hyeonjoon Shin. {\it Fermionic $T$-duality and Morita Equivalence}, arXiv:1101.0473, 2011.

%\bibitem{morita_hopf_galois}S. Caenepeel, S. Crivei, A. Marcus, M. Takeuchi. {\it Morita equivalences induced by bimodules over Hopf-Galois extensions.} arXiv:math/0608572, 2007.

%\bibitem{cheng_li:gauge} Cheng, T.-P.; Li, L.-F. {\it Gauge Theory of Elementary Particle Physics}. Oxford University Press. ISBN 0-19-851961-3. 1983.

%\bibitem{chavel:riemann} Isaac Chavel. {\it Riemannian Geometry: A Modern Introduction} (Cambridge Studies in Advanced Mathematics) Paperback � July 6, 2006.

%\bibitem{chun-yen:separability} Chun-Yen Chou. {\it Notes on the Separability of $C^*$-Algebras.} TAIWANESE JOURNAL OF MATHEMATICS Vol. 16, No. 2, pp. 555-559, April 2012 This paper is available online at http://journal.taiwanmathsoc.org.tw, 2012.

%\bibitem{cohn_measure} Donald L. Cohn \textit{Measure Theory}. Basel: Birkh\"auser,  � 457 p. Second edition. ISBN 978-1-4614-6955-1. 2013.


%\bibitem{connes:c_alg_dg} Alain Connes. {\it $C^*$-algebras and differential geometry}. arXiv:hep-th/0101093, 2001.

%\bibitem{clare_crisp_higson:adj_hilb} Pierre Clare, Tyrone Crisp, Nigel Higson {\it Adjoint functors between categories of Hilbert modules}.  arXiv:1409.8656, 2014.

%\bibitem{connes:grav}A. Connes. {\it Gravity coupled with matter and foundation of noncommutative geometry}. Commun. Math. Phys. 182 (1996), 155�176. 1996.

%\bibitem{connes:ncg94} Alain Connes. {\it Noncommutative Geometry}, Academic Press, San Diego, CA,  661 p., ISBN 0-12-185860-X, 1994.

%\bibitem{connes_landi:isospectral} Alain Connes, Giovanni Landi. {\it Noncommutative Manifolds the Instanton Algebra and Isospectral Deformations}, arXiv:math/0011194, 2001.

%\bibitem{connes_marcolli:motives}
%Alain Connes, Matilde Marcolli. {\it Noncommutative Geometry, Quantum Fields and Motives},  American Mathematical Society, Colloquium Publications, 2008.

% \bibitem{connes_moscovici:local_index} A. Connes and H. Moscovici, {\it The local index theorem in noncommutative geometry"}. Geom. and Funct. Anal., 1996.


%\bibitem{cuntz_quillen:alg_ext} Joachim Cuntz, Daniel Quillen.  {\it Algebra extensions and nonsingularity}, J. Amer. Math. Soc. 8 251-289, 1995

%\bibitem{davis_kirk_at}James F. Davis. Paul Kirk. {\it Lecture Notes in Algebraic Topology}. Department of Mathematics, Indiana University, Blooming- ton, IN 47405, 2001.
%\bibitem{dixmier_tr}J.Dixmier. {\it Traces sur les $C^*$-algebras}. Ann. Inst. Fourier, 13, 1(1963), 219-262, 1963.


%\bibitem{engelking:general_topology} Ryszard Engelking. \textit{General topology}, PWN, Warsaw. 1977.


%\bibitem{spectral_action_nt}Driss Essouabri, Bruno Iochum, Cyril Levy, and Andrzej Sitarz. \textit{Spectral action on noncommutative torus.} Journal of Noncommutative Geometry, 2 (2008), 53�123, 2008. 


%\bibitem{frank:frames} Michael Frank , David R. Larson, {\it Frames in Hilbert $C^*$-modules and $C^*$-algebras}, arXiv:math/0010189, 2000.


%\bibitem{moyal_spectral} V. Gayral, J. M. Gracia-Bond\'{i}a, B. Iochum, T. Sch\"{u}cker, J. C. Varilly. {\it Moyal Planes are Spectral Triples}. arXiv:hep-th/0307241, 2003.

%\bibitem{gilkey:odd_space}P.B. Gilkey. {\it The eta invariant and the $K$-theory of odd dimensional spherical space forms}.Inventiones mathematicae, Springer-Verlag, 1984.

%\bibitem{varilly_bondia:phobos}Jos\'e M. Gracia-Bondia, Joseph C. Varilly.  \textit{Algebras of Distributions suitable for phase-space quantum mechanics. I}. Escuela de Matem\'{a}atica, Universidad de Costa Rica,San Jos\'e, Costa Rica J. Math. Phys 29 (1988), 869-879, 1988.


%\bibitem{elliot:an} Elliott H. Lieb, Michael Loss. \textit{Analysis}, American Mathematical Soc., 2001.



%\bibitem{nicolas_ginoux:dirac_spectrum}Nicolas Ginoux. {\it The Dirac Spectrum.} Springer, Jun 11, 2009.

%\bibitem{varilly_bondia} Jos\'e M. Gracia-Bondia, Joseph C. Varilly, Hector Figueroa, {\it Elements of Noncommutative Geometry}, Springer, 2001.   92  96    142

%\bibitem{green_schwarz_witten:superstring} {\it Superstring Theory: Volume 2, Loop Amplitudes, Anomalies and Phenomenology}. (Cambridge Monographs on Mathematical Physics)  by Michael B. Green, John H. Schwarz, Edward Witten. 1988.


%\bibitem{gross_gauge}David J. Gross. {\it Gauge Theory-Past, Present, and Future?} Joseph Henry Luborutoties, Ainceton University, Princeton, NJ 08544, USA. (Received November 3,1992).

%\bibitem{ful:gr_repr} Fulton William, Harris Joe. {\it Representation theory. A first course} Graduate Texts in Mathematics, Readings in Mathematics 129, New York: Springer-Verlag. 1991.

%\bibitem{halmos:set} Paul R.  Halmos {\it Naive Set Theory.} D. Van Nostrand Company, Inc., Prineston, N.J., 1960.


%\bibitem{helemsky:qfa} A. Ya. Helemsky. {\it Quantum Functional Analysis. Non-Coordinate Approach.} Providence, R.I. : American Mathematical Society, 2010.

%\bibitem{ivankov:infinite_cov_pr} Petr Ivankov.  {\it Infinite Noncommutative Covering Projections}.  arXiv:1405.1859, 2014.
 
% \bibitem{ivankov:inv_lim}  Petr R. Ivankov. {\it Inverse Limits of Noncommutative Covering Projections},  arXiv:1412.3431, 2014.
 

% \bibitem{ivankov:nc_cov_k_hom}Petr Ivankov. {\it Noncommutative covering projections and $K$-homology}, 	arXiv:1402.0775, 2014.
 

 %\bibitem{ivankov:nc_wilson_lines} Petr Ivankov.  {\it Noncommutative Generalization of Wilson Lines}. arXiv:1408.4101, 2014.

%\bibitem{hajac:toknotes}{\it Lecture notes on noncommutative geometry and quantum groups}, Edited by Piotr M. Hajac.


%\bibitem{ivankov:nc_cov_k_hom}Petr Ivankov. {\it Noncommutative covering projections and $K$-homology}, 	arXiv:1402.0775, 2014.

%\bibitem{ivankov:uni_nc_cov}
%Petr R. Ivankov. {\it Universal covering space of the noncommutative torus},
%arXiv:1401.6748, 2014.

%\bibitem{kakariadis:corr}Evgenios T.A. Kakariadis, Elias G. Katsoulis, {\it Operator algebras and $C^*$-correspondences: A survey.} 	arXiv:1210.6067, 2012.

%\bibitem{kaku:loc}Kaku, M. {\it Locality in the gauge-covariant field theory of strings}. Phys. Lett. 162B, 97. Kaku, M. 1986.

%\bibitem{karaali:ha} Gizem Karaali {\it On Hopf Algebras and Their Generalizations}, arXiv:math/0703441, 2007.

%\bibitem{karoubi:k} M. Karoubi. {\it K-theory, An Introduction.} Springer-Verlag 1978.

%\bibitem{kastler:connes_lott} Daniel Kastler, Thomas Schucker, {\it The Standard Model a la Connes-Lott}, arXiv:hep-th/9412185, 1994.


%\bibitem{koba_nomi:fgd} S. Kobayashi, K. Nomizu. {\it Foundations of Differential Geometry}. Volume 1. Interscience publishers a division of John Willey \& Sons, New York - London. 1963.

%\bibitem{lee:smooth} John M. Lee. {\it Introduction to Smooth Manifolds}. University of Washington. Department of Mathematics. Version 3.0, December 31, 2000.

%\bibitem{demeyer:genreal_galois}DeMeyer, F. R. \textit{Some notes on the general Galois theory of rings}. Osaka J. Math.2 (1965), 117-127, 1965.

%\bibitem{mitchener:c_cat}Paul D. Mitchener. \textit{$C^*$-categories}. Odense University,  September 13, 2001.



%\bibitem{milne:etale}J.S. Milne. {\it \'Etale cohomology.} Princeton Univ. Press  1980.

%\bibitem{miyashita_fin_outer_gal} Y\^oichi Miyashita, {\it Finite outer Galois theory of noncommutative rings}. Department of Mathematics, Hokkaido, University, 1966.

%\bibitem{bram:atricle} Bram Mesland. {\it Unbounded bivariant $K$-theory and correspondences in noncommutative geometry} arXiv:0904.4383, 2009.


%\bibitem{miyashita_fin_outer_gal} Y\^oichi Miyashita, {\it Finite outer Galois theory of noncommutative rings}. Department of Mathematics, Hokkaido, University, 1966.

%\bibitem{miyashita_infin_outer_gal} Y\^oichi Miyashita, {\it Locally finite outer Galois theory}. Department of Mathematics, Hokkaido, University, 1967.

%\bibitem{muhly_williams:groupoid_ctr}  Paul S. Muhly and Dana P. Williams. {\it Continuous trace groupoid $C^*$-algebras.}, Math. Scand. 1990

\bibitem{ivankov:qncstr} Petr Ivankov. \textit{Coverings of Spectral Triples}, arXiv:1705.08651, 2017.

\bibitem{ivankov:qnc} Petr Ivankov. \textit{Quantization of noncompact coverings}, arXiv:1702.07918, 2017.


\bibitem{mimuta_toda_lie}Mamoru Mimura, Hiroshi Toda, \textit{Topology of Lie Groups, I and II}, American Mathematical Soc., 1991.

%\bibitem{munkres:topology} James R. Munkres. {\it Topology.} Prentice Hall, Incorporated, 2000.


%\bibitem{murphy}G.J. Murphy. {\it $C^*$-Algebras and Operator Theory.}Academic Press 1990.

%\bibitem{Pa1} {W.~L.~Paschke}, Inner product modules over B*-algebras,  {\it Trans.~Amer.~Math.~Soc.} {\bf 182}(1973), 443-468.
  
  \bibitem{neshveyev_tuset_qg} Sergey Neshveyev, Lars Tuset,	\textit{Compact Quantum Groups and Their Representation Categories}. 	Collection SMF.: Cours sp\'{e}cialis\'{e}s (V 20), ISSN 1284-6090, Cours Specialises, 	Cours Sp\'{e}cialis\'{e}s Collection SMF (V 20), Documents math�matiques, ISSN 1629-4939, Amer Mathematical Society, 2013.
  
  
 	\bibitem{pavlov_troisky:cov} Alexander Pavlov, Evgenij Troitsky. {\it Quantization of branched coverings.}   Russ. J. Math. Phys. (2011) 18: 338. doi:10.1134/S1061920811030071, 2011.
 

%\bibitem{pedersen:ca_aut}Pedersen G.K. {\it $C^*$-algebras and their automorphism groups}. London ; New York : Academic Press, 1979.

%\bibitem{ros_scho:kt_uct} Jonathan Rosenberg, Claude Schochet, {\it The K\"unneth theorem and the universal coefficient theorem for Kasparov's generalized K -functor}, Duke Math. J. Volume 55, Number 2 1987.

%\bibitem{phillips:inv_lim_app} N. Christopher Phillips {\it Inverse Limits of $C^*$ - algebras and Applications.} University of California at Los Angeles, Los Angeles, CA 90024, 1991

%\bibitem{reed_simon:mp_1}Michael Reed, Barry Simon. {\it Methods of modern mathematical physics 1: Functional Analysis}. Academic Press, 1972.

%\bibitem{switzer:at} Switzer R M, {\it Algebraic Topology - Homotopy and Homology},
%Springer. 2002



%\bibitem{adams:infinite_loop_spaces} J. F. Adams. {\it Infinite loop spaces}. Ann. of Math. Studies no. 90, Princeton Univ. Press, Princeton, N. J., 1978

%\bibitem{phillips:c_infty_loop} N. Christopher Philllips. {\it $C^{\infty}$ Loop Algebras and Noncommutative Bott Periodicity}. Transactions of the American Matematical Society, Volume 325, Number 2, June 1991

%\bibitem{sitarz:equiv} Andrzej Sitarz {\it Equivariant spectral triples}, Noncommutative Geometry and Quantum Groups (Piotr M. Hajac and Wieslaw Pusz, eds.), Banach Center Publ., vol 61, Polish Acad. Sci., pp. 231-268,  Warsaw 2003



%\bibitem{cuntz:o_n} J. Cuntz, {\it Simple $C^*$ - algebras generated by isometries}, Comm. Math. Phys. 57:2, 1977

%\bibitem{cuntz:k_o_n} J. Cuntz, {\it$K$ - theory of certain $C^*$ - algebras}, Ann. of Math. (2), 113:1 1981


%\bibitem{Cohn:68} Paul~Moritz Cohn. {\it {M}orita equivalence and duality}, Queen Mary College   Mathematics Notes, Dillon's Q.M.C.\ Bookshop, London, 1968.


%\bibitem{bourbaki_sp:gt} N. Bourbaki, {\it General Topology}. Chapters 1-4, Springer, Sep 18, 1998

%\bibitem{williams_sp:morita_cont_trace_alg} Iain Raeburn, Dana P. Williams. {\it Morita Equivalence and Continuous-Trace $C^*$-Algebras}. American Mathematical Soc., 1998

%\bibitem{dixmier_tr}J.Dixmier. {\it Traces sur les $C^*$-algebras}. Ann. Inst. Fourier, 13, 1(1963), 219-262, 1963

%\bibitem{baum_higson_schik:kh}Paul Baum, Nigel Higson, and Thomas Schick. {\it On the Equivalence of Geometric and Analytic $K$-Homology}. Pure and Applied Mathematics Quarterly Volume 3, Number 1 (Special Issue: In honor of Robert MacPherson, Part 3 of 3) 1-24, 2007

%\bibitem{meyer:morita} Ralf Meyer. {\it Morita Equivalence In Algebra And Geometry.} math.berkeley.edu/~alanw/277papers/meyer.tex, 1997



%\bibitem{rumynin_hopf_galois_ci} Dmitriy Rumynin  {\it Hopf-Galois extensions with central invariants.}  arXiv:q-alg/9707021 1997

%\bibitem{dixmier_a_r} Jacques Dixmier {\it Les C*-alg\`{e}bres et leurs repr\'esentations} 2e \'ed. Gauthier-Villars in Paris 1969



%\bibitem{rieffel_finite_g} Marc A. Reiffel, {\it Actions of Finite Groups on $C^*$ - Algebras}. 	Department of Mathematics University of California Berkeley. Cal. 94720 U.S.A. 1980.

%\bibitem{rieffel_morita}Marc A. Reiffel, {\it Morita equivalence for $C^*$-algebras and $W^*$-algebras }, Journal of Pure and Applied Algebra 5 (1974), 51-96. 1974.

%\bibitem{dixmier_douady_d} Claude Schochet, {\it Dixmier-Douady for Dummies}. 	arXiv:0902.2025 2009.


%\bibitem{rennie:smooth_nonunital} A. Rennie, {\it Smoothness and locality for nonunital spectral triples}, $K$-Theory 28 (2003), 127�165. 2003.

%\bibitem{schweitzer:m_local_frechet} L. B. Schweitzer. {\it  A short proof that $M_n(A)$ is local if $A$ is local and Fr\'echet}, Internat. J. Math. 3 (1992), 581-589. MR 93i:46082. 1992.


%\bibitem{spanier:at}E.H. Spanier. {\it Algebraic Topology.} McGraw-Hill. New York 1966.

\bibitem{woronowicz:su2} S.L. Woronowicz. \textit{Twisted SU(2) Group. An Example of a Non-Commutative Differential Calculus}	PubL RIMS, Kyoto Univ. 23 (1987), 117-181, 1987.


%	\bibitem{takeda:inductive} Zir\^{o} Takeda. \textit{Inductive limit and infinite direct product of operator algebras.} Tohoku Math. J. (2) 	Volume 7, Number 1-2 (1955), 67-86. 1955.

%\bibitem{takesaki:oa_ii} Takesaki, Masamichi. {\it Theory of Operator Algebras II } Encyclopaedia of Mathematical Sciences, 2003. 

%\bibitem{thomsen:ho_type_uhf} Klaus Thomsen. {\it The homotopy type of the group of automorphisms of a $UHF$-algebra}. Journal of Functional Analysis. Volume 72, Issue 1, May 1987.


%\bibitem{takeuchi:inf_out_cov}Takeuchi, Yasuji {\it Infinite outer Galois theory of non commutative rings} Osaka J. Math. Volume 3, Number 2, 1966.

%\bibitem{inikolaev:c_bundles} Igor Nikolaev, {\it Topology of the $C^*$ algebra bundles}. Centre interuniversitaire de recherche en g\'eom\'etrie diff\'erentielle et topologie UQAM Montr\'eal H3C 3P8 Canada 1999.




%\bibitem{bass} H. Bass. {\it Algebraic K-theory.} W.A. Benjamin, Inc. 1968.

%\bibitem{thompsen:homtop}Klaus Thompsen. {\it Homotopy classes of * - homomorphisms between stable $C^*$ - algebras and their muliplier algebras.} Duke Matematical Journal (C) August 1990.




%\bibitem{blackadar:oa}
%B. Blackadar. {\it Operator Algebras Theory of $C^*$ Algebras and von Neumann Algebras}. Springer-Verlag Berlin Heidelberg 2006




%\bibitem{murre:fund}
%J.P. Murre. {\it Lectures on An Introduction to Grothendieck's  Theory of the Fundamental Group.} Notes by S. Anantharaman, Tata Institute of Fundamental Research, Bombay, 1967.


%\bibitem{connes_marcolli::motives} Alain Connes Matilde Marcolli. {\it Noncommutative Geometry, Quantum Fields and Motives.} Preliminatry version. www.alainconnes.org/docs/bookwebfinal.pdf

%\bibitem{mesland::unbounded_biviariant} Bram Mesland. {\it Unbounded biviariant $K$-theory and correspondences in noncommutative geometry}. arXiv:0904.4383. 2009.



%\bibitem{connes:ng} A. Connes. {\it Noncommutative Geometry.} Academic Press, London, 1994.

%\bibitem{brown_green_rieffel:morita_stable} Lawrence G. Brown, Philip Green, and Marc A. Rieffel.  {\it Stable isomorphism and strong Morita equivalence of $C^∗$  -algebras.} Source: Pacific J. Math. Volume 71, Number 2 , 349-363, 1977


%\bibitem{faith:I} C. Faith. Algebra: {\it Rings, Modules and Cathegories I}. Springer-Verlag 1973




%\bibitem{varilly:noncom} J.C. V\'arilly. {\it An Introduction to Noncommutative Geometry}. EMS 2006.

%\bibitem{Wegge-Olsen} {N.~E.~Wegge-Olsen}, {\it K-theory and C*-algebras -- a Friendly Approach}, Oxford University Press, Oxford, England,   1993.
   
%\bibitem{weil:basic_number_theory}Andre Weil {\it Basic Number Theory}. Springer 1995

%\bibitem{Partha_quantum_su} Partha Sarathi Chakraborty, Arupkumar Pal. {\it Equivariant spectral triples on the quantum $SU(2)$ group.} arXiv:math.KT/0201004, 2003.

%\bibitem{geom_anal_k_homology} Paul Baum, Nigel Higson, and Thomas Schick {\it On the Equivalence of Geometric and Analytic K-Homology} arXiv:math/0701484, 2009.






%\bibitem{connesdebois:3dsphere}
%A. Connes, M. Dubois-Violette. Moduli space and structure of
%noncommutative 3-spheres.  LPT-ORSAY 03-34 ; IHES/M/03/56.
%Lett.Math.Phys. 66 91-121. 2003.

%\bibitem{conneslandi:isospectal}
%A. Connes, G. Landi. Noncommutative Manifolds the Instanton Algebra
%and Isospectral Deformations, math.QA/0011194, 2000.

%\bibitem{suprsym:qt}
%J. Fr\"ohlich, O. Grandjean, A. Recknagel. Supersymmetric Quantum
%Theory and (Non-Commutative) Differential Geometry, ETH-TH/96-45
%1996.

%\bibitem{ivankov:fund}
%P.R. Ivankov, N.P. Ivankov. The Noncommutative Geometry
%Generalization of Fundamental Group. arXiv:math.KT/0604508, 2006

%\bibitem{johnstone:topos}
%P.T. Johnstone. Topos Theory, L. M. S. Monographs no. 10, Academic
%Press 1977.

%\bibitem{lang}
%S. Lang. Algebra. Addison-Wesley Publishing Company, Reading, Mass
%1965.

%\bibitem{reconstr}
%A. Rennie, J.C. V\'arilly. Reconstruction of Manifolds in
%Noncommutative Geomery. \newline arXiv:math/0610418v3 [math.OA] 24
%Mar 2007.


%\bibitem{varilly:lecture}
%J.C. V\'arilly. Dirac operators and Spectral Geometry. Lecture notes
%by Pave{\l} Witkowsky from Warshaw Noncommutative Geometry, January
%2006.%http://ncg.mimuw.edu.pl/index.php?option=com_docman&task=doc_download&gid=10&Itemid=58

%\bibitem{wolf:const_curv} Wolf, J. {\it Spaces of constant curvature}. New York: McGraw-Hill, 1967.

%\bibitem{raimar_wulkenhaar:nc_spectral_triple}
%Raimar Wulkenhaar.{\it Non-compact spectral triples with finite volume}. 	arXiv:0907.1351, 2009.


\bibitem{dijkhuizen:so_doublecov} Dijkhuizen, Mathijs S. \textit{The double covering of the quantum group $SO_q(3)$}. Rend. Circ. Mat. Palermo (2) Suppl. (1994), 47-57. MR1344000, Zbl 0833.17019. 1994.

\bibitem{podles:so_su} Piotr Podle\'{s}. \textit{Symmetries of quantum spaces. Subgroups and quotient spaces of quantum SU(2) and SO(3) groups.} Comm. Math. Phys. Volume 170, Number 1 (1995), 1-20. 1995.

\end{thebibliography}

\end{document}
\endinput

