%% bare_adv.tex
%% V1.4b
%% 2015/08/26
%% by Michael Shell
%% See: 
%% http://www.michaelshell.org/
%% for current contact information.
%%
%% This is a skeleton file demonstrating the advanced use of IEEEtran.cls
%% (requires IEEEtran.cls version 1.8b or later) with an IEEE Computer
%% Society journal paper.
%%
%% Support sites:
%% http://www.michaelshell.org/tex/ieeetran/
%% http://www.ctan.org/pkg/ieeetran
%% and
%% http://www.ieee.org/

%%*************************************************************************
%% Legal Notice:
%% This code is offered as-is without any warranty either expressed or
%% implied; without even the implied warranty of MERCHANTABILITY or
%% FITNESS FOR A PARTICULAR PURPOSE! 
%% User assumes all risk.
%% In no event shall the IEEE or any contributor to this code be liable for
%% any damages or losses, including, but not limited to, incidental,
%% consequential, or any other damages, resulting from the use or misuse
%% of any information contained here.
%%
%% All comments are the opinions of their respective authors and are not
%% necessarily endorsed by the IEEE.
%%
%% This work is distributed under the LaTeX Project Public License (LPPL)
%% ( http://www.latex-project.org/ ) version 1.3, and may be freely used,
%% distributed and modified. A copy of the LPPL, version 1.3, is included
%% in the base LaTeX documentation of all distributions of LaTeX released
%% 2003/12/01 or later.
%% Retain all contribution notices and credits.
%% ** Modified files should be clearly indicated as such, including  **
%% ** renaming them and changing author support contact information. **
%%*************************************************************************


% *** Authors should verify (and, if needed, correct) their LaTeX system  ***
% *** with the testflow diagnostic prior to trusting their LaTeX platform ***
% *** with production work. The IEEE's font choices and paper sizes can   ***
% *** trigger bugs that do not appear when using other class files.       ***                          ***
% The testflow support page is at:
% http://www.michaelshell.org/tex/testflow/


% IEEEtran V1.7 and later provides for these CLASSINPUT macros to allow the
% user to reprogram some IEEEtran.cls defaults if needed. These settings
% override the internal defaults of IEEEtran.cls regardless of which class
% options are used. Do not use these unless you have good reason to do so as
% they can result in nonIEEE compliant documents. User beware. ;)
%
%\newcommand{\CLASSINPUTbaselinestretch}{1.0} % baselinestretch
%\newcommand{\CLASSINPUTinnersidemargin}{1in} % inner side margin
%\newcommand{\CLASSINPUToutersidemargin}{1in} % outer side margin
%\newcommand{\CLASSINPUTtoptextmargin}{1in}   % top text margin
%\newcommand{\CLASSINPUTbottomtextmargin}{1in}% bottom text margin




%
\documentclass[10pt,journal,compsoc]{IEEEtran}
\usepackage{amsmath}
% If IEEEtran.cls has not been installed into the LaTeX system files,
% manually specify the path to it like:
% \documentclass[10pt,journal,compsoc]{../sty/IEEEtran}


% For Computer Society journals, IEEEtran defaults to the use of 
% Palatino/Palladio as is done in IEEE Computer Society journals.
% To go back to Times Roman, you can use this code:
%\renewcommand{\rmdefault}{ptm}\selectfont





% Some very useful LaTeX packages include:
% (uncomment the ones you want to load)



% *** MISC UTILITY PACKAGES ***
%
%\usepackage{ifpdf}
% Heiko Oberdiek's ifpdf.sty is very useful if you need conditional
% compilation based on whether the output is pdf or dvi.
% usage:
% \ifpdf
%   % pdf code
% \else
%   % dvi code
% \fi
% The latest version of ifpdf.sty can be obtained from:
% http://www.ctan.org/pkg/ifpdf
% Also, note that IEEEtran.cls V1.7 and later provides a builtin
% \ifCLASSINFOpdf conditional that works the same way.
% When switching from latex to pdflatex and vice-versa, the compiler may
% have to be run twice to clear warning/error messages.






% *** CITATION PACKAGES ***
%
\ifCLASSOPTIONcompsoc
% The IEEE Computer Society needs nocompress option
% requires cite.sty v4.0 or later (November 2003)
\usepackage[nocompress]{cite}
\else
% normal IEEE
\usepackage{cite}
\fi
% cite.sty was written by Donald Arseneau
% V1.6 and later of IEEEtran pre-defines the format of the cite.sty package
% \cite{} output to follow that of the IEEE. Loading the cite package will
% result in citation numbers being automatically sorted and properly
% "compressed/ranged". e.g., [1], [9], [2], [7], [5], [6] without using
% cite.sty will become [1], [2], [5]--[7], [9] using cite.sty. cite.sty's
% \cite will automatically add leading space, if needed. Use cite.sty's
% noadjust option (cite.sty V3.8 and later) if you want to turn this off
% such as if a citation ever needs to be enclosed in parenthesis.
% cite.sty is already installed on most LaTeX systems. Be sure and use
% version 5.0 (2009-03-20) and later if using hyperref.sty.
% The latest version can be obtained at:
% http://www.ctan.org/pkg/cite
% The documentation is contained in the cite.sty file itself.
%
% Note that some packages require special options to format as the Computer
% Society requires. In particular, Computer Society  papers do not use
% compressed citation ranges as is done in typical IEEE papers
% (e.g., [1]-[4]). Instead, they list every citation separately in order
% (e.g., [1], [2], [3], [4]). To get the latter we need to load the cite
% package with the nocompress option which is supported by cite.sty v4.0
% and later.





% *** GRAPHICS RELATED PACKAGES ***
%
\ifCLASSINFOpdf
 \usepackage[pdftex]{graphicx}
% declare the path(s) where your graphic files are
 \graphicspath{{./png/}{../jpeg/}}
% and their extensions so you won't have to specify these with
% every instance of \includegraphics
\DeclareGraphicsExtensions{.pdf,.jpeg,.png}
\else
% or other class option (dvipsone, dvipdf, if not using dvips). graphicx
% will default to the driver specified in the system graphics.cfg if no
% driver is specified.
 %\usepackage[dvips]{graphicx}
% declare the path(s) where your graphic files are
% \graphicspath{{.}}
% and their extensions so you won't have to specify these with
% every instance of \includegraphics
% \DeclareGraphicsExtensions{.png}
\fi
% graphicx was written by David Carlisle and Sebastian Rahtz. It is
% required if you want graphics, photos, etc. graphicx.sty is already
% installed on most LaTeX systems. The latest version and documentation
% can be obtained at: 
% http://www.ctan.org/pkg/graphicx
% Another good source of documentation is "Using Imported Graphics in
% LaTeX2e" by Keith Reckdahl which can be found at:
% http://www.ctan.org/pkg/epslatex
%
% latex, and pdflatex in dvi mode, support graphics in encapsulated
% postscript (.eps) format. pdflatex in pdf mode supports graphics
% in .pdf, .jpeg, .png and .mps (metapost) formats. Users should ensure
% that all non-photo figures use a vector format (.eps, .pdf, .mps) and
% not a bitmapped formats (.jpeg, .png). The IEEE frowns on bitmapped formats
% which can result in "jaggedy"/blurry rendering of lines and letters as
% well as large increases in file sizes.
%
% You can find documentation about the pdfTeX application at:
% http://www.tug.org/applications/pdftex





% *** MATH PACKAGES ***
%
%\usepackage{amsmath}
% A popular package from the American Mathematical Society that provides
% many useful and powerful commands for dealing with mathematics.
%
% Note that the amsmath package sets \interdisplaylinepenalty to 10000
% thus preventing page breaks from occurring within multiline equations. Use:
%\interdisplaylinepenalty=2500
% after loading amsmath to restore such page breaks as IEEEtran.cls normally
% does. amsmath.sty is already installed on most LaTeX systems. The latest
% version and documentation can be obtained at:
% http://www.ctan.org/pkg/amsmath





% *** SPECIALIZED LIST PACKAGES ***
%\usepackage{acronym}
% acronym.sty was written by Tobias Oetiker. This package provides tools for
% managing documents with large numbers of acronyms. (You don't *have* to
% use this package - unless you have a lot of acronyms, you may feel that
% such package management of them is bit of an overkill.)
% Do note that the acronym environment (which lists acronyms) will have a
% problem when used under IEEEtran.cls because acronym.sty relies on the
% description list environment - which IEEEtran.cls has customized for
% producing IEEE style lists. A workaround is to declared the longest
% label width via the IEEEtran.cls \IEEEiedlistdecl global control:
%
% \renewcommand{\IEEEiedlistdecl}{\IEEEsetlabelwidth{SONET}}
% \begin{acronym}
	%
	% \end{acronym}
% \renewcommand{\IEEEiedlistdecl}{\relax}% remember to reset \IEEEiedlistdecl
%
% instead of using the acronym environment's optional argument.
% The latest version and documentation can be obtained at:
% http://www.ctan.org/pkg/acronym


%\usepackage{algorithmic}
% algorithmic.sty was written by Peter Williams and Rogerio Brito.
% This package provides an algorithmic environment fo describing algorithms.
% You can use the algorithmic environment in-text or within a figure
% environment to provide for a floating algorithm. Do NOT use the algorithm
% floating environment provided by algorithm.sty (by the same authors) or
% algorithm2e.sty (by Christophe Fiorio) as the IEEE does not use dedicated
% algorithm float types and packages that provide these will not provide
% correct IEEE style captions. The latest version and documentation of
% algorithmic.sty can be obtained at:
% http://www.ctan.org/pkg/algorithms
% Also of interest may be the (relatively newer and more customizable)
% algorithmicx.sty package by Szasz Janos:
% http://www.ctan.org/pkg/algorithmicx




% *** ALIGNMENT PACKAGES ***
%
%\usepackage{array}
% Frank Mittelbach's and David Carlisle's array.sty patches and improves
% the standard LaTeX2e array and tabular environments to provide better
% appearance and additional user controls. As the default LaTeX2e table
% generation code is lacking to the point of almost being broken with
% respect to the quality of the end results, all users are strongly
% advised to use an enhanced (at the very least that provided by array.sty)
% set of table tools. array.sty is already installed on most systems. The
% latest version and documentation can be obtained at:
% http://www.ctan.org/pkg/array


%\usepackage{mdwmath}
%\usepackage{mdwtab}
% Also highly recommended is Mark Wooding's extremely powerful MDW tools,
% especially mdwmath.sty and mdwtab.sty which are used to format equations
% and tables, respectively. The MDWtools set is already installed on most
% LaTeX systems. The lastest version and documentation is available at:
% http://www.ctan.org/pkg/mdwtools


% IEEEtran contains the IEEEeqnarray family of commands that can be used to
% generate multiline equations as well as matrices, tables, etc., of high
% quality.


%\usepackage{eqparbox}
% Also of notable interest is Scott Pakin's eqparbox package for creating
% (automatically sized) equal width boxes - aka "natural width parboxes".
% Available at:
% http://www.ctan.org/pkg/eqparbox




% *** SUBFIGURE PACKAGES ***
%\ifCLASSOPTIONcompsoc
%  \usepackage[caption=false,font=footnotesize,labelfont=sf,textfont=sf]{subfig}
%\else
%  \usepackage[caption=false,font=footnotesize]{subfig}
%\fi
% subfig.sty, written by Steven Douglas Cochran, is the modern replacement
% for subfigure.sty, the latter of which is no longer maintained and is
% incompatible with some LaTeX packages including fixltx2e. However,
% subfig.sty requires and automatically loads Axel Sommerfeldt's caption.sty
% which will override IEEEtran.cls' handling of captions and this will result
% in non-IEEE style figure/table captions. To prevent this problem, be sure
% and invoke subfig.sty's "caption=false" package option (available since
% subfig.sty version 1.3, 2005/06/28) as this is will preserve IEEEtran.cls
% handling of captions.
% Note that the Computer Society format requires a sans serif font rather
% than the serif font used in traditional IEEE formatting and thus the need
% to invoke different subfig.sty package options depending on whether
% compsoc mode has been enabled.
%
% The latest version and documentation of subfig.sty can be obtained at:
% http://www.ctan.org/pkg/subfig




% *** FLOAT PACKAGES ***
%
%\usepackage{fixltx2e}
% fixltx2e, the successor to the earlier fix2col.sty, was written by
% Frank Mittelbach and David Carlisle. This package corrects a few problems
% in the LaTeX2e kernel, the most notable of which is that in current
% LaTeX2e releases, the ordering of single and double column floats is not
% guaranteed to be preserved. Thus, an unpatched LaTeX2e can allow a
% single column figure to be placed prior to an earlier double column
% figure.
% Be aware that LaTeX2e kernels dated 2015 and later have fixltx2e.sty's
% corrections already built into the system in which case a warning will
% be issued if an attempt is made to load fixltx2e.sty as it is no longer
% needed.
% The latest version and documentation can be found at:
% http://www.ctan.org/pkg/fixltx2e


%\usepackage{stfloats}
% stfloats.sty was written by Sigitas Tolusis. This package gives LaTeX2e
% the ability to do double column floats at the bottom of the page as well
% as the top. (e.g., "\begin{figure*}[!b]" is not normally possible in
	% LaTeX2e). It also provides a command:
	%\fnbelowfloat
	% to enable the placement of footnotes below bottom floats (the standard
	% LaTeX2e kernel puts them above bottom floats). This is an invasive package
	% which rewrites many portions of the LaTeX2e float routines. It may not work
	% with other packages that modify the LaTeX2e float routines. The latest
	% version and documentation can be obtained at:
	% http://www.ctan.org/pkg/stfloats
	% Do not use the stfloats baselinefloat ability as the IEEE does not allow
	% \baselineskip to stretch. Authors submitting work to the IEEE should note
	% that the IEEE rarely uses double column equations and that authors should try
	% to avoid such use. Do not be tempted to use the cuted.sty or midfloat.sty
	% packages (also by Sigitas Tolusis) as the IEEE does not format its papers in
	% such ways.
	% Do not attempt to use stfloats with fixltx2e as they are incompatible.
	% Instead, use Morten Hogholm'a dblfloatfix which combines the features
	% of both fixltx2e and stfloats:
	%
	% \usepackage{dblfloatfix}
	% The latest version can be found at:
	% http://www.ctan.org/pkg/dblfloatfix
	
	
	%\ifCLASSOPTIONcaptionsoff
	%  \usepackage[nomarkers]{endfloat}
	% \let\MYoriglatexcaption\caption
	% \renewcommand{\caption}[2][\relax]{\MYoriglatexcaption[\#2]{\#2}}
	%\fi
	% endfloat.sty was written by James Darrell McCauley, Jeff Goldberg and 
	% Axel Sommerfeldt. This package may be useful when used in conjunction with 
	% IEEEtran.cls'  captionsoff option. Some IEEE journals/societies require that
	% submissions have lists of figures/tables at the end of the paper and that
	% figures/tables without any captions are placed on a page by themselves at
	% the end of the document. If needed, the draftcls IEEEtran class option or
	% \CLASSINPUTbaselinestretch interface can be used to increase the line
	% spacing as well. Be sure and use the nomarkers option of endfloat to
	% prevent endfloat from "marking" where the figures would have been placed
	% in the text. The two hack lines of code above are a slight modification of
	% that suggested by in the endfloat docs (section 8.4.1) to ensure that
	% the full captions always appear in the list of figures/tables - even if
	% the user used the short optional argument of \caption[]{}.
	% IEEE papers do not typically make use of \caption[]'s optional argument,
	% so this should not be an issue. A similar trick can be used to disable
	% captions of packages such as subfig.sty that lack options to turn off
	% the subcaptions:
	% For subfig.sty:
	% \let\MYorigsubfloat\subfloat
	% \renewcommand{\subfloat}[2][\relax]{\MYorigsubfloat[]{\#2}}
	% However, the above trick will not work if both optional arguments of
	% the \subfloat command are used. Furthermore, there needs to be a
	% description of each subfigure *somewhere* and endfloat does not add
	% subfigure captions to its list of figures. Thus, the best approach is to
	% avoid the use of subfigure captions (many IEEE journals avoid them anyway)
	% and instead reference/explain all the subfigures within the main caption.
	% The latest version of endfloat.sty and its documentation can obtained at:
	% http://www.ctan.org/pkg/endfloat
	%
	% The IEEEtran \ifCLASSOPTIONcaptionsoff conditional can also be used
	% later in the document, say, to conditionally put the References on a 
	% page by themselves.
	
	
	
	
	
	% *** PDF, URL AND HYPERLINK PACKAGES ***
	%
	%\usepackage{url}
	% url.sty was written by Donald Arseneau. It provides better support for
	% handling and breaking URLs. url.sty is already installed on most LaTeX
	% systems. The latest version and documentation can be obtained at:
	% http://www.ctan.org/pkg/url
	% Basically, \url{my_url_here}.
	
	
	% NOTE: PDF thumbnail features are not required in IEEE papers
	%       and their use requires extra complexity and work.
	%\ifCLASSINFOpdf
	%  \usepackage[pdftex]{thumbpdf}
	%\else
	%  \usepackage[dvips]{thumbpdf}
	%\fi
	% thumbpdf.sty and its companion Perl utility were written by Heiko Oberdiek.
	% It allows the user a way to produce PDF documents that contain fancy
	% thumbnail images of each of the pages (which tools like acrobat reader can
	% utilize). This is possible even when using dvi->ps->pdf workflow if the
	% correct thumbpdf driver options are used. thumbpdf.sty incorporates the
	% file containing the PDF thumbnail information (filename.tpm is used with
	% dvips, filename.tpt is used with pdftex, where filename is the base name of
	% your tex document) into the final ps or pdf output document. An external
	% utility, the thumbpdf *Perl script* is needed to make these .tpm or .tpt
	% thumbnail files from a .ps or .pdf version of the document (which obviously
	% does not yet contain pdf thumbnails). Thus, one does a:
	% 
	% thumbpdf filename.pdf 
	%
	% to make a filename.tpt, and:
	%
	% thumbpdf --mode dvips filename.ps
	%
	% to make a filename.tpm which will then be loaded into the document by
	% thumbpdf.sty the NEXT time the document is compiled (by pdflatex or
	% latex->dvips->ps2pdf). Users must be careful to regenerate the .tpt and/or
	% .tpm files if the main document changes and then to recompile the
	% document to incorporate the revised thumbnails to ensure that thumbnails
	% match the actual pages. It is easy to forget to do this!
	% 
	% Unix systems come with a Perl interpreter. However, MS Windows users
	% will usually have to install a Perl interpreter so that the thumbpdf
	% script can be run. The Ghostscript PS/PDF interpreter is also required.
	% See the thumbpdf docs for details. The latest version and documentation
	% can be obtained at.
	% http://www.ctan.org/pkg/thumbpdf
	
	
	% NOTE: PDF hyperlink and bookmark features are not required in IEEE
	%       papers and their use requires extra complexity and work.
	% *** IF USING HYPERREF BE SURE AND CHANGE THE EXAMPLE PDF ***
	% *** TITLE/SUBJECT/AUTHOR/KEYWORDS INFO BELOW!!           ***
	\newcommand\MYhyperrefoptions{bookmarks=true,bookmarksnumbered=true,
		pdfpagemode={UseOutlines},plainpages=false,pdfpagelabels=true,
		colorlinks=true,linkcolor={black},citecolor={black},urlcolor={black},
		pdftitle={The Concept of the Universal Language and its Applications},%<!CHANGE!
		pdfsubject={Typesetting},%<!CHANGE!
		pdfauthor={Michael D. Shell},%<!CHANGE!
		pdfkeywords={Computer Society, IEEEtran, journal, LaTeX, paper,
			template}}%<^!CHANGE!
	%\ifCLASSINFOpdf
	%\usepackage[\MYhyperrefoptions,pdftex]{hyperref}
	%\else
	%\usepackage[\MYhyperrefoptions,breaklinks=true,dvips]{hyperref}
	%\usepackage{breakurl}
	%\fi
	% One significant drawback of using hyperref under DVI output is that the
	% LaTeX compiler cannot break URLs across lines or pages as can be done
	% under pdfLaTeX's PDF output via the hyperref pdftex driver. This is
	% probably the single most important capability distinction between the
	% DVI and PDF output. Perhaps surprisingly, all the other PDF features
	% (PDF bookmarks, thumbnails, etc.) can be preserved in
	% .tex->.dvi->.ps->.pdf workflow if the respective packages/scripts are
	% loaded/invoked with the correct driver options (dvips, etc.). 
	% As most IEEE papers use URLs sparingly (mainly in the references), this
	% may not be as big an issue as with other publications.
	%
	% That said, Vilar Camara Neto created his breakurl.sty package which
	% permits hyperref to easily break URLs even in dvi mode.
	% Note that breakurl, unlike most other packages, must be loaded
	% AFTER hyperref. The latest version of breakurl and its documentation can
	% be obtained at:
	% http://www.ctan.org/pkg/breakurl
	% breakurl.sty is not for use under pdflatex pdf mode.
	%
	% The advanced features offer by hyperref.sty are not required for IEEE
	% submission, so users should weigh these features against the added
	% complexity of use.
	% The package options above demonstrate how to enable PDF bookmarks
	% (a type of table of contents viewable in Acrobat Reader) as well as
	% PDF document information (title, subject, author and keywords) that is
	% viewable in Acrobat reader's Document_Properties menu. PDF document
	% information is also used extensively to automate the cataloging of PDF
	% documents. The above set of options ensures that hyperlinks will not be
	% colored in the text and thus will not be visible in the printed page,
	% but will be active on "mouse over". USING COLORS OR OTHER HIGHLIGHTING
	% OF HYPERLINKS CAN RESULT IN DOCUMENT REJECTION BY THE IEEE, especially if
	% these appear on the "printed" page. IF IN DOUBT, ASK THE RELEVANT
	% SUBMISSION EDITOR. You may need to add the option hypertexnames=false if
	% you used duplicate equation numbers, etc., but this should not be needed
	% in normal IEEE work.
	% The latest version of hyperref and its documentation can be obtained at:
	% http://www.ctan.org/pkg/hyperref
	
	
	
	
	
	% *** Do not adjust lengths that control margins, column widths, etc. ***
	% *** Do not use packages that alter fonts (such as pslatex).         ***
	% There should be no need to do such things with IEEEtran.cls V1.6 and later.
	% (Unless specifically asked to do so by the journal or conference you plan
	% to submit to, of course. )
	
	
	% correct bad hyphenation here
	\hyphenation{op-tical net-works semi-conduc-tor}
	
	
	\begin{document}
		%
		% paper title
		% Titles are generally capitalized except for words such as a, an, and, as,
		% at, but, by, for, in, nor, of, on, or, the, to and up, which are usually
		% not capitalized unless they are the first or last word of the title.
		% Linebreaks \\ can be used within to get better formatting as desired.
		% Do not put math or special symbols in the title.
		\title{Asynchronous Animation of Realistic Motion}
		%
		%
		% author names and IEEE memberships
		% note positions of commas and nonbreaking spaces ( ~ ) LaTeX will not break
		% a structure at a ~ so this keeps an author's name from being broken across
		% two lines.
		% use \thanks{} to gain access to the first footnote area
		% a separate \thanks must be used for each paragraph as LaTeX2e's \thanks
		% was not built to handle multiple paragraphs
		%
		%
		%\IEEEcompsocitemizethanks is a special \thanks that produces the bulleted
		% lists the Computer Society journals use for "first footnote" author
		% affiliations. Use \IEEEcompsocthanksitem which works much like \item
		% for each affiliation group. When not in compsoc mode,
		% \IEEEcompsocitemizethanks becomes like \thanks and
		% \IEEEcompsocthanksitem becomes a line break with idention. This
		% facilitates dual compilation, although admittedly the differences in the
		% desired content of \author between the different types of papers makes a
		% one-size-fits-all approach a daunting prospect. For instance, compsoc 
		% journal papers have the author affiliations above the "Manuscript
		% received ..."  text while in non-compsoc journals this is reversed. Sigh.
		
		\author{Petr~Ivankov,~\IEEEmembership{!!!Member,~IEEE,}
			John~Doe,~\IEEEmembership{Fellow,~OSA,}
			and~Max~Shestov\IEEEmembership{!!!Life~Fellow,~IEEE}% <-this % stops a space
			\IEEEcompsocitemizethanks{\IEEEcompsocthanksitem P. Ivankov !!!was with the Department
				of Electrical and Computer Engineering, Georgia Institute of Technology, Atlanta,
				GA, 30332.\protect\\
				% note need leading \protect in front of \\ to get a newline within \thanks as
				% \\ is fragile and will error, could use \hfil\break instead.
				E-mail: see http://www.michaelshell.org/contact.html
				\IEEEcompsocthanksitem M. Shestov and M. Shestov are with !!! Anonymous University.}% <-this % stops a space
			\thanks{Manuscript received April 19, 2005; revised August 26, 2015.}}
		
		% note the % following the last \IEEEmembership and also \thanks - 
		% these prevent an unwanted space from occurring between the last author name
		% and the end of the author line. i.e., if you had this:
		% 
		% \author{....lastname \thanks{...} \thanks{...} }
		%                     ^------------^------------^----Do not want these spaces!
		%
		% a space would be appended to the last name and could cause every name on that
		% line to be shifted left slightly. This is one of those "LaTeX things". For
		% instance, "\textbf{A} \textbf{B}" will typeset as "A B" not "AB". To get
		% "AB" then you have to do: "\textbf{A}\textbf{B}"
		% \thanks is no different in this regard, so shield the last } of each \thanks
	% that ends a line with a % and do not let a space in before the next \thanks.
	% Spaces after \IEEEmembership other than the last one are OK (and needed) as
	% you are supposed to have spaces between the names. For what it is worth,
	% this is a minor point as most people would not even notice if the said evil
	% space somehow managed to creep in.
	
	
	
	% The paper headers
	\markboth{Journal of \LaTeX\ Class Files,~Vol.~14, No.~8, August~2015}%
	{Shell \MakeLowercase{\textit{et al.}}: Bare Advanced Demo of IEEEtran.cls for IEEE Computer Society Journals}
	% The only time the second header will appear is for the odd numbered pages
	% after the title page when using the twoside option.
	% 
	% *** Note that you probably will NOT want to include the author's ***
	% *** name in the headers of peer review papers.                   ***
	% You can use \ifCLASSOPTIONpeerreview for conditional compilation here if
	% you desire.
	
	
	
	% The publisher's ID mark at the bottom of the page is less important with
	% Computer Society journal papers as those publications place the marks
	% outside of the main text columns and, therefore, unlike regular IEEE
	% journals, the available text space is not reduced by their presence.
	% If you want to put a publisher's ID mark on the page you can do it like
	% this:
	%\IEEEpubid{0000--0000/00\$00.00~\copyright~2015 IEEE}
	% or like this to get the Computer Society new two part style.
	%\IEEEpubid{\makebox[\columnwidth]{\hfill 0000--0000/00/\$00.00~\copyright~2015 IEEE}%
		%\hspace{\columnsep}\makebox[\columnwidth]{Published by the IEEE Computer Society\hfill}}
	% Remember, if you use this you must call \IEEEpubidadjcol in the second
	% column for its text to clear the IEEEpubid mark (Computer Society journal
	% papers don't need this extra clearance.)
	
	
	
	% use for special paper notices
	%\IEEEspecialpapernotice{(Invited Paper)}
	
	
	
	% for Computer Society papers, we must declare the abstract and index terms
	% PRIOR to the title within the \IEEEtitleabstractindextext IEEEtran
	% command as these need to go into the title area created by \maketitle.
	% As a general rule, do not put math, special symbols or citations
	% in the abstract or keywords.
	\IEEEtitleabstractindextext{%
		\begin{abstract}
		Here we consider the animation of realistic motion of described objects described by complicated system of differential equations. Both synchronous and asynchronous methods are being used.
		\end{abstract}
		
		% Note that keywords are not normally used for peerreview papers.
		\begin{IEEEkeywords}
			Computer Society, IEEE, IEEEtran, journal, \LaTeX, paper, template.
	\end{IEEEkeywords}}
	
	
	% make the title area
	\maketitle
	
	
	% To allow for easy dual compilation without having to reenter the
	% abstract/keywords data, the \IEEEtitleabstractindextext text will
	% not be used in maketitle, but will appear (i.e., to be "transported")
	% here as \IEEEdisplaynontitleabstractindextext when compsoc mode
	% is not selected <OR> if conference mode is selected - because compsoc
	% conference papers position the abstract like regular (non-compsoc)
	% papers do!
	\IEEEdisplaynontitleabstractindextext
	
	% \IEEEdisplaynontitleabstractindextext has no effect when using
	% compsoc under a non-conference mode.
	
	
	% For peer review papers, you can put extra information on the cover
	% page as needed:
	% \ifCLASSOPTIONpeerreview
	% \begin{center} \bfseries EDICS Category: 3-BBND \end{center}
	% \fi
	%
	% For peerreview papers, this IEEEtran command inserts a page break and
	% creates the second title. It will be ignored for other modes.
	\IEEEpeerreviewmaketitle
	
	
	\ifCLASSOPTIONcompsoc
	\IEEEraisesectionheading{\section{Introduction}\label{sec:introduction1}}
	\else
	\section{Introduction}
	\label{sec:introduction2}
	\fi
	% Computer Society journal (but not conference!) papers do something unusual
	% with the very first section heading (almost always called "Introduction").
	% They place it ABOVE the main text! IEEEtran.cls does not automatically do
	% this for you, but you can achieve this effect with the provided
	% \IEEEraisesectionheading{} command. Note the need to keep any \label that
	% is to refer to the section immediately after \section in the above as
	% \IEEEraisesectionheading puts \section within a raised box.
	
	
	
	
	% The very first letter is a 2 line initial drop letter followed
	% by the rest of the first word in caps (small caps for compsoc).
	% 
	% form to use if the first word consists of a single letter:
	% \IEEEPARstart{A}{demo} file is ....
	% 
	% form to use if you need the single drop letter followed by
	% normal text (unknown if ever used by the IEEE):
	% \IEEEPARstart{A}{}demo file is ....
	% 
	% Some journals put the first two words in caps:
	% \IEEEPARstart{T}{his demo} file is ....
	% 
	% Here we have the typical use of a "T" for an initial drop letter
	% and "HIS" in caps to complete the first word.
	\IEEEPARstart {T}HE realistic motion is usually described by ordinary differential equations (ODE). Solution of ordinary differential equations mostly requires {numerical methods}. Some equations assume a lot of calculations. For example motion model of artificial Earth's contains model of {Earth's gravity} and {Earth's atmosphere }. Numerical methods calculate motion parameters at discrete values of time $t_1$, $t_2$,$t_3$, ... The difference $dt = t_{i+1}-t_i$ is called a step size. Reduction of step size supplies a more accurate solution of ODE. However there is the optimal value of $dt_0$ such that there is no substantial difference between solutions given by step sizes $dt_{\text{opt}}$ and $dt < dt_{\text{opt}}$. Value of $dt_{\text{opt}}$ depends on ODE. Moreover this value is not a constant. In case of robot pilot control value of $dt_{\text{opt}}$ is bigger then in case of manual control. A lot of tasks require animation. If $dt_{\text{opt}}$ is small then animation can be performed by following way. Computer solves ODE step by step and simultaneously updates 3D images. This method is in fact synchronous animation which is well known long time ago. However if $dt_{\text{opt}}$ is not small then synchronous animation occurs considerable jumps of 3D images. There are following ways to avoid this problem: 
	\begin{itemize}
		\item Reduction of step size; 
		\item Application of asynchronous animation.
	\end{itemize}
First way sometimes is not good because it requires substantial enlargement of calculation. Asynchronous animation is explained by following picture. 
\section{Types of animation}	
\subsection{Asynchronous animation}

Real trajectory of 3D object is replaced with {piece wise linear} trajectory, which is close to real. However synchronous animation corresponds to {piece wise constant function} which is presented below.

\subsection{Synchronous animation}

Synchronous animation occurs substantial jumps of cube, jumps of asynchronous animation are scarcely visible. Present day technologies support a lot of asynchronous animation methods with a powerful hardware support. In this article the WPF animation is being used as example. However it can be easy adapted for other types of animation.

\section{Math Background}	
Uniform motion from 3D point $p_1 = (x_1, y_1, z_1)$ to $p_2 = (x_2, y_2, z_2)$ can be represented by following elementary expression
$$
p(t) = (x_1(1 - t) + x_2t, y_1(1 - t) + y_2t, z_1(1 - t) + z_2t) 
$$
where $t$ is a current progress. Uniform rotational motion can be easy described by {Quaternion}. Uniform rotation from quaternion $Q^1 = Q_0^1 + Q_1^1i + Q_2^1j + Q_3^1k$ to  $Q^2 = Q_0^2 + Q_1^2i + Q_2^2j + Q_3^2k$
quaternion 
$$
Q(t) = Q_1Q^{\text{uniform}}(t)
$$
where $Q^{\text{uniform}}$ corresponds to uniform rotation. Since uniform rotation is supported by WPF we do not need explicit calculation of $Q^{\text{uniform}}$. We need {axis of rotation and rotation angle} only. These parameters can be given from a following relative rotation quaternion
$$
Q^{\text{relative}}= Q^{1*}Q^2
$$
where $Q_0^{1*} - Q_1^1i - Q_2^1j - Q_3^1k$.
Otherwise
$$
Q^{\text{relative}}= \text{cos} (a/2) + r_x\text{sin}(a/2)i + r_y\text{sin}(a/2)i + r_z\text{sin}(a/2)k.
$$

Where a is a full rotation angle, direction of vector r = (rx, ry, rz) coincides with rotation axis. If r= 0 (resp. r is close to 0) then rotation is abscent (resp. can be neglected). Above equations are implemented by Uniform6DMotion class which is contained is source code. This class is independent from WPF or any other animation implementation.

3. WPF Implementation of the Uniform 6D Motion

Uniform 6D motion is implemented by Uniform6DTransformation which is a {custom animation class}. Following code snippet explains its logics.
/// <summary>
/// Custom animation class for uniform 6D motion
/// </summary>
public class Uniform6DTransformation : AnimationTimeline
{
	
}
Above code means that a transformatin is decomoposed to
* Uniform 3D translation; 
* Constant 3D rotation; 
* Uniform 3D rotation. 
4. General Asynchronous Animation Algorithm

Asynchronous animation architecture contains abstract level which corresponds to following interface.
/// <summary>
/// Driver of animation
/// </summary>
public interface IAnimationDriver
{
	/// <summary>
	/// Starts animation
	/// </summary>
	/// <param name="collection">Collection of components</param>
	/// <param name="reasons">Reasons</param>
	/// <param name="animationType">Type of animation</param>
	/// <param name="pause">Pause</param>
	/// <param name="timeScale">Time scale</param>
	/// <param name="realTime">The "real time" sign</param>
	/// <param name="absoluteTime">The "absolute time" sign</param>
	/// <returns>Animation asynchronous calculation</returns>
	object StartAnimation(IComponentCollection collection, string[] reasons,
	Enums.AnimationType animationType,
	TimeSpan pause, double timeScale, bool realTime, bool absoluteTime);
	
	/// <summary>
	/// Supports asynchronous
	/// </summary>
	bool SuppotrsAsynchronous
	{
		get;
	}
}

/// <summary>
/// Type of action
/// </summary>
public enum ActionType
{
	/// <summary>
	/// Calculation
	/// </summary>
	Calculation,
	/// <summary>
	/// Animation
	/// </summary>
	Animation
}


/// <summary>
/// Type of animation
/// </summary>
public enum AnimationType
{
	/// <summary>
	/// Synchronous
	/// </summary>
	Synchronous,
	
	/// <summary>
	/// Asynchronous
	/// </summary>
	Asynchronous
}
The StartAnimation function returns object which implements following interface.
/// <summary>
/// Asynchronous calculation
/// </summary>
public interface IAsynchronousCalculation
{
	/// <summary>
	/// Starts itself
	/// </summary>
	/// <param name="time"></param>
	void Start(double time);
	
	/// <summary>
	/// Step
	/// </summary>
	Action<double> Step
	{
		get;
	}
	
	/// <summary>
	/// Stops itself
	/// </summary>
	void Interrupt();
	
	/// <summary>
	/// Suspends itself
	/// </summary>
	void Suspend();
	
	/// <summary>
	/// The "is running" sign
	/// </summary>
	bool IsRunning
	{
		get;
	}
	
	/// <summary>
	/// Suspend event
	/// </summary>
	event Action OnSuspend;
	
	/// <summary>
	/// Finish event
	/// </summary>
	event Action Finish;
	
	/// <summary>
	/// Interrupt
	/// </summary>
	event Action OnInterrupt;
	
}
Above interface is not intended for animation only. Following table contains different implementations of this interface.
N  Class name  Purpose 
1  PauseAsynchronousCalculation  Synchronous animation 
2  WpfAsynchronousAnimatedCalculation  Asynchronous animation for WPF 
3  WpfAsynchronousRealtimeAnimatedCalculation  Asynchronous real-time animation for WPF 

The real-time mode is described below.

The PauseAsynchronousCalculation does not depend on WPF or other animation technology. This class just performs a pause. Following code explains this circumstance.
/// <summary>
/// Pause asynchronous calculation
/// </summary>
public class PauseAsynchronousCalculation : IAsynchronousCalculation
{
	\#region Fields
	
	private Action<double> pause;
	
	// ...
	
	/// <summary>
	/// Constructor
	/// </summary>
	/// <param name="pauseSpan">Pause</param>
	public PauseAsynchronousCalculation(TimeSpan pauseSpan)
	{
		// Pause action
		pause = (double time) =>
		{
			Thread.Sleep(pauseSpan);
		};
	}
	
	// ...
	
	Action<double> IAsynchronousCalculation.Step
	{
		get { return pause; }
	}
	
	// ...
}    
Every step of synchronous animation assumes following operations:
* Calculation of motion parameters; 
* Showing of 3D objects with new 6D positions; 
* Time pause. 
Asynchronous animation assumes one thread of calculation and many threads for animation. Every step of asynchronous animation calculation contains following operations:
* Calculation of motion parameters; 
* Enqueuing motion parameters and current time to every animation thread. 
Every animation thread performs following operations:
* 1. Dequeues motion parameters and current time; 
* 2. Performs calculations described in {section 2}. 
* 3. Calls {IAnimatable.BeginAnimation method}. 
* 4. Callback of animation completed event ({Timeline.Completed event}) checks queue. If the queue is not empty then it goes to operation 1. 
5. Calculation of motion parameters

Calculation of motion parameters is already described in {my previous article}. However I overlap some text from my previous article for convenience.

A {kinematics} domain contains following basic types:
* 3D Position (IPosition interface);  
* 3D Orientation (IOrientation interface; 
* Standard 3D Position (Position class which implements IPosition interface);  
* 3D Reference frame (ReferenceFrame class which implements both IPosition and IOrientation);  
* Holder 3D Reference frame (IReferenceFrame interface) ; 
* Reference frame binding (ReferenceFrameArrow class which implements ICategoryArrow interface). 
Source (resp. target) of ReferenceFrameArrow is always IPosition (resp. IReferenceFrame). This arrow means that coordinates of IPosition are relative with respect to IReferenceFrame. Following code represents these types:
/// <summary>
/// 3D Position
/// </summary>
public interface IPosition
{
	
	/// <summary>
	/// Absolute position coordinates
	/// </summary>
	double[] Position
	{
		get;
	}
	
	/// <summary>
	/// Parent frame
	/// </summary>
	IReferenceFrame Parent
	{
		get;
		set;
	}
	
	/// <summary>
	/// Position parameters
	/// </summary>
	object Parameters
	{
		get;
		set;
	}
	
	/// <summary>
	/// Updates itself
	/// </summary>
	void Update();
}
/// <summary>
/// Object with orientation
/// </summary>
public interface IOrientation
{
	/// <summary>
	/// Orientation quaternion
	/// </summary>
	double[] Quaternion
	{
		get;
	}
	
	/// <summary>
	/// Orientation matrix
	/// </summary>
	double[,] Matrix
	{
		get;
	}
}
/// <summary>
/// Standard position
/// </summary>
public class Position : IPosition, IChildrenObject
{
	
	\#region Fields
	
	/// <summary>
	/// Parent frame
	/// </summary>
	protected IReferenceFrame parent;
	
	/// <summary>
	/// Own position
	/// </summary>
	protected double[] own = new double[] { 0, 0, 0 };
	
	/// <summary>
	/// Relatyive position
	/// </summary>
	protected double[] position = new double[3];
	
	/// <summary>
	/// Parameters
	/// </summary>
	protected object parameters;
	
	/// <summary>
	/// Children objects
	/// </summary>
	protected IAssociatedObject[] ch = new IAssociatedObject[1];
	
	\#endregion
	
	\#region Ctor
	
	/// <summary>
	/// Default constructor
	/// </summary>
	protected Position()
	{
	}
	
	/// <summary>
	/// Constructor
	/// </summary>
	/// <param name="position">Position coordinates</param>
	public Position(double[] position)
	{
		for (int i = 0; i < own.Length; i++)
		{
			own[i] = position[i];
		}
	}
	
	\#endregion
	
	\#region IPosition Members
	
	double[] IPosition.Position
	{
		get { return position; }
	}
	
	/// <summary>
	/// Parent frame
	/// </summary>
	public virtual IReferenceFrame Parent
	{
		get
		{
			return parent;
		}
		set
		{
			parent = value;
		}
	}
	
	/// <summary>
	/// Position parameters
	/// </summary>
	public virtual object Parameters
	{
		get
		{
			return parameters;
		}
		set
		{
			parameters = value;
			if (value is IAssociatedObject)
			{
				IAssociatedObject ao = value as IAssociatedObject;
				ch[0] = ao;
			}
		}
	}
	
	/// <summary>
	/// Updates itself
	/// </summary>
	public virtual void Update()
	{
		Update(BaseFrame);
	}
	
	\#endregion
	
	// ...
}
/// <summary>
/// Reference frame
/// </summary>
public class ReferenceFrame : IPosition, IOrientation
{
	
	\#region Fields
	
	/// <summary>
	/// Orientation quaternion
	/// </summary>
	protected double[] quaternion = new double[] { 1, 0, 0, 0 };
	
	/// <summary>
	/// Absolute position
	/// </summary>
	protected double[] position = new double[] { 0, 0, 0 };
	
	/// <summary>
	/// Orientation matrix
	/// </summary>
	protected double[,] matrix = new double[,] { { 1, 0, 0 }, { 0, 1, 0 }, { 0, 0, 1 } };
	
	/// <summary>
	/// Auxiliary array
	/// </summary>
	protected double[,] qq = new double[4, 4];
	
	/// <summary>
	/// Auxiliary array
	/// </summary>
	protected double[] p = new double[3];
	
	/// <summary>
	/// Parent frame
	/// </summary>
	protected IReferenceFrame parent;
	
	/// <summary>
	/// Parameters
	/// </summary>
	protected object parameters;
	
	/// <summary>
	/// Auxliary position
	/// </summary>
	private double[] auxPos = new double[3];
	
	\#endregion
	
	\#region Ctor
	
	/// <summary>
	/// Constructor
	/// </summary>
	public ReferenceFrame()
	{
	}
	
	/// <summary>
	/// Constructor
	/// </summary>
	/// <param name="b">Auxiliary</param>
	private ReferenceFrame(bool b)
	{
	}
	
	/// <summary>
	/// Absolute position
	/// </summary>
	public double[] Position
	{
		get { return position; }
	}
	
	/// <summary>
	/// Parent frame
	/// </summary>
	public virtual IReferenceFrame Parent
	{
		get
		{
			return parent;
		}
		set
		{
			parent = value;
		}
	}
	
	/// <summary>
	/// Position parameters
	/// </summary>
	public virtual object Parameters
	{
		get
		{
			return parameters;
		}
		set
		{
			parameters = value;
		}
	}
	
	/// <summary>
	/// Updates itself
	/// </summary>
	public virtual void Update()
	{
		ReferenceFrame p = ParentFrame;
		position = p.Position;
		quaternion = p.quaternion;
		matrix = p.matrix;
	}
	
	\#endregion
	
	\#region IOrientation Members
	
	/// <summary>
	/// Orientation quaternion
	/// </summary>
	public double[] Quaternion
	{
		get { return quaternion; }
	}
	
	/// <summary>
	/// Orientation matrix
	/// </summary>
	public double[,] Matrix
	{
		get { return matrix; }
	}
	
	\#endregion
	
	// ...
	
}
/// <summary>
/// Reference frame holder
/// </summary>
public interface IReferenceFrame : IPosition
{
	/// <summary>
	/// Own frame
	/// </summary>
	ReferenceFrame Own
	{
		get;
	}
	
	/// <summary>
	/// Children objects
	/// </summary>
	List<IPosition> Children
	{
		get;
	}
	
}
/// <summary>
/// Link of relative frame
/// </summary>
[Serializable()]
public class ReferenceFrameArrow : CategoryArrow, ISerializable, IRemovableObject
{
	\#region Fields
	
	IPosition source;
	
	IReferenceFrame target;
	
	
	/// <summary>
	/// Default constructor
	/// </summary>
	public ReferenceFrameArrow()
	{
		
	}
	
	
	/// <summary>
	/// Deserialization constructor
	/// </summary>
	/// <param name="info">Serialization info</param>
	/// <param name="context">Streaming context</param>
	protected ReferenceFrameArrow(SerializationInfo info, StreamingContext context)
	{
	}
	void ISerializable.GetObjectData(SerializationInfo info, StreamingContext context)
	{
	}
	
	/// <summary>
	/// The source of this arrow
	/// </summary>
	public override ICategoryObject Source
	{
		get
		{
			return source as ICategoryObject;
		}
		set
		{
			IPosition position = value.GetSource<IPosition>();
			if (position.Parent != null)
			{
				throw new CategoryException("Root", this);
			}
			source = position;
		}
	}
	
	/// <summary>
	/// The target of this arrow
	/// </summary>
	public override ICategoryObject Target
	{
		get
		{
			return target as ICategoryObject;
		}
		set
		{
			IReferenceFrame rf = value.GetTarget<IReferenceFrame>();
			IAssociatedObject sa = source as IAssociatedObject;
			IAssociatedObject ta = value as IAssociatedObject;
			INamedComponent ns = sa.Object as INamedComponent;
			INamedComponent nt = ta.Object as INamedComponent;
			target = rf;
			source.Parent = target;
			target.Children.Add(source);
		}
	}
	
	
	\#endregion
	
	
	
	//...
	
}
Some code is omitted, a reader can find it in source files. Let us consider a following picture:

POINT AND FRAME

Type of the Point is Position and the Frame implements IReferenceFrame interface. The Link arrow means that motion of Point is relative with respect to Frame. Absolute coordinates of Point are calculated by following way: 

Xa = XF + A11Xr + A12Yr + A13Zr;

Ya = YF + A21Xr + A22Yr + A23Zr;

Za = ZF + A31Xr + A32Yr + A33Zr.

where 
* Xa, Ya, Za absolute coordinates of Point  
* XF, YF, ZF absolute coordinates of Frame  
* Xr, Yr, Zr relative coordinates of Point  
* A11,..., A33 - elements of {3D rotation matrix}. 
6. Animated objects

There is a many to many relation of following objects:
* 3D shapes ({Visual3D class}); 
* Virtual videocameras ({PerspectiveCamera class}). 
Following movies contains 7 shapes and 5 cameras with independent motion

{[Helicopter]}

Helicoper contains following shapes:
* Fuselage; 
* Tail rotor; 
* 5 blades of main rotor. 
Independent motion of blades is due to {main rotor assembly}

[helicopter Rotor]

WPF allows usage of single Visual3D in single camera only. So we need for copies of 3D shapes. Above picture contains 5 * 7 + 5 - 40 animatable ({IAnimatable interface}), they are 5 cameras and 5 copies for each 7 Visual3D objects. However simulation of motion is performed for 5 + 7 = 12 objects (5 cameras and 7 3D shapes). So we need 12 objects of the {ReferenceFrame} type. Following interface is responsible for interoperability between animated objects and objects of 6D motion simulation.
/// <summary>
/// Linear forecast
/// </summary>
public interface ILinear6DForecast
{
	
	/// <summary>
	/// Frame
	/// </summary>
	ReferenceFrame ReferenceFrame
	{
		get;
	}
	
	
	/// <summary>
	/// Forecast time
	/// </summary>
	TimeSpan ForecastTime
	{
		get;
		set;
	}
	
	/// <summary>
	/// Error of coordinate
	/// </summary>
	double CoordinateError
	{
		get;
		set;
	}
	
	/// <summary>
	/// Error of angle
	/// </summary>
	double AngleError
	{
		get;
		set;
	}
}
Above interface does not depend on 3D graphics technology, following interface is a WPF version.
/// <summary>
/// Animated object
/// </summary>
interface IAnimatedObject : ILinear6DForecast
{
	/// <summary>
	/// Initialze animation
	/// </summary>
	void InitAnimation(AnimationType animationType);
	
	/// <summary>
	/// Initialze animation
	/// </summary>
	/// <param name="animationType">Type of animation</param>
	/// <param name="changeFrameTime">Time of frame change</param>
	void InitRealtime(AnimationType animationType, double[] changeFrameTime);
	
	/// <summary>
	/// Children
	/// </summary>
	AnimatableWrapper[] Children
	{
		get;
	}
	
	/// <summary>
	/// Change event
	/// </summary>
	event Action Change;
	
	/// <summary>
	/// Stops animation
	/// </summary>
	void StopAnimation();
	
	/// <summary>
	/// On stop action
	/// </summary>
	event Action OnStop;
	
	/// <summary>
	/// Supports animation events
	/// </summary>
	bool SupportsAnimationEvents
	{
		get;
		set;
	}
	
}
The Children property contains wrappers of animatable ({IAnimatable interface}). In previous picture every 3D shape contains 5 children, every perspective camera has a single child. Following code contains wrapper of animatable object.
/// <summary>
/// Wrapper of animatable object
/// </summary>
public class AnimatableWrapper : IDisposable
{
	
	private IAnimatable animatable;
	
	private DependencyProperty dependencyProperty;
	
	private object value;
	
	private Action change = () => { };
	
	private IAnimatedObject animatedObject;
	
	private Uniform6DTransformation transformation;
	
	private Action finish = () => { };
	
	double[] auxQuaternion = new double[4];
	
	bool isStopped = false;
	
	object loc = new object();
	internal AnimatableWrapper(IAnimatable animatable, DependencyProperty dependencyProperty,
	IAnimatedObject animatedObject, bool realtime, double[] changeFrameTime)
	{
		this.animatable = animatable;
		this.dependencyProperty = dependencyProperty;
		value = (animatable as DependencyObject).GetValue(dependencyProperty);
		this.animatedObject = animatedObject;
		transformation = new Uniform6DTransformation(this,animatedObject.ReferenceFrame,
		realtime, changeFrameTime, animatedObject.ForecastTime);
	}
	
	/// <summary>
	/// Animatable
	/// </summary>
	public IAnimatable Animatable
	{
		get
		{
			return animatable;
		}
	}
	
	/// <summary>
	/// Dependency property
	/// </summary>
	public DependencyProperty DependencyProperty
	{
		get
		{
			return dependencyProperty;
		}
	}
	
	
	
	internal IAnimatedObject Animated
	{
		get
		{
			return animatedObject;
		}
	}
	
	internal void Stop()
	{
		lock (loc)
		{
			transformation.Stop();
			isStopped = true;
			finish();
		}
	}
	
	internal void StartRealtime(double time, DateTime start)
	{
		transformation.StartRealtime(time, start);
	}
	
	internal void Init(double[] coord, double[] quaternion)
	{
		transformation.Init(coord, quaternion);
	}
	
	internal void StartAnimation(double[] coord, double[] quaternion, DateTime start)
	{
		transformation.StartAnimation(coord, quaternion, start);
	}
	
	
	internal void Enqueue(Tuple<TimeSpan, double[], double[]> parameters)
	{
		transformation.Enqueue(parameters);
	}
	
	internal void Finish()
	{
		finish();
	}
	
	internal event Action OnFinish
	{
		add { finish += value; }
		remove { finish -= value; }
	}
	
	internal  Action Event
	{
		get
		{
			return transformation.Event;
		}
	}
	
	void IDisposable.Dispose()
	{
		(animatable as DependencyObject).SetValue(dependencyProperty, value);
		animatedObject.Change -= change;
	}
	
}
Above class contains a transformation field of the {Uniform6DTransformation : AnimationTimeline} type. This field is responsible for animation.

7. Real-time

Recently I wrote an {article devoted to real-time}. This section can be regarded as an animation extension of the real-time simulation. Above animation is explained by following picture.

[Asynchronous animation]

However above scheme is not sufficient for real-time because animation lag behind simulation. Sufficient scheme is presented below.

[Real-time]

This scheme implies usage of the {linear prediction}. Both {velocity vector} and {angular velocity} vectors supply necessary data for linear prediction. My software supports calculation of velocity and angular velocity for different engineering problems (See {here} and {here}). Now these calculations are used for the asynchronous animation. Following interfaces are responsible for calculation of velocity vectors.
/// <summary>
/// Object with linear velocity
/// </summary>
public interface IVelocity
{
	/// <summary>
	/// Linear velocity
	/// </summary>
	double[] Velocity
	{
		get;
	}
	
}

/// <summary>
/// Object that have angular velocity
/// </summary>
public interface IAngularVelocity
{
	/// <summary>
	/// Angular velocity of object
	/// </summary>
	double[] Omega
	{
		get;
	}
}
Following code explains application of these interfaces for linear prediction.
/// <summary>
/// Reference frame
/// </summary>
ReferenceFrame frame;

/// <summary>
/// Velocity object
/// </summary>
IVelocity velocity;

/// <summary>
/// Angular velocity object
/// </summary>
IAngularVelocity angularVelocity;

// ...


/// <summary>
/// Initialization of prediction
/// </summary>
public void InitializePrediction(double forecastTime)
{
	if (!(frame is IVelocity))
	{
		throw new Exception("Frame does not support velocity");
	}
	if (!(frame is IAngularVelocity))
	{
		throw new Exception("Frame does not support angular velocity");
	}
	velocity = frame as IVelocity;
	angularVelocity = frame as IAngularVelocity;
}

/// <summary>
/// Initialization of prediction
/// </summary>
public void InitializePrediction(double forecastTime)
{
	if (!(frame is IVelocity))
	{
		throw new Exception("Frame does not support velocity");
	}
	if (!(frame is IAngularVelocity))
	{
		throw new Exception("Frame does not support angular velocity");
	}
	velocity = frame as IVelocity;
	angularVelocity = frame as IAngularVelocity;
}

/// <summary>
/// Linear prediction
/// </summary>
/// <param name="time">Current time</param>
private void LinearPrediction(double time)
{
	lastTime = time;
	double delta = time - changeFrameTime[0]; // Forecast time
	double[] coord = frame.Position;          // Coordinates of frame
	double[] v = velocity.Velocity;           // Velocity vector
	for (int i = 0; i < 3; i++)
	{
		auxVectorFrame[i] = coord[i] + v[i] * delta; // Linear prediction of coordinates
	}
	double[] omega = angularVelocity.Omega; // Angular velocity vector
	double mod = omega[0] * omega[0] + omega[1] * omega[1] + omega[2] * omega[2];
	mod = Math.Sqrt(mod);
	Array.Copy(omega, 0, auxQuarterInter, 1, 3); // Calculation of "shift" quaternion
	double angle = 0.5 * mod * delta;
	double s = Math.Sin(angle);
	double c = Math.Cos(angle);
	auxQuarterInter[0] = c;
	double smod = s / mod;
	for (int i = 1; i < 3; i++)
	{
		auxQuarterInter[i] *= smod;
	}
	QuaternionMultiply(frame.Quaternion, auxQuarterInter, auxQuaterFrame); // Calculation of quaternion
}
8. Examples

8.1 Animation of a Cube

This sample is rather demo than realistic. Motion of cube is described by following finite formulas.

[Cube motion]

Above formulas are used as coordinates and components of 3D orientation quaternion.

[Cube motion parameters]

Following movie represents animation of the cube.

{[Modes of anination]}
* {Download cube sample 630 KB} 
8.2 Animation of a Helicopter

Motion of helicopter is also rather demo and it also is defined by finite formulas. But it includes 7 3D shapes and 5 virtual cameras. Following movie represents motion of helicopter.

{[Helicopter]}
* {Download helicopter sample 50 KB} 
* {Download additional files 247 KB} 
8.3 Animation of Artificial Earth's Satellite.

This sample is quite realistic because it uses following ingredients:
* Ordinary differential equations; 
* Model of {Earth's gravity}; 
* Model of {Earth's atmosphere}. 
Description of math model details is {here}. Following movie represents this animation.

{[Orbital movie]}
* {Download orbital motion sample 457 KB} 
* {Download additional files 247 KB} 
8.4 Manual Control of an Airplane

This sample is close to realistic. It uses real-time manual control of an airplane. Motion model of airplane is quite realistic because it uses realistic model atmosphere and an {aerodynamics} model. However manual control is not realistic. Instead a pilot wheel the "{roll pitch} robot pilot" is used. The {ForcedEventData} class is used for manual control. Object of this class performs following operations:
* Manual input of parameters ({roll and pitch}); 
* Raising of input event. 
User can update required values of {pith and roll}. Any update occurs a {transient state}. Following chart contains two transient states.

[Transient state]
. 

Real-time simulation is based on {event driven} calculations. Following scheme of events is used for airplane simulation.

[Events]

Our calculation is driven by Event collection object which is simultaneously a generator of events and an event handler. It is an object of the {EventCollection} class. The Event collection is a "sum" of Timer, Transition and Pilot Control events. The Timer raises one event per 2000 ms = 2 s. The Pilot Control raises event of user manual control. The Transition raises one event per 0.2 s during transition state. The Transition is an object of the TransientProcessEvent class. This class is simultaneously a source of event and event handler. Following code explains how does this class works.
/// <summary>
/// Transient State Event
/// </summary>
[Serializable()]
public class TransientProcessEvent : CategoryObject, IEvent, IEventHandler, ISerializable
{
	
	TimeSpan[] spans;
	
	private Action ev = () => { };
	
	object loc = new object();
	
	bool isEnabled = false;
	
	DateTime previous = DateTime.Now;
	
	bool isRunning = false;
	
	// ...
	
	
	/// <summary>
	/// Default constructor
	/// </summary>
	public TransientProcessEvent()
	{
	}
	
	// ...
	
	
	event Action IEvent.Event
	{
		add { ev += value; }
		remove { ev -= value; }
	}
	
	// ...
	
	
	void IEventHandler.Add(IEvent ev)
	{
		events.Add(ev);
		onAdd(ev);
	}
	
	// ...
	
	/// <summary>
	/// Enanble/Disable operation
	/// </summary>
	void Enable()
	{
		lock (loc)
		{
			isRunning = false;
			if (isEnabled)
			{
				previous = DateTime.Now;
				foreach (IEvent ev in events)
				{
					ev.Event += EventHandler; // Adds event handler
				}
				return;
			}
			foreach (IEvent ev in events)
			{
				ev.Event -= EventHandler;    // Removes event handler
			}
		}
	}
	
	/// <summary>
	/// Event handler
	/// </summary>
	void EventHandler()
	{
		if (isRunning)
		{
			previous = DateTime.Now;
			return;
		}
		lock (loc)
		{
			if (isRunning)
			{
				return;
			}
			isRunning = true;
			currentStep = 0;
			Action act = AsyncEvent; // Asynchronous event
			act.BeginInvoke(null, null);
		}
	}
	
	/// <summary>
	/// Asynchronous event
	/// </summary>
	void AsyncEvent()
	{
		for (int i = 0; i < spans.Length; i++)
		{
			Thread.Sleep(spans[i]); // Sleep
			ev();                   // Raise event
		}
	}
	
	
}
The AsyncEvent performs cyclic sleep operation and raising of event. Following movies explains how does it works. 

{[transient]}

After user action (mouse click/move) we have 10 updates distinguished by update per 0.2 s. In passive mode we have one update per 2 s. Following movie shows this sample.

{[Avia movie]}
* {Download sample of manually controlled airplane 9.6 MB} 
Points of Interest

I am a software developer since 1977. During development of this article I first time encoutered with deadlock.

	
	
\section{Conversion of 6D Kinematics}
section{Conclusion}
	The conclusion goes here.
	
	
	
	
	
	% if have a single appendix:
	%\appendix[Proof of the Zonklar Equations]
	% or
	%\appendix  % for no appendix heading
	% do not use \section anymore after \appendix, only \section*
	% is possibly needed
	
	% use appendices with more than one appendix
	% then use \section to start each appendix
	% you must declare a \section before using any
	% \subsection or using \label (\appendices by itself
	% starts a section numbered zero.)
	%
	
	
	\appendices
	\section{Proof of the First Zonklar Equation}
	Appendix one text goes here.
	
	% you can choose not to have a title for an appendix
	% if you want by leaving the argument blank
	\section{}
	Appendix two text goes here.
	
	
	% use section* for acknowledgment
	\ifCLASSOPTIONcompsoc
	% The Computer Society usually uses the plural form
	\section*{Acknowledgments}
	\else
	% regular IEEE prefers the singular form
	\section*{Acknowledgment}
	\fi
	
	
	The authors would like to thank...
	
	
	% Can use something like this to put references on a page
	% by themselves when using endfloat and the captionsoff option.
	\ifCLASSOPTIONcaptionsoff
	\newpage
	\fi
	
	
	
	% trigger a \newpage just before the given reference
	% number - used to balance the columns on the last page
	% adjust value as needed - may need to be readjusted if
	% the document is modified later
	%\IEEEtriggeratref{8}
	% The "triggered" command can be changed if desired:
	%\IEEEtriggercmd{\enlargethispage{-5in}}
	
	% references section
	
	% can use a bibliography generated by BibTeX as a .bbl file
	% BibTeX documentation can be easily obtained at:
	% http://mirror.ctan.org/biblio/bibtex/contrib/doc/
	% The IEEEtran BibTeX style support page is at:
	% http://www.michaelshell.org/tex/ieeetran/bibtex/
	%\bibliographystyle{IEEEtran}
	% argument is your BibTeX string definitions and bibliography database(s)
	%\bibliography{IEEEabrv,../bib/paper}
	%
	% <OR> manually copy in the resultant .bbl file
	% set second argument of \begin to the number of references
	% (used to reserve space for the reference number labels box)
	\begin{thebibliography}{1}
		\bibitem{goldblatt:topoi} Robert Goldblatt. \textit{Topoi: The Categorial Analysis of Logic}. Revised edition of XLVII 445. Studies in logic and the foundations of mathematics, vol. 98. North-Holland, Amsterdam, New York, and Oxford, 1984, xvi + 551 pp. 1984.
		
			\bibitem{cockett:dg}
	J.R.B. Cockett and G.S.H. Cruttwelly \emph{Differential structure, tangent structure, and SDG}, Applied Categorical Structures  April 2014, 2014
	\bibitem{kobayashi_nomizu:diff_geom} S. Kobayashi, K. Nomizu. {\it Foundations of Differential Geometry}. Volume 1. Interscience publishers a division of John Willey \& Sons, New York - London. 1963.
		
		\bibitem{IEEEhowto:kopka}
		H.~Kopka and P.~W. Daly, \emph{A Guide to {\LaTeX}}, 3rd~ed.\hskip 1em plus
		0.5em minus 0.4em\relax Harlow, England: Addison-Wesley, 1999.
		
	\end{thebibliography}
	
	% biography section
	% 
	% If you have an EPS/PDF photo (graphicx package needed) extra braces are
	% needed around the contents of the optional argument to biography to prevent
	% the LaTeX parser from getting confused when it sees the complicated
	% \includegraphics command within an optional argument. (You could create
	% your own custom macro containing the \includegraphics command to make things
	% simpler here.)
	%\begin{IEEEbiography}[{\includegraphics[width=1in,height=1.25in,clip,keepaspectratio]{mshell}}]{Michael Shell}
	% or if you just want to reserve a space for a photo:
	
	\begin{IEEEbiography}{Michael Shell}
		Biography text here.
	\end{IEEEbiography}
	
	% if you will not have a photo at all:
	\begin{IEEEbiographynophoto}{John Doe}
		Biography text here.
	\end{IEEEbiographynophoto}
	
	% insert where needed to balance the two columns on the last page with
	% biographies
	%\newpage
	
	\begin{IEEEbiographynophoto}{Jane Doe}
		Biography text here.
	\end{IEEEbiographynophoto}
	
	% You can push biographies down or up by placing
	% a \vfill before or after them. The appropriate
	% use of \vfill depends on what kind of text is
	% on the last page and whether or not the columns
	% are being equalized.
	
	%\vfill
	
	% Can be used to pull up biographies so that the bottom of the last one
	% is flush with the other column.
	%\enlargethispage{-5in}
	
	
	
	% that's all folks
\end{document}

